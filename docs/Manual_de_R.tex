% Options for packages loaded elsewhere
\PassOptionsToPackage{unicode}{hyperref}
\PassOptionsToPackage{hyphens}{url}
%
\documentclass[
]{book}
\usepackage{amsmath,amssymb}
\usepackage{lmodern}
\usepackage{ifxetex,ifluatex}
\ifnum 0\ifxetex 1\fi\ifluatex 1\fi=0 % if pdftex
  \usepackage[T1]{fontenc}
  \usepackage[utf8]{inputenc}
  \usepackage{textcomp} % provide euro and other symbols
\else % if luatex or xetex
  \usepackage{unicode-math}
  \defaultfontfeatures{Scale=MatchLowercase}
  \defaultfontfeatures[\rmfamily]{Ligatures=TeX,Scale=1}
\fi
% Use upquote if available, for straight quotes in verbatim environments
\IfFileExists{upquote.sty}{\usepackage{upquote}}{}
\IfFileExists{microtype.sty}{% use microtype if available
  \usepackage[]{microtype}
  \UseMicrotypeSet[protrusion]{basicmath} % disable protrusion for tt fonts
}{}
\makeatletter
\@ifundefined{KOMAClassName}{% if non-KOMA class
  \IfFileExists{parskip.sty}{%
    \usepackage{parskip}
  }{% else
    \setlength{\parindent}{0pt}
    \setlength{\parskip}{6pt plus 2pt minus 1pt}}
}{% if KOMA class
  \KOMAoptions{parskip=half}}
\makeatother
\usepackage{xcolor}
\IfFileExists{xurl.sty}{\usepackage{xurl}}{} % add URL line breaks if available
\IfFileExists{bookmark.sty}{\usepackage{bookmark}}{\usepackage{hyperref}}
\hypersetup{
  pdftitle={Manual de R},
  pdfauthor={Freddy Hernández; Olga Usuga},
  hidelinks,
  pdfcreator={LaTeX via pandoc}}
\urlstyle{same} % disable monospaced font for URLs
\usepackage{color}
\usepackage{fancyvrb}
\newcommand{\VerbBar}{|}
\newcommand{\VERB}{\Verb[commandchars=\\\{\}]}
\DefineVerbatimEnvironment{Highlighting}{Verbatim}{commandchars=\\\{\}}
% Add ',fontsize=\small' for more characters per line
\usepackage{framed}
\definecolor{shadecolor}{RGB}{248,248,248}
\newenvironment{Shaded}{\begin{snugshade}}{\end{snugshade}}
\newcommand{\AlertTok}[1]{\textcolor[rgb]{0.94,0.16,0.16}{#1}}
\newcommand{\AnnotationTok}[1]{\textcolor[rgb]{0.56,0.35,0.01}{\textbf{\textit{#1}}}}
\newcommand{\AttributeTok}[1]{\textcolor[rgb]{0.77,0.63,0.00}{#1}}
\newcommand{\BaseNTok}[1]{\textcolor[rgb]{0.00,0.00,0.81}{#1}}
\newcommand{\BuiltInTok}[1]{#1}
\newcommand{\CharTok}[1]{\textcolor[rgb]{0.31,0.60,0.02}{#1}}
\newcommand{\CommentTok}[1]{\textcolor[rgb]{0.56,0.35,0.01}{\textit{#1}}}
\newcommand{\CommentVarTok}[1]{\textcolor[rgb]{0.56,0.35,0.01}{\textbf{\textit{#1}}}}
\newcommand{\ConstantTok}[1]{\textcolor[rgb]{0.00,0.00,0.00}{#1}}
\newcommand{\ControlFlowTok}[1]{\textcolor[rgb]{0.13,0.29,0.53}{\textbf{#1}}}
\newcommand{\DataTypeTok}[1]{\textcolor[rgb]{0.13,0.29,0.53}{#1}}
\newcommand{\DecValTok}[1]{\textcolor[rgb]{0.00,0.00,0.81}{#1}}
\newcommand{\DocumentationTok}[1]{\textcolor[rgb]{0.56,0.35,0.01}{\textbf{\textit{#1}}}}
\newcommand{\ErrorTok}[1]{\textcolor[rgb]{0.64,0.00,0.00}{\textbf{#1}}}
\newcommand{\ExtensionTok}[1]{#1}
\newcommand{\FloatTok}[1]{\textcolor[rgb]{0.00,0.00,0.81}{#1}}
\newcommand{\FunctionTok}[1]{\textcolor[rgb]{0.00,0.00,0.00}{#1}}
\newcommand{\ImportTok}[1]{#1}
\newcommand{\InformationTok}[1]{\textcolor[rgb]{0.56,0.35,0.01}{\textbf{\textit{#1}}}}
\newcommand{\KeywordTok}[1]{\textcolor[rgb]{0.13,0.29,0.53}{\textbf{#1}}}
\newcommand{\NormalTok}[1]{#1}
\newcommand{\OperatorTok}[1]{\textcolor[rgb]{0.81,0.36,0.00}{\textbf{#1}}}
\newcommand{\OtherTok}[1]{\textcolor[rgb]{0.56,0.35,0.01}{#1}}
\newcommand{\PreprocessorTok}[1]{\textcolor[rgb]{0.56,0.35,0.01}{\textit{#1}}}
\newcommand{\RegionMarkerTok}[1]{#1}
\newcommand{\SpecialCharTok}[1]{\textcolor[rgb]{0.00,0.00,0.00}{#1}}
\newcommand{\SpecialStringTok}[1]{\textcolor[rgb]{0.31,0.60,0.02}{#1}}
\newcommand{\StringTok}[1]{\textcolor[rgb]{0.31,0.60,0.02}{#1}}
\newcommand{\VariableTok}[1]{\textcolor[rgb]{0.00,0.00,0.00}{#1}}
\newcommand{\VerbatimStringTok}[1]{\textcolor[rgb]{0.31,0.60,0.02}{#1}}
\newcommand{\WarningTok}[1]{\textcolor[rgb]{0.56,0.35,0.01}{\textbf{\textit{#1}}}}
\usepackage{longtable,booktabs,array}
\usepackage{calc} % for calculating minipage widths
% Correct order of tables after \paragraph or \subparagraph
\usepackage{etoolbox}
\makeatletter
\patchcmd\longtable{\par}{\if@noskipsec\mbox{}\fi\par}{}{}
\makeatother
% Allow footnotes in longtable head/foot
\IfFileExists{footnotehyper.sty}{\usepackage{footnotehyper}}{\usepackage{footnote}}
\makesavenoteenv{longtable}
\usepackage{graphicx}
\makeatletter
\def\maxwidth{\ifdim\Gin@nat@width>\linewidth\linewidth\else\Gin@nat@width\fi}
\def\maxheight{\ifdim\Gin@nat@height>\textheight\textheight\else\Gin@nat@height\fi}
\makeatother
% Scale images if necessary, so that they will not overflow the page
% margins by default, and it is still possible to overwrite the defaults
% using explicit options in \includegraphics[width, height, ...]{}
\setkeys{Gin}{width=\maxwidth,height=\maxheight,keepaspectratio}
% Set default figure placement to htbp
\makeatletter
\def\fps@figure{htbp}
\makeatother
\setlength{\emergencystretch}{3em} % prevent overfull lines
\providecommand{\tightlist}{%
  \setlength{\itemsep}{0pt}\setlength{\parskip}{0pt}}
\setcounter{secnumdepth}{5}
\usepackage{booktabs}
\usepackage[spanish]{babel}
\decimalpoint
\selectlanguage{spanish}

% Comandos para escribir nombres de paquetes, programas y codigos
\newcommand{\pkg}[1]{{\normalfont\fontseries{b}\selectfont #1}}
\let\proglang=\textsf
\let\code=\texttt


\usepackage{booktabs}
\usepackage{longtable}
\usepackage[bf,singlelinecheck=off]{caption}

\usepackage{framed,color}
\definecolor{shadecolor}{RGB}{248,248,248}

\renewcommand{\textfraction}{0.05}
\renewcommand{\topfraction}{0.8}
\renewcommand{\bottomfraction}{0.8}
\renewcommand{\floatpagefraction}{0.75}

\renewenvironment{quote}{\begin{VF}}{\end{VF}}
\let\oldhref\href
\renewcommand{\href}[2]{#2\footnote{\url{#1}}}

\ifxetex
  \usepackage{letltxmacro}
  \setlength{\XeTeXLinkMargin}{1pt}
  \LetLtxMacro\SavedIncludeGraphics\includegraphics
  \def\includegraphics#1#{% #1 catches optional stuff (star/opt. arg.)
    \IncludeGraphicsAux{#1}%
  }%
  \newcommand*{\IncludeGraphicsAux}[2]{%
    \XeTeXLinkBox{%
      \SavedIncludeGraphics#1{#2}%
    }%
  }%
\fi

\makeatletter
\newenvironment{kframe}{%
\medskip{}
\setlength{\fboxsep}{.8em}
 \def\at@end@of@kframe{}%
 \ifinner\ifhmode%
  \def\at@end@of@kframe{\end{minipage}}%
  \begin{minipage}{\columnwidth}%
 \fi\fi%
 \def\FrameCommand##1{\hskip\@totalleftmargin \hskip-\fboxsep
 \colorbox{shadecolor}{##1}\hskip-\fboxsep
     % There is no \\@totalrightmargin, so:
     \hskip-\linewidth \hskip-\@totalleftmargin \hskip\columnwidth}%
 \MakeFramed {\advance\hsize-\width
   \@totalleftmargin\z@ \linewidth\hsize
   \@setminipage}}%
 {\par\unskip\endMakeFramed%
 \at@end@of@kframe}
\makeatother

\renewenvironment{Shaded}{\begin{kframe}}{\end{kframe}}

%%%%%

\newenvironment{rmdblock}[1]
  {
  \begin{itemize}
  \renewcommand{\labelitemi}{
    \raisebox{-.7\height}[0pt][0pt]{
      {\setkeys{Gin}{width=3em,keepaspectratio}\includegraphics{images/#1}}
    }
  }
  \setlength{\fboxsep}{1em}
  \begin{kframe}
  \item
  }
  {
  \end{kframe}
  \end{itemize}
  }
\newenvironment{rmdnote}
  {\begin{rmdblock}{note}}
  {\end{rmdblock}}
\newenvironment{rmdcaution}
  {\begin{rmdblock}{caution}}
  {\end{rmdblock}}
\newenvironment{rmdimportant}
  {\begin{rmdblock}{important}}
  {\end{rmdblock}}
\newenvironment{rmdtip}
  {\begin{rmdblock}{tip}}
  {\end{rmdblock}}
\newenvironment{rmdwarning}
  {\begin{rmdblock}{warning}}
  {\end{rmdblock}}


%%%%%


\usepackage{makeidx}
\makeindex

\urlstyle{tt}

\usepackage{amsthm}
\makeatletter
\def\thm@space@setup{%
  \thm@preskip=8pt plus 2pt minus 4pt
  \thm@postskip=\thm@preskip
}
\makeatother

\frontmatter
\ifluatex
  \usepackage{selnolig}  % disable illegal ligatures
\fi
\usepackage[]{natbib}
\bibliographystyle{apalike}

\title{Manual de R}
\author{Freddy Hernández \and Olga Usuga}
\date{2021-07-26}

\begin{document}
\maketitle

% you may need to leave a few empty pages before the dedication page

%\cleardoublepage\newpage\thispagestyle{empty}\null
%\cleardoublepage\newpage\thispagestyle{empty}\null
%\cleardoublepage\newpage
\thispagestyle{empty}

\begin{center}

Gracias a Dios por todo lo que me ha dado.

%\includegraphics{images/dedication.pdf}
\end{center}

\setlength{\abovedisplayskip}{-5pt}
\setlength{\abovedisplayshortskip}{-5pt}

{
\setcounter{tocdepth}{1}
\tableofcontents
}
\hypertarget{prefacio}{%
\chapter*{Prefacio}\label{prefacio}}
\addcontentsline{toc}{chapter}{Prefacio}

\begin{center}\includegraphics[width=0.33\linewidth]{images/portada} \end{center}

Este libro fue creado con la intención de apoyar el aprendizaje del lenguaje de programación R en estudiantes de pregrado, especialización, maestría e investigadores, que necesiten realizar análisis estadísticos. En este libro se explica de una forma sencilla la utilidad de la principales funciones para realizar análisis estadístico.

\hypertarget{estructura-del-libro}{%
\section*{Estructura del libro}\label{estructura-del-libro}}
\addcontentsline{toc}{section}{Estructura del libro}

El libro está estructurado de la siguiente manera.

En el capítulo \ref{intro} se presenta una breve introducción sobre el lenguaje de programación R; en el capítulo \ref{objetos} se explican los tipos de objetos más comunes en R; en el capítulo \ref{estilo} se muestran las normas de estilo sugeridas para escribir código en R; el capítulo \ref{funbas} presenta las funciones básicas que todo usuario debe conocer para usar con éxito R; el capítulo \ref{creafun} trata sobre cómo crear funciones; el capítulo \ref{read} muestra como leer bases de datos desde R; en el capítulo \ref{tablas} se ilustra la forma para construir tablas de frecuencia; en el capítulo \ref{central} se muestra como obtener las diversas medidas de tendencial central para variables cuantitativas, el capítulo \ref{varia} muestra como calcular las medidas de variabilidad, en el capítulo \ref{posi} se ilustra cómo usar las funciones para obtener medidas de posición; en el capítulo \ref{correl} se muestra como obtener medidas de correlación entre pares de variables; en los capítulos \ref{discretas} y \ref{continuas} se tratan los temas de distribuciones discretas y continuas; en el capítulo \ref{loglik} se aborda el tema de verosimilitud; en el capítulo \ref{aproxint} se muestra el tema de aproximación de integrales.

\hypertarget{informaciuxf3n-del-software-y-convenciones}{%
\section*{Información del software y convenciones}\label{informaciuxf3n-del-software-y-convenciones}}
\addcontentsline{toc}{section}{Información del software y convenciones}

Para realizar este libro se usaron los paquetes de R \textbf{knitr}\index{knitr} \citep{xie2015} y \textbf{bookdown}\index{bookdown} \citep{R-bookdown}, estos paquetes permiten construir todo el libro desde R y sirven para incluir código que se ejecute de forma automática incluyendo las salidas y gráficos.

En todo el libro se presentarán códigos que el lector puede copiar y pegar en su consola de R para obtener los mismos resultados aquí presentados. Los códigos se destacan en una caja de color beis (o beige) similar a la mostrada a continuación.

\begin{Shaded}
\begin{Highlighting}[]
\DecValTok{4} \SpecialCharTok{+} \DecValTok{6}
\NormalTok{a }\OtherTok{\textless{}{-}} \FunctionTok{c}\NormalTok{(}\DecValTok{1}\NormalTok{, }\DecValTok{5}\NormalTok{, }\DecValTok{6}\NormalTok{)}
\DecValTok{5} \SpecialCharTok{*}\NormalTok{ a}
\DecValTok{1}\SpecialCharTok{:}\DecValTok{10}
\end{Highlighting}
\end{Shaded}

Los resultados o salidas obtenidos de cualquier código se destacan con dos símbolos de númeral (\texttt{\#\#}) al inicio de cada línea o renglón, esto quiere decir que todo lo que inicie con \texttt{\#\#} son resultados obtenidos y el usuario \textbf{NO} los debe copiar. Abajo se muestran los resultados obtenidos luego de correr el código anterior.

\begin{verbatim}
## [1] 10
\end{verbatim}

\begin{verbatim}
## [1]  5 25 30
\end{verbatim}

\begin{verbatim}
##  [1]  1  2  3  4  5  6  7  8  9 10
\end{verbatim}

\hypertarget{bloques-informativos}{%
\section*{Bloques informativos}\label{bloques-informativos}}
\addcontentsline{toc}{section}{Bloques informativos}

En varias partes del libro usaremos bloques informativos para resaltar algún aspecto importante. Abajo se encuentra un ejemplo de los bloques y su significado.

\begin{rmdnote}
Nota aclaratoria.
\end{rmdnote}

\begin{rmdtip}
Sugerencia.
\end{rmdtip}

\begin{rmdwarning}
Advertencia.
\end{rmdwarning}

\hypertarget{agradecimientos}{%
\section*{Agradecimientos}\label{agradecimientos}}
\addcontentsline{toc}{section}{Agradecimientos}

Agradecemos enormemente a todos los estudiantes, profesores e investigadores que han leído este libro y nos han retroalimentado con comentarios valiosos para mejorar el documento.

\begin{flushright}
Freddy Hernández Barajas

Olga Cecilia Usuga Manco
\end{flushright}

\hypertarget{sobre-los-autores}{%
\chapter*{Sobre los autores}\label{sobre-los-autores}}
\addcontentsline{toc}{chapter}{Sobre los autores}

Freddy Hernández Barajas es profesor asistente de la Universidad Nacional de Colombia adscrito a la Escuela de Estadística de la Facultad de Ciencias.

Olga Cecilia Usuga Manco es profesora asociada de la Universidad de Antioquia adscrita al Departamento de Ingeniería Industrial de la Facultad de Ingeniería.

\hypertarget{intro}{%
\chapter{Introducción}\label{intro}}

\hypertarget{oruxedgenes}{%
\section{Orígenes}\label{oruxedgenes}}

R es un lenguaje de programación usado para realizar procedimientos estadísticos y gráficos de alto nivel, este lenguaje fue creado en 1993 por los profesores e investigadores Robert Gentleman y Ross Ihaka. Inicialmente el lenguaje se usó para apoyar los cursos que tenían a su cargo los profesores, pero luego de ver la utilidad de la herramienta desarrollada, decidieron colocar copias de R en StatLib. A partir de 1995 el código fuente de R está disponible bajo licencia GNU GPL para sistemas operativos Windows, Macintosh y distribuciones Unix/Linux. La comunidad de usuarios de R en el mundo es muy grande y los usuarios cuentan con diferentes espacios para interactuar, a continuación una lista no exhaustiva de los sitios más populares relacionados con R:

\begin{itemize}
\tightlist
\item
  \href{https://www.r-bloggers.com/}{Rbloggers}.
\item
  \href{http://r-es.org/}{Comunidad hispana de R}.
\item
  \href{http://r.789695.n4.nabble.com/}{Nabble}.
\item
  \href{http://r-br.2285057.n4.nabble.com/}{Foro en portugués}.
\item
  \href{http://stackoverflow.com/questions/tagged/r}{Stackoverflow}.
\item
  \href{http://stats.stackexchange.com/questions/tagged/r}{Cross Validated}.
\item
  \href{https://stat.ethz.ch/mailman/listinfo/r-help}{R-Help Mailing List}.
\item
  \href{http://blog.revolutionanalytics.com/}{Revolutions}.
\item
  \href{https://www.r-statistics.com/}{R-statistics blog}.
\item
  \href{https://rdatamining.wordpress.com/}{RDataMining}.
\end{itemize}

En la siguiente figura están Robert Gentleman (izquierda) y Ross Ihaka (derecha) creadores de R.

\hypertarget{instalaciuxf3n}{%
\section{Instalación}\label{instalaciuxf3n}}

Para realizar la instalación de R usted debe visitar la página del CRAN (Comprehensive R Archive Network) disponible en este \href{https://cran.r-project.org/}{enlace}. Una vez ingrese a la página encontrará un cuadro similar al mostrado en la siguiente figura donde encontrará los enlaces de la instalación para los sistemas operativos Linux, Mac y Windows.

Supongamos que se desea instalar R en Windows, para esto se debe dar clic sobre el hiperenlace \textbf{Download R for Windows}. Una vez hecho esto se abrirá una página con el contenido mostrado en la siguiente figura. Luego se debe dar clic sobre el hiperenlace \textbf{install R for the first time}.

Luego de esto se abrirá otra página con un encabezado similar al mostrado en la siguiente figura. Al momento de capturar la figura la versión actual de R era 3.2.5 pero con certeza usted tendrá disponible la versión actualizada. Una vez allí uste debe dar clic sobre \textbf{Download R 3.2.5 for Windows} como es señalado por la flecha verde. Luego de esto se descargará el instalador R en el computador el cual deberá ser instalado con las opciones que vienen por defecto.

Se recomienda observar el siguiente video didáctico de instalación de R disponible en este \href{https://youtu.be/rzw1E6HxBFY}{enlace} para facilitar la tarea de instalación.

\hypertarget{apariencia-del-programa}{%
\section{Apariencia del programa}\label{apariencia-del-programa}}

Una vez que esté instalado R en su computador, usted podrá acceder a él por la lista de programas o por medio del acceso directo que quedó en el escritorio, en la siguiente figura se muestra la apariencia del acceso directo para ingresar a R.

Al abrir R aparecerá en la pantalla de su computador algo similar a lo que está en la siguiente figura. La ventana izquierda se llama consola y es donde se ingresan las instrucciones, una vez que se construye un gráfico se activa otra ventana llamada ventana gráfica. Cualquier usuario puede modificar la posición y tamaños de estas ventanas, puede cambiar el tipo y tamaño de las letras en la consola, para hacer esto se deben explorar las opciones de \textbf{editar} en la barra de herramientas.

\hypertarget{objetos}{%
\chapter{Tipos de objetos}\label{objetos}}

En R existen varios tipos de objectos que permiten que el usuario pueda almacenar la información para realizar procedimientos estadísticos y gráficos. Los principales objetos en R son vectores, matrices, arreglos, marcos de datos y listas. A continuación se presentan las características de estos objetos y la forma para crearlos.

\hypertarget{variables}{%
\section{Variables}\label{variables}}

Las variables sirven para almacenar un valor que luego vamos a utilizar en algún procedimiento.

Para hacer la asignación de un valor a alguna variable se utiliza el operador \texttt{\textless{}-} entre el valor y el nombre de la variable. A continuación un ejemplo sencillo.

\begin{Shaded}
\begin{Highlighting}[]
\NormalTok{x }\OtherTok{\textless{}{-}} \DecValTok{5}
\DecValTok{2} \SpecialCharTok{*}\NormalTok{ x }\SpecialCharTok{+} \DecValTok{3}
\end{Highlighting}
\end{Shaded}

\begin{verbatim}
## [1] 13
\end{verbatim}

En el siguiente ejemplo se crea la variable \texttt{pais} y se almacena el nombre Colombia, luego se averigua el número de caracteres de la variable \texttt{pais}.

\begin{Shaded}
\begin{Highlighting}[]
\NormalTok{pais }\OtherTok{\textless{}{-}} \StringTok{"Colombia"}
\FunctionTok{nchar}\NormalTok{(pais)}
\end{Highlighting}
\end{Shaded}

\begin{verbatim}
## [1] 8
\end{verbatim}

\hypertarget{vectores}{%
\section{\texorpdfstring{Vectores \index{vector} \label{vector}}{Vectores  }}\label{vectores}}

Los vectores vectores son arreglos ordenados en los cuales se puede almacenar información de tipo numérico (variable cuantitativa), alfanumérico (variable cualitativa) o lógico (\texttt{TRUE} o \texttt{FALSE}), pero no mezclas de éstos. La función de R para crear un vector es \texttt{c()} y que significa concatenar; dentro de los paréntesis de esta función se ubica la información a almacenar. Una vez construído el vector se acostumbra a etiquetarlo con un nombre corto y representativo de la información que almacena, la asignación se hace por medio del operador \texttt{\textless{}-} entre el nombre y el vector.

A continuación se presenta un ejemplo de cómo crear tres vectores que contienen las respuestas de cinco personas a tres preguntas que se les realizaron.

\begin{Shaded}
\begin{Highlighting}[]
\NormalTok{edad }\OtherTok{\textless{}{-}} \FunctionTok{c}\NormalTok{(}\DecValTok{15}\NormalTok{, }\DecValTok{19}\NormalTok{, }\DecValTok{13}\NormalTok{, }\ConstantTok{NA}\NormalTok{, }\DecValTok{20}\NormalTok{)}
\NormalTok{deporte }\OtherTok{\textless{}{-}} \FunctionTok{c}\NormalTok{(}\ConstantTok{TRUE}\NormalTok{, }\ConstantTok{TRUE}\NormalTok{, }\ConstantTok{NA}\NormalTok{, }\ConstantTok{FALSE}\NormalTok{, }\ConstantTok{TRUE}\NormalTok{)}
\NormalTok{comic\_fav }\OtherTok{\textless{}{-}} \FunctionTok{c}\NormalTok{(}\ConstantTok{NA}\NormalTok{, }\StringTok{\textquotesingle{}Superman\textquotesingle{}}\NormalTok{, }\StringTok{\textquotesingle{}Batman\textquotesingle{}}\NormalTok{, }\ConstantTok{NA}\NormalTok{, }\StringTok{\textquotesingle{}Batman\textquotesingle{}}\NormalTok{)}
\end{Highlighting}
\end{Shaded}

El vector \texttt{edad} es un vector cuantitativo y contiene las edades de las 5 personas. En la cuarta posición del vector se colocó el símbolo \texttt{NA} que significa \textbf{Not Available} debido a que no se registró la edad para esa persona. Al hacer una asignación se acostumbra a dejar un espacio antes y después del operador \texttt{\textless{}-} de asignación. El segundo vector es llamado \texttt{deporte} y es un vector lógico que almacena las respuestas a la pregunta de si la persona practica deporte, nuevamente aquí hay un \texttt{NA} para la tercera persona. El último vector \texttt{comic\_fav} contiene la información del cómic favorito de cada persona, como esta variable es cualitativa es necesario usar las comillas \texttt{\textquotesingle{}\ \textquotesingle{}} para encerrar las respuestas.

\begin{rmdwarning}
Cuando se usa \texttt{NA} para representar una información \textbf{Not Available} no se deben usar comillas.
\end{rmdwarning}

\begin{rmdnote}
Es posible usar comillas sencillas \texttt{\textquotesingle{}foo\textquotesingle{}} o comillas dobles \texttt{"foo"} para ingresar valores de una variable cualitativa.
\end{rmdnote}

Si se desea ver lo que está almacenado en cada uno de estos vectores, se debe escribir en la consola de R el nombre de uno de los objetos y luego se presiona la tecla \textbf{enter} o \textbf{intro}, al realizar esto lo que se obtiene se muestra a continuación.

\begin{Shaded}
\begin{Highlighting}[]
\NormalTok{edad}
\end{Highlighting}
\end{Shaded}

\begin{verbatim}
## [1] 15 19 13 NA 20
\end{verbatim}

\begin{Shaded}
\begin{Highlighting}[]
\NormalTok{deporte}
\end{Highlighting}
\end{Shaded}

\begin{verbatim}
## [1]  TRUE  TRUE    NA FALSE  TRUE
\end{verbatim}

\begin{Shaded}
\begin{Highlighting}[]
\NormalTok{comic\_fav}
\end{Highlighting}
\end{Shaded}

\begin{verbatim}
## [1] NA         "Superman" "Batman"   NA         "Batman"
\end{verbatim}

\begin{rmdnote}
Una variable es un vector de longitud uno.
\end{rmdnote}

\hypertarget{cuxf3mo-extraer-elementos-de-un-vector}{%
\subsection{¿Cómo extraer elementos de un vector?}\label{cuxf3mo-extraer-elementos-de-un-vector}}

Para extraer un elemento almacenado dentro un vector se usan los corchetes \texttt{{[}{]}} y dentro de ellos la posición o posiciones que interesan.

\hypertarget{ejemplo}{%
\subsection*{Ejemplo}\label{ejemplo}}
\addcontentsline{toc}{subsection}{Ejemplo}

Si queremos extraer la edad de la tercera persona escribimos el nombre del vector y luego \texttt{{[}3{]}} para indicar la tercera posición de \texttt{edad}, a continuación el código.

\begin{Shaded}
\begin{Highlighting}[]
\NormalTok{edad[}\DecValTok{3}\NormalTok{]}
\end{Highlighting}
\end{Shaded}

\begin{verbatim}
## [1] 13
\end{verbatim}

Si queremos conocer el cómic favorito de la segunda y quinta persona, escribimos el nombre del vector y luego, dentro de los corchetes, escribimos otro vector con las posiciones 2 y 5 que nos interesan así \texttt{{[}c(2,\ 5){]}}, a continuación el código.

\begin{Shaded}
\begin{Highlighting}[]
\NormalTok{comic\_fav[}\FunctionTok{c}\NormalTok{(}\DecValTok{2}\NormalTok{, }\DecValTok{5}\NormalTok{)]}
\end{Highlighting}
\end{Shaded}

\begin{verbatim}
## [1] "Superman" "Batman"
\end{verbatim}

Si nos interesan las respuestas de la práctica de deporte, excepto la de la persona 3, usamos \texttt{{[}-3{]}} luego del nombre del vector para obtener todo, excepto la tercera posición.

\begin{Shaded}
\begin{Highlighting}[]
\NormalTok{deporte[}\SpecialCharTok{{-}}\DecValTok{3}\NormalTok{]}
\end{Highlighting}
\end{Shaded}

\begin{verbatim}
## [1]  TRUE  TRUE FALSE  TRUE
\end{verbatim}

\begin{rmdwarning}
Si desea extraer varios posiciones de un vector NUNCA escriba esto: \texttt{mivector{[}2,\ 5,\ 7{]}}. Tiene que crear un vector con las posiciones y luego colocarlo dentro de los corchetes así: \texttt{mivector{[}c(2,\ 5,\ 7){]}}
\end{rmdwarning}

\hypertarget{matrices}{%
\section{Matrices}\label{matrices}}

Las matrices \index{matrices} son arreglos rectangulares de filas y columnas con información numérica, alfanumérica o lógica. Para construir una matriz se usa la función \texttt{matrix(\ )}. Por ejemplo, para crear una matriz de 4 filas y 5 columnas (de dimensión \(4 \times 5\)) con los primeros 20 números positivos se escribe el código siguiente en la consola.

\begin{Shaded}
\begin{Highlighting}[]
\NormalTok{mimatriz }\OtherTok{\textless{}{-}} \FunctionTok{matrix}\NormalTok{(}\AttributeTok{data=}\DecValTok{1}\SpecialCharTok{:}\DecValTok{20}\NormalTok{, }\AttributeTok{nrow=}\DecValTok{4}\NormalTok{, }\AttributeTok{ncol=}\DecValTok{5}\NormalTok{, }\AttributeTok{byrow=}\ConstantTok{FALSE}\NormalTok{)}
\end{Highlighting}
\end{Shaded}

El argumento \texttt{data} de la función sirve para indicar los datos que se van a almacenar en la matriz, los argumentos \texttt{nrow} y \texttt{ncol} sirven para definir la dimensión de la matriz y por último el argumento \texttt{byrow} sirve para indicar si la información contenida en \texttt{data} se debe ingresar por filas o no. Para observar lo que quedó almacenado en el objeto \texttt{mimatriz} se escribe en la consola el nombre del objeto seguido de la tecla \textbf{enter} o \textbf{intro}.

\begin{Shaded}
\begin{Highlighting}[]
\NormalTok{mimatriz}
\end{Highlighting}
\end{Shaded}

\begin{verbatim}
##      [,1] [,2] [,3] [,4] [,5]
## [1,]    1    5    9   13   17
## [2,]    2    6   10   14   18
## [3,]    3    7   11   15   19
## [4,]    4    8   12   16   20
\end{verbatim}

\hypertarget{cuxf3mo-extraer-elementos-de-una-matriz}{%
\subsection{¿Cómo extraer elementos de una matriz?}\label{cuxf3mo-extraer-elementos-de-una-matriz}}

Al igual que en el caso de los vectores, para extraer elementos almacenados dentro de una matriz se usan los corchetes \texttt{{[}\ ,\ {]}} y dentro, separado por una coma, el número de fila(s) y el número de columna(s) que nos interesan.

\hypertarget{ejemplo-1}{%
\subsection*{Ejemplo}\label{ejemplo-1}}
\addcontentsline{toc}{subsection}{Ejemplo}

Si queremos extraer el valor almacenado en la fila 3 y columna 4 usamos el siguiente código.

\begin{Shaded}
\begin{Highlighting}[]
\NormalTok{mimatriz[}\DecValTok{3}\NormalTok{, }\DecValTok{4}\NormalTok{]}
\end{Highlighting}
\end{Shaded}

\begin{verbatim}
## [1] 15
\end{verbatim}

Si queremos recuperar \textbf{toda} la fila 2 usamos el siguiente código.

\begin{Shaded}
\begin{Highlighting}[]
\NormalTok{mimatriz[}\DecValTok{2}\NormalTok{, ]  }\CommentTok{\# No se escribe nada luego de la coma}
\end{Highlighting}
\end{Shaded}

\begin{verbatim}
## [1]  2  6 10 14 18
\end{verbatim}

Si queremos recuperar \textbf{toda} la columna 5 usamos el siguiente código.

\begin{Shaded}
\begin{Highlighting}[]
\NormalTok{mimatriz[, }\DecValTok{5}\NormalTok{]  }\CommentTok{\# No se escribe nada antes de la coma}
\end{Highlighting}
\end{Shaded}

\begin{verbatim}
## [1] 17 18 19 20
\end{verbatim}

Si queremos recuperar la matriz original sin las columnas 2 y 4 usamos el siguiente código.

\begin{Shaded}
\begin{Highlighting}[]
\NormalTok{mimatriz[, }\SpecialCharTok{{-}}\FunctionTok{c}\NormalTok{(}\DecValTok{2}\NormalTok{, }\DecValTok{4}\NormalTok{)]  }\CommentTok{\# Las columnas como vector}
\end{Highlighting}
\end{Shaded}

\begin{verbatim}
##      [,1] [,2] [,3]
## [1,]    1    9   17
## [2,]    2   10   18
## [3,]    3   11   19
## [4,]    4   12   20
\end{verbatim}

Si queremos recuperar la matriz original sin la fila 1 ni columna 3 usamos el siguiente código.

\begin{Shaded}
\begin{Highlighting}[]
\NormalTok{mimatriz[}\SpecialCharTok{{-}}\DecValTok{1}\NormalTok{, }\SpecialCharTok{{-}}\DecValTok{3}\NormalTok{]  }\CommentTok{\# Signo de menos para eliminar}
\end{Highlighting}
\end{Shaded}

\begin{verbatim}
##      [,1] [,2] [,3] [,4]
## [1,]    2    6   14   18
## [2,]    3    7   15   19
## [3,]    4    8   16   20
\end{verbatim}

\hypertarget{arreglos}{%
\section{\texorpdfstring{Arreglos \index{arreglo} \index{array}}{Arreglos  }}\label{arreglos}}

Un arreglo es una matriz de varias dimensiones con información numérica, alfanumérica o lógica. Para construir una arreglo se usa la función \texttt{array(\ )}. Por ejemplo, para crear un arreglo de \(3 \times 4 \times 2\) con las primeras 24 letras minúsculas del alfabeto se escribe el siguiente código.

\begin{Shaded}
\begin{Highlighting}[]
\NormalTok{miarray }\OtherTok{\textless{}{-}} \FunctionTok{array}\NormalTok{(}\AttributeTok{data=}\NormalTok{letters[}\DecValTok{1}\SpecialCharTok{:}\DecValTok{24}\NormalTok{], }\AttributeTok{dim=}\FunctionTok{c}\NormalTok{(}\DecValTok{3}\NormalTok{, }\DecValTok{4}\NormalTok{, }\DecValTok{2}\NormalTok{))}
\end{Highlighting}
\end{Shaded}

El argumento \texttt{data} de la función sirve para indicar los datos que se van a almacenar en el arreglo y el argumento \texttt{dim} sirve para indicar las dimensiones del arreglo. Para observar lo que quedó almacenado en el objeto \texttt{miarray} se escribe en la consola lo siguiente.

\begin{Shaded}
\begin{Highlighting}[]
\NormalTok{miarray}
\end{Highlighting}
\end{Shaded}

\begin{verbatim}
## , , 1
## 
##      [,1] [,2] [,3] [,4]
## [1,] "a"  "d"  "g"  "j" 
## [2,] "b"  "e"  "h"  "k" 
## [3,] "c"  "f"  "i"  "l" 
## 
## , , 2
## 
##      [,1] [,2] [,3] [,4]
## [1,] "m"  "p"  "s"  "v" 
## [2,] "n"  "q"  "t"  "w" 
## [3,] "o"  "r"  "u"  "x"
\end{verbatim}

\hypertarget{cuxf3mo-extraer-elementos-de-un-arreglo}{%
\subsection{¿Cómo extraer elementos de un arreglo?}\label{cuxf3mo-extraer-elementos-de-un-arreglo}}

Para recuperar elementos almacenados en un arreglo se usan también corchetes, y dentro de los corchetes, las coordenadas del objeto de interés.

\hypertarget{ejemplo-2}{%
\subsection*{Ejemplo}\label{ejemplo-2}}
\addcontentsline{toc}{subsection}{Ejemplo}

Si queremos extraer la letra almacenada en la fila 1 y columna 3 de la segunda capa de \texttt{miarray} usamos el siguiente código.

\begin{Shaded}
\begin{Highlighting}[]
\NormalTok{miarray[}\DecValTok{1}\NormalTok{, }\DecValTok{3}\NormalTok{, }\DecValTok{2}\NormalTok{]  }\CommentTok{\# El orden es importante}
\end{Highlighting}
\end{Shaded}

\begin{verbatim}
## [1] "s"
\end{verbatim}

Si queremos extraer la segunda capa completa usamos el siguiente código.

\begin{Shaded}
\begin{Highlighting}[]
\NormalTok{miarray[,, }\DecValTok{2}\NormalTok{]  }\CommentTok{\# No se coloca nada en las primeras posiciones}
\end{Highlighting}
\end{Shaded}

\begin{verbatim}
##      [,1] [,2] [,3] [,4]
## [1,] "m"  "p"  "s"  "v" 
## [2,] "n"  "q"  "t"  "w" 
## [3,] "o"  "r"  "u"  "x"
\end{verbatim}

Si queremos extraer la tercera columna de todas las capas usamos el siguiente código.

\begin{Shaded}
\begin{Highlighting}[]
\NormalTok{miarray[, }\DecValTok{3}\NormalTok{,]  }\CommentTok{\# No se coloca nada en las primeras posiciones}
\end{Highlighting}
\end{Shaded}

\begin{verbatim}
##      [,1] [,2]
## [1,] "g"  "s" 
## [2,] "h"  "t" 
## [3,] "i"  "u"
\end{verbatim}

\hypertarget{marco-de-datos}{%
\section{\texorpdfstring{Marco de datos \index{marco de datos} \index{data.frame}}{Marco de datos  }}\label{marco-de-datos}}

El marco de datos marco de datos o \emph{data frame} es uno de los objetos más utilizados porque permite agrupar vectores con información de diferente tipo (numérica, alfanumérica o lógica) en un mismo objeto, la única restricción es que los vectores deben tener la misma longitud. Para crear un marco de datos se usa la función \texttt{data.frame(\ )}, como ejemplo vamos a crear un marco de datos con los vectores \texttt{edad}, \texttt{deporte} y \texttt{comic\_fav} definidos anteriormente.

\begin{Shaded}
\begin{Highlighting}[]
\NormalTok{mimarco }\OtherTok{\textless{}{-}} \FunctionTok{data.frame}\NormalTok{(edad, deporte, comic\_fav)}
\end{Highlighting}
\end{Shaded}

Una vez creado el objeto \texttt{mimarco} podemos ver el objeto escribiendo su nombre en la consola, a continuación se muestra lo que se obtiene.

\begin{Shaded}
\begin{Highlighting}[]
\NormalTok{mimarco}
\end{Highlighting}
\end{Shaded}

\begin{verbatim}
##   edad deporte comic_fav
## 1   15    TRUE      <NA>
## 2   19    TRUE  Superman
## 3   13      NA    Batman
## 4   NA   FALSE      <NA>
## 5   20    TRUE    Batman
\end{verbatim}

De la salida anterior vemos que el marco de datos tiene 3 variables (columnas) cuyos nombres coinciden con los nombres de los vectores creados anteriormente, los números consecutivos al lado izquierdo son sólo de referencia y permiten identificar la información para cada persona en la base de datos.

\hypertarget{cuxf3mo-extraer-elementos-de-un-marco-de-datos}{%
\subsection{¿Cómo extraer elementos de un marco de datos?}\label{cuxf3mo-extraer-elementos-de-un-marco-de-datos}}

Para recuperar las variables (columnas) almacenadas en un marco de datos se puede usar el operador \texttt{\$}, corchetes simples \texttt{{[}{]}} o corchetes dobles \texttt{{[}{[}{]}{]}}. A continuación algunos ejemplos para entender las diferencias entre estas opciones.

\hypertarget{ejemplo-3}{%
\subsection*{Ejemplo}\label{ejemplo-3}}
\addcontentsline{toc}{subsection}{Ejemplo}

Si queremos extraer la variable \texttt{deporte} del marco de datos \texttt{mimarco} como un vector usamos el siguiente código.

\begin{Shaded}
\begin{Highlighting}[]
\NormalTok{mimarco}\SpecialCharTok{$}\NormalTok{deporte  }\CommentTok{\# Se recomienda si el nombre es corto}
\end{Highlighting}
\end{Shaded}

\begin{verbatim}
## [1]  TRUE  TRUE    NA FALSE  TRUE
\end{verbatim}

Otra forma de recuperar la variable \texttt{deporte} como vector es indicando el número de la columna donde se encuentra la variable.

\begin{Shaded}
\begin{Highlighting}[]
\NormalTok{mimarco[, }\DecValTok{2}\NormalTok{]  }\CommentTok{\# Se recomienda si recordamos su ubicacion}
\end{Highlighting}
\end{Shaded}

\begin{verbatim}
## [1]  TRUE  TRUE    NA FALSE  TRUE
\end{verbatim}

Otra forma de extraer la variable \texttt{deporte} como vector es usando \texttt{{[}{[}{]}{]}} y dentro el nombre de la variable.

\begin{Shaded}
\begin{Highlighting}[]
\NormalTok{mimarco[[}\StringTok{"deporte"}\NormalTok{]]}
\end{Highlighting}
\end{Shaded}

\begin{verbatim}
## [1]  TRUE  TRUE    NA FALSE  TRUE
\end{verbatim}

Si usamos \texttt{mimarco{[}"deporte"{]}} el resultado es la variable \texttt{deporte} pero en forma de marco de datos, no en forma vectorial.

\begin{Shaded}
\begin{Highlighting}[]
\NormalTok{mimarco[}\StringTok{"deporte"}\NormalTok{]}
\end{Highlighting}
\end{Shaded}

\begin{verbatim}
##   deporte
## 1    TRUE
## 2    TRUE
## 3      NA
## 4   FALSE
## 5    TRUE
\end{verbatim}

Si queremos extraer un marco de datos sólo con las variables deporte y edad podemos usar el siguiente código.

\begin{Shaded}
\begin{Highlighting}[]
\NormalTok{mimarco[}\FunctionTok{c}\NormalTok{(}\StringTok{"deporte"}\NormalTok{, }\StringTok{"edad"}\NormalTok{)]}
\end{Highlighting}
\end{Shaded}

\begin{verbatim}
##   deporte edad
## 1    TRUE   15
## 2    TRUE   19
## 3      NA   13
## 4   FALSE   NA
## 5    TRUE   20
\end{verbatim}

Por otra, si queremos la \texttt{edad} de las personas que están en las posiciones 2 hasta 4 usamos el siguiente código.

\begin{Shaded}
\begin{Highlighting}[]
\NormalTok{mimarco[}\DecValTok{2}\SpecialCharTok{:}\DecValTok{4}\NormalTok{, }\DecValTok{1}\NormalTok{]}
\end{Highlighting}
\end{Shaded}

\begin{verbatim}
## [1] 19 13 NA
\end{verbatim}

\hypertarget{cuxf3mo-extraer-subconjuntos-de-un-marco-de-datos}{%
\subsection{\texorpdfstring{¿Cómo extraer subconjuntos de un marco de datos? \index{subset}}{¿Cómo extraer subconjuntos de un marco de datos? }}\label{cuxf3mo-extraer-subconjuntos-de-un-marco-de-datos}}

Para extraer partes de un marco de datos se puede utilizar la función \texttt{subset(x,\ subset,\ select)}. El parámetro \texttt{x} sirve para indicar el marco de datos original, el parámetro \texttt{subset} sirve para colocar la condición y el parámetro \texttt{select} sirve para quedarnos sólo con algunas de las variables del marco de datos. A continuación varios ejemplos de la función \texttt{subset} para ver su utilidad.

\hypertarget{ejemplos}{%
\subsection*{Ejemplos}\label{ejemplos}}
\addcontentsline{toc}{subsection}{Ejemplos}

Si queremos el marco de datos \texttt{mimarco} sólo con las personas que SI practican deporte usamos el siguiente código.

\begin{Shaded}
\begin{Highlighting}[]
\FunctionTok{subset}\NormalTok{(mimarco, }\AttributeTok{subset=}\NormalTok{deporte }\SpecialCharTok{==} \ConstantTok{TRUE}\NormalTok{)}
\end{Highlighting}
\end{Shaded}

\begin{verbatim}
##   edad deporte comic_fav
## 1   15    TRUE      <NA>
## 2   19    TRUE  Superman
## 5   20    TRUE    Batman
\end{verbatim}

Si queremos el marco de datos \texttt{mimarco} sólo con las personas mayores o iguales a 17 años usamos el siguiente código.

\begin{Shaded}
\begin{Highlighting}[]
\FunctionTok{subset}\NormalTok{(mimarco, }\AttributeTok{subset=}\NormalTok{edad }\SpecialCharTok{\textgreater{}=} \DecValTok{17}\NormalTok{)}
\end{Highlighting}
\end{Shaded}

\begin{verbatim}
##   edad deporte comic_fav
## 2   19    TRUE  Superman
## 5   20    TRUE    Batman
\end{verbatim}

Si queremos el submarco con deporte y comic de las personas menores de 20 años usamos el siguiente código.

\begin{Shaded}
\begin{Highlighting}[]
\FunctionTok{subset}\NormalTok{(mimarco, }\AttributeTok{subset=}\NormalTok{edad }\SpecialCharTok{\textless{}} \DecValTok{20}\NormalTok{, }\AttributeTok{select=}\FunctionTok{c}\NormalTok{(}\StringTok{\textquotesingle{}deporte\textquotesingle{}}\NormalTok{, }\StringTok{\textquotesingle{}comic\_fav\textquotesingle{}}\NormalTok{))}
\end{Highlighting}
\end{Shaded}

\begin{verbatim}
##   deporte comic_fav
## 1    TRUE      <NA>
## 2    TRUE  Superman
## 3      NA    Batman
\end{verbatim}

Si queremos el marco de datos \texttt{mimarco} sólo con las personas menores de 20 años y que SI practican deporte usamos el siguiente código.

\begin{Shaded}
\begin{Highlighting}[]
\FunctionTok{subset}\NormalTok{(mimarco, }\AttributeTok{subset=}\NormalTok{edad }\SpecialCharTok{\textless{}} \DecValTok{20} \SpecialCharTok{\&}\NormalTok{ deporte }\SpecialCharTok{==} \ConstantTok{TRUE}\NormalTok{)}
\end{Highlighting}
\end{Shaded}

\begin{verbatim}
##   edad deporte comic_fav
## 1   15    TRUE      <NA>
## 2   19    TRUE  Superman
\end{verbatim}

\hypertarget{ejemplo-4}{%
\subsection*{Ejemplo}\label{ejemplo-4}}
\addcontentsline{toc}{subsection}{Ejemplo}

Leer la base de datos medidas del cuerpo disponible en este enlace \url{https://raw.githubusercontent.com/fhernanb/datos/master/medidas_cuerpo}. Extraer de esta base de datos una sub-base o subconjunto que contenga sólo la edad, peso, altura y sexo de aquellos que miden más de 185 cm y pesan más de 80 kg.

\begin{Shaded}
\begin{Highlighting}[]
\NormalTok{url }\OtherTok{\textless{}{-}} \StringTok{\textquotesingle{}https://raw.githubusercontent.com/fhernanb/datos/master/medidas\_cuerpo\textquotesingle{}}
\NormalTok{dt1 }\OtherTok{\textless{}{-}} \FunctionTok{read.table}\NormalTok{(url, }\AttributeTok{header=}\NormalTok{T)}
\FunctionTok{dim}\NormalTok{(dt1)  }\CommentTok{\# Para conocer la dimensión de la base original}
\end{Highlighting}
\end{Shaded}

\begin{verbatim}
## [1] 36  6
\end{verbatim}

\begin{Shaded}
\begin{Highlighting}[]
\NormalTok{dt2 }\OtherTok{\textless{}{-}} \FunctionTok{subset}\NormalTok{(}\AttributeTok{x=}\NormalTok{dt1, }\AttributeTok{subset=}\NormalTok{altura }\SpecialCharTok{\textgreater{}} \DecValTok{185} \SpecialCharTok{\&}\NormalTok{ peso }\SpecialCharTok{\textgreater{}} \DecValTok{80}\NormalTok{,}
              \AttributeTok{select=}\FunctionTok{c}\NormalTok{(}\StringTok{\textquotesingle{}sexo\textquotesingle{}}\NormalTok{, }\StringTok{\textquotesingle{}edad\textquotesingle{}}\NormalTok{, }\StringTok{\textquotesingle{}peso\textquotesingle{}}\NormalTok{, }\StringTok{\textquotesingle{}altura\textquotesingle{}}\NormalTok{))}
\NormalTok{dt2  }\CommentTok{\# Para mostrar la base de datos final}
\end{Highlighting}
\end{Shaded}

\begin{verbatim}
##      sexo edad peso altura
## 1  Hombre   43 87.3  188.0
## 6  Hombre   33 85.9  188.0
## 15 Hombre   30 98.2  190.5
\end{verbatim}

Al almacenar la nueva base de datos en el objeto \texttt{dt2} se puede manipular este nuevo objeto para realizar los análisis de interés.

\hypertarget{listas}{%
\section{\texorpdfstring{Listas \index{lista} \index{list}}{Listas  }}\label{listas}}

Las listas son otro tipo de objeto muy usado para almacenar objetos de diferente tipo. La instrucción para crear una lista es \texttt{list(\ )}. A continuación vamos a crear una lista que contiene tres objetos: un vector con 5 números aleatorios llamado \texttt{mivector}, una matriz de dimensión \(6 \times 2\) con los primeros doce números enteros positivos llamada \texttt{matriz2} y el tercer objeto será el marco de datos \texttt{mimarco} creado en el apartado anterior. Las instrucciones para crear la lista requerida se muestran a continuación.

\begin{Shaded}
\begin{Highlighting}[]
\FunctionTok{set.seed}\NormalTok{(}\DecValTok{12345}\NormalTok{)}
\NormalTok{mivector }\OtherTok{\textless{}{-}} \FunctionTok{runif}\NormalTok{(}\AttributeTok{n=}\DecValTok{5}\NormalTok{)}
\NormalTok{matriz2 }\OtherTok{\textless{}{-}} \FunctionTok{matrix}\NormalTok{(}\AttributeTok{data=}\DecValTok{1}\SpecialCharTok{:}\DecValTok{12}\NormalTok{, }\AttributeTok{ncol=}\DecValTok{6}\NormalTok{)}
\NormalTok{milista }\OtherTok{\textless{}{-}} \FunctionTok{list}\NormalTok{(}\AttributeTok{E1=}\NormalTok{mivector, }\AttributeTok{E2=}\NormalTok{matriz2, }\AttributeTok{E3=}\NormalTok{mimarco)}
\end{Highlighting}
\end{Shaded}

La función \texttt{set.seed} de la línea número 1 sirve para fijar la semilla de tal manera que los números aleatorios generados en la segunda línea con la función \texttt{runif} sean siempre los mismos. En la última línea del código anterior se construye la lista, dentro de la función \texttt{list} se colocan los tres objetos \texttt{mivector}, \texttt{matriz2} y \texttt{mimarco}. Es posible colocarle un nombre especial a cada uno de los elementos de la lista, en este ejemplo se colocaron los nombres \texttt{E1}, \texttt{E2} y \texttt{E3} para cada uno de los tres elementos. Para observar lo que quedó almacenado en la lista se escribe \texttt{milista} en la consola y el resultado se muestra a continuación.

\begin{Shaded}
\begin{Highlighting}[]
\NormalTok{milista}
\end{Highlighting}
\end{Shaded}

\begin{verbatim}
## $E1
## [1] 0.7209039 0.8757732 0.7609823 0.8861246 0.4564810
## 
## $E2
##      [,1] [,2] [,3] [,4] [,5] [,6]
## [1,]    1    3    5    7    9   11
## [2,]    2    4    6    8   10   12
## 
## $E3
##   edad deporte comic_fav
## 1   15    TRUE      <NA>
## 2   19    TRUE  Superman
## 3   13      NA    Batman
## 4   NA   FALSE      <NA>
## 5   20    TRUE    Batman
\end{verbatim}

\hypertarget{cuxf3mo-extraer-elementos-de-una-lista}{%
\subsection{¿Cómo extraer elementos de una lista?}\label{cuxf3mo-extraer-elementos-de-una-lista}}

Para recuperar los elementos almacenadas en una lista se usa el operador \texttt{\$}, corchetes dobles \texttt{{[}{[}{]}{]}} o corchetes sencillos \texttt{{[}{]}}. A continuación unos ejemplos para entender cómo extraer elementos de una lista.

\hypertarget{ejemplos-1}{%
\subsection*{Ejemplos}\label{ejemplos-1}}
\addcontentsline{toc}{subsection}{Ejemplos}

Si queremos la matriz almacenada con el nombre de \texttt{E2} dentro del objeto \texttt{milista} se puede usar el siguiente código.

\begin{Shaded}
\begin{Highlighting}[]
\NormalTok{milista}\SpecialCharTok{$}\NormalTok{E2}
\end{Highlighting}
\end{Shaded}

\begin{verbatim}
##      [,1] [,2] [,3] [,4] [,5] [,6]
## [1,]    1    3    5    7    9   11
## [2,]    2    4    6    8   10   12
\end{verbatim}

Es posible indicar la posición del objeto en lugar del nombre, para eso se usan los corchetes dobles.

\begin{Shaded}
\begin{Highlighting}[]
\NormalTok{milista[[}\DecValTok{2}\NormalTok{]]}
\end{Highlighting}
\end{Shaded}

\begin{verbatim}
##      [,1] [,2] [,3] [,4] [,5] [,6]
## [1,]    1    3    5    7    9   11
## [2,]    2    4    6    8   10   12
\end{verbatim}

El resultado obtenido con \texttt{milista\$E2} y \texttt{milista{[}{[}2{]}{]}} es \textbf{exactamente} el mismo. Vamos ahora a solicitar la posición 2 pero usando corchetes sencillos.

\begin{Shaded}
\begin{Highlighting}[]
\NormalTok{milista[}\DecValTok{2}\NormalTok{]}
\end{Highlighting}
\end{Shaded}

\begin{verbatim}
## $E2
##      [,1] [,2] [,3] [,4] [,5] [,6]
## [1,]    1    3    5    7    9   11
## [2,]    2    4    6    8   10   12
\end{verbatim}

La apariencia de este último resultado es similar, no igual, al encontrado al usar \texttt{\$} y \texttt{{[}{[}{]}{]}}. Para ver la diferencia vamos a pedir la clase a la que pertenecen los tres últimos objetos usando la función \texttt{class}. A continuación el código usado.

\begin{Shaded}
\begin{Highlighting}[]
\FunctionTok{class}\NormalTok{(milista}\SpecialCharTok{$}\NormalTok{E2)}
\end{Highlighting}
\end{Shaded}

\begin{verbatim}
## [1] "matrix" "array"
\end{verbatim}

\begin{Shaded}
\begin{Highlighting}[]
\FunctionTok{class}\NormalTok{(milista[[}\DecValTok{2}\NormalTok{]])}
\end{Highlighting}
\end{Shaded}

\begin{verbatim}
## [1] "matrix" "array"
\end{verbatim}

\begin{Shaded}
\begin{Highlighting}[]
\FunctionTok{class}\NormalTok{(milista[}\DecValTok{2}\NormalTok{])}
\end{Highlighting}
\end{Shaded}

\begin{verbatim}
## [1] "list"
\end{verbatim}

De lo anterior se observa claramente que cuando usamos \texttt{\$} o \texttt{{[}{[}{]}{]}} el resultado es el objeto almacenado, una matriz. Cuando usamos \texttt{{[}{]}} el resultado es una \textbf{lista} cuyo contenido es el objeto almacendado.

\begin{rmdwarning}
Al manipular listas con \texttt{\$} y \texttt{{[}{[}{]}{]}} se obtienen los objetos ahí almacenados, al manipular listas con \texttt{{[}{]}} se obtiene una lista.
\end{rmdwarning}

\hypertarget{ejercicios}{%
\section*{EJERCICIOS}\label{ejercicios}}
\addcontentsline{toc}{section}{EJERCICIOS}

Use funciones o procedimientos (varias líneas) de R para responder cada una de las siguientes preguntas.

\begin{enumerate}
\def\labelenumi{\arabic{enumi}.}
\item
  Construya un vector con la primeras 20 letras MAYÚSCULAS usando la función LETTERS.
\item
  Construya una matriz de \(10 \times 10\) con los primeros 100 números positivos pares.
\item
  Construya una matriz identidad de dimension \(3 \times 3\). Recuerde que una matriz identidad tiene sólo unos en la diagonal principal y los demás elementos son cero.
\item
  Construya una lista con los anteriores tres objetos creados.
\item
  Construya un marco de datos o data frame con las respuestas de 3 personas a las preguntas: (a) ¿Cuál es su edad en años? (b) ¿Tipo de música que más le gusta? (c) ¿Tiene usted pareja sentimental estable?
\item
  ¿Cuál es el error al correr el siguiente código? ¿A qué se debe?
\end{enumerate}

\begin{Shaded}
\begin{Highlighting}[]
\NormalTok{edad }\OtherTok{\textless{}{-}} \FunctionTok{c}\NormalTok{(}\DecValTok{15}\NormalTok{, }\DecValTok{19}\NormalTok{, }\DecValTok{13}\NormalTok{, }\ConstantTok{NA}\NormalTok{, }\DecValTok{20}\NormalTok{)}
\NormalTok{deporte }\OtherTok{\textless{}{-}} \FunctionTok{c}\NormalTok{(}\ConstantTok{TRUE}\NormalTok{, }\ConstantTok{TRUE}\NormalTok{, }\ConstantTok{NA}\NormalTok{, }\ConstantTok{FALSE}\NormalTok{, }\ConstantTok{TRUE}\NormalTok{)}
\NormalTok{comic\_fav }\OtherTok{\textless{}{-}} \FunctionTok{c}\NormalTok{(}\ConstantTok{NA}\NormalTok{, }\StringTok{\textquotesingle{}Superman\textquotesingle{}}\NormalTok{, }\StringTok{\textquotesingle{}Batman\textquotesingle{}}\NormalTok{, }\ConstantTok{NA}\NormalTok{, }\StringTok{\textquotesingle{}Batman\textquotesingle{}}\NormalTok{)}
\FunctionTok{matrix}\NormalTok{(edad, deporte, comic\_fav)}
\end{Highlighting}
\end{Shaded}

\hypertarget{estilo}{%
\chapter{Guía de estilo}\label{estilo}}

Así como en el español existen reglas ortográficas, la escritura de códigos en R también tiene unas reglas que se recomienda seguir para evitar confusiones. Tener una buena guía de estilo es importante para que el código creado por usted sea fácilmente entendido por sus lectores \citep{rpackages}. No existe una única y mejor guía de estilo para escritura en R, sin embargo aquí vamos a mostrar unas sugerencias basadas en la guía llamada \href{https://style.tidyverse.org/}{The tidyverse style guidee}.

\hypertarget{nombres-de-los-archivos}{%
\section{Nombres de los archivos}\label{nombres-de-los-archivos}}

Se sugiere que el nombre usado para nombrar un archivo tenga sentido y que termine con extensión ``.R''. A continuación dos ejemplos de como nombrar bien y mal un archivo.

\begin{itemize}
\tightlist
\item
  Mal: \texttt{analisis\_icfes.R}
\item
  Bien: \texttt{ju89HR56\_74.R}
\end{itemize}

\hypertarget{nombres-de-los-objetos}{%
\section{Nombres de los objetos}\label{nombres-de-los-objetos}}

Se recomienda usar los símbolos \texttt{\_} dentro de los nombres de objetos.

\begin{itemize}
\tightlist
\item
  Para las variables es preferible usar letras minúsculas (\texttt{pesomaiz} o \texttt{peso\_maiz}) o utilizar la notación camello iniciando en minúscula (\texttt{pesoMaiz}).
\item
  Para las funciones se recomienda usar la notación camello iniciando todas la palabras en mayúscula (\texttt{PlotRes}).
\item
  Para los nombres de las constantes se recomienda que inicien con la letra k (\texttt{kPrecioBus}).
\end{itemize}

\hypertarget{longitud-de-una-luxednea-de-cuxf3digo}{%
\section{Longitud de una línea de código}\label{longitud-de-una-luxednea-de-cuxf3digo}}

Se recomienda que cada línea tenga como máximo 80 caracteres. Si una línea es muy larga se debe cortar siempre por una coma.

\hypertarget{espacios}{%
\section{Espacios}\label{espacios}}

Use espacios alrededor de todos los operadores binarios (=, +, -, \textless-, etc.). Los espacios alrededor del símbolo \texttt{=} son opcionales cuando se usan para ingresar valores dentro de una función. Así como en español, nunca coloque espacio antes de una coma, pero siempre use espacio luego de una coma. A continuación ejemplos de buenas y malas prácticas.

\begin{Shaded}
\begin{Highlighting}[]
\NormalTok{tab }\OtherTok{\textless{}{-}} \FunctionTok{table}\NormalTok{(df[df}\SpecialCharTok{$}\NormalTok{days }\SpecialCharTok{\textless{}} \DecValTok{0}\NormalTok{, }\DecValTok{2}\NormalTok{])  }\CommentTok{\# Bien}
\NormalTok{tot }\OtherTok{\textless{}{-}} \FunctionTok{sum}\NormalTok{(x[, }\DecValTok{1}\NormalTok{])                }\CommentTok{\# Bien}
\NormalTok{tot }\OtherTok{\textless{}{-}} \FunctionTok{sum}\NormalTok{(x[}\DecValTok{1}\NormalTok{, ])                }\CommentTok{\# Bien}
\NormalTok{tab }\OtherTok{\textless{}{-}} \FunctionTok{table}\NormalTok{(df[df}\SpecialCharTok{$}\NormalTok{days}\SpecialCharTok{\textless{}}\DecValTok{0}\NormalTok{, }\DecValTok{2}\NormalTok{])    }\CommentTok{\# Faltan espacios alrededor \textquotesingle{}\textless{}\textquotesingle{} }
\NormalTok{tab }\OtherTok{\textless{}{-}} \FunctionTok{table}\NormalTok{(df[df}\SpecialCharTok{$}\NormalTok{days }\SpecialCharTok{\textless{}} \DecValTok{0}\NormalTok{,}\DecValTok{2}\NormalTok{])   }\CommentTok{\# Falta espacio luego de coma}
\NormalTok{tab }\OtherTok{\textless{}{-}} \FunctionTok{table}\NormalTok{(df[df}\SpecialCharTok{$}\NormalTok{days }\SpecialCharTok{\textless{}} \DecValTok{0}\NormalTok{ , }\DecValTok{2}\NormalTok{]) }\CommentTok{\# Sobra espacio antes de coma}
\NormalTok{tab}\OtherTok{\textless{}{-}} \FunctionTok{table}\NormalTok{(df[df}\SpecialCharTok{$}\NormalTok{days }\SpecialCharTok{\textless{}} \DecValTok{0}\NormalTok{, }\DecValTok{2}\NormalTok{])   }\CommentTok{\# Falta espacio antes de \textquotesingle{}\textless{}{-}\textquotesingle{}}
\NormalTok{tab}\OtherTok{\textless{}{-}}\FunctionTok{table}\NormalTok{(df[df}\SpecialCharTok{$}\NormalTok{days }\SpecialCharTok{\textless{}} \DecValTok{0}\NormalTok{, }\DecValTok{2}\NormalTok{])    }\CommentTok{\# Falta espacio alrededor de \textquotesingle{}\textless{}{-}\textquotesingle{}}
\NormalTok{tot }\OtherTok{\textless{}{-}} \FunctionTok{sum}\NormalTok{(x[,}\DecValTok{1}\NormalTok{])                 }\CommentTok{\# Falta espacio luego de coma}
\NormalTok{tot }\OtherTok{\textless{}{-}} \FunctionTok{sum}\NormalTok{(x[}\DecValTok{1}\NormalTok{,])                 }\CommentTok{\# Falta espacio luego de coma}
\end{Highlighting}
\end{Shaded}

Otra buena práctica es colocar espacio antes de un paréntesis excepto cuando se llama una función.

\begin{Shaded}
\begin{Highlighting}[]
\ControlFlowTok{if}\NormalTok{ (debug)    }\CommentTok{\# Correcto}
\ControlFlowTok{if}\NormalTok{(debug)     }\CommentTok{\# Funciona pero no se recomienda}
\FunctionTok{colMeans}\NormalTok{ (x)  }\CommentTok{\# Funciona pero no se recomienda}
\end{Highlighting}
\end{Shaded}

Espacios extras pueden ser usados si con esto se mejora la apariencia del código, ver el ejemplo siguiente.

\begin{Shaded}
\begin{Highlighting}[]
\FunctionTok{plot}\NormalTok{(}\AttributeTok{x    =}\NormalTok{ x.coord,}
     \AttributeTok{y    =}\NormalTok{ data.mat[, }\FunctionTok{MakeColName}\NormalTok{(metric, ptiles[}\DecValTok{1}\NormalTok{], }\StringTok{"roiOpt"}\NormalTok{)],}
     \AttributeTok{ylim =}\NormalTok{ ylim,}
     \AttributeTok{xlab =} \StringTok{"dates"}\NormalTok{,}
     \AttributeTok{ylab =}\NormalTok{ metric,}
     \AttributeTok{main =}\NormalTok{ (}\FunctionTok{paste}\NormalTok{(metric, }\StringTok{" for 3 samples "}\NormalTok{, }\AttributeTok{sep =} \StringTok{""}\NormalTok{)))}
\end{Highlighting}
\end{Shaded}

No coloque espacios alrededor del código que esté dentro de paréntesis \texttt{(\ )} o corchetes \texttt{{[}\ {]}}, la única excepción es luego de una coma, ver el ejemplo siguiente.

\begin{Shaded}
\begin{Highlighting}[]
\ControlFlowTok{if}\NormalTok{ (condicion)    }\CommentTok{\# Correcto }
\NormalTok{x[}\DecValTok{1}\NormalTok{, ]            }\CommentTok{\# Correcto}
\ControlFlowTok{if}\NormalTok{ ( condicion )  }\CommentTok{\# Sobran espacios alrededor de condicion}
\NormalTok{x[}\DecValTok{1}\NormalTok{,]             }\CommentTok{\# Se necesita espacio luego de coma}
\end{Highlighting}
\end{Shaded}

Los signos de agrupación llaves \texttt{\{\ \}} se utilizan para agrupar bloques de código y se recomienda que nunca una llave abierta \texttt{\{} esté sola en una línea; una llave cerrada \texttt{\}} si debe ir sola en su propia línea. Se pueden omitir las llaves cuando el bloque de instrucciones esté formado por una sola línea pero esa línea de código NO debe ir en la misma línea de la condición. A continuación dos ejemplos de lo que se recomienda.

\begin{Shaded}
\begin{Highlighting}[]
\ControlFlowTok{if}\NormalTok{ (}\FunctionTok{is.null}\NormalTok{(ylim)) \{                     }\CommentTok{\# Correcto}
\NormalTok{  ylim }\OtherTok{\textless{}{-}} \FunctionTok{c}\NormalTok{(}\DecValTok{0}\NormalTok{, }\FloatTok{0.06}\NormalTok{)}
\NormalTok{\}}

\ControlFlowTok{if}\NormalTok{ (}\FunctionTok{is.null}\NormalTok{(ylim))                       }\CommentTok{\# Correcto}
\NormalTok{  ylim }\OtherTok{\textless{}{-}} \FunctionTok{c}\NormalTok{(}\DecValTok{0}\NormalTok{, }\FloatTok{0.06}\NormalTok{)}

\ControlFlowTok{if}\NormalTok{ (}\FunctionTok{is.null}\NormalTok{(ylim)) ylim }\OtherTok{\textless{}{-}} \FunctionTok{c}\NormalTok{(}\DecValTok{0}\NormalTok{, }\FloatTok{0.06}\NormalTok{)    }\CommentTok{\# Aceptable}

\ControlFlowTok{if}\NormalTok{ (}\FunctionTok{is.null}\NormalTok{(ylim))                       }\CommentTok{\# No se recomienda}
\NormalTok{\{        }
\NormalTok{  ylim }\OtherTok{\textless{}{-}} \FunctionTok{c}\NormalTok{(}\DecValTok{0}\NormalTok{, }\FloatTok{0.06}\NormalTok{)}
\NormalTok{\}}
    
\ControlFlowTok{if}\NormalTok{ (}\FunctionTok{is.null}\NormalTok{(ylim)) \{ylim }\OtherTok{\textless{}{-}} \FunctionTok{c}\NormalTok{(}\DecValTok{0}\NormalTok{, }\FloatTok{0.06}\NormalTok{)\}}
\CommentTok{\# Frente a la llave \{ no debe ir nada}
\CommentTok{\# la llave de cierre \} debe ir sola}
\end{Highlighting}
\end{Shaded}

La sentencia else debe ir siempre entre llaves \texttt{\}\ \{}, ver el siguiente ejemplo.

\begin{Shaded}
\begin{Highlighting}[]
\ControlFlowTok{if}\NormalTok{ (condition) \{         }
\NormalTok{  one or more lines}
\NormalTok{\} }\ControlFlowTok{else}\NormalTok{ \{                 }\CommentTok{\# Correcto}
\NormalTok{  one or more lines}
\NormalTok{\}}


\ControlFlowTok{if}\NormalTok{ (condition) \{         }
\NormalTok{  one or more lines}
\NormalTok{\}}
\ControlFlowTok{else}\NormalTok{ \{                   }\CommentTok{\# Incorrecto}
\NormalTok{  one or more lines}
\NormalTok{\}}


\ControlFlowTok{if}\NormalTok{ (condition)           }
\NormalTok{  one line}
\ControlFlowTok{else}                     \CommentTok{\# Incorrecto}
\NormalTok{  one line}
\end{Highlighting}
\end{Shaded}

\hypertarget{asignaciuxf3n}{%
\section{Asignación}\label{asignaciuxf3n}}

Para realizar asignaciones se recomienda usar el símbolo \texttt{\textless{}-}, el símbolo de igualdad \texttt{=} no se recomienda usarlo para asignaciones.

\begin{Shaded}
\begin{Highlighting}[]
\NormalTok{x }\OtherTok{\textless{}{-}} \DecValTok{5}  \CommentTok{\# Correcto}
\NormalTok{x }\OtherTok{=} \DecValTok{5}   \CommentTok{\# No recomendado}
\end{Highlighting}
\end{Shaded}

Para una explicación más detallada sobre el símbolo de asignación se recomienda visitar este \href{http://www.win-vector.com/blog/2016/12/the-case-for-using-in-r/}{enlace}.

\hypertarget{punto-y-coma}{%
\section{Punto y coma}\label{punto-y-coma}}

No se recomienda colocar varias instrucciones separadas por \texttt{;} en la misma línea, aunque funciona dificulta la revisión del código.

\begin{Shaded}
\begin{Highlighting}[]
\NormalTok{n }\OtherTok{\textless{}{-}} \DecValTok{100}\NormalTok{; y }\OtherTok{\textless{}{-}} \FunctionTok{rnorm}\NormalTok{(n, }\AttributeTok{mean=}\DecValTok{5}\NormalTok{); }\FunctionTok{hist}\NormalTok{(y)  }\CommentTok{\# No se recomienda}

\NormalTok{n }\OtherTok{\textless{}{-}} \DecValTok{100}                                  \CommentTok{\# Correcto}
\NormalTok{y }\OtherTok{\textless{}{-}} \FunctionTok{rnorm}\NormalTok{(n, }\AttributeTok{mean=}\DecValTok{5}\NormalTok{)}
\FunctionTok{hist}\NormalTok{(y)}
\end{Highlighting}
\end{Shaded}

A pesar de la anterior advertencia es posible que en este libro usemos el \texttt{;} en algunas ocasiones, si lo hacemos es para ahorrar espacio en la presentación del código.

\hypertarget{ingresando-datos-a-r}{%
\chapter{Ingresando datos a R}\label{ingresando-datos-a-r}}

En este capítulo se muestra como ingresar datos a R.

\hypertarget{usando-la-consola}{%
\section{Usando la consola}\label{usando-la-consola}}

La función \texttt{readline} básica de R sirve para escribir un mensaje en la consola y solicitar al usuario una información que luego se puede utilizar para realizar alguna operación.

Abajo se muestra un código de R que el lector puede copiar y pegar en un script. El código se debe ejecutar línea por línea y no en bloque.

La primera línea solicita el nombre del usuario y lo almacena automáticamente en la variable \texttt{my\_name}. La segunda línea solicita la edad y la almacena en la variable \texttt{my\_age}. La tercera se asegura que la edad se convierta a un número entero. La cuarta instrucción, escrita en varias líneas, saluda y entrega la edad del usuario en el próximo año.

\begin{Shaded}
\begin{Highlighting}[]
\NormalTok{my\_name }\OtherTok{\textless{}{-}} \FunctionTok{readline}\NormalTok{(}\AttributeTok{prompt=}\StringTok{"Ingrese su nombre: "}\NormalTok{)}
\NormalTok{my\_age  }\OtherTok{\textless{}{-}} \FunctionTok{readline}\NormalTok{(}\AttributeTok{prompt=}\StringTok{"Ingrese su edad en años: "}\NormalTok{)}
\NormalTok{my\_age  }\OtherTok{\textless{}{-}} \FunctionTok{as.integer}\NormalTok{(my\_age) }\CommentTok{\# convert character into integer}

\FunctionTok{print}\NormalTok{(}\FunctionTok{paste}\NormalTok{(}\StringTok{"Hola,"}\NormalTok{, my\_name, }
            \StringTok{"el año siguiente usted tendra"}\NormalTok{, }
\NormalTok{            my\_age }\SpecialCharTok{+} \DecValTok{1}\NormalTok{, }
            \StringTok{"años de edad."}\NormalTok{))}
\end{Highlighting}
\end{Shaded}

\hypertarget{usando-ventana-emergente-con-svdialogs}{%
\section{\texorpdfstring{Usando ventana emergente con \textbf{svDialogs}}{Usando ventana emergente con svDialogs}}\label{usando-ventana-emergente-con-svdialogs}}

El paquete \textbf{svDialogs} se puede utilizar para crear ventanas emergentes con un mensaje y solicitando información que luego se puede utilizar para realizar alguna operación.

Abajo se muestra un código de R que el lector puede copiar y pegar en un script. El código se puede ejecutar en bloque.

Este ejemplo hace lo mismo que el ejemplo anterior pero con la ventaja de abrir una ventana emergente para mostrar un mensaje y solicitar alguna información. Lo primero que se debe hacer es cargar el paquete \textbf{svDialogs}, si aún no lo ha instalado puede hacer escribiendo

\begin{Shaded}
\begin{Highlighting}[]
\FunctionTok{install.packages}\NormalTok{(}\StringTok{"svDialogs"}\NormalTok{) }\CommentTok{\# Para instalar el paquete}
\FunctionTok{library}\NormalTok{(svDialogs)            }\CommentTok{\# Para usar el paquete}

\NormalTok{my\_name }\OtherTok{\textless{}{-}} \FunctionTok{dlgInput}\NormalTok{(}\AttributeTok{message=}\StringTok{"Ingrese su nombre: "}\NormalTok{)}\SpecialCharTok{$}\NormalTok{res}
\NormalTok{my\_age  }\OtherTok{\textless{}{-}} \FunctionTok{dlgInput}\NormalTok{(}\AttributeTok{message=}\StringTok{"Ingrese su edad en años: "}\NormalTok{)}\SpecialCharTok{$}\NormalTok{res}
\NormalTok{my\_age  }\OtherTok{\textless{}{-}} \FunctionTok{as.integer}\NormalTok{(my\_age) }\CommentTok{\# convert character into integer}

\FunctionTok{print}\NormalTok{(}\FunctionTok{paste}\NormalTok{(}\StringTok{"Hola,"}\NormalTok{, my\_name, }
            \StringTok{"el año siguiente usted tendá"}\NormalTok{, }
\NormalTok{            my\_age }\SpecialCharTok{+} \DecValTok{1}\NormalTok{, }
            \StringTok{"años de edad."}\NormalTok{))}
\end{Highlighting}
\end{Shaded}

En la siguiente figura se muestran las cajas solicitando la información.

Abajo la salida luego de correr todo el código del ejemplo.

\begin{verbatim}
## [1] "Hola, Pedro el año siguiente usted tendr 26 años de edad."
\end{verbatim}

\hypertarget{botones-para-responder}{%
\section{Botones para responder}\label{botones-para-responder}}

La función \texttt{winDialog} del paquete básico \_\_utils\_ sirve para crear botones de diálogo en Windows solamente. A continuación se muestra la forma de generar los 4 tipos de botones.

\begin{Shaded}
\begin{Highlighting}[]
\FunctionTok{library}\NormalTok{(utils)}

\FunctionTok{winDialog}\NormalTok{(}\AttributeTok{type=}\StringTok{"ok"}\NormalTok{, }\AttributeTok{message=}\StringTok{"¿Usted quiere BORRAR el archivo?"}\NormalTok{)}
\FunctionTok{winDialog}\NormalTok{(}\AttributeTok{type=}\StringTok{"okcancel"}\NormalTok{, }\AttributeTok{message=}\StringTok{"¿Usted quiere BORRAR el archivo?"}\NormalTok{)}
\FunctionTok{winDialog}\NormalTok{(}\AttributeTok{type=}\StringTok{"yesno"}\NormalTok{, }\AttributeTok{message=}\StringTok{"¿Usted quiere BORRAR el archivo?"}\NormalTok{)}
\FunctionTok{winDialog}\NormalTok{(}\AttributeTok{type=}\StringTok{"yesnocancel"}\NormalTok{, }\AttributeTok{message=}\StringTok{"¿Usted quiere BORRAR el archivo?"}\NormalTok{)}
\end{Highlighting}
\end{Shaded}

En la siguiente figura están las imágenes de los 4 tipos de botones.

A continuación se muestra un ejemplo de cómo usar un botón para preguntar y luego imprimir en la consola un mensaje dependiendo de la respuesta. Copie todo el siguiente código en la consola y vea el resultado.

\begin{Shaded}
\begin{Highlighting}[]
\NormalTok{answer }\OtherTok{\textless{}{-}} \FunctionTok{winDialog}\NormalTok{(}\AttributeTok{type=}\StringTok{"yesno"}\NormalTok{, }\AttributeTok{mess=}\StringTok{"¿Le sirvió mi sugerencia?"}\NormalTok{)}
\ControlFlowTok{if}\NormalTok{ (answer}\SpecialCharTok{==}\StringTok{\textquotesingle{}YES\textquotesingle{}}\NormalTok{) \{}\FunctionTok{print}\NormalTok{(}\StringTok{\textquotesingle{}Excelente!\textquotesingle{}}\NormalTok{)\} }\ControlFlowTok{else}\NormalTok{ \{}\FunctionTok{print}\NormalTok{(}\StringTok{\textquotesingle{}Lástima\textquotesingle{}}\NormalTok{)\}}
\end{Highlighting}
\end{Shaded}

\hypertarget{funbas}{%
\chapter{Funciones básicas de R}\label{funbas}}

En este capítulo se presentará lo que es una función y se mostrarán varias funciones básicas que son útiles para realizar diversas tareas.

\hypertarget{quuxe9-es-una-funciuxf3n-de-r}{%
\section{¿Qué es una función de R?}\label{quuxe9-es-una-funciuxf3n-de-r}}

En la figura de abajo se muestra una ilustración de lo que es una función o máquina general. Hay unas entradas (\emph{inputs}) que luego son procesadas dentro de la caja para generar unas salidas (\emph{outputs}). Un ejemplo de una función o máquina muy común en nuestras casas es la licuadora. Si a una licuadora le ingresamos leche, fresas, azúcar y hielo, el resultado será un delicioso jugo de fresa.

Las funciones en R se caracterizan por un nombre corto y que dé una idea de lo que hace la función. Los elementos que pueden ingresar (\emph{inputs}) a la función se llaman \textbf{parámetros} o \textbf{argumentos} y se ubican dentro de paréntesis, el cuerpo de la función se ubica dentro de llaves y es ahí donde se procesan los \emph{inputs} para convertirlos en \emph{outputs}, a continuación se muestra la estructura general de una función.

\begin{Shaded}
\begin{Highlighting}[]
\FunctionTok{nombre\_de\_funcion}\NormalTok{(parametro1, parametro2, ...) \{}
\NormalTok{  tareas internas}
\NormalTok{  tareas internas}
\NormalTok{  tareas internas}
\NormalTok{  salida}
\NormalTok{\}}
\end{Highlighting}
\end{Shaded}

Cuando usamos una función sólo debemos escribir bien el nombre e ingresar correctamente los parámetros de la función, el cuerpo de la función ni lo vemos ni lo debemos modificar. A continuación se presenta un ejemplo de cómo usar la función \texttt{mean} para calcular un promedio.

\begin{Shaded}
\begin{Highlighting}[]
\NormalTok{notas }\OtherTok{\textless{}{-}} \FunctionTok{c}\NormalTok{(}\FloatTok{4.0}\NormalTok{, }\FloatTok{1.3}\NormalTok{, }\FloatTok{3.8}\NormalTok{, }\FloatTok{2.0}\NormalTok{)  }\CommentTok{\# Notas de un estudiante}
\FunctionTok{mean}\NormalTok{(notas)}
\end{Highlighting}
\end{Shaded}

\begin{verbatim}
## [1] 2.775
\end{verbatim}

\hypertarget{operadores-de-asignaciuxf3n}{%
\section{Operadores de asignación}\label{operadores-de-asignaciuxf3n}}

En R se pueden hacer asignación de varias formas, a continuación se presentan los operadores disponibles para tal fin.

\begin{itemize}
\tightlist
\item
  \texttt{\textless{}-} este es el operador de asignación a izquierda, es el más usado y recomendado.
\item
  \texttt{-\textgreater{}} este es el operador de asignación a derecha, no es frecuente su uso.
\item
  \texttt{=} el símbolo igual sirve para hacer asignaciones pero \textbf{NO} se recomienda usarlo.
\item
  \texttt{\textless{}\textless{}-} este es un operador de asignación global y sólo debe ser usado por usuarios avanzados.
\end{itemize}

\hypertarget{ejemplo-5}{%
\subsection*{Ejemplo}\label{ejemplo-5}}
\addcontentsline{toc}{subsection}{Ejemplo}

Almacene los valores 5.3, 4.6 y 25 en los objetos \texttt{a}, \texttt{b} y \texttt{age} respectivamente, use diferentes símbolos de asignación.

Para hacer lo solicitado se podría usar el siguiente código.

\begin{Shaded}
\begin{Highlighting}[]
\NormalTok{a }\OtherTok{\textless{}{-}} \FloatTok{5.3} \CommentTok{\# Recomended}
\FloatTok{4.6} \OtherTok{{-}\textgreater{}}\NormalTok{ b }\CommentTok{\# It is not usual}
\NormalTok{age }\OtherTok{=} \DecValTok{25} \CommentTok{\# Not recomended}
\end{Highlighting}
\end{Shaded}

\begin{rmdimportant}
Aunque una asignación se puede hacer de tres formas diferentes, se recomienda sólo usar el símbolo \texttt{\textless{}-}.
\end{rmdimportant}

\hypertarget{operaciones-buxe1sicas}{%
\section{Operaciones básicas}\label{operaciones-buxe1sicas}}

En R se pueden hacer diversas operaciones usando operadores binarios. Este tipo de operadores se denomina binarios porque actuan entre dos objetos, a continuación el listado.

\begin{itemize}
\tightlist
\item
  \texttt{+} operador binario para sumar.
\item
  \texttt{-} operador binario para restar.
\item
  \texttt{*} operador binario para multiplicar.
\item
  \texttt{/} operador binario para dividir.
\item
  \texttt{\^{}} operador binario para potencia.
\item
  \texttt{\%/\%} operador binario para obtener el cociente en una división (número entero).
\item
  \texttt{\%\%} operador binario para obtener el residuo en una división.
\end{itemize}

A continuación se presentan ejemplos de cómo usar las anteriores funciones.

\begin{Shaded}
\begin{Highlighting}[]
\DecValTok{6} \SpecialCharTok{+} \DecValTok{4}  \CommentTok{\# Para sumar dos números}
\end{Highlighting}
\end{Shaded}

\begin{verbatim}
## [1] 10
\end{verbatim}

\begin{Shaded}
\begin{Highlighting}[]
\NormalTok{a }\OtherTok{\textless{}{-}} \FunctionTok{c}\NormalTok{(}\DecValTok{1}\NormalTok{, }\DecValTok{3}\NormalTok{, }\DecValTok{2}\NormalTok{)}
\NormalTok{b }\OtherTok{\textless{}{-}} \FunctionTok{c}\NormalTok{(}\DecValTok{2}\NormalTok{, }\DecValTok{0}\NormalTok{, }\DecValTok{1}\NormalTok{)  }\CommentTok{\# a y b de la misma dimensión}
\NormalTok{a }\SpecialCharTok{+}\NormalTok{ b  }\CommentTok{\# Para sumar los vectores a y b miembro a miembro}
\end{Highlighting}
\end{Shaded}

\begin{verbatim}
## [1] 3 3 3
\end{verbatim}

\begin{Shaded}
\begin{Highlighting}[]
\NormalTok{a }\SpecialCharTok{{-}}\NormalTok{ b  }\CommentTok{\# Para restar dos vectores a y b miembro a miembro}
\end{Highlighting}
\end{Shaded}

\begin{verbatim}
## [1] -1  3  1
\end{verbatim}

\begin{Shaded}
\begin{Highlighting}[]
\NormalTok{a }\SpecialCharTok{*}\NormalTok{ b  }\CommentTok{\# Para multiplicar}
\end{Highlighting}
\end{Shaded}

\begin{verbatim}
## [1] 2 0 2
\end{verbatim}

\begin{Shaded}
\begin{Highlighting}[]
\NormalTok{a }\SpecialCharTok{/}\NormalTok{ b  }\CommentTok{\# Para dividir}
\end{Highlighting}
\end{Shaded}

\begin{verbatim}
## [1] 0.5 Inf 2.0
\end{verbatim}

\begin{Shaded}
\begin{Highlighting}[]
\NormalTok{a }\SpecialCharTok{\^{}}\NormalTok{ b  }\CommentTok{\# Para potencia}
\end{Highlighting}
\end{Shaded}

\begin{verbatim}
## [1] 1 1 2
\end{verbatim}

\begin{Shaded}
\begin{Highlighting}[]
\DecValTok{7} \SpecialCharTok{\%/\%} \DecValTok{3}  \CommentTok{\# Para saber las veces que cabe 3 en 7}
\end{Highlighting}
\end{Shaded}

\begin{verbatim}
## [1] 2
\end{verbatim}

\begin{Shaded}
\begin{Highlighting}[]
\DecValTok{7} \SpecialCharTok{\%\%} \DecValTok{3}  \CommentTok{\# Para saber el residuo al dividir 7 entre 3}
\end{Highlighting}
\end{Shaded}

\begin{verbatim}
## [1] 1
\end{verbatim}

\hypertarget{pruebas-luxf3gicas}{%
\section{Pruebas lógicas}\label{pruebas-luxf3gicas}}

En R se puede verificar si un objeto cumple una condición dada, a continuación el listado de las pruebas usuales.

\begin{itemize}
\tightlist
\item
  \texttt{\textless{}} para saber si un número es menor que otro.
\item
  \texttt{\textgreater{}} para saber si un número es mayor que otro.
\item
  \texttt{==} para saber si un número es igual que otro.
\item
  \texttt{\textless{}=} para saber si un número es menor o igual que otro.
\item
  \texttt{\textgreater{}=} para saber si un número es mayor o igual que otro.
\end{itemize}

A continuación se presentan ejemplos de cómo usar las anteriores funciones.

\begin{Shaded}
\begin{Highlighting}[]
\DecValTok{5} \SpecialCharTok{\textless{}} \DecValTok{12}  \CommentTok{\# ¿Será 5 menor que 12?}
\end{Highlighting}
\end{Shaded}

\begin{verbatim}
## [1] TRUE
\end{verbatim}

\begin{Shaded}
\begin{Highlighting}[]
\CommentTok{\# Comparando objetos}
\NormalTok{x }\OtherTok{\textless{}{-}} \DecValTok{5}
\NormalTok{y }\OtherTok{\textless{}{-}} \DecValTok{20} \SpecialCharTok{/} \DecValTok{4}
\NormalTok{x }\SpecialCharTok{==}\NormalTok{ y  }\CommentTok{\# ¿Será x igual a y?}
\end{Highlighting}
\end{Shaded}

\begin{verbatim}
## [1] TRUE
\end{verbatim}

\begin{Shaded}
\begin{Highlighting}[]
\CommentTok{\# Usando vectores}
\NormalTok{a }\OtherTok{\textless{}{-}} \FunctionTok{c}\NormalTok{(}\DecValTok{1}\NormalTok{, }\DecValTok{3}\NormalTok{, }\DecValTok{2}\NormalTok{)}
\NormalTok{b }\OtherTok{\textless{}{-}} \FunctionTok{c}\NormalTok{(}\DecValTok{2}\NormalTok{, }\DecValTok{0}\NormalTok{, }\DecValTok{1}\NormalTok{)}
\NormalTok{a }\SpecialCharTok{\textgreater{}}\NormalTok{ b  }\CommentTok{\# Comparación término a término}
\end{Highlighting}
\end{Shaded}

\begin{verbatim}
## [1] FALSE  TRUE  TRUE
\end{verbatim}

\begin{Shaded}
\begin{Highlighting}[]
\NormalTok{a }\SpecialCharTok{==}\NormalTok{ b  }\CommentTok{\# Comparación de igualdad término a término}
\end{Highlighting}
\end{Shaded}

\begin{verbatim}
## [1] FALSE FALSE FALSE
\end{verbatim}

\hypertarget{ejemplo-6}{%
\subsection*{Ejemplo}\label{ejemplo-6}}
\addcontentsline{toc}{subsection}{Ejemplo}

Crear un vector con los números de 1 a 17 y extrater los números que son mayores o iguales a 12.

Primero se crear el vector \texttt{x} con los elementos del 1 al 17. La prueba lógica \texttt{x\ \textgreater{}=\ 12} se usa para evaluar la condición, el resultado es un vector de 17 posiciones con valores de \texttt{TRUE} o \texttt{FALSE} dependiendo de si la condición se cumple o no. Este vector lógico se coloca dentro de \texttt{x{[}\ {]}} para que al evaluar \texttt{x{[}x\ \textgreater{}=\ 12{]}} sólo aparezcan los valores del vector original que SI cumplen la condición. El código necesario se muestra a continuación.

\begin{Shaded}
\begin{Highlighting}[]
\NormalTok{x }\OtherTok{\textless{}{-}} \DecValTok{1}\SpecialCharTok{:}\DecValTok{17}  \CommentTok{\# Se crea el vector}
\NormalTok{x[x }\SpecialCharTok{\textgreater{}=} \DecValTok{12}\NormalTok{]  }\CommentTok{\# Se solicitan los valores que cumplen la condición}
\end{Highlighting}
\end{Shaded}

\begin{verbatim}
## [1] 12 13 14 15 16 17
\end{verbatim}

\hypertarget{ejemplo-7}{%
\subsection*{Ejemplo}\label{ejemplo-7}}
\addcontentsline{toc}{subsection}{Ejemplo}

Retome el marco de datos \texttt{mimarco} construído en la sección 2.4 y use una prueba lógica para extraer la información de las personas que tienen una edad superior o igual a 15 años.

Inicialmente vamos a construir nuevamente el objeto \texttt{mimarco} de la sección 2.4 usando el siguiente código.

\begin{Shaded}
\begin{Highlighting}[]
\NormalTok{mimarco }\OtherTok{\textless{}{-}} \FunctionTok{data.frame}\NormalTok{(}\AttributeTok{edad =} \FunctionTok{c}\NormalTok{(}\DecValTok{15}\NormalTok{, }\DecValTok{19}\NormalTok{, }\DecValTok{13}\NormalTok{, }\ConstantTok{NA}\NormalTok{, }\DecValTok{20}\NormalTok{), }
                      \AttributeTok{deporte =} \FunctionTok{c}\NormalTok{(}\ConstantTok{TRUE}\NormalTok{, }\ConstantTok{TRUE}\NormalTok{, }\ConstantTok{NA}\NormalTok{, }\ConstantTok{FALSE}\NormalTok{, }\ConstantTok{TRUE}\NormalTok{),}
                      \AttributeTok{comic\_fav =} \FunctionTok{c}\NormalTok{(}\ConstantTok{NA}\NormalTok{, }\StringTok{\textquotesingle{}Superman\textquotesingle{}}\NormalTok{, }\StringTok{\textquotesingle{}Batman\textquotesingle{}}\NormalTok{, }\ConstantTok{NA}\NormalTok{, }\StringTok{\textquotesingle{}Batman\textquotesingle{}}\NormalTok{))}

\NormalTok{mimarco  }\CommentTok{\# Para ver el contenido de mimarco}
\end{Highlighting}
\end{Shaded}

\begin{verbatim}
##   edad deporte comic_fav
## 1   15    TRUE      <NA>
## 2   19    TRUE  Superman
## 3   13      NA    Batman
## 4   NA   FALSE      <NA>
## 5   20    TRUE    Batman
\end{verbatim}

Para extraer de \texttt{mimarco} la información de las personas que tienen una edad superior o igual a 15 años se coloca dentro de corchetes la condición \texttt{mimarco\$edad\ \textgreater{}=\ 15}, esto servirá para chequear cuáles de las edades del vector \texttt{mimarco\$ead} cumplen la condición. El resultado de evaluar \texttt{mimarco\$edad\ \textgreater{}=\ 15} será un vector lógico (\texttt{TRUE} o \texttt{FALSE}), que al ser colocado dentro de \texttt{mimarco{[},{]}}, entregará la información de las personas que cumplen la condición. A continuación el código para extraer la información solicitada.

\begin{Shaded}
\begin{Highlighting}[]
\NormalTok{mimarco[mimarco}\SpecialCharTok{$}\NormalTok{edad }\SpecialCharTok{\textgreater{}=} \DecValTok{15}\NormalTok{, ]}
\end{Highlighting}
\end{Shaded}

\begin{verbatim}
##    edad deporte comic_fav
## 1    15    TRUE      <NA>
## 2    19    TRUE  Superman
## NA   NA      NA      <NA>
## 5    20    TRUE    Batman
\end{verbatim}

De la salida anterior se observa que 4 personas de las 5 cumplean la condición.

\begin{rmdwarning}
Note que la condición \texttt{mimarco\$edad\ \textgreater{}=\ 15} se debe ubicar \textbf{antes} de la coma para obtener todos individuos que cumplen con la condición.
\end{rmdwarning}

\hypertarget{operadores-luxf3gicos}{%
\section{Operadores lógicos}\label{operadores-luxf3gicos}}

En R están disponibles los operadores lógicos negación, conjunción y disyunción. A continuación el listado de los operadores entre los elementos \texttt{x} e \texttt{y}.

\begin{Shaded}
\begin{Highlighting}[]
\SpecialCharTok{!}\NormalTok{x  }\CommentTok{\# Negación de x}
\NormalTok{x }\SpecialCharTok{\&}\NormalTok{ y  }\CommentTok{\# Conjunción entre x e y}
\NormalTok{x }\SpecialCharTok{\&\&}\NormalTok{ y}
\NormalTok{x }\SpecialCharTok{|}\NormalTok{ y  }\CommentTok{\# Disyunción entre x e y}
\NormalTok{x }\SpecialCharTok{||}\NormalTok{ y}
\FunctionTok{xor}\NormalTok{(x, y)}
\end{Highlighting}
\end{Shaded}

A continuación se presentan ejemplos de cómo usar el símbolo de negación \texttt{!}.

\begin{Shaded}
\begin{Highlighting}[]
\NormalTok{ans }\OtherTok{\textless{}{-}} \FunctionTok{c}\NormalTok{(}\ConstantTok{TRUE}\NormalTok{, }\ConstantTok{FALSE}\NormalTok{, }\ConstantTok{TRUE}\NormalTok{)}
\SpecialCharTok{!}\NormalTok{ans  }\CommentTok{\# Negando las respuestas almacenadas en ans}
\end{Highlighting}
\end{Shaded}

\begin{verbatim}
## [1] FALSE  TRUE FALSE
\end{verbatim}

\begin{Shaded}
\begin{Highlighting}[]
\NormalTok{x }\OtherTok{\textless{}{-}} \FunctionTok{c}\NormalTok{(}\DecValTok{5}\NormalTok{, }\FloatTok{1.5}\NormalTok{, }\DecValTok{2}\NormalTok{, }\DecValTok{3}\NormalTok{, }\DecValTok{2}\NormalTok{)}
\SpecialCharTok{!}\NormalTok{(x }\SpecialCharTok{\textless{}} \FloatTok{2.5}\NormalTok{)  }\CommentTok{\# Negando los resultados de una prueba}
\end{Highlighting}
\end{Shaded}

\begin{verbatim}
## [1]  TRUE FALSE FALSE  TRUE FALSE
\end{verbatim}

A continuación se presentan ejemplos de cómo aplicar la conjunción \texttt{\&} y \texttt{\&\&}.

\begin{Shaded}
\begin{Highlighting}[]
\NormalTok{x }\OtherTok{\textless{}{-}} \FunctionTok{c}\NormalTok{(}\DecValTok{5}\NormalTok{, }\FloatTok{1.5}\NormalTok{, }\DecValTok{2}\NormalTok{)  }\CommentTok{\# Se construyen dos vectores para la prueba}
\NormalTok{y }\OtherTok{\textless{}{-}} \FunctionTok{c}\NormalTok{(}\DecValTok{4}\NormalTok{, }\DecValTok{6}\NormalTok{, }\DecValTok{3}\NormalTok{)}

\NormalTok{x }\SpecialCharTok{\textless{}} \DecValTok{4}  \CommentTok{\# ¿Serán los elementos de x menores que 4?}
\end{Highlighting}
\end{Shaded}

\begin{verbatim}
## [1] FALSE  TRUE  TRUE
\end{verbatim}

\begin{Shaded}
\begin{Highlighting}[]
\NormalTok{y }\SpecialCharTok{\textgreater{}} \DecValTok{5}  \CommentTok{\# ¿Serán los elementos de y mayores que 5?}
\end{Highlighting}
\end{Shaded}

\begin{verbatim}
## [1] FALSE  TRUE FALSE
\end{verbatim}

\begin{Shaded}
\begin{Highlighting}[]
\NormalTok{x }\SpecialCharTok{\textless{}} \DecValTok{4} \SpecialCharTok{\&}\NormalTok{ y }\SpecialCharTok{\textgreater{}} \DecValTok{5}  \CommentTok{\# Conjunción entre las pruebas anteriores.}
\end{Highlighting}
\end{Shaded}

\begin{verbatim}
## [1] FALSE  TRUE FALSE
\end{verbatim}

\begin{Shaded}
\begin{Highlighting}[]
\NormalTok{x }\SpecialCharTok{\textless{}} \DecValTok{4} \SpecialCharTok{\&\&}\NormalTok{ y }\SpecialCharTok{\textgreater{}} \DecValTok{5}  \CommentTok{\# Conjunción vectorial}
\end{Highlighting}
\end{Shaded}

\begin{verbatim}
## [1] FALSE
\end{verbatim}

Note las diferencias entre los dos últimos ejemplos, cuando se usa \texttt{\&} se hace una prueba término a término y el resultado es un vector, cuando se usa \texttt{\&\&} se aplica la conjunción al vector de resultados obtenido con \texttt{\&}.

\hypertarget{ejemplo-8}{%
\subsection*{Ejemplo}\label{ejemplo-8}}
\addcontentsline{toc}{subsection}{Ejemplo}

Retome el marco de datos \texttt{mimarco} construído en la sección 2.4 y use una prueba lógica para extraer la información de las personas que tienen una edad superior o igual a 15 años y que practican deporte.

Aquí interesa extraer la información de los individuos que cumplen dos condiciones simultáneamente, aquellos con edad \(\geq\) 15 y que SI practiquen deporte. El código necesario para obtener la información solicitada es el siguiente.

\begin{Shaded}
\begin{Highlighting}[]
\NormalTok{mimarco[mimarco}\SpecialCharTok{$}\NormalTok{edad }\SpecialCharTok{\textgreater{}=} \DecValTok{15} \SpecialCharTok{\&}\NormalTok{ mimarco}\SpecialCharTok{$}\NormalTok{deporte }\SpecialCharTok{==} \ConstantTok{TRUE}\NormalTok{, ]}
\end{Highlighting}
\end{Shaded}

\begin{verbatim}
##   edad deporte comic_fav
## 1   15    TRUE      <NA>
## 2   19    TRUE  Superman
## 5   20    TRUE    Batman
\end{verbatim}

De la anterior salida se observa que sólo 3 de las 5 personas cumplen ambas condiciones.

\begin{rmdtip}
La función \texttt{with} es útil porque nos permite realizar algún procedimiento en relación de un objeto, escribiendo menos y de una forma más natural.
\end{rmdtip}

Una forma alternativa para escribir lo anterior usando la función \texttt{with} es la siguiente.

\begin{Shaded}
\begin{Highlighting}[]
\FunctionTok{with}\NormalTok{(mimarco, mimarco[edad }\SpecialCharTok{\textgreater{}=} \DecValTok{15} \SpecialCharTok{\&}\NormalTok{ deporte }\SpecialCharTok{==} \ConstantTok{TRUE}\NormalTok{, ])}
\end{Highlighting}
\end{Shaded}

\begin{verbatim}
##   edad deporte comic_fav
## 1   15    TRUE      <NA>
## 2   19    TRUE  Superman
## 5   20    TRUE    Batman
\end{verbatim}

Al usar \texttt{with} sólo se tuvo que escribir el objeto \texttt{mimarco} dos veces. Cuando hay muchas condiciones o cuando el objeto tiene un nombre largo es aconsejable usar \texttt{with}.

\hypertarget{funciones-sobre-vectores}{%
\section{Funciones sobre vectores}\label{funciones-sobre-vectores}}

En R podemos destacar las siguientes funciones básicas sobre vectores numéricos.

\begin{itemize}
\tightlist
\item
  \texttt{min}: para obtener el mínimo de un vector.
\item
  \texttt{max}: para obtener el máximo de un vector.
\item
  \texttt{length}: para determinar la longitud de un vector.
\item
  \texttt{range}: para obtener el rango de valores de un vector, entrega el mínimo y máximo.
\item
  \texttt{sum}: entrega la suma de todos los elementos del vector.
\item
  \texttt{prod}: multiplica todos los elementos del vector.
\item
  \texttt{which.min}: nos entrega la posición en donde está el valor mínimo del vector.
\item
  \texttt{which.max}: nos da la posición del valor máximo del vector.
\item
  \texttt{rev}: invierte un vector.
\end{itemize}

\hypertarget{ejemplo-9}{%
\subsection*{Ejemplo}\label{ejemplo-9}}
\addcontentsline{toc}{subsection}{Ejemplo}

Construir en vector llamado \texttt{myvec} con los siguientes elementos: 5, 3, 2, 1, 2, 0, NA, 0, 9, 6. Luego aplicar todas las funciones anteriores para verificar el funcionamiento de las mismas.

\begin{Shaded}
\begin{Highlighting}[]
\NormalTok{myvec }\OtherTok{\textless{}{-}} \FunctionTok{c}\NormalTok{(}\DecValTok{5}\NormalTok{, }\DecValTok{3}\NormalTok{, }\DecValTok{2}\NormalTok{, }\DecValTok{1}\NormalTok{, }\DecValTok{2}\NormalTok{, }\DecValTok{0}\NormalTok{, }\ConstantTok{NA}\NormalTok{, }\DecValTok{0}\NormalTok{, }\DecValTok{9}\NormalTok{, }\DecValTok{6}\NormalTok{)}
\NormalTok{myvec}
\end{Highlighting}
\end{Shaded}

\begin{verbatim}
##  [1]  5  3  2  1  2  0 NA  0  9  6
\end{verbatim}

\begin{Shaded}
\begin{Highlighting}[]
\FunctionTok{min}\NormalTok{(myvec)  }\CommentTok{\# Opss, no aparece el mínimo que es Cero.}
\end{Highlighting}
\end{Shaded}

\begin{verbatim}
## [1] NA
\end{verbatim}

\begin{Shaded}
\begin{Highlighting}[]
\FunctionTok{min}\NormalTok{(myvec, }\AttributeTok{na.rm=}\ConstantTok{TRUE}\NormalTok{)  }\CommentTok{\# Usamos na.rm = TRUE para remover el NA}
\end{Highlighting}
\end{Shaded}

\begin{verbatim}
## [1] 0
\end{verbatim}

\begin{Shaded}
\begin{Highlighting}[]
\FunctionTok{max}\NormalTok{(myvec, }\AttributeTok{na.rm=}\NormalTok{T)  }\CommentTok{\# Para obtener el valor máximo}
\end{Highlighting}
\end{Shaded}

\begin{verbatim}
## [1] 9
\end{verbatim}

\begin{Shaded}
\begin{Highlighting}[]
\FunctionTok{range}\NormalTok{(myvec, }\AttributeTok{na.rm=}\NormalTok{T)  }\CommentTok{\# Genera min y max simultáneamente}
\end{Highlighting}
\end{Shaded}

\begin{verbatim}
## [1] 0 9
\end{verbatim}

\begin{Shaded}
\begin{Highlighting}[]
\FunctionTok{sum}\NormalTok{(myvec, }\AttributeTok{na.rm=}\NormalTok{T)  }\CommentTok{\# La suma de los valores internos}
\end{Highlighting}
\end{Shaded}

\begin{verbatim}
## [1] 28
\end{verbatim}

\begin{Shaded}
\begin{Highlighting}[]
\FunctionTok{prod}\NormalTok{(myvec, }\AttributeTok{na.rm=}\NormalTok{T)  }\CommentTok{\# El productor de los valores internos}
\end{Highlighting}
\end{Shaded}

\begin{verbatim}
## [1] 0
\end{verbatim}

\begin{Shaded}
\begin{Highlighting}[]
\FunctionTok{which.min}\NormalTok{(myvec)  }\CommentTok{\# Posición del valor mínimo 0 en el vector}
\end{Highlighting}
\end{Shaded}

\begin{verbatim}
## [1] 6
\end{verbatim}

\begin{Shaded}
\begin{Highlighting}[]
\FunctionTok{which.max}\NormalTok{(myvec)  }\CommentTok{\# Posición del valor máximo 9 en el vector}
\end{Highlighting}
\end{Shaded}

\begin{verbatim}
## [1] 9
\end{verbatim}

De las dos últimas líneas podemos destacar lo siguiente:

\begin{enumerate}
\def\labelenumi{\arabic{enumi}.}
\tightlist
\item
  \textbf{NO es necesario} usar \texttt{na.rm\ =\ TRUE} para remover el \texttt{NA} dentro de las funciones \texttt{which.min} ni \texttt{which.max}.
\item
  El valor mínimo 0 aparece en las posiciones 6 y 8 pero la función \texttt{which.min} sólo entrega la posición del primer valor mínimo dentro del vector.
\end{enumerate}

\hypertarget{funciones-matemuxe1ticas}{%
\section{Funciones matemáticas}\label{funciones-matemuxe1ticas}}

Otras funciones básicas muy utilizadas en estadística son: \texttt{sin,\ cos,\ tan,\ asin,\ acos,\ atan,\ atan2,\ log,\ logb,\ log10,\ exp,\ sqrt,\ abs}. A continuación algunos ejemplos de las anteriores funciones.

\textbf{Ejemplos de medidas trigonométricas}

\begin{Shaded}
\begin{Highlighting}[]
\NormalTok{angulos }\OtherTok{\textless{}{-}} \FunctionTok{c}\NormalTok{(}\DecValTok{0}\NormalTok{, pi}\SpecialCharTok{/}\DecValTok{2}\NormalTok{, pi)}
\FunctionTok{sin}\NormalTok{(angulos)}
\end{Highlighting}
\end{Shaded}

\begin{verbatim}
## [1] 0.000000e+00 1.000000e+00 1.224606e-16
\end{verbatim}

\begin{Shaded}
\begin{Highlighting}[]
\FunctionTok{tan}\NormalTok{(angulos)}
\end{Highlighting}
\end{Shaded}

\begin{verbatim}
## [1]  0.000000e+00  1.633124e+16 -1.224647e-16
\end{verbatim}

\textbf{Ejemplos de logaritmos}

\begin{Shaded}
\begin{Highlighting}[]
\FunctionTok{log}\NormalTok{(}\DecValTok{100}\NormalTok{)}
\end{Highlighting}
\end{Shaded}

\begin{verbatim}
## [1] 4.60517
\end{verbatim}

\begin{Shaded}
\begin{Highlighting}[]
\FunctionTok{log10}\NormalTok{(}\DecValTok{100}\NormalTok{)}
\end{Highlighting}
\end{Shaded}

\begin{verbatim}
## [1] 2
\end{verbatim}

\begin{Shaded}
\begin{Highlighting}[]
\FunctionTok{logb}\NormalTok{(}\DecValTok{125}\NormalTok{, }\AttributeTok{base=}\DecValTok{5}\NormalTok{)}
\end{Highlighting}
\end{Shaded}

\begin{verbatim}
## [1] 3
\end{verbatim}

\textbf{Ejemplos de exponencial}

\begin{Shaded}
\begin{Highlighting}[]
\FunctionTok{exp}\NormalTok{(}\DecValTok{1}\NormalTok{)}
\end{Highlighting}
\end{Shaded}

\begin{verbatim}
## [1] 2.718282
\end{verbatim}

\begin{Shaded}
\begin{Highlighting}[]
\FunctionTok{exp}\NormalTok{(}\DecValTok{2}\NormalTok{)}
\end{Highlighting}
\end{Shaded}

\begin{verbatim}
## [1] 7.389056
\end{verbatim}

\begin{Shaded}
\begin{Highlighting}[]
\FunctionTok{exp}\NormalTok{(}\DecValTok{1}\SpecialCharTok{:}\DecValTok{3}\NormalTok{)}
\end{Highlighting}
\end{Shaded}

\begin{verbatim}
## [1]  2.718282  7.389056 20.085537
\end{verbatim}

\textbf{Ejemplos de raices}

\begin{Shaded}
\begin{Highlighting}[]
\FunctionTok{sqrt}\NormalTok{(}\DecValTok{49}\NormalTok{)  }\CommentTok{\# Raiz cuadrada de 49}
\end{Highlighting}
\end{Shaded}

\begin{verbatim}
## [1] 7
\end{verbatim}

\begin{Shaded}
\begin{Highlighting}[]
\DecValTok{27} \SpecialCharTok{\^{}}\NormalTok{ (}\DecValTok{1}\SpecialCharTok{/}\DecValTok{3}\NormalTok{)  }\CommentTok{\# Raiz cúbica de 27}
\end{Highlighting}
\end{Shaded}

\begin{verbatim}
## [1] 3
\end{verbatim}

\textbf{Ejemplos de valor absoluto}

\begin{Shaded}
\begin{Highlighting}[]
\FunctionTok{abs}\NormalTok{(}\FloatTok{2.5}\NormalTok{)}
\end{Highlighting}
\end{Shaded}

\begin{verbatim}
## [1] 2.5
\end{verbatim}

\begin{Shaded}
\begin{Highlighting}[]
\FunctionTok{abs}\NormalTok{(}\SpecialCharTok{{-}}\FloatTok{3.6}\NormalTok{)}
\end{Highlighting}
\end{Shaded}

\begin{verbatim}
## [1] 3.6
\end{verbatim}

\hypertarget{funciuxf3n-seq}{%
\section{\texorpdfstring{Función \texttt{seq}}{Función seq}}\label{funciuxf3n-seq}}

En R podemos crear secuencias de números de una forma sencilla usando la función \texttt{seq}, la estructura de esta función es:

\begin{Shaded}
\begin{Highlighting}[]
\FunctionTok{seq}\NormalTok{(}\AttributeTok{from=}\DecValTok{1}\NormalTok{, }\AttributeTok{to=}\DecValTok{1}\NormalTok{, by, length.out)}
\end{Highlighting}
\end{Shaded}

Los argumentos de esta función son:

\begin{itemize}
\tightlist
\item
  \texttt{from}: valor de inicio de la secuencia.
\item
  \texttt{to}: valor de fin de la secuencia, no siempre se alcanza.
\item
  \texttt{by}: incremento de la secuencia.
\item
  \texttt{length.out}: longitud deseado de la secuencia.
\end{itemize}

\hypertarget{ejemplo-10}{%
\subsection*{Ejemplo}\label{ejemplo-10}}
\addcontentsline{toc}{subsection}{Ejemplo}

Construya las siguientes tres secuencias usando la función \texttt{seq}.

\begin{itemize}
\tightlist
\item
  Once valores igualmente espaciados desde 0 hasta 1.
\item
  Una secuencia de dos en dos comenzando en 1.
\item
  Una secuencia desde 1 con un salto de \(\pi\) y sin pasar del número 9.
\end{itemize}

El código necesario para obtener las secuencias se muestra a continuación.

\begin{Shaded}
\begin{Highlighting}[]
\FunctionTok{seq}\NormalTok{(}\AttributeTok{from=}\DecValTok{0}\NormalTok{, }\AttributeTok{to=}\DecValTok{1}\NormalTok{, }\AttributeTok{length.out =} \DecValTok{11}\NormalTok{)}
\end{Highlighting}
\end{Shaded}

\begin{verbatim}
##  [1] 0.0 0.1 0.2 0.3 0.4 0.5 0.6 0.7 0.8 0.9 1.0
\end{verbatim}

\begin{Shaded}
\begin{Highlighting}[]
\FunctionTok{seq}\NormalTok{(}\AttributeTok{from=}\DecValTok{1}\NormalTok{, }\AttributeTok{to=}\DecValTok{9}\NormalTok{, }\AttributeTok{by=}\DecValTok{2}\NormalTok{)  }\CommentTok{\# matches \textquotesingle{}end\textquotesingle{}}
\end{Highlighting}
\end{Shaded}

\begin{verbatim}
## [1] 1 3 5 7 9
\end{verbatim}

\begin{Shaded}
\begin{Highlighting}[]
\FunctionTok{seq}\NormalTok{(}\AttributeTok{from=}\DecValTok{1}\NormalTok{, }\AttributeTok{to=}\DecValTok{9}\NormalTok{, }\AttributeTok{by=}\NormalTok{pi) }\CommentTok{\# stays below \textquotesingle{}end\textquotesingle{}}
\end{Highlighting}
\end{Shaded}

\begin{verbatim}
## [1] 1.000000 4.141593 7.283185
\end{verbatim}

\begin{rmdnote}
En R existe el operador binario \texttt{:} que sirve para construir secuencias de uno en uno fácilmente.
\end{rmdnote}

Revise los siguientes ejemplos para entender el funcionamiento del operador \texttt{:}.

\begin{Shaded}
\begin{Highlighting}[]
\DecValTok{2}\SpecialCharTok{:}\DecValTok{8}
\end{Highlighting}
\end{Shaded}

\begin{verbatim}
## [1] 2 3 4 5 6 7 8
\end{verbatim}

\begin{Shaded}
\begin{Highlighting}[]
\DecValTok{3}\SpecialCharTok{:{-}}\DecValTok{5}
\end{Highlighting}
\end{Shaded}

\begin{verbatim}
## [1]  3  2  1  0 -1 -2 -3 -4 -5
\end{verbatim}

\begin{Shaded}
\begin{Highlighting}[]
\NormalTok{pi}\SpecialCharTok{:}\DecValTok{6}  \CommentTok{\# real sequence}
\end{Highlighting}
\end{Shaded}

\begin{verbatim}
## [1] 3.141593 4.141593 5.141593
\end{verbatim}

\begin{Shaded}
\begin{Highlighting}[]
\DecValTok{6}\SpecialCharTok{:}\NormalTok{pi  }\CommentTok{\# integer sequence}
\end{Highlighting}
\end{Shaded}

\begin{verbatim}
## [1] 6 5 4
\end{verbatim}

\hypertarget{funciuxf3n-rep}{%
\section{\texorpdfstring{Función \texttt{rep}}{Función rep}}\label{funciuxf3n-rep}}

En R podemos crear repeticiones usando la función \texttt{rep}, la estructura de esta función es:

\begin{Shaded}
\begin{Highlighting}[]
\FunctionTok{rep}\NormalTok{(x, }\AttributeTok{times=}\DecValTok{1}\NormalTok{, }\AttributeTok{length.out=}\ConstantTok{NA}\NormalTok{, }\AttributeTok{each=}\DecValTok{1}\NormalTok{)}
\end{Highlighting}
\end{Shaded}

Los argumentos de esta función son:

\begin{itemize}
\tightlist
\item
  \texttt{x}: vector con los elementos a repetir.
\item
  \texttt{times}: número de veces que el vector \texttt{x} se debe repetir.
\item
  \texttt{length.out}: longitud deseada para el vector resultante.
\item
  \texttt{each}: número de veces que cada elemento de \texttt{x} se debe repetir.
\end{itemize}

\hypertarget{ejemplo-11}{%
\subsection*{Ejemplo}\label{ejemplo-11}}
\addcontentsline{toc}{subsection}{Ejemplo}

Construya las siguientes repeticiones usando la función \texttt{rep}, no lo haga ingresando número por número.

\begin{itemize}
\tightlist
\item
  1 2 3 4 1 2 3 4
\item
  1 1 2 2 3 3 4 4
\item
  1 1 2 3 3 4
\item
  1 1 2 2 3 3 4 4
\end{itemize}

La clave para construir una repetición es descrubir la semilla o elemento que se repite. Las instrucciones para obtener las repeticiones anteriores se muestra a continuación.

\begin{Shaded}
\begin{Highlighting}[]
\FunctionTok{rep}\NormalTok{(}\AttributeTok{x=}\DecValTok{1}\SpecialCharTok{:}\DecValTok{4}\NormalTok{, }\AttributeTok{times=}\DecValTok{2}\NormalTok{)}
\end{Highlighting}
\end{Shaded}

\begin{verbatim}
## [1] 1 2 3 4 1 2 3 4
\end{verbatim}

\begin{Shaded}
\begin{Highlighting}[]
\FunctionTok{rep}\NormalTok{(}\AttributeTok{x=}\DecValTok{1}\SpecialCharTok{:}\DecValTok{4}\NormalTok{, }\AttributeTok{times=}\FunctionTok{c}\NormalTok{(}\DecValTok{2}\NormalTok{,}\DecValTok{2}\NormalTok{,}\DecValTok{2}\NormalTok{,}\DecValTok{2}\NormalTok{))}
\end{Highlighting}
\end{Shaded}

\begin{verbatim}
## [1] 1 1 2 2 3 3 4 4
\end{verbatim}

\begin{Shaded}
\begin{Highlighting}[]
\FunctionTok{rep}\NormalTok{(}\AttributeTok{x=}\DecValTok{1}\SpecialCharTok{:}\DecValTok{4}\NormalTok{, }\AttributeTok{times=}\FunctionTok{c}\NormalTok{(}\DecValTok{2}\NormalTok{,}\DecValTok{1}\NormalTok{,}\DecValTok{2}\NormalTok{,}\DecValTok{1}\NormalTok{))}
\end{Highlighting}
\end{Shaded}

\begin{verbatim}
## [1] 1 1 2 3 3 4
\end{verbatim}

\begin{Shaded}
\begin{Highlighting}[]
\FunctionTok{rep}\NormalTok{(}\AttributeTok{x=}\DecValTok{1}\SpecialCharTok{:}\DecValTok{4}\NormalTok{, }\AttributeTok{each=}\DecValTok{2}\NormalTok{)}
\end{Highlighting}
\end{Shaded}

\begin{verbatim}
## [1] 1 1 2 2 3 3 4 4
\end{verbatim}

\hypertarget{ejemplo-12}{%
\subsection*{Ejemplo}\label{ejemplo-12}}
\addcontentsline{toc}{subsection}{Ejemplo}

La función \texttt{rep} es muy versátil, observe los siguientes 4 ejemplos y saque una conclusión de cada uno de ellos.

\begin{Shaded}
\begin{Highlighting}[]
\FunctionTok{rep}\NormalTok{(}\AttributeTok{x=}\DecValTok{1}\SpecialCharTok{:}\DecValTok{4}\NormalTok{, }\AttributeTok{each=}\DecValTok{2}\NormalTok{)}
\end{Highlighting}
\end{Shaded}

\begin{verbatim}
## [1] 1 1 2 2 3 3 4 4
\end{verbatim}

\begin{Shaded}
\begin{Highlighting}[]
\FunctionTok{rep}\NormalTok{(}\AttributeTok{x=}\DecValTok{1}\SpecialCharTok{:}\DecValTok{4}\NormalTok{, }\AttributeTok{each=}\DecValTok{2}\NormalTok{, }\AttributeTok{len=}\DecValTok{4}\NormalTok{)    }\CommentTok{\# first 4 only.}
\end{Highlighting}
\end{Shaded}

\begin{verbatim}
## [1] 1 1 2 2
\end{verbatim}

\begin{Shaded}
\begin{Highlighting}[]
\FunctionTok{rep}\NormalTok{(}\AttributeTok{x=}\DecValTok{1}\SpecialCharTok{:}\DecValTok{4}\NormalTok{, }\AttributeTok{each=}\DecValTok{2}\NormalTok{, }\AttributeTok{len=}\DecValTok{10}\NormalTok{)   }\CommentTok{\# 8 integers plus two recycled 1\textquotesingle{}s.}
\end{Highlighting}
\end{Shaded}

\begin{verbatim}
##  [1] 1 1 2 2 3 3 4 4 1 1
\end{verbatim}

\begin{Shaded}
\begin{Highlighting}[]
\FunctionTok{rep}\NormalTok{(}\AttributeTok{x=}\DecValTok{1}\SpecialCharTok{:}\DecValTok{4}\NormalTok{, }\AttributeTok{each=}\DecValTok{2}\NormalTok{, }\AttributeTok{times=}\DecValTok{3}\NormalTok{)  }\CommentTok{\# length 24, 3 complete replications}
\end{Highlighting}
\end{Shaded}

\begin{verbatim}
##  [1] 1 1 2 2 3 3 4 4 1 1 2 2 3 3 4 4 1 1 2 2 3 3 4 4
\end{verbatim}

\hypertarget{funciones-round-ceiling-floor-y-trunc}{%
\section{\texorpdfstring{Funciones \texttt{round}, \texttt{ceiling}, \texttt{floor} y \texttt{trunc}}{Funciones round, ceiling, floor y trunc}}\label{funciones-round-ceiling-floor-y-trunc}}

Existen 4 funciones útiles para modificar u obtener información de un número, estas funciones son \texttt{round}, \texttt{ceiling}, \texttt{floor} y \texttt{trunc}.

\begin{itemize}
\tightlist
\item
  \texttt{round(x,\ digits)}: sirve para redondear un número según los dígitos indicados.
\item
  \texttt{ceiling(x)}: entrega el mínimo entero mayor o igual que \texttt{x}.
\item
  \texttt{floor(x)}: entrega el máximo entero menor o igual que \texttt{x}.
\item
  \texttt{trunc(x)}: entrega la parte entera de un número \texttt{x}.
\end{itemize}

\hypertarget{ejemplo-13}{%
\subsection*{Ejemplo}\label{ejemplo-13}}
\addcontentsline{toc}{subsection}{Ejemplo}

Aplique las funciones \texttt{round}, \texttt{ceiling}, \texttt{floor} y \texttt{trunc} a un valor positivo y a un valor negativo para inspeccionar los resultados.

A continuación el código de prueba para un número positivo cualquiera.

\begin{Shaded}
\begin{Highlighting}[]
\NormalTok{x }\OtherTok{\textless{}{-}} \FloatTok{5.34896}  \CommentTok{\# Número positivo elegido}
\FunctionTok{round}\NormalTok{(x, }\AttributeTok{digits=}\DecValTok{3}\NormalTok{)}
\end{Highlighting}
\end{Shaded}

\begin{verbatim}
## [1] 5.349
\end{verbatim}

\begin{Shaded}
\begin{Highlighting}[]
\FunctionTok{ceiling}\NormalTok{(x)}
\end{Highlighting}
\end{Shaded}

\begin{verbatim}
## [1] 6
\end{verbatim}

\begin{Shaded}
\begin{Highlighting}[]
\FunctionTok{floor}\NormalTok{(x)}
\end{Highlighting}
\end{Shaded}

\begin{verbatim}
## [1] 5
\end{verbatim}

\begin{Shaded}
\begin{Highlighting}[]
\FunctionTok{trunc}\NormalTok{(x)}
\end{Highlighting}
\end{Shaded}

\begin{verbatim}
## [1] 5
\end{verbatim}

A continuación las pruebas con un número negativo cualquiera.

\begin{Shaded}
\begin{Highlighting}[]
\NormalTok{x }\OtherTok{\textless{}{-}} \SpecialCharTok{{-}}\FloatTok{4.26589}  \CommentTok{\# Número negativo elegido}
\FunctionTok{round}\NormalTok{(x, }\AttributeTok{digits=}\DecValTok{3}\NormalTok{)}
\end{Highlighting}
\end{Shaded}

\begin{verbatim}
## [1] -4.266
\end{verbatim}

\begin{Shaded}
\begin{Highlighting}[]
\FunctionTok{ceiling}\NormalTok{(x)}
\end{Highlighting}
\end{Shaded}

\begin{verbatim}
## [1] -4
\end{verbatim}

\begin{Shaded}
\begin{Highlighting}[]
\FunctionTok{floor}\NormalTok{(x)}
\end{Highlighting}
\end{Shaded}

\begin{verbatim}
## [1] -5
\end{verbatim}

\begin{Shaded}
\begin{Highlighting}[]
\FunctionTok{trunc}\NormalTok{(x)}
\end{Highlighting}
\end{Shaded}

\begin{verbatim}
## [1] -4
\end{verbatim}

\hypertarget{funciones-sort-y-rank}{%
\section{\texorpdfstring{Funciones \texttt{sort} y \texttt{rank}}{Funciones sort y rank}}\label{funciones-sort-y-rank}}

Las funciones \texttt{sort} y \texttt{rank} son útiles para ordenar los elementos de un vector o para saber las posiciones que ocuarían los elementos de un vector al ser ordenado. La estructura de las dos funciones es la siguiente.

\begin{Shaded}
\begin{Highlighting}[]
\FunctionTok{sort}\NormalTok{(x, }\AttributeTok{decreasing =} \ConstantTok{FALSE}\NormalTok{)}
\FunctionTok{rank}\NormalTok{(x)}
\end{Highlighting}
\end{Shaded}

En el parámetro \texttt{x} se ingresa el vector y el parámetro \texttt{decreasing} sirva para indicar si el ordenamiento es de menor a mayor (por defecto es este) o de mayor a menor.

\hypertarget{ejemplo-14}{%
\subsection*{Ejemplo}\label{ejemplo-14}}
\addcontentsline{toc}{subsection}{Ejemplo}

Considere el vector \texttt{x} que tiene los siguientes elementos: 2, 3, 6, 4, 9 y 5. Ordene el vector de menor a mayor, de mayor a menor y por último encuentre la posición que ocupan los elementos de \texttt{x} si se ordenaran de menor a mayor.

\begin{Shaded}
\begin{Highlighting}[]
\NormalTok{x }\OtherTok{\textless{}{-}} \FunctionTok{c}\NormalTok{(}\DecValTok{2}\NormalTok{, }\DecValTok{3}\NormalTok{, }\DecValTok{6}\NormalTok{, }\DecValTok{4}\NormalTok{, }\DecValTok{9}\NormalTok{, }\DecValTok{5}\NormalTok{)}
\FunctionTok{sort}\NormalTok{(x)}
\end{Highlighting}
\end{Shaded}

\begin{verbatim}
## [1] 2 3 4 5 6 9
\end{verbatim}

\begin{Shaded}
\begin{Highlighting}[]
\FunctionTok{sort}\NormalTok{(x, }\AttributeTok{decreasing=}\ConstantTok{TRUE}\NormalTok{)}
\end{Highlighting}
\end{Shaded}

\begin{verbatim}
## [1] 9 6 5 4 3 2
\end{verbatim}

\begin{Shaded}
\begin{Highlighting}[]
\FunctionTok{rank}\NormalTok{(x)}
\end{Highlighting}
\end{Shaded}

\begin{verbatim}
## [1] 1 2 5 3 6 4
\end{verbatim}

\hypertarget{ejercicios-1}{%
\section*{EJERCICIOS}\label{ejercicios-1}}
\addcontentsline{toc}{section}{EJERCICIOS}

Use funciones o procedimientos (varias líneas) de R para responder cada una de las siguientes preguntas.

\begin{enumerate}
\def\labelenumi{\arabic{enumi}.}
\tightlist
\item
  ¿Qué cantidad de dinero sobra al repartir 10000\$ entre 3 personas?
\item
  ¿Es el número 4560 divisible por 3?
\item
  Construya un vector con los números enteros del 2 al 87. ¿Cuáles de esos números son divisibles por 7?
\item
  Construya dos vectores, el primero con los números enteros desde 7 hasta 3, el segundo vector con los primeros cinco números positivos divisibles por 5. Sea A la condición de ser par en el primer vector. Sea B la condición de ser mayor que 10 en el segundo vector. ¿En cuál de las 5 posiciones se cumple A y B simultáneamente?
\item
  Consulte \href{https://github.com/fhernanb/Manual-de-R/blob/master/images/anecdota_gauss.PNG}{este enlace} en el cual hay una anéctoda de Gauss niño. Use R para obtener el resultado de la suma solicitada por el profesor del niño Gauss.
\item
  Construya un vector con los siguientes elementos: 1, -4, 5, 9, -4. Escriba un procedimiento para extraer \textbf{las posiciones} donde está el valor mínimo en el vector.
\item
  Calcular \(8!\)
\item
  Evaluar la siguiente suma \(\sum_{i=3}^{i=7}e^i\)
\item
  Evaluar la siguiente productoria \(\prod_{i=1}^{i=10}\log\sqrt{i}\)
\item
  Construya un vector cualquiera e inviertalo, es decir, que el primer elemento quede de último, el segundo de penúltimo y así sucesivamente. Compare su resultado con el de la función \texttt{rev}.
\item
  Create the vector: \(1, 2, 3, \ldots, 19, 20\).
\item
  Create the vector: \(20, 19, \ldots , 2, 1\).
\item
  Create the vector: \(1, -2, 3, -4, 5, -6, \ldots, 19, -20\).
\item
  Create the vector: \(0.1^3, 0.2^1, 0.1^6, 0.2^4, . . . , 0.1^{36}, 0.2^{34}\).
\item
  Calculate the following: \(\sum_{i=10}^{100}(i^3+4i^2)\) and \(\sum_{i=1}^{25}\left( \frac{2^i}{i} + \frac{3^i}{i^2} \right)\).
\item
  Read the data set available in: \url{https://raw.githubusercontent.com/fhernanb/datos/master/Paises.txt}
\item
  Use a code to obtain the number of variables of the data set.
\item
  Use a code to obtain the number of countries in the data set.
\item
  Which is the country with the higher population?
\item
  Which is the country with the lowest literacy rate?
\item
  ¿Qué valor de verdad tiene la siguiente afirmación? ``Los resultados de la función \texttt{floor} y \texttt{trunc} son siempre los mismos''.
\end{enumerate}

En R hay unas bases de datos incluídas, una de ellas es la base de datos llamada \texttt{mtcars}. Para conocer las variables que están en \texttt{mtcars} usted puede escribir en la consola \texttt{?mtcars} o también \texttt{help(mtcars)}. De la base \texttt{mtcars} obtenga bases de datos que cumplan las siguientes condiciones.

\begin{enumerate}
\def\labelenumi{\arabic{enumi}.}
\setcounter{enumi}{21}
\tightlist
\item
  Autos que tengan un rendimiento menor a 18 millas por galón de combustible.
\item
  Autos que tengan 4 cilindros.
\item
  Autos que pesen más de 2500 libras y tengan transmisión manual.
\end{enumerate}

\hypertarget{bucles}{%
\chapter{Instrucciones de control}\label{bucles}}

En R se disponen de varias instrucciones de control para facilitar los procedimientos que un usuario debe realizar. A continuación se explican esas instrucciones de control.

\hypertarget{instrucciuxf3n-if}{%
\section{\texorpdfstring{Instrucción \texttt{if}}{Instrucción if}}\label{instrucciuxf3n-if}}

Esta instrucción sirve para realizar un conjunto de operaciones \textbf{si} se cumple cierta condición. A continuación se muestra la estructura básica de uso.

\begin{Shaded}
\begin{Highlighting}[]
\ControlFlowTok{if}\NormalTok{ (condicion) \{}
\NormalTok{  operación }\DecValTok{1}
\NormalTok{  operación }\DecValTok{2}
\NormalTok{  ...}
\NormalTok{  operación final}
\NormalTok{\}}
\end{Highlighting}
\end{Shaded}

\hypertarget{ejemplo-15}{%
\subsection*{Ejemplo}\label{ejemplo-15}}
\addcontentsline{toc}{subsection}{Ejemplo}

Una secretaria recibe la información del salario básico semanal de un empleado y las horas trabajadas durante la semana por ese empleado. El salario básico es la remuneración por 40 horas de labor por semana, las horas extra son pagadas a ciencuenta mil pesos. Escriba el procedimiento en R que debe usar la secretaria para calcular el salario semanal de un empleado que trabajó 45 horas y tiene salario básico de un millon de pesos.

El código para calcular el salario final del empleado es el siguiente:

\begin{Shaded}
\begin{Highlighting}[]
\NormalTok{sal }\OtherTok{\textless{}{-}} \DecValTok{1}  \CommentTok{\# Salario básico por semana}
\NormalTok{hlab }\OtherTok{\textless{}{-}} \DecValTok{45}   \CommentTok{\# Horas laboradas por semana}

\ControlFlowTok{if}\NormalTok{(hlab }\SpecialCharTok{\textgreater{}} \DecValTok{40}\NormalTok{) \{}
\NormalTok{  hext }\OtherTok{\textless{}{-}}\NormalTok{ hlab }\SpecialCharTok{{-}} \DecValTok{40}
\NormalTok{  salext }\OtherTok{\textless{}{-}}\NormalTok{ hext }\SpecialCharTok{*} \FloatTok{0.05}
\NormalTok{  sal }\OtherTok{\textless{}{-}}\NormalTok{ sal }\SpecialCharTok{+}\NormalTok{ salext}
\NormalTok{\}}

\NormalTok{sal  }\CommentTok{\# Salario semanal}
\end{Highlighting}
\end{Shaded}

\begin{verbatim}
## [1] 1.25
\end{verbatim}

\hypertarget{instrucciuxf3n-if-else}{%
\section{\texorpdfstring{Instrucción \texttt{if} \texttt{else}}{Instrucción if else}}\label{instrucciuxf3n-if-else}}

Esta instrucción sirve para realizar un conjunto de operaciones cuando \textbf{NO} se cumple cierta condición evaluada por un \texttt{if}. A continuación se muestra la estructura básica de uso.

\begin{Shaded}
\begin{Highlighting}[]
\ControlFlowTok{if}\NormalTok{ (condicion) \{}
\NormalTok{  operación }\DecValTok{1}
\NormalTok{  operación }\DecValTok{2}
\NormalTok{  ...}
\NormalTok{  operación final}
\NormalTok{\}}
\ControlFlowTok{else}\NormalTok{ \{}
\NormalTok{  operación }\DecValTok{1}
\NormalTok{  operación }\DecValTok{2}
\NormalTok{  ...}
\NormalTok{  operación final}
\NormalTok{\}}
\end{Highlighting}
\end{Shaded}

\hypertarget{instrucciuxf3n-ifelse}{%
\section{\texorpdfstring{Instrucción \texttt{ifelse}}{Instrucción ifelse}}\label{instrucciuxf3n-ifelse}}

Se recomienda usar la instrucción \texttt{ifelse} cuando hay una sola instrucción para el caso \texttt{if} y para el caso \texttt{else}. A continuación se muestra la estructura básica de uso.

\begin{Shaded}
\begin{Highlighting}[]
\FunctionTok{ifelse}\NormalTok{(condición, operación SI cumple, operación NO cumple)}
\end{Highlighting}
\end{Shaded}

\hypertarget{ejemplo-16}{%
\subsection*{Ejemplo}\label{ejemplo-16}}
\addcontentsline{toc}{subsection}{Ejemplo}

Suponga que usted recibe un vector de números enteros, escriba un procedimiento que diga si cada elemento del vector es par o impar.

\begin{Shaded}
\begin{Highlighting}[]
\NormalTok{x }\OtherTok{\textless{}{-}} \FunctionTok{c}\NormalTok{(}\DecValTok{5}\NormalTok{, }\DecValTok{3}\NormalTok{, }\DecValTok{2}\NormalTok{, }\DecValTok{8}\NormalTok{, }\SpecialCharTok{{-}}\DecValTok{4}\NormalTok{, }\DecValTok{1}\NormalTok{)}

\FunctionTok{ifelse}\NormalTok{(x }\SpecialCharTok{\%\%} \DecValTok{2} \SpecialCharTok{==} \DecValTok{0}\NormalTok{, }\StringTok{\textquotesingle{}Es par\textquotesingle{}}\NormalTok{, }\StringTok{\textquotesingle{}Es impar\textquotesingle{}}\NormalTok{)}
\end{Highlighting}
\end{Shaded}

\begin{verbatim}
## [1] "Es impar" "Es impar" "Es par"   "Es par"   "Es par"   "Es impar"
\end{verbatim}

\hypertarget{instrucciuxf3n-for}{%
\section{\texorpdfstring{Instrucción \texttt{for}}{Instrucción for}}\label{instrucciuxf3n-for}}

La instrucción \texttt{for} es muy útil para repetir un procedimiento cierta cantidad de veces. A continuación se muestra la estructura básica de uso.

\begin{Shaded}
\begin{Highlighting}[]
\ControlFlowTok{for}\NormalTok{ (i }\ControlFlowTok{in}\NormalTok{ secuencia) \{}
\NormalTok{  operación }\DecValTok{1}
\NormalTok{  operación }\DecValTok{2}
\NormalTok{  ...}
\NormalTok{  operación final}
\NormalTok{\}}
\end{Highlighting}
\end{Shaded}

\hypertarget{ejemplo-17}{%
\subsection*{Ejemplo}\label{ejemplo-17}}
\addcontentsline{toc}{subsection}{Ejemplo}

Escriba un procedimiento para crear 10 muestras de tamaño 100 de una distribución uniforme entre uno y tres. Para cada una de las muestra, se debe contar el número de elementos de la muestra que fueron mayores o iguales a 2.5.

\begin{Shaded}
\begin{Highlighting}[]
\NormalTok{nrep }\OtherTok{\textless{}{-}} \DecValTok{10}  \CommentTok{\# Número de repeticiones}
\NormalTok{n }\OtherTok{\textless{}{-}} \DecValTok{100}    \CommentTok{\# Tamaño de la muestra}
\NormalTok{conteo }\OtherTok{\textless{}{-}} \FunctionTok{numeric}\NormalTok{(nrep)  }\CommentTok{\# Vector para almacenar el conteo}

\ControlFlowTok{for}\NormalTok{ (i }\ControlFlowTok{in} \DecValTok{1}\SpecialCharTok{:}\NormalTok{nrep) \{}
\NormalTok{  x }\OtherTok{\textless{}{-}} \FunctionTok{runif}\NormalTok{(}\AttributeTok{n=}\NormalTok{n, }\AttributeTok{min=}\DecValTok{1}\NormalTok{, }\AttributeTok{max=}\DecValTok{3}\NormalTok{)}
\NormalTok{  conteo[i] }\OtherTok{\textless{}{-}} \FunctionTok{sum}\NormalTok{(x }\SpecialCharTok{\textgreater{}=} \FloatTok{2.5}\NormalTok{)}
\NormalTok{\}}

\NormalTok{conteo  }\CommentTok{\# Para obtener el conteo}
\end{Highlighting}
\end{Shaded}

\begin{verbatim}
##  [1] 24 37 28 26 30 18 29 23 19 19
\end{verbatim}

\hypertarget{instrucciuxf3n-while}{%
\section{\texorpdfstring{Instrucción \texttt{while}}{Instrucción while}}\label{instrucciuxf3n-while}}

La instrucción \texttt{while} es muy útil para repetir un procedimiento siempre que se cumple una condición. A continuación se muestra la estructura básica de uso.

\begin{Shaded}
\begin{Highlighting}[]
\ControlFlowTok{while}\NormalTok{ (condición) \{}
\NormalTok{  operación }\DecValTok{1}
\NormalTok{  operación }\DecValTok{2}
\NormalTok{  ...}
\NormalTok{  operación final}
\NormalTok{\}}
\end{Highlighting}
\end{Shaded}

\hypertarget{ejemplo-18}{%
\subsection*{Ejemplo}\label{ejemplo-18}}
\addcontentsline{toc}{subsection}{Ejemplo}

Suponga que se lanza una moneda en la cual el resultado es cara o sello. Escribir un procedimiento que simule lanzamientos hasta que el número de caras obtenidas sea 5. El procedimiento debe entregar el historial de lanzamientos.

Para simular el lanzamiento de \textbf{una} moneda se puede usar la función \texttt{sample} y definiendo el vector \texttt{resultados} con \texttt{size=1} para simular un lanzamiento, a continuación el código y tres pruebas ilustrativas.

\begin{Shaded}
\begin{Highlighting}[]
\NormalTok{resultados }\OtherTok{\textless{}{-}} \FunctionTok{c}\NormalTok{(}\StringTok{\textquotesingle{}Cara\textquotesingle{}}\NormalTok{, }\StringTok{\textquotesingle{}Sello\textquotesingle{}}\NormalTok{)}
\FunctionTok{sample}\NormalTok{(}\AttributeTok{x=}\NormalTok{resultados, }\AttributeTok{size=}\DecValTok{1}\NormalTok{)  }\CommentTok{\# Prueba 1}
\end{Highlighting}
\end{Shaded}

\begin{verbatim}
## [1] "Sello"
\end{verbatim}

Una vez seamos capaces de simular un lanzamiento podemos escribir el procedimiento para generar tantos lanzamientos hasta que se cumpla la condición. El código mostrado abajo permite hacer lo solicitado.

\begin{Shaded}
\begin{Highlighting}[]
\NormalTok{num.lanza }\OtherTok{\textless{}{-}} \DecValTok{0}     \CommentTok{\# Contador de lanzamientos}
\NormalTok{num.caras }\OtherTok{\textless{}{-}} \DecValTok{0}     \CommentTok{\# Contados de caras obtenidas}
\NormalTok{historial }\OtherTok{\textless{}{-}} \ConstantTok{NULL}  \CommentTok{\# Vector vacío para almacenar}

\ControlFlowTok{while}\NormalTok{ (num.caras }\SpecialCharTok{\textless{}} \DecValTok{5}\NormalTok{) \{}
\NormalTok{  res }\OtherTok{\textless{}{-}} \FunctionTok{sample}\NormalTok{(}\AttributeTok{x=}\NormalTok{resultados, }\AttributeTok{size=}\DecValTok{1}\NormalTok{)}
\NormalTok{  num.lanza }\OtherTok{\textless{}{-}}\NormalTok{ num.lanza }\SpecialCharTok{+} \DecValTok{1}
\NormalTok{  historial[num.lanza] }\OtherTok{\textless{}{-}}\NormalTok{ res}
  \ControlFlowTok{if}\NormalTok{ (res }\SpecialCharTok{==} \StringTok{\textquotesingle{}Cara\textquotesingle{}}\NormalTok{) \{}
\NormalTok{    num.caras }\OtherTok{\textless{}{-}}\NormalTok{ num.caras }\SpecialCharTok{+} \DecValTok{1}
\NormalTok{  \}}
\NormalTok{\}}

\NormalTok{historial}
\end{Highlighting}
\end{Shaded}

\begin{verbatim}
##  [1] "Sello" "Sello" "Sello" "Sello" "Cara"  "Cara"  "Sello" "Sello" "Cara" 
## [10] "Cara"  "Cara"
\end{verbatim}

\begin{Shaded}
\begin{Highlighting}[]
\NormalTok{num.lanza}
\end{Highlighting}
\end{Shaded}

\begin{verbatim}
## [1] 11
\end{verbatim}

\begin{rmdnote}
La instrucción \texttt{for} se usa cuando sabemos el número de veces que se debe repetir el procedimiento, mientras que la instrucción \texttt{while} se usa cuando debemos repetir un procedimiento cuando se cumpla una condición.
\end{rmdnote}

\hypertarget{instrucciuxf3n-repeat}{%
\section{\texorpdfstring{Instrucción \texttt{repeat}}{Instrucción repeat}}\label{instrucciuxf3n-repeat}}

La instrucción \texttt{while} es muy útil para repetir un procedimiento siempre que se cumple una condición. A continuación se muestra la estructura básica de uso.

\begin{Shaded}
\begin{Highlighting}[]
\ControlFlowTok{repeat}\NormalTok{ \{}
\NormalTok{  operación }\DecValTok{1}
\NormalTok{  operación }\DecValTok{2}
\NormalTok{  ...}
\NormalTok{  operación final}
  \ControlFlowTok{if}\NormalTok{ (condición) }\ControlFlowTok{break}
\NormalTok{\}}
\end{Highlighting}
\end{Shaded}

\hypertarget{ejemplo-19}{%
\subsection*{Ejemplo}\label{ejemplo-19}}
\addcontentsline{toc}{subsection}{Ejemplo}

Escribir un procedimiento para ir aumentando de uno en uno el valor de \texttt{x} hasta que \texttt{x} sea igual a siete El procedimiento debe imprimir por pantalla la secuencia de valores de \texttt{x}.

\begin{Shaded}
\begin{Highlighting}[]
\NormalTok{x }\OtherTok{\textless{}{-}} \DecValTok{3}  \CommentTok{\# Valor de inicio}

\ControlFlowTok{repeat}\NormalTok{ \{}
   \FunctionTok{print}\NormalTok{(x)}
\NormalTok{   x }\OtherTok{\textless{}{-}}\NormalTok{  x }\SpecialCharTok{+} \DecValTok{1}
   \ControlFlowTok{if}\NormalTok{ (x }\SpecialCharTok{==} \DecValTok{8}\NormalTok{) \{}
     \ControlFlowTok{break}
\NormalTok{   \}}
\NormalTok{\}}
\end{Highlighting}
\end{Shaded}

\begin{verbatim}
## [1] 3
## [1] 4
## [1] 5
## [1] 6
## [1] 7
\end{verbatim}

\begin{rmdtip}
La instrucción \texttt{break} sirve para salir de un procedimiento iterativo.
\end{rmdtip}

\hypertarget{creafun}{%
\chapter{Creación de funciones en R}\label{creafun}}

Uno de los atractivos de R es la gran cantidad de funciones que existen para realizar diversos procedimientos. En este capítulo se explica al lector la forma de crear sus propias funciones para que pueda realizar diversas tareas y así logre explotar el potencial que ofrece R.

\hypertarget{quuxe9-es-una-funciuxf3n-en-r}{%
\section{¿Qué es una función en R?}\label{quuxe9-es-una-funciuxf3n-en-r}}

Una función es un conjunto de instrucciones que convierten las entradas (\emph{inputs}) en resultados (\emph{outputs}) deseados. En la siguiente figura se muestra una ilustración de lo que es una función.

\hypertarget{partes-de-una-funciuxf3n-en-r}{%
\section{Partes de una función en R}\label{partes-de-una-funciuxf3n-en-r}}

Las partes de una función son:

\begin{itemize}
\tightlist
\item
  Entradas o argumentos: sirven para ingresar información necesaria para realizar el procedimiento de la función. Los argumentos pueden estar vacíos y a la espera de que el usuario ingrese valores, o pueden tener valores por defecto, esto significa que si el usuario no ingresa un valor, la función usará el valor por defecto. Una función puede tener o no argumentos de entrada, en los ejemplos se mostrarán estos casos.
\item
  Cuerpo: está formado por un conjunto de instrucciones que transforman las entradas en las salidas deseadas. Si el cuerpo de la función está formado por varias instrucciones éstas deben ir entre llaves \texttt{\{\ \}}.
\item
  Salidas: son los resultados de la función. Toda función debe tener al menos un resultado. Si una función entrega varios tipos de objetos se acostumbra a organizarlos en una lista que puede manejar los diferentes tipos de objetos.
\end{itemize}

A continuación se muestra la estructura general de una función en R.

\begin{Shaded}
\begin{Highlighting}[]
\NormalTok{nombre\_de\_funcion }\OtherTok{\textless{}{-}} \ControlFlowTok{function}\NormalTok{(par1, par2, ...) \{}
\NormalTok{  cuerpo}
\NormalTok{  cuerpo}
\NormalTok{  cuerpo}
\NormalTok{  cuerpo}
  \FunctionTok{return}\NormalTok{(resultado)}
\NormalTok{\}}
\end{Highlighting}
\end{Shaded}

A continuación se mostrarán varios ejemplos sencillos para que el lector aprenda a construir funciones en R.

\hypertarget{ejemplo-20}{%
\subsection*{Ejemplo}\label{ejemplo-20}}
\addcontentsline{toc}{subsection}{Ejemplo}

Construir una función que reciba dos números y que entregue la suma de estos números.

\textbf{Solución}

Lo primero es elegir un nombre apropiado para la función, aquí se usó el nombre \texttt{suma} porque así se tiene una idea clara de lo que hace la función. La función suma recibe dos parámetros, \texttt{x} representa el primer valor ingresado mientras que \texttt{y} representa el segundo. El cuerpo de la función está formado por dos líneas, en la primera se crea el objeto \texttt{resultado} en el cual se almanacena el valor de la suma, en la segunda línea se le indica a R que queremos que retorne el valor de la suma almacenada en el objeto \texttt{resultado}. A continuación se muestra el código para crear la función solicitada.

\begin{Shaded}
\begin{Highlighting}[]
\NormalTok{suma }\OtherTok{\textless{}{-}} \ControlFlowTok{function}\NormalTok{(x, y) \{}
\NormalTok{  resultado }\OtherTok{\textless{}{-}}\NormalTok{ x }\SpecialCharTok{+}\NormalTok{ y}
  \FunctionTok{return}\NormalTok{(resultado)}
\NormalTok{\}}
\end{Highlighting}
\end{Shaded}

Para usar la función creada sólo se debe ejecutar, vamos a obtener la suma de los valores 4 y 6 usando la función \texttt{suma}, a continuación el código necesario.

\begin{Shaded}
\begin{Highlighting}[]
\FunctionTok{suma}\NormalTok{(}\AttributeTok{x=}\DecValTok{4}\NormalTok{, }\AttributeTok{y=}\DecValTok{6}\NormalTok{)}
\end{Highlighting}
\end{Shaded}

\begin{verbatim}
## [1] 10
\end{verbatim}

Para funciones simples como la anterior es posible escribirlas en forma más compacta. Es posible reducir el cuerpo de la función de 2 líneas a sólo una línea solicitándole a R que retorne directamente la suma sin almacenarla en ningún objeto. A continuación la función \texttt{suma} modificada.

\begin{Shaded}
\begin{Highlighting}[]
\NormalTok{suma }\OtherTok{\textless{}{-}} \ControlFlowTok{function}\NormalTok{(x, y) \{}
  \FunctionTok{return}\NormalTok{(x }\SpecialCharTok{+}\NormalTok{ y)}
\NormalTok{\}}

\FunctionTok{suma}\NormalTok{(}\AttributeTok{x=}\DecValTok{4}\NormalTok{, }\AttributeTok{y=}\DecValTok{6}\NormalTok{)  }\CommentTok{\# Probando la función}
\end{Highlighting}
\end{Shaded}

\begin{verbatim}
## [1] 10
\end{verbatim}

Debido a que la función \texttt{suma} tiene un cuerpo muy reducido es posible escribirla en forma más compacta, en una sola línea. A continuación se muestra el código para reescribir la función.

\begin{Shaded}
\begin{Highlighting}[]
\NormalTok{suma }\OtherTok{\textless{}{-}} \ControlFlowTok{function}\NormalTok{(x, y) x }\SpecialCharTok{+}\NormalTok{ y}

\FunctionTok{suma}\NormalTok{(}\AttributeTok{x=}\DecValTok{4}\NormalTok{, }\AttributeTok{y=}\DecValTok{6}\NormalTok{)  }\CommentTok{\# Probando la función}
\end{Highlighting}
\end{Shaded}

\begin{verbatim}
## [1] 10
\end{verbatim}

\hypertarget{ejemplo-21}{%
\subsection*{Ejemplo}\label{ejemplo-21}}
\addcontentsline{toc}{subsection}{Ejemplo}

Construir una función que genere números aleatorios entre cero y uno hasta que la suma de éstos números supere por primera vez el valor de 3. La función debe entregar la cantidad de números aleatorios generados para que se cumpla la condición.

\textbf{Solución}

Vamos a llamar la función solicitada con el nombre \texttt{fun1}, esta función \textbf{NO} necesita ningún parámetro de entrada. El valor de 3 que está en la condición puede ir dentro del cuerpo y por eso no se necesitan parámetros para esta función. En el cuerpo de la función se genera un vector con un número aleatorio y luego se chequea si la suma de sus elementos es menor de 3, si se cumple que la suma es menor que 3 se siguen generando números que se almacenan en el vector \texttt{num}. Una vez que la suma exceda el valor de 3 NO se ingresa al \texttt{while} y se pide la longitud del vector o el valor de \texttt{veces} solicitado. A continuación el código de la función.

\begin{Shaded}
\begin{Highlighting}[]
\NormalTok{fun1 }\OtherTok{\textless{}{-}} \ControlFlowTok{function}\NormalTok{() \{}
\NormalTok{  num }\OtherTok{\textless{}{-}} \FunctionTok{runif}\NormalTok{(}\DecValTok{1}\NormalTok{)}
\NormalTok{  veces }\OtherTok{\textless{}{-}} \DecValTok{1}
  \ControlFlowTok{while}\NormalTok{ (}\FunctionTok{sum}\NormalTok{(num) }\SpecialCharTok{\textless{}} \DecValTok{3}\NormalTok{) \{}
\NormalTok{    veces }\OtherTok{\textless{}{-}}\NormalTok{ veces }\SpecialCharTok{+} \DecValTok{1}
\NormalTok{    num[veces] }\OtherTok{\textless{}{-}} \FunctionTok{runif}\NormalTok{(}\DecValTok{1}\NormalTok{)}
\NormalTok{  \}}
  \FunctionTok{return}\NormalTok{(veces)}
\NormalTok{\}}

\FunctionTok{fun1}\NormalTok{()  }\CommentTok{\# primera prueba}
\end{Highlighting}
\end{Shaded}

\begin{verbatim}
## [1] 8
\end{verbatim}

\hypertarget{ejemplo-22}{%
\subsection*{Ejemplo}\label{ejemplo-22}}
\addcontentsline{toc}{subsection}{Ejemplo}

Construir una función que, dado un número entero positivo (cota) ingresado por el usuario, genere números aleatorios entre cero y uno hasta que la suma de los números generados exceda por primera vez la cota. La función debe entregar un vector con los números aleatorios, la suma y la cantidad de números aleatorios. Si el usuario no ingresa el valor de la cota, se debe asumir igual a 1.

\textbf{Solución}

La función aquí solicitada es similar a la construída en el ejemplo anterior. La función \texttt{fun2} tiene un sólo parámetro con el valor por defecto, si el usuario no ingresa valor a este parámetro, se asumirá el valor de uno. El cuerpo de la función es similar al anterior. Como la función debe entregar un vector y dos números, se construye la lista \texttt{resultado} que almacena los tres objetos solicitados. A continuación el código para función solicitada.

\begin{Shaded}
\begin{Highlighting}[]
\NormalTok{fun2 }\OtherTok{\textless{}{-}} \ControlFlowTok{function}\NormalTok{(}\AttributeTok{cota=}\DecValTok{1}\NormalTok{) \{}
\NormalTok{  num }\OtherTok{\textless{}{-}} \FunctionTok{runif}\NormalTok{(}\DecValTok{1}\NormalTok{)}
  \ControlFlowTok{while}\NormalTok{ (}\FunctionTok{sum}\NormalTok{(num) }\SpecialCharTok{\textless{}}\NormalTok{ cota) \{}
\NormalTok{    num }\OtherTok{\textless{}{-}} \FunctionTok{c}\NormalTok{(num, }\FunctionTok{runif}\NormalTok{(}\DecValTok{1}\NormalTok{))}
\NormalTok{  \}}
\NormalTok{  resultado }\OtherTok{\textless{}{-}} \FunctionTok{list}\NormalTok{(}\AttributeTok{vector=}\NormalTok{num,}
                    \AttributeTok{suma=}\FunctionTok{sum}\NormalTok{(num),}
                    \AttributeTok{cantidad=}\FunctionTok{length}\NormalTok{(num))}
  \FunctionTok{return}\NormalTok{(resultado)}
\NormalTok{\}}
\end{Highlighting}
\end{Shaded}

Probando la función con cota de uno.

\begin{Shaded}
\begin{Highlighting}[]
\FunctionTok{fun2}\NormalTok{()}
\end{Highlighting}
\end{Shaded}

\begin{verbatim}
## $vector
## [1] 0.8523376 0.4814579
## 
## $suma
## [1] 1.333796
## 
## $cantidad
## [1] 2
\end{verbatim}

Probando la función con cota de tres.

\begin{Shaded}
\begin{Highlighting}[]
\FunctionTok{fun2}\NormalTok{(}\AttributeTok{cota=}\DecValTok{3}\NormalTok{)}
\end{Highlighting}
\end{Shaded}

\begin{verbatim}
## $vector
## [1] 0.7703864 0.6567623 0.5173527 0.7785944 0.6926085
## 
## $suma
## [1] 3.415704
## 
## $cantidad
## [1] 5
\end{verbatim}

\hypertarget{ejemplo-23}{%
\subsection*{Ejemplo}\label{ejemplo-23}}
\addcontentsline{toc}{subsection}{Ejemplo}

Construya una función que reciba dos números de la recta real y que entregue el punto médio de estos números. El resultado debe ser un mensaje por pantalla.

\textbf{Solución}

El punto médio entre dos valores es la suma de los números divido entre dos. La función \texttt{cat} sirve para concatenar objetos y presentarlos por pantalla. A continuación el código para la función requerida.

\begin{Shaded}
\begin{Highlighting}[]
\NormalTok{medio }\OtherTok{\textless{}{-}} \ControlFlowTok{function}\NormalTok{(a, b) \{}
\NormalTok{  medio }\OtherTok{\textless{}{-}}\NormalTok{ (a }\SpecialCharTok{+}\NormalTok{ b) }\SpecialCharTok{/} \DecValTok{2}
  \FunctionTok{cat}\NormalTok{(}\StringTok{"El punto medio de los valores"}\NormalTok{, a, }\StringTok{"y"}\NormalTok{, b,}
      \StringTok{"ingresados es"}\NormalTok{, medio)}
\NormalTok{\}}

\FunctionTok{medio}\NormalTok{(}\AttributeTok{a=}\SpecialCharTok{{-}}\DecValTok{3}\NormalTok{, }\AttributeTok{b=}\SpecialCharTok{{-}}\DecValTok{1}\NormalTok{)  }\CommentTok{\# Probando la función}
\end{Highlighting}
\end{Shaded}

\begin{verbatim}
## El punto medio de los valores -3 y -1 ingresados es -2
\end{verbatim}

\begin{rmdnote}
La función \texttt{cat} es muy útil para presentar resultados por pantalla. Consulte la ayuda de la función para ver otros ejemplos.
\end{rmdnote}

\hypertarget{ejercicios-2}{%
\section*{EJERCICIOS}\label{ejercicios-2}}
\addcontentsline{toc}{section}{EJERCICIOS}

Construir funciones en R que realicen lo solicitado.

\begin{enumerate}
\def\labelenumi{\arabic{enumi}.}
\item
  Construya una función que reciba dos números reales \texttt{a} y \texttt{b}, la función debe decir cuál es el mayor de ellos.
\item
  Escriba una función llamada \texttt{media} que calcule la media muestral de un vector numérico \texttt{x} ingresado a la función. A continuación la fórmula para calcular la media muestral.
\end{enumerate}

\[\bar{x}=\frac{\sum_{i=1}^n x_i}{n}\]

Nota: no puede usar la función \texttt{mean(\ )}.

\begin{enumerate}
\def\labelenumi{\arabic{enumi}.}
\setcounter{enumi}{2}
\item
  Construya una función que encuentre las raíces de una ecuación de segundo grado. El usuario debe suministrar los coeficientes \texttt{a}, \texttt{b} y \texttt{c} de la ecuación \(ax^2+bx+c=0\) y la función debe entregar las raíces.
\item
  Escribir una función que calcule la velocidad de un proyectil dado que el usuario ingresa la distancia recorrida en Km y el tiempo necesario en minutos. Expresar el resultado se debe entregar en metros/segundo, recuerde que
\end{enumerate}

\[velocidad = \frac{espacio}{tiempo}\]
5. Escribir una función que reciba dos valores \(a\) y \(b\) y que los intercambie. Es decir, si ingresa \(a=4\) y \(b=9\) que la función entregue \(a=9\) y \(b=4\).

\begin{enumerate}
\def\labelenumi{\arabic{enumi}.}
\setcounter{enumi}{5}
\item
  Construya una función a la cual le ingrese el salario por hora y el número de horas trabajadas durante una semana por un trabajador. La función debe calcular el salario neto.
\item
  Construya una función llamada \texttt{precio} que calcule el precio total de sacar A fotocopias y B impresiones, sabiendo que los precios son 50 y 100 pesos para A y B respectivamente si el cliente es un estudiante, y de 75 y 150 para A y B si el cliente es un profesor. La función debe tener dos argumentos cuantitativos (\texttt{A} y \texttt{B}) y el argumento lógico \texttt{estudiante} que por defecto tenga el valor de \texttt{TRUE}. Use la estructura mostrada abajo.
\end{enumerate}

\begin{Shaded}
\begin{Highlighting}[]
\NormalTok{precio }\OtherTok{\textless{}{-}} \ControlFlowTok{function}\NormalTok{(A, B, }\AttributeTok{estudiante=}\ConstantTok{TRUE}\NormalTok{) \{}
\NormalTok{  ...}
\NormalTok{  ...}
\NormalTok{  ...}
  \FunctionTok{return}\NormalTok{(precio.total)}
\NormalTok{\}}
\end{Highlighting}
\end{Shaded}

\begin{enumerate}
\def\labelenumi{\arabic{enumi}.}
\setcounter{enumi}{7}
\tightlist
\item
  Construya una función llamada \texttt{salario} que le ingrese el salario por hora y el número de horas trabajadas durante una semana por un trabajador. La función debe calcular el salario neto semanal, teniendo en cuenta que si el número de horas trabajadas durante la semana es mayor de 48, esas horas de demás se consideran horas extras y tienen un 35\% de recargo. Imprima el salario neto. Use la estructura mostrada abajo.
\end{enumerate}

\begin{Shaded}
\begin{Highlighting}[]
\NormalTok{salario }\OtherTok{\textless{}{-}} \ControlFlowTok{function}\NormalTok{(num.horas, valor.hora) \{}
\NormalTok{  ...}
\NormalTok{  ...}
\NormalTok{  ...}
  \FunctionTok{return}\NormalTok{(salario.neto)}
\NormalTok{\}}
\end{Highlighting}
\end{Shaded}

\begin{enumerate}
\def\labelenumi{\arabic{enumi}.}
\setcounter{enumi}{8}
\tightlist
\item
  Construya una función llamada \texttt{nota} que calcule la nota obtenida por un alumno en una evaluación de tres puntos cuya ponderación o importancia son 20\%, 30\% y 50\% para los puntos I, II y III respectivamente. Adicionalmente la función debe generar un mensaje sobre si el estudiante aprobó la evaluación o no. El usuario debe ingresar las notas individuales de los tres puntos y la función debe entregar la nota final de la evaluación. Use la estructura mostrada abajo.
\end{enumerate}

\begin{Shaded}
\begin{Highlighting}[]
\NormalTok{nota }\OtherTok{\textless{}{-}} \ControlFlowTok{function}\NormalTok{(p1, p2, p3) \{}
\NormalTok{  ...}
\NormalTok{  ...}
\NormalTok{  ...}
\NormalTok{\}}
\end{Highlighting}
\end{Shaded}

\begin{enumerate}
\def\labelenumi{\arabic{enumi}.}
\setcounter{enumi}{9}
\tightlist
\item
  Escriba una función llamada \texttt{minimo} que permita obtener el valor mínimo de un vector numérico. No puede usar ninguna de las funciones básicas de R como \texttt{which.min()}, \texttt{which.max()}, \texttt{order()}, \texttt{min(\ )}, \texttt{max(\ )}, \texttt{sort(\ )} u \texttt{order(\ )}. Use la estructura mostrada abajo.
\end{enumerate}

\begin{Shaded}
\begin{Highlighting}[]
\NormalTok{minimo }\OtherTok{\textless{}{-}} \ControlFlowTok{function}\NormalTok{(x) \{}
\NormalTok{  ...}
\NormalTok{  ...}
  \FunctionTok{return}\NormalTok{(minimo)}
\NormalTok{\}}
\end{Highlighting}
\end{Shaded}

\begin{enumerate}
\def\labelenumi{\arabic{enumi}.}
\setcounter{enumi}{10}
\tightlist
\item
  Construya una función que calcule las coordenadas del punto medio \(M\) entre dos puntos \(A\) y \(B\). Vea la siguiente figura para una ilustración. ¿Cuáles cree usted que deben ser los parámetros de entrada de la función?
\end{enumerate}

\hypertarget{read}{%
\chapter{Lectura de bases de datos}\label{read}}

En este capítulo se mostrará cómo leer una base de datos externa hacia R.

\hypertarget{quuxe9-es-una-base-de-datos}{%
\section{¿Qué es una base de datos?}\label{quuxe9-es-una-base-de-datos}}

Una base de datos es un arreglo ordenado de variables numéricas, lógicas y cualitativas.

En la siguiente figura se ilustran los elementos de una base de datos.

\hypertarget{en-quuxe9-formato-almacenar-una-base-de-datos}{%
\section{¿En qué formato almacenar una base de datos?}\label{en-quuxe9-formato-almacenar-una-base-de-datos}}

Usualmente los archivos con la información para ser leídos por R se pueden almacenar en formato:

\begin{itemize}
\tightlist
\item
  plano con extensión \textbf{.txt} o,
\item
  Excel con extensión \textbf{.csv}.
\end{itemize}

En las secciones siguientes se mostrará cómo almacenar datos en los dos formatos para ser leídos en R. En el Cuadro \ref{tab:dt1} se presenta una base de datos pequeña, tres observaciones y tres variables, que nos servirá como ejemplo para mostrar cómo se debe almacenar la información.

\begin{table}

\caption{\label{tab:dt1}Ejemplo de una base de datos simple.}
\centering
\begin{tabular}[t]{rll}
\toprule
Edad & Fuma & Pais\\
\midrule
35 & TRUE & Colombia\\
46 & TRUE & Francia\\
23 & FALSE & Malta\\
\bottomrule
\end{tabular}
\end{table}

\hypertarget{almacenamiento-de-informaciuxf3n-en-excel}{%
\subsection{Almacenamiento de información en Excel}\label{almacenamiento-de-informaciuxf3n-en-excel}}

Para almacenar la información del Cuadro \ref{tab:dt1} en Excel, abrimos un archivo nuevo archivo de Excel y copiamos la información tal como se muestra en la figura de abajo. Se debe iniciar en la parte superior izquierda, no se deben dejar filas vacías, no se debe colorear, no se deben colocar bordes ni nada, se ingresa la información sin embellecer el contenido. Por último se guarda el archivo en la carpeta deseada y al momento de nombrar el archivo se debe modificar la opción tipo de archivo a \textbf{csv (delimitado por comas)}.

\begin{rmdwarning}
Recuerde que el archivo de Excel se debe guardar con extensión .csv.
\end{rmdwarning}

\hypertarget{almacenamiento-de-informaciuxf3n-en-bloc-de-notas}{%
\subsection{Almacenamiento de información en bloc de notas}\label{almacenamiento-de-informaciuxf3n-en-bloc-de-notas}}

Para almacenar la información del Cuadro \ref{tab:dt1} en un bloc de notas abrimos un archivo nuevo de bloc de notas y copiamos la información tal como se muestra en la figura de abajo. Se copian los nombres de las variables o los datos separados por un espacio obtenido con la tecla tabuladora, cada línea se finaliza con un \emph{enter}. Para guardar el archivo se recomienda que el cursor quede al inicio de una línea vacía. En la figura de abajo se señala la posición del cursor con la flecha roja, a pesar de que no éxiste línea número 5, el curso debe quedar al inicio de esa línea número 5.

Es posible mejorar la apariencia de la información almacenada en el bloc de notas si, en lugar de usar espacios con la barra espaciadora, se colocan los espacios con la barra tabuladora, así la información se ve más organizada y se puede chequear fácilmente la información ingresada. En la siguiente figura se muestra la información para el ejemplo, claramente se nota la organización de la información.

\begin{rmdtip}
Una buena práctica es usar la barra tabuladora para separar, eso permite que la información se vea ordenada.
\end{rmdtip}

\hypertarget{funciuxf3n-read.table}{%
\section{\texorpdfstring{Función \texttt{read.table}}{Función read.table}}\label{funciuxf3n-read.table}}

La función \texttt{read.table} se puede usar para leer bases de datos hacia R. La estructura de la función con los parámetros más comunes de uso es la siguiente.

\begin{Shaded}
\begin{Highlighting}[]
\FunctionTok{read.table}\NormalTok{(file, header, sep, dec)}
\end{Highlighting}
\end{Shaded}

Los argumentos de la función \texttt{read.table} son:

\begin{itemize}
\tightlist
\item
  \texttt{file}: nombre o ruta donde están alojados los datos. Puede ser un url o una dirección del computador. Es también posible usar \texttt{file.choose()} para que se abra un ventana y adjuntar el archivo deseado manualmente.
\item
  \texttt{header}: valor lógico, se usa \texttt{TRUE} si la primera línea de la base de datos tiene los nombres de las variables, caso contrario se usa \texttt{FALSE}.
\item
  \texttt{sep}: tipo de separación interna para los datos dentro del archivo. Los valores usuales para este parámetros son:

  \begin{itemize}
  \tightlist
  \item
    \texttt{sep=\textquotesingle{},\textquotesingle{}} si el archivo tiene extensión .csv.
  \item
    \texttt{sep=\textquotesingle{}\textquotesingle{}} si el archivo es bloc de notas con espacios por la barra \textbf{espaciadora}.
  \item
    \texttt{sep=\textquotesingle{}\textbackslash{}t\textquotesingle{}} si el archivo es bloc de notas con espacios por la barra \textbf{tabuladora}.
  \end{itemize}
\item
  \texttt{dec}: símbolo con el cual están indicados los decimales.
\end{itemize}

\hypertarget{ejemplo-24}{%
\subsection*{Ejemplo}\label{ejemplo-24}}
\addcontentsline{toc}{subsection}{Ejemplo}

Crear la base de datos del Cuadro \ref{tab:dt1} en Excel y bloc de notas para practicar la lectura de base de datos desde R.

\textbf{Solución}

Lo primero que se debe hacer para realizar lo solicitado es construir tres archivos (uno de Excel y dos bloc de notas) igual a los mostrados en las figuras anteriores. Vamos a suponer que los nombres para cada uno de ellos son \texttt{base1.csv}, \texttt{base2.txt} y \texttt{base3.txt} respectivamente.

\hypertarget{para-excel}{%
\subsubsection*{Para Excel}\label{para-excel}}
\addcontentsline{toc}{subsubsection}{Para Excel}

Para leer el archivo de Excel llamado \texttt{base1.csv} podemos usar el siguiente código.

\begin{Shaded}
\begin{Highlighting}[]
\NormalTok{datos }\OtherTok{\textless{}{-}} \FunctionTok{read.table}\NormalTok{(}\AttributeTok{file=}\StringTok{\textquotesingle{}C:/Users/mi\_usuario/Desktop/base1.csv\textquotesingle{}}\NormalTok{,}
                    \AttributeTok{header=}\ConstantTok{TRUE}\NormalTok{, }\AttributeTok{sep=}\StringTok{\textquotesingle{},\textquotesingle{}}\NormalTok{)}
\NormalTok{datos}
\end{Highlighting}
\end{Shaded}

La dirección \texttt{file=\textquotesingle{}C:/Users/mi\_usuario/Desktop/base1.csv\textquotesingle{}} le indica a R en qué lugar del computador debe buscar el archivo, note que se debe usar el símbolo \texttt{/} para que sea un dirección válida. Substituya la dirección del código anterior con la dirección donde se encuentra su archivo para que pueda leer la base de datos.

Si no se conoce la ubicación del archivo a leer o si la dirección es muy extensa, se puede usar \texttt{file.choose()} para que se abra una ventana y así adjuntar manualmente el archivo. A continuación se muestra el código para hacerlo de esta manera.

\begin{Shaded}
\begin{Highlighting}[]
\NormalTok{datos }\OtherTok{\textless{}{-}} \FunctionTok{read.table}\NormalTok{(}\FunctionTok{file.choose}\NormalTok{(), }\AttributeTok{header=}\ConstantTok{TRUE}\NormalTok{, }\AttributeTok{sep=}\StringTok{\textquotesingle{},\textquotesingle{}}\NormalTok{)}
\NormalTok{datos}
\end{Highlighting}
\end{Shaded}

\hypertarget{para-bloc-de-notas-con-barra-espaciadora}{%
\subsubsection*{Para bloc de notas con barra espaciadora}\label{para-bloc-de-notas-con-barra-espaciadora}}
\addcontentsline{toc}{subsubsection}{Para bloc de notas con barra espaciadora}

Para leer el archivo de Excel llamado \texttt{base2.txt} podemos usar el siguiente código.

\begin{Shaded}
\begin{Highlighting}[]
\NormalTok{datos }\OtherTok{\textless{}{-}} \FunctionTok{read.table}\NormalTok{(}\AttributeTok{file=}\StringTok{\textquotesingle{}C:/Users/mi\_usuario/Desktop/base2.txt\textquotesingle{}}\NormalTok{,}
                    \AttributeTok{header=}\ConstantTok{TRUE}\NormalTok{, }\AttributeTok{sep=}\StringTok{\textquotesingle{}\textquotesingle{}}\NormalTok{)}
\NormalTok{datos}
\end{Highlighting}
\end{Shaded}

\hypertarget{para-bloc-de-notas-con-barra-tabuladora}{%
\subsubsection*{Para bloc de notas con barra tabuladora}\label{para-bloc-de-notas-con-barra-tabuladora}}
\addcontentsline{toc}{subsubsection}{Para bloc de notas con barra tabuladora}

Para leer el archivo de Excel llamado \texttt{base3.txt} podemos usar el siguiente código.

\begin{Shaded}
\begin{Highlighting}[]
\NormalTok{datos }\OtherTok{\textless{}{-}} \FunctionTok{read.table}\NormalTok{(}\AttributeTok{file=}\StringTok{\textquotesingle{}C:/Users/mi\_usuario/Desktop/base3.txt\textquotesingle{}}\NormalTok{,}
                    \AttributeTok{header=}\ConstantTok{TRUE}\NormalTok{, }\AttributeTok{sep=}\StringTok{\textquotesingle{}}\SpecialCharTok{\textbackslash{}t}\StringTok{\textquotesingle{}}\NormalTok{)}
\NormalTok{datos}
\end{Highlighting}
\end{Shaded}

\begin{rmdnote}
El usuario puede usar indiferentemente \texttt{file=\textquotesingle{}C:/Users/bla/bla\textquotesingle{}} o \texttt{file.choose()} para ingresar el archivo, con la práctica se aprende a decidir cuando conviene una u otra forma.
\end{rmdnote}

\begin{rmdwarning}
Un error frecuente es escribir la dirección o ubicación del archivo usando \texttt{\textbackslash{}}, lo correcto es usar \texttt{/}.
\end{rmdwarning}

\hypertarget{ejemplo-25}{%
\subsection*{Ejemplo}\label{ejemplo-25}}
\addcontentsline{toc}{subsection}{Ejemplo}

Leer la base de datos sobre apartamentos usados en la ciudad de Medellín que está disponible en la página web cuya url es: \url{https://raw.githubusercontent.com/fhernanb/datos/master/aptos2015}

\textbf{Solución}

Para leer la base de datos desde una url usamos el siguiente código.

\begin{Shaded}
\begin{Highlighting}[]
\NormalTok{enlace }\OtherTok{\textless{}{-}} \StringTok{\textquotesingle{}https://raw.githubusercontent.com/fhernanb/datos/master/aptos2015\textquotesingle{}}
\NormalTok{datos }\OtherTok{\textless{}{-}} \FunctionTok{read.table}\NormalTok{(}\AttributeTok{file=}\NormalTok{enlace, }\AttributeTok{header=}\ConstantTok{TRUE}\NormalTok{)}
\end{Highlighting}
\end{Shaded}

La base de datos ingresada queda en el marco de datos llamado \texttt{datos} y ya está disponible para usarla.

\hypertarget{lectura-de-bases-de-datos-en-excel}{%
\section{Lectura de bases de datos en Excel}\label{lectura-de-bases-de-datos-en-excel}}

Algunas veces los datos están disponibles en un archivo estándar de Excel, y dentro de cada archivo hojas con la información a utilizar. En estos casos se recomienda usar el paquete \textbf{readxl} \citep{R-readxl} y en particular la función \texttt{readxl}. A continuación un ejemplo de cómo proceder en estos casos.

\hypertarget{ejemplo-26}{%
\subsection*{Ejemplo}\label{ejemplo-26}}
\addcontentsline{toc}{subsection}{Ejemplo}

En este \href{https://github.com/fhernanb/datos/blob/master/BD_Excel.xlsx}{enlace} está disponible un archivo de Excel llamado BD\_Excel.xlxs, una vez se ha abierto la página donde está alojado el archivo, se debe descargar y guardar en alguna carpeta. El archivo contiene dos bases de datos muy pequeñas, en la primera hoja llamada \textbf{Hijos} está la información de un grupo de niños y en la segunda hoja llamada \textbf{Padres} está la información de los padres. ¿Cómo se pueden leer las dos bases de datos?

\textbf{Solución}

Lo primero que se debe hacer es instalar el paquete \textbf{readxl}, la instalación de cualquier paquete en un computador se hace una sola vez y éste quedará instalado para ser usado las veces que se requiera. La función para instalar un paquete cualquiera es \texttt{install.packages}, a continuación se muestra el código necesario para instalar el paquete \textbf{readxl}.

\begin{Shaded}
\begin{Highlighting}[]
\FunctionTok{install.packages}\NormalTok{(}\StringTok{"readxl"}\NormalTok{)}
\end{Highlighting}
\end{Shaded}

Una vez instalado el paquete es necesario cargarlo, la función para cargar el paquete en la sesión actual de R es \texttt{library}. La instrucción para cargar el paquete es la siguiente:

\begin{Shaded}
\begin{Highlighting}[]
\FunctionTok{library}\NormalTok{(readxl)}
\end{Highlighting}
\end{Shaded}

\begin{rmdwarning}
La instalación de un paquete con \texttt{install.packages} se hace sólo una vez y no más. Cargar el paquete con \texttt{library} en la sesión actual se debe hacer siempre que se vaya a usar el paquete.
\end{rmdwarning}

Luego de haber cargado el paquete \textbf{readxl} se puede usar la función \texttt{read\_xl} para leer la información contenida en las hojas. A continuación el código para crear la base de datos \texttt{hijos} contenida en el archivo BD\_Excel.xlsx.

\begin{Shaded}
\begin{Highlighting}[]
\NormalTok{hijos }\OtherTok{\textless{}{-}} \FunctionTok{read\_excel}\NormalTok{(}\FunctionTok{file.choose}\NormalTok{(), }\AttributeTok{sheet=}\StringTok{\textquotesingle{}Hijos\textquotesingle{}}\NormalTok{)}
\FunctionTok{as.data.frame}\NormalTok{(hijos)  }\CommentTok{\# Para ver el contenido}
\end{Highlighting}
\end{Shaded}

\begin{verbatim}
##   Edad Grado    ComicFav
## 1    8     2    Superman
## 2    6     1      Batman
## 3    9     3      Batman
## 4   10     5 Bob Esponja
## 5    8     4      Batman
## 6    9     4 Bob Esponja
\end{verbatim}

A continuación el código para crear la base de datos \texttt{padres} contenida en el archivo BD\_Excel.xlsx.

\begin{Shaded}
\begin{Highlighting}[]
\NormalTok{padres }\OtherTok{\textless{}{-}} \FunctionTok{read\_excel}\NormalTok{(}\StringTok{\textquotesingle{}BD\_Excel.xlsx\textquotesingle{}}\NormalTok{, }\AttributeTok{sheet=}\StringTok{\textquotesingle{}Padres\textquotesingle{}}\NormalTok{)}
\FunctionTok{as.data.frame}\NormalTok{(padres)  }\CommentTok{\# Para ver el contenido}
\end{Highlighting}
\end{Shaded}

\begin{verbatim}
##   Edad   EstCivil NumHijos
## 1   45    Soltero        1
## 2   50     Casado        0
## 3   35     Casado        3
## 4   65 Divorciado        1
\end{verbatim}

La función \texttt{read\_excel} tiene otros parámetros adicionales útiles para leer bases de datos, se recomienda consultar la ayuda de la función escribiendo en la consola \texttt{help(read\_excel)}.

\hypertarget{ejercicios-3}{%
\section*{EJERCICIOS}\label{ejercicios-3}}
\addcontentsline{toc}{section}{EJERCICIOS}

Realice los siguiente ejercicios propuestos.

\begin{enumerate}
\def\labelenumi{\arabic{enumi}.}
\tightlist
\item
  En el Cuadro \ref{tab:toy} se presenta una base de datos sencilla. Almacene la información del cuadro en dos archivos diferentes, en Excel y en bloc de notas. Lea los dos archivos con la función \texttt{read.table} y compare los resultados obtenidos con la del Cuadro \ref{tab:toy} fuente.
\end{enumerate}

\begin{table}

\caption{\label{tab:toy}Base de datos para practicar lectura.}
\centering
\begin{tabular}[t]{llrr}
\toprule
Fuma & Pasatiempo & Num\_hermanos & Mesada\\
\midrule
Si & Lectura & 0 & 4500\\
Si & NA & 2 & 2600\\
No & Correr & 4 & 1000\\
No & Correr & NA & 3990\\
Si & TV & 3 & 2570\\
\addlinespace
No & TV & 1 & 2371\\
Si & Correr & 1 & 1389\\
NA & Correr & 0 & 4589\\
Si & Lectura & 2 & NA\\
\bottomrule
\end{tabular}
\end{table}

\begin{enumerate}
\def\labelenumi{\arabic{enumi}.}
\setcounter{enumi}{1}
\tightlist
\item
  En la url \url{https://raw.githubusercontent.com/fhernanb/datos/master/medidas_cuerpo} están disponibles los datos sobre medidas corporales para un grupo de estudiante de la universidad, use la función \texttt{read.table} para leer la base de datos.
\end{enumerate}

\hypertarget{tablas}{%
\chapter{Tablas de frecuencia}\label{tablas}}

Las tablas de frecuencia son muy utilizadas en estadística y R permite crear tablas de una forma sencilla. En este capítulo se explican las principales funciones para la elaboración de tablas.

\hypertarget{tabla-de-contingencia-con-table}{%
\section{\texorpdfstring{Tabla de contingencia con \texttt{table} \index{table}}{Tabla de contingencia con table }}\label{tabla-de-contingencia-con-table}}

La función \texttt{table} sirve para construir tablas de frecuencia de una vía, a continuación la estrctura de la función.

\begin{Shaded}
\begin{Highlighting}[]
\FunctionTok{table}\NormalTok{(..., exclude, useNA)}
\end{Highlighting}
\end{Shaded}

Los parámetros de la función son:

\begin{itemize}
\tightlist
\item
  \texttt{...} espacio para ubicar los nombres de los objetos (variables o vectores) para los cuales se quiere construir la tabla.
\item
  \texttt{exclude}: vector con los niveles a remover de la tabla. Si \texttt{exclude=NULL} implica que se desean ver los \texttt{NA}, lo que equivale a \texttt{useNA\ =\ \textquotesingle{}always\textquotesingle{}}.
\item
  \texttt{useNA}: instrucción de lo que se desea con los \texttt{NA}. Hay tres posibles valores para este parámetro: \texttt{\textquotesingle{}no\textquotesingle{}} si no se desean usar, \texttt{\textquotesingle{}ifany\textquotesingle{}} y \texttt{\textquotesingle{}always\textquotesingle{}} si se desean incluir.
\end{itemize}

\hypertarget{ejemplo-tabla-de-frecuencia-de-una-vuxeda}{%
\subsection*{Ejemplo: tabla de frecuencia de una vía}\label{ejemplo-tabla-de-frecuencia-de-una-vuxeda}}
\addcontentsline{toc}{subsection}{Ejemplo: tabla de frecuencia de una vía}

Considere el vector \texttt{fuma} mostrado a continuación y construya una tabla de frecuencias absolutas para los niveles de la variable frecuencia de fumar.

\begin{Shaded}
\begin{Highlighting}[]
\NormalTok{fuma }\OtherTok{\textless{}{-}} \FunctionTok{c}\NormalTok{(}\StringTok{\textquotesingle{}Frecuente\textquotesingle{}}\NormalTok{, }\StringTok{\textquotesingle{}Nunca\textquotesingle{}}\NormalTok{, }\StringTok{\textquotesingle{}A veces\textquotesingle{}}\NormalTok{, }\StringTok{\textquotesingle{}A veces\textquotesingle{}}\NormalTok{, }\StringTok{\textquotesingle{}A veces\textquotesingle{}}\NormalTok{,}
          \StringTok{\textquotesingle{}Nunca\textquotesingle{}}\NormalTok{, }\StringTok{\textquotesingle{}Frecuente\textquotesingle{}}\NormalTok{, }\ConstantTok{NA}\NormalTok{, }\StringTok{\textquotesingle{}Frecuente\textquotesingle{}}\NormalTok{, }\ConstantTok{NA}\NormalTok{, }\StringTok{\textquotesingle{}hola\textquotesingle{}}\NormalTok{, }
          \StringTok{\textquotesingle{}Nunca\textquotesingle{}}\NormalTok{, }\StringTok{\textquotesingle{}Hola\textquotesingle{}}\NormalTok{, }\StringTok{\textquotesingle{}Frecuente\textquotesingle{}}\NormalTok{, }\StringTok{\textquotesingle{}Nunca\textquotesingle{}}\NormalTok{)}
\end{Highlighting}
\end{Shaded}

A continuación se muestra el código para crear la tabla de frecuencias para la variable \texttt{fuma}.

\begin{Shaded}
\begin{Highlighting}[]
\FunctionTok{table}\NormalTok{(fuma)}
\end{Highlighting}
\end{Shaded}

\begin{verbatim}
## fuma
##   A veces Frecuente      hola      Hola     Nunca 
##         3         4         1         1         4
\end{verbatim}

De la tabla anterior vemos que NO aparece el conteo de los \texttt{NA}, para obtenerlo usamos lo siguiente.

\begin{Shaded}
\begin{Highlighting}[]
\FunctionTok{table}\NormalTok{(fuma, }\AttributeTok{useNA=}\StringTok{\textquotesingle{}always\textquotesingle{}}\NormalTok{)}
\end{Highlighting}
\end{Shaded}

\begin{verbatim}
## fuma
##   A veces Frecuente      hola      Hola     Nunca      <NA> 
##         3         4         1         1         4         2
\end{verbatim}

Vemos que hay dos niveles errados en la tabla anterior, \texttt{Hola} y \texttt{hola}. Para construir la tabla sin esos niveles errados usamos lo siguiente.

\begin{Shaded}
\begin{Highlighting}[]
\FunctionTok{table}\NormalTok{(fuma, }\AttributeTok{exclude=}\FunctionTok{c}\NormalTok{(}\StringTok{\textquotesingle{}Hola\textquotesingle{}}\NormalTok{, }\StringTok{\textquotesingle{}hola\textquotesingle{}}\NormalTok{))}
\end{Highlighting}
\end{Shaded}

\begin{verbatim}
## fuma
##   A veces Frecuente     Nunca      <NA> 
##         3         4         4         2
\end{verbatim}

Por último construyamos la tabla sin los niveles errados y los \texttt{NA}, a esta última tabla la llamaremos \texttt{tabla1} para luego poder usarla. Las instrucciones para hacer esto son las siguientes.

\begin{Shaded}
\begin{Highlighting}[]
\NormalTok{tabla1 }\OtherTok{\textless{}{-}} \FunctionTok{table}\NormalTok{(fuma, }\AttributeTok{exclude=}\FunctionTok{c}\NormalTok{(}\StringTok{\textquotesingle{}Hola\textquotesingle{}}\NormalTok{, }\StringTok{\textquotesingle{}hola\textquotesingle{}}\NormalTok{, }\ConstantTok{NA}\NormalTok{))}
\NormalTok{tabla1}
\end{Highlighting}
\end{Shaded}

\begin{verbatim}
## fuma
##   A veces Frecuente     Nunca 
##         3         4         4
\end{verbatim}

\begin{rmdnote}
Al crear una tabla con la instrucción \texttt{table(var1,\ var2)}, la variable 1 quedará por filas mientras que la variable 2 estará en las columnas.
\end{rmdnote}

\hypertarget{ejemplo-tabla-de-frecuencia-de-dos-vuxedas}{%
\subsection*{Ejemplo: tabla de frecuencia de dos vías}\label{ejemplo-tabla-de-frecuencia-de-dos-vuxedas}}
\addcontentsline{toc}{subsection}{Ejemplo: tabla de frecuencia de dos vías}

Considere otro vector \texttt{sexo} mostrado a continuación y construya una tabla de frecuencias absolutas para ver cómo se relaciona el sexo con fumar del ejemplo anterior.

\begin{Shaded}
\begin{Highlighting}[]
\NormalTok{sexo }\OtherTok{\textless{}{-}} \FunctionTok{c}\NormalTok{(}\StringTok{\textquotesingle{}Hombre\textquotesingle{}}\NormalTok{, }\StringTok{\textquotesingle{}Hombre\textquotesingle{}}\NormalTok{, }\StringTok{\textquotesingle{}Hombre\textquotesingle{}}\NormalTok{, }\ConstantTok{NA}\NormalTok{, }\StringTok{\textquotesingle{}Mujer\textquotesingle{}}\NormalTok{,}
          \StringTok{\textquotesingle{}Casa\textquotesingle{}}\NormalTok{, }\StringTok{\textquotesingle{}Mujer\textquotesingle{}}\NormalTok{, }\StringTok{\textquotesingle{}Mujer\textquotesingle{}}\NormalTok{, }\StringTok{\textquotesingle{}Mujer\textquotesingle{}}\NormalTok{, }\StringTok{\textquotesingle{}Hombre\textquotesingle{}}\NormalTok{, }\StringTok{\textquotesingle{}Mujer\textquotesingle{}}\NormalTok{, }
          \StringTok{\textquotesingle{}Hombre\textquotesingle{}}\NormalTok{, }\ConstantTok{NA}\NormalTok{, }\StringTok{\textquotesingle{}Mujer\textquotesingle{}}\NormalTok{, }\StringTok{\textquotesingle{}Mujer\textquotesingle{}}\NormalTok{)}
\end{Highlighting}
\end{Shaded}

Para construir la tabla solicitada usamos el siguiente código.

\begin{Shaded}
\begin{Highlighting}[]
\FunctionTok{table}\NormalTok{(sexo, fuma)}
\end{Highlighting}
\end{Shaded}

\begin{verbatim}
##         fuma
## sexo     A veces Frecuente hola Hola Nunca
##   Casa         0         0    0    0     1
##   Hombre       1         1    0    0     2
##   Mujer        1         3    1    0     1
\end{verbatim}

De la tabla anterior vemos que aparecen niveles errados en fuma y en sexo, para retirarlos usamos el siguiente código incluyendo en el parámetro \texttt{exclude} un vector con los niveles que \textbf{NO} deseamos en la tabla.

\begin{Shaded}
\begin{Highlighting}[]
\NormalTok{tabla2 }\OtherTok{\textless{}{-}} \FunctionTok{table}\NormalTok{(sexo, fuma, }\AttributeTok{exclude=}\FunctionTok{c}\NormalTok{(}\StringTok{\textquotesingle{}Hola\textquotesingle{}}\NormalTok{, }\StringTok{\textquotesingle{}hola\textquotesingle{}}\NormalTok{, }\StringTok{\textquotesingle{}Casa\textquotesingle{}}\NormalTok{, }\ConstantTok{NA}\NormalTok{))}
\NormalTok{tabla2}
\end{Highlighting}
\end{Shaded}

\begin{verbatim}
##         fuma
## sexo     A veces Frecuente Nunca
##   Hombre       1         1     2
##   Mujer        1         3     1
\end{verbatim}

\hypertarget{funciuxf3n-prop.table}{%
\section{\texorpdfstring{Función \texttt{prop.table} \index{prop.table}}{Función prop.table }}\label{funciuxf3n-prop.table}}

La función \texttt{prop.table} se utiliza para crear tablas de frecuencia relativa a partir de tablas de frecuencia absoluta, la estructura de la función se muestra a continuación.

\begin{Shaded}
\begin{Highlighting}[]
\FunctionTok{prop.table}\NormalTok{(x, }\AttributeTok{margin=}\ConstantTok{NULL}\NormalTok{)}
\end{Highlighting}
\end{Shaded}

\begin{itemize}
\tightlist
\item
  \texttt{x}: tabla de frecuencia.
\item
  \texttt{margin}: valor de 1 si se desean proporciones por filas, 2 si se desean por columnas, \texttt{NULL} si se desean frecuencias globales.
\end{itemize}

\hypertarget{ejemplo-tabla-de-frecuencia-relativa-de-una-vuxeda}{%
\subsection*{Ejemplo: tabla de frecuencia relativa de una vía}\label{ejemplo-tabla-de-frecuencia-relativa-de-una-vuxeda}}
\addcontentsline{toc}{subsection}{Ejemplo: tabla de frecuencia relativa de una vía}

Obtener la tabla de frencuencia relativa para la \texttt{tabla1}.

Para obtener la tabla solicitada se usa el siguiente código.

\begin{Shaded}
\begin{Highlighting}[]
\FunctionTok{prop.table}\NormalTok{(}\AttributeTok{x=}\NormalTok{tabla1)}
\end{Highlighting}
\end{Shaded}

\begin{verbatim}
## fuma
##   A veces Frecuente     Nunca 
## 0.2727273 0.3636364 0.3636364
\end{verbatim}

\hypertarget{ejemplo-tabla-de-frecuencia-relativa-de-dos-vuxedas}{%
\subsection*{Ejemplo: tabla de frecuencia relativa de dos vías}\label{ejemplo-tabla-de-frecuencia-relativa-de-dos-vuxedas}}
\addcontentsline{toc}{subsection}{Ejemplo: tabla de frecuencia relativa de dos vías}

Obtener la tabla de frencuencia relativa para la \texttt{tabla2}.

Si se desea la tabla de frecuencias relativas global se usa el siguiente código. El resultado se almacena en el objeto \texttt{tabla3} para ser usado luego.

\begin{Shaded}
\begin{Highlighting}[]
\NormalTok{tabla3 }\OtherTok{\textless{}{-}} \FunctionTok{prop.table}\NormalTok{(}\AttributeTok{x=}\NormalTok{tabla2)}
\NormalTok{tabla3}
\end{Highlighting}
\end{Shaded}

\begin{verbatim}
##         fuma
## sexo       A veces Frecuente     Nunca
##   Hombre 0.1111111 0.1111111 0.2222222
##   Mujer  0.1111111 0.3333333 0.1111111
\end{verbatim}

Si se desea la tabla de frecuencias relativas marginal por \textbf{columnas} se usa el siguiente código.

\begin{Shaded}
\begin{Highlighting}[]
\NormalTok{tabla4 }\OtherTok{\textless{}{-}} \FunctionTok{prop.table}\NormalTok{(}\AttributeTok{x=}\NormalTok{tabla2, }\AttributeTok{margin=}\DecValTok{2}\NormalTok{)}
\NormalTok{tabla4}
\end{Highlighting}
\end{Shaded}

\begin{verbatim}
##         fuma
## sexo       A veces Frecuente     Nunca
##   Hombre 0.5000000 0.2500000 0.6666667
##   Mujer  0.5000000 0.7500000 0.3333333
\end{verbatim}

\hypertarget{funciuxf3n-addmargins}{%
\section{\texorpdfstring{Función \texttt{addmargins} \index{addmargins}}{Función addmargins }}\label{funciuxf3n-addmargins}}

Esta función se puede utilizar para agregar los totales por filas o por columnas a una tabla de frecuencia absoluta o relativa. La estructura de la función es la siguiente.

\begin{Shaded}
\begin{Highlighting}[]
\FunctionTok{addmargins}\NormalTok{(A, margin)}
\end{Highlighting}
\end{Shaded}

\begin{itemize}
\tightlist
\item
  \texttt{A}: tabla de frecuencia.
\item
  \texttt{margin}: valor de 1 si se desean proporciones por columnas, 2 si se desean por filas, \texttt{NULL} si se desean frecuencias globales.
\end{itemize}

\hypertarget{ejemplo-27}{%
\subsection*{Ejemplo}\label{ejemplo-27}}
\addcontentsline{toc}{subsection}{Ejemplo}

Obtener las tablas \texttt{tabla3} y \texttt{tabla4} con los totales margines global y por columnas respectivamente.

Para hacer lo solicitado usamos las siguientes instrucciones.

\begin{Shaded}
\begin{Highlighting}[]
\FunctionTok{addmargins}\NormalTok{(tabla3)}
\end{Highlighting}
\end{Shaded}

\begin{verbatim}
##         fuma
## sexo       A veces Frecuente     Nunca       Sum
##   Hombre 0.1111111 0.1111111 0.2222222 0.4444444
##   Mujer  0.1111111 0.3333333 0.1111111 0.5555556
##   Sum    0.2222222 0.4444444 0.3333333 1.0000000
\end{verbatim}

\begin{Shaded}
\begin{Highlighting}[]
\FunctionTok{addmargins}\NormalTok{(tabla4, }\AttributeTok{margin=}\DecValTok{1}\NormalTok{)}
\end{Highlighting}
\end{Shaded}

\begin{verbatim}
##         fuma
## sexo       A veces Frecuente     Nunca
##   Hombre 0.5000000 0.2500000 0.6666667
##   Mujer  0.5000000 0.7500000 0.3333333
##   Sum    1.0000000 1.0000000 1.0000000
\end{verbatim}

\begin{rmdwarning}
Note que los valores de 1 y 2 en el parámetro \texttt{margin} de las funciones \texttt{prop.table} y \texttt{addmargins} significan lo contrario.
\end{rmdwarning}

\hypertarget{funciuxf3n-hist}{%
\section{\texorpdfstring{Función \texttt{hist} \index{hist}}{Función hist }}\label{funciuxf3n-hist}}

Construir tablas de frecuencias para variables cuantitativas es necesario en muchos procedimientos estadísticos, la función \texttt{hist} sirve para obtener este tipo de tablas. La estructura de la función es la siguiente.

\begin{Shaded}
\begin{Highlighting}[]
\FunctionTok{hist}\NormalTok{(x, }\AttributeTok{breaks=}\StringTok{\textquotesingle{}Sturges\textquotesingle{}}\NormalTok{, }\AttributeTok{include.lowest=}\ConstantTok{TRUE}\NormalTok{, }\AttributeTok{right=}\ConstantTok{TRUE}\NormalTok{, }
     \AttributeTok{plot=}\ConstantTok{FALSE}\NormalTok{)}
\end{Highlighting}
\end{Shaded}

Los parámetros de la función son:

\begin{itemize}
\tightlist
\item
  \texttt{x}: vector numérico.
\item
  \texttt{breaks}: vector con los límites de los intervalos. Si no se especifica se usar la regla de Sturges para definir el número de intervalos y el ancho.
\item
  \texttt{include.lowest}: valor lógico, si \texttt{TRUE} una observación \(x_i\) que coincida con un límite de intervalo será ubicada en el intervalo izquierdo, si \texttt{FALSE} será incluída en el intervalo a la derecha.
\item
  \texttt{right}: valor lógico, si \texttt{TRUE} los intervalos serán cerrados a derecha de la forma \((lim_{inf}, lim_{sup}]\), si es \texttt{FALSE} serán abiertos a derecha.
\item
  \texttt{plot}: valor lógico, si \texttt{FALSE} sólo se obtiene la tabla de frecuencias mientras que con \texttt{TRUE} se obtiene la representación gráfica llamada histograma.
\end{itemize}

\hypertarget{ejemplo-28}{%
\subsection*{Ejemplo}\label{ejemplo-28}}
\addcontentsline{toc}{subsection}{Ejemplo}

Genere 200 observaciones aleatorias de una distribución normal con media \(\mu=170\) y desviación \(\sigma=5\), luego construya una tabla de frecuencias para la muestra obtenida usando (a) la regla de Sturges y (b) tres intervalos con límites 150, 170, 180 y 190.

Primero se construye el vector \texttt{x} con las observaciones de la distribución normal por medio de la función \texttt{rnorm} y se especifica la media y desviación solicitada. Luego se aplica la función \texttt{hist} con el parámetro \texttt{breaks=\textquotesingle{}Sturges\textquotesingle{}}, a continuación el código utilizado.

\begin{Shaded}
\begin{Highlighting}[]
\NormalTok{x }\OtherTok{\textless{}{-}} \FunctionTok{rnorm}\NormalTok{(}\AttributeTok{n=}\DecValTok{200}\NormalTok{, }\AttributeTok{mean=}\DecValTok{170}\NormalTok{, }\AttributeTok{sd=}\DecValTok{5}\NormalTok{)}

\NormalTok{res1 }\OtherTok{\textless{}{-}} \FunctionTok{hist}\NormalTok{(}\AttributeTok{x=}\NormalTok{x, }\AttributeTok{breaks=}\StringTok{\textquotesingle{}Sturges\textquotesingle{}}\NormalTok{, }\AttributeTok{plot=}\ConstantTok{FALSE}\NormalTok{)}
\NormalTok{res1}
\end{Highlighting}
\end{Shaded}

\begin{verbatim}
## $breaks
## [1] 155 160 165 170 175 180 185
## 
## $counts
## [1]  4 30 61 71 26  8
## 
## $density
## [1] 0.004 0.030 0.061 0.071 0.026 0.008
## 
## $mids
## [1] 157.5 162.5 167.5 172.5 177.5 182.5
## 
## $xname
## [1] "x"
## 
## $equidist
## [1] TRUE
## 
## attr(,"class")
## [1] "histogram"
\end{verbatim}

El objeto \texttt{res1} es una lista donde se encuentra la información de la tabla de frecuencias para \texttt{x}. Esa lista tiene en el elemento \texttt{breaks} los límites inferior y superior de los intervalos y en el elemento \texttt{counts} están las frecuencias de cada uno de los intervalos.

Para obtener las frecuencias de tres intervalos con límites 150, 170, 180 y 190 se especifica en el parámetros \texttt{breaks} los límites. El código para obtener la segunda tabla de frecuencias se muestra a continuación.

\begin{Shaded}
\begin{Highlighting}[]
\NormalTok{res2 }\OtherTok{\textless{}{-}} \FunctionTok{hist}\NormalTok{(}\AttributeTok{x=}\NormalTok{x, }\AttributeTok{plot=}\ConstantTok{FALSE}\NormalTok{, }
             \AttributeTok{breaks=}\FunctionTok{c}\NormalTok{(}\DecValTok{150}\NormalTok{, }\DecValTok{170}\NormalTok{, }\DecValTok{180}\NormalTok{, }\DecValTok{190}\NormalTok{))}
\NormalTok{res2}
\end{Highlighting}
\end{Shaded}

\begin{verbatim}
## $breaks
## [1] 150 170 180 190
## 
## $counts
## [1] 95 97  8
## 
## $density
## [1] 0.02375 0.04850 0.00400
## 
## $mids
## [1] 160 175 185
## 
## $xname
## [1] "x"
## 
## $equidist
## [1] FALSE
## 
## attr(,"class")
## [1] "histogram"
\end{verbatim}

\hypertarget{ejemplo-29}{%
\subsection*{Ejemplo}\label{ejemplo-29}}
\addcontentsline{toc}{subsection}{Ejemplo}

Construya el vector \texttt{x} con los siguientes elementos: 1.0, 1.2, 1.3, 2.0, 2.5, 2.7, 3.0 y 3.4. Obtenga varias tablas de frecuencia con la función \texttt{hist} variando los parámetros \texttt{include.lowest} y \texttt{right}. Use como límite de los intervalos los valores 1, 2, 3 y 4.

Lo primero que debemos hacer es crear el vector \texttt{x} solicitado así:

\begin{Shaded}
\begin{Highlighting}[]
\NormalTok{x }\OtherTok{\textless{}{-}} \FunctionTok{c}\NormalTok{(}\FloatTok{1.1}\NormalTok{, }\FloatTok{1.2}\NormalTok{, }\FloatTok{1.3}\NormalTok{, }\FloatTok{2.0}\NormalTok{, }\FloatTok{2.0}\NormalTok{, }\FloatTok{2.5}\NormalTok{, }\FloatTok{2.7}\NormalTok{, }\FloatTok{3.0}\NormalTok{, }\FloatTok{3.4}\NormalTok{)}
\end{Highlighting}
\end{Shaded}

En la Figura \ref{fig:dots} se muestran los 9 puntos y con color azul se representan los límites de los intervalos.

\begin{figure}
\centering
\includegraphics{Manual_de_R_files/figure-latex/dots-1.pdf}
\caption{\label{fig:dots}Ubicación de los puntos del ejemplo con límites en color azul.}
\end{figure}

A continuación se presenta el código para obtener la tabla de frecuencia usando \texttt{rigth=TRUE}, los resultados se almacenan en el objeto \texttt{res3} y se solicitan sólo los dos primeros elementos que corresponden a los límites y frecuencias.

\begin{Shaded}
\begin{Highlighting}[]
\NormalTok{res3 }\OtherTok{\textless{}{-}} \FunctionTok{hist}\NormalTok{(x, }\AttributeTok{breaks=}\FunctionTok{c}\NormalTok{(}\DecValTok{1}\NormalTok{, }\DecValTok{2}\NormalTok{, }\DecValTok{3}\NormalTok{, }\DecValTok{4}\NormalTok{), }\AttributeTok{right=}\ConstantTok{TRUE}\NormalTok{, }\AttributeTok{plot=}\ConstantTok{FALSE}\NormalTok{)}
\NormalTok{res3[}\DecValTok{1}\SpecialCharTok{:}\DecValTok{2}\NormalTok{]}
\end{Highlighting}
\end{Shaded}

\begin{verbatim}
## $breaks
## [1] 1 2 3 4
## 
## $counts
## [1] 5 3 1
\end{verbatim}

Ahora vamos a repetir la tabla pero usando \texttt{rigth=FALSE} para ver la diferencia, en \texttt{res4} están los resultados.

\begin{Shaded}
\begin{Highlighting}[]
\NormalTok{res4 }\OtherTok{\textless{}{-}} \FunctionTok{hist}\NormalTok{(x, }\AttributeTok{breaks=}\FunctionTok{c}\NormalTok{(}\DecValTok{1}\NormalTok{, }\DecValTok{2}\NormalTok{, }\DecValTok{3}\NormalTok{, }\DecValTok{4}\NormalTok{), }\AttributeTok{right=}\ConstantTok{FALSE}\NormalTok{, }\AttributeTok{plot=}\ConstantTok{FALSE}\NormalTok{)}
\NormalTok{res4[}\DecValTok{1}\SpecialCharTok{:}\DecValTok{2}\NormalTok{]}
\end{Highlighting}
\end{Shaded}

\begin{verbatim}
## $breaks
## [1] 1 2 3 4
## 
## $counts
## [1] 3 4 2
\end{verbatim}

Al comparar los últimos dos resultados vemos que la primera frecuencia es 5 cuando \texttt{right=TRUE} porque los intervalos se consideran cerrados a la derecha.

Ahora vamos a construir una tabla de frecuencia usando \texttt{FALSE} para los parámetros \texttt{include.lowest} y \texttt{right}.

\begin{Shaded}
\begin{Highlighting}[]
\NormalTok{res5 }\OtherTok{\textless{}{-}} \FunctionTok{hist}\NormalTok{(x, }\AttributeTok{breaks=}\FunctionTok{c}\NormalTok{(}\DecValTok{1}\NormalTok{, }\DecValTok{2}\NormalTok{, }\DecValTok{3}\NormalTok{, }\DecValTok{4}\NormalTok{),}
             \AttributeTok{include.lowest=}\ConstantTok{FALSE}\NormalTok{, }\AttributeTok{right=}\ConstantTok{FALSE}\NormalTok{,}
             \AttributeTok{plot=}\ConstantTok{FALSE}\NormalTok{)}
\NormalTok{res5[}\DecValTok{1}\SpecialCharTok{:}\DecValTok{2}\NormalTok{]}
\end{Highlighting}
\end{Shaded}

\begin{verbatim}
## $breaks
## [1] 1 2 3 4
## 
## $counts
## [1] 3 4 2
\end{verbatim}

De este último resultado se ve claramente el efecto de los parámetros \texttt{include.lowest} y \texttt{right} en la construcción de tablas de frecuencia.

\hypertarget{ejercicios-4}{%
\section*{EJERCICIOS}\label{ejercicios-4}}
\addcontentsline{toc}{section}{EJERCICIOS}

Use funciones o procedimientos (varias líneas) de R para responder cada una de las siguientes preguntas.

En el Cuadro \ref{tab:toy} se presenta una base de datos sencilla. Lea la base de datos usando la funcion \texttt{read.table} y construya lo que se solicita a continuación.

\begin{enumerate}
\def\labelenumi{\arabic{enumi}.}
\tightlist
\item
  Construya una tabla de frecuencia absoluta para la variable pasatiempo.
\item
  Construya una tabla de frecuencia relativa para la variable fuma.
\item
  Construya una tabla de frecuencia relativa para las variables pasatiempo y fuma.
\item
  ¿Qué porcentaje de de los que no fuman tienen como pasatiempo la lectura.
\item
  ¿Qué porcentaje de los que corren no fuman?
\end{enumerate}

\hypertarget{central}{%
\chapter{Medidas de tendencia central}\label{central}}

En este capítulo se mostrará cómo obtener las diferentes medidas de tendencia central con R.

Para ilustrar el uso de las funciones se utilizará una base de datos llamada \textbf{medidas del cuerpo}, esta base de datos cuenta con 6 variables registradas a un grupo de 36 estudiantes de la universidad. Las variables son:

\begin{enumerate}
\def\labelenumi{\arabic{enumi}.}
\tightlist
\item
  \texttt{edad} del estudiante (años),
\item
  \texttt{peso} del estudiante (kilogramos),
\item
  \texttt{altura} del estudiante (centímetros),
\item
  \texttt{sexo} del estudiante (Hombre, Mujer),
\item
  \texttt{muneca}: perímetro de la muñeca derecha (centímetros),
\item
  \texttt{biceps}: perímetro del biceps derecho (centímetros).
\end{enumerate}

A continuación se presenta el código para definir la url donde están los datos, para cargar la base de datos en R y para mostrar por pantalla un encabezado (usando \texttt{head}) de la base de datos.

\begin{Shaded}
\begin{Highlighting}[]
\NormalTok{url }\OtherTok{\textless{}{-}} \StringTok{\textquotesingle{}https://raw.githubusercontent.com/fhernanb/datos/master/medidas\_cuerpo\textquotesingle{}}
\NormalTok{datos }\OtherTok{\textless{}{-}} \FunctionTok{read.table}\NormalTok{(}\AttributeTok{file=}\NormalTok{url, }\AttributeTok{header=}\NormalTok{T)}
\FunctionTok{head}\NormalTok{(datos)  }\CommentTok{\# Para ver el encabezado de la base de datos}
\end{Highlighting}
\end{Shaded}

\begin{verbatim}
##   edad peso altura   sexo muneca biceps
## 1   43 87.3  188.0 Hombre   12.2   35.8
## 2   65 80.0  174.0 Hombre   12.0   35.0
## 3   45 82.3  176.5 Hombre   11.2   38.5
## 4   37 73.6  180.3 Hombre   11.2   32.2
## 5   55 74.1  167.6 Hombre   11.8   32.9
## 6   33 85.9  188.0 Hombre   12.4   38.5
\end{verbatim}

\hypertarget{media}{%
\section{\texorpdfstring{Media \index{media} \index{mean}}{Media  }}\label{media}}

Para calcular la media de una variable cuantitativa se usa la función \texttt{mean}. Los argumentos básicos de la función \texttt{mean} son dos y se muestran a continuación.

\begin{Shaded}
\begin{Highlighting}[]
\FunctionTok{mean}\NormalTok{(x, }\AttributeTok{na.rm =} \ConstantTok{FALSE}\NormalTok{)}
\end{Highlighting}
\end{Shaded}

En el parámetro \texttt{x} se indica la variable de interés para la cual se quiere calcular la media, el parámetro \texttt{na.rm} es un valor lógico que en caso de ser \texttt{TRUE}, significa que se deben remover las observaciones con \texttt{NA}, el valor por defecto para este parámetro es \texttt{FALSE}.

\hypertarget{ejemplo-30}{%
\subsection*{Ejemplo}\label{ejemplo-30}}
\addcontentsline{toc}{subsection}{Ejemplo}

Suponga que queremos obtener la altura media del grupo de estudiantes.

Para encontrar la media general se usa la función \texttt{mean} sobre el vector númerico \texttt{datos\$altura}.

\begin{Shaded}
\begin{Highlighting}[]
\FunctionTok{mean}\NormalTok{(}\AttributeTok{x=}\NormalTok{datos}\SpecialCharTok{$}\NormalTok{altura)}
\end{Highlighting}
\end{Shaded}

\begin{verbatim}
## [1] 171.5556
\end{verbatim}

Del anterior resultado podemos decir que la estatura media o promedio de los estudiantes es 171.5555556 centímetros.

\hypertarget{ejemplo-31}{%
\subsection*{Ejemplo}\label{ejemplo-31}}
\addcontentsline{toc}{subsection}{Ejemplo}

Suponga que ahora queremos la altura media pero diferenciando por sexo.

Para hacer esto se debe primero dividir o partir el vector de altura según los niveles de la variable sexo, esto se consigue por medio de la función \texttt{split} y el resultado será una lista con tantos elementos como niveles tenga la variable sexo. Luego a cada uno de los elementos de la lista se le aplica la función \texttt{mean} con la ayuda de \texttt{sapply} o \texttt{tapply}. A continuación el código completo para obtener las alturas medias para hombres y mujeres.

\begin{Shaded}
\begin{Highlighting}[]
\FunctionTok{sapply}\NormalTok{(}\FunctionTok{split}\NormalTok{(}\AttributeTok{x=}\NormalTok{datos}\SpecialCharTok{$}\NormalTok{altura, }\AttributeTok{f=}\NormalTok{datos}\SpecialCharTok{$}\NormalTok{sexo), mean)}
\end{Highlighting}
\end{Shaded}

\begin{verbatim}
##   Hombre    Mujer 
## 179.0778 164.0333
\end{verbatim}

El resultado es un vector con dos elementos, vemos que la altura media para hombres es 179.0777778 centímetros y que para las mujeres es de 164.0333333 centímetros.

¿Qué sucede si se usa \texttt{tapply} en lugar de \texttt{sapply}? Substituya en el código anterior la función \texttt{sapply} por \texttt{tapply} y observe la diferencia entre los resultados.

\hypertarget{ejemplo-32}{%
\subsection*{Ejemplo}\label{ejemplo-32}}
\addcontentsline{toc}{subsection}{Ejemplo}

Suponga que se tiene el vector \texttt{edad} con las edades de siete personas y supóngase que para el individuo cinco no se tiene información de su edad, eso significa que el vector tendrá un \texttt{NA} en la quinta posición.

¿Cuál será la edad promedio del grupo de personas?

\begin{Shaded}
\begin{Highlighting}[]
\NormalTok{edad }\OtherTok{\textless{}{-}} \FunctionTok{c}\NormalTok{(}\DecValTok{18}\NormalTok{, }\DecValTok{23}\NormalTok{, }\DecValTok{26}\NormalTok{, }\DecValTok{32}\NormalTok{, }\ConstantTok{NA}\NormalTok{, }\DecValTok{32}\NormalTok{, }\DecValTok{29}\NormalTok{)}
\FunctionTok{mean}\NormalTok{(}\AttributeTok{x=}\NormalTok{edad)}
\end{Highlighting}
\end{Shaded}

\begin{verbatim}
## [1] NA
\end{verbatim}

Al correr el código anterior se obtiene un error y es debido al símbolo \texttt{NA} en la quinta posición. Para calcular la media sólo con los datos de los cuales se tiene información, se incluye el argumento \texttt{na.rm\ =\ TRUE} para que R remueva los \texttt{NA}. El código correcto a usar en este caso es:

\begin{Shaded}
\begin{Highlighting}[]
\FunctionTok{mean}\NormalTok{(}\AttributeTok{x=}\NormalTok{edad, }\AttributeTok{na.rm=}\ConstantTok{TRUE}\NormalTok{)}
\end{Highlighting}
\end{Shaded}

\begin{verbatim}
## [1] 26.66667
\end{verbatim}

De este último resultado se obtiene que la edad promedio de los individuos es 26.67 años.

\hypertarget{mediana}{%
\section{\texorpdfstring{Mediana \index{mediana} \index{median}}{Mediana  }}\label{mediana}}

Para calcular la mediana de una variable cantitativa se usa la función \texttt{median}. Los argumentos básicos de la función \texttt{median} son dos y se muestran a continuación.

\begin{Shaded}
\begin{Highlighting}[]
\FunctionTok{median}\NormalTok{(x, }\AttributeTok{na.rm =} \ConstantTok{FALSE}\NormalTok{)}
\end{Highlighting}
\end{Shaded}

En el parámetro \texttt{x} se indica la variable de interés para la cual se quiere calcular la mediana, el parámetro \texttt{na.rm} es un valor lógico que en caso de ser \texttt{TRUE}, significa que se deben remover las observaciones con \texttt{NA}, el valor por defecto para este parámetro es \texttt{FALSE}.

\hypertarget{ejemplo-33}{%
\subsection*{Ejemplo}\label{ejemplo-33}}
\addcontentsline{toc}{subsection}{Ejemplo}

Calcular la edad mediana para los estudiantes de la base de datos.

Para obtener la mediana usamos el siguiente código:

\begin{Shaded}
\begin{Highlighting}[]
\FunctionTok{median}\NormalTok{(}\AttributeTok{x=}\NormalTok{datos}\SpecialCharTok{$}\NormalTok{edad)}
\end{Highlighting}
\end{Shaded}

\begin{verbatim}
## [1] 28
\end{verbatim}

y obtenemos que la mitad de los estudiantes tienen edades mayores o iguales a 28 años.

El resultado anterior se pudo haber obtenido con la función \texttt{quantile} e indicando que se desea el cuantil 50 así:

\begin{Shaded}
\begin{Highlighting}[]
\FunctionTok{quantile}\NormalTok{(}\AttributeTok{x=}\NormalTok{datos}\SpecialCharTok{$}\NormalTok{edad, }\AttributeTok{probs=}\FloatTok{0.5}\NormalTok{)}
\end{Highlighting}
\end{Shaded}

\begin{verbatim}
## 50% 
##  28
\end{verbatim}

\hypertarget{moda}{%
\section{\texorpdfstring{Moda \index{moda}}{Moda }}\label{moda}}

La moda de una variable cuantitativa corresponde a valor o valores que más se repiten, una forma sencilla de encontrar la moda es construir una tabla de frecuencias y observar los valores con mayor frecuencia.

\hypertarget{ejemplo-34}{%
\subsection*{Ejemplo}\label{ejemplo-34}}
\addcontentsline{toc}{subsection}{Ejemplo}

Calcular la moda para la variable edad de la base de datos de estudiantes.

Se construye la tabla con la función \texttt{table} y se crea el objeto \texttt{tabla} para almacenarla.

\begin{Shaded}
\begin{Highlighting}[]
\NormalTok{tabla }\OtherTok{\textless{}{-}} \FunctionTok{table}\NormalTok{(datos}\SpecialCharTok{$}\NormalTok{edad)}
\NormalTok{tabla}
\end{Highlighting}
\end{Shaded}

\begin{verbatim}
## 
## 19 20 21 22 23 24 25 26 28 29 30 32 33 35 37 40 43 45 51 55 65 
##  1  1  1  3  2  1  5  3  2  1  2  1  1  2  3  1  2  1  1  1  1
\end{verbatim}

Al mirar con detalle la tabla anterior se observa que el valor que más se repite es la edad de 25 años en 5 ocasiones. Si la tabla hubiese sido mayor, la inspección visual nos podría tomar unos segundos o hasta minutos y podríamos equivocarnos, por esa razón es mejor ordenar los resultados de la tabla.

Para observar los valores con mayor frecuencia de la tabla se puede ordenar la tabla usando la función \texttt{sort} de la siguiente manera:

\begin{Shaded}
\begin{Highlighting}[]
\FunctionTok{sort}\NormalTok{(tabla, }\AttributeTok{decreasing=}\ConstantTok{TRUE}\NormalTok{)}
\end{Highlighting}
\end{Shaded}

\begin{verbatim}
## 
## 25 22 26 37 23 28 30 35 43 19 20 21 24 29 32 33 40 45 51 55 65 
##  5  3  3  3  2  2  2  2  2  1  1  1  1  1  1  1  1  1  1  1  1
\end{verbatim}

De esta manera se ve fácilmente que la variable edad es unimodal con valor de 25 años.

\hypertarget{varia}{%
\chapter{Medidas de variabilidad}\label{varia}}

En este capítulo se mostrará cómo obtener las diferentes medidas de variabilidad con R.

Para ilustrar el uso de las funciones se utilizará la base de datos llamada \textbf{aptos2015}, esta base de datos cuenta con 11 variables registradas a apartamentos usados en la ciudad de Medellín. Las variables de la base de datos son:

\begin{enumerate}
\def\labelenumi{\arabic{enumi}.}
\tightlist
\item
  \texttt{precio}: precio de venta del apartamento (millones de pesos),
\item
  \texttt{mt2}: área del apartamento (\(m^2\)),
\item
  \texttt{ubicacion}: lugar de ubicación del aparamentos en la ciudad (cualitativa),
\item
  \texttt{estrato}: nivel socioeconómico donde está el apartamento (2 a 6),
\item
  \texttt{alcobas}: número de alcobas del apartamento,
\item
  \texttt{banos}: número de baños del apartamento,
\item
  \texttt{balcon}: si el apartamento tiene balcón (si o no),
\item
  \texttt{parqueadero}: si el apartamento tiene parqueadero (si o no),
\item
  \texttt{administracion}: valor mensual del servicio de administración (millones de pesos),
\item
  \texttt{avaluo}: valor del apartamento en escrituras (millones de pesos),
\item
  \texttt{terminado}: si el apartamento se encuentra terminado (si o no).
\end{enumerate}

A continuación se presenta el código para definir la url donde están los datos, para cargar la base de datos en R y para mostrar por pantalla un encabezado (usando \texttt{head}) de la base de datos.

\begin{Shaded}
\begin{Highlighting}[]
\NormalTok{url }\OtherTok{\textless{}{-}} \StringTok{\textquotesingle{}https://raw.githubusercontent.com/fhernanb/datos/master/aptos2015\textquotesingle{}}
\NormalTok{datos }\OtherTok{\textless{}{-}} \FunctionTok{read.table}\NormalTok{(}\AttributeTok{file=}\NormalTok{url, }\AttributeTok{header=}\NormalTok{T)}
\FunctionTok{head}\NormalTok{(datos)  }\CommentTok{\# Para ver el encabezado de la base de datos}
\end{Highlighting}
\end{Shaded}

\begin{verbatim}
##   precio   mt2 ubicacion estrato alcobas banos balcon parqueadero
## 1     79 43.16     norte       3       3     1     si          si
## 2     93 56.92     norte       2       2     1     si          si
## 3    100 66.40     norte       3       2     2     no          no
## 4    123 61.85     norte       2       3     2     si          si
## 5    135 89.80     norte       4       3     2     si          no
## 6    140 71.00     norte       3       3     2     no          si
##   administracion   avaluo terminado
## 1          0.050 14.92300        no
## 2          0.069 27.00000        si
## 3          0.000 15.73843        no
## 4          0.130 27.00000        no
## 5          0.000 39.56700        si
## 6          0.120 31.14551        si
\end{verbatim}

\hypertarget{rango}{%
\section{\texorpdfstring{Rango \index{rango} \index{range}}{Rango  }}\label{rango}}

Para calcular el rango de una variable cuantitativa se usa la función \texttt{range}. Los argumentos básicos de la función \texttt{range} son dos y se muestran abajo.

\begin{Shaded}
\begin{Highlighting}[]
\FunctionTok{range}\NormalTok{(x, }\AttributeTok{na.rm =} \ConstantTok{FALSE}\NormalTok{)}
\end{Highlighting}
\end{Shaded}

En el parámetro \texttt{x} se indica la variable de interés para la cual se quiere calcular el rango, el parámetro \texttt{na.rm} es un valor lógico que en caso de ser \texttt{TRUE}, significa que se deben remover las observaciones con \texttt{NA}, el valor por defecto para este parámetro es \texttt{FALSE}.

La función \texttt{range} entrega el valor mínimo y máximo de la variable que se ingresó, para obtener el valor de rango se debe restar del valor máximo el valor mínimo.

\hypertarget{ejemplo-35}{%
\subsection*{Ejemplo}\label{ejemplo-35}}
\addcontentsline{toc}{subsection}{Ejemplo}

Suponga que queremos obtener el rango para la variable precio de los apartamentos.

\textbf{Solución}

Para obtener el rango usamos el siguiente código.

\begin{Shaded}
\begin{Highlighting}[]
\FunctionTok{range}\NormalTok{(datos}\SpecialCharTok{$}\NormalTok{precio)}
\end{Highlighting}
\end{Shaded}

\begin{verbatim}
## [1]   25 1700
\end{verbatim}

Otra forma de escribir el código anterior de forma secuencial es utilizando el operador pipe \texttt{\%\textgreater{}\%}. Este operador se puede leer como ``entonces'' y permite escribir código que cuenta una historia.

\begin{Shaded}
\begin{Highlighting}[]
\FunctionTok{library}\NormalTok{(dplyr) }\CommentTok{\# Para cargar el paquete dplyr}
\NormalTok{datos }\SpecialCharTok{\%\textgreater{}\%} \FunctionTok{select}\NormalTok{(precio) }\SpecialCharTok{\%\textgreater{}\%} \FunctionTok{range}\NormalTok{()}
\end{Highlighting}
\end{Shaded}

\begin{verbatim}
## [1]   25 1700
\end{verbatim}

\begin{rmdnote}
El código de arriba se puede leer así: Tome los datos, entonces seleccione el precio, entonces calcule el rango.
\end{rmdnote}

De la salida anterior podemos ver que los precios de los apartamentos van desde 25 hasta 1700 millones de pesos, es decir, el rango de la variable precio es sería igual 1700-25=1675 millones de pesos.

\hypertarget{ejemplo-36}{%
\subsection*{Ejemplo}\label{ejemplo-36}}
\addcontentsline{toc}{subsection}{Ejemplo}

Suponga que queremos obtener nuevamente el rango para la variable precio de los apartamentos pero diferenciando por el estrato.

\textbf{Solución}

Para calcular el rango (max-min) para el precio pero diferenciando por el estrato podemos usar el siguiente código.

\begin{Shaded}
\begin{Highlighting}[]
\NormalTok{datos }\SpecialCharTok{\%\textgreater{}\%} 
  \FunctionTok{group\_by}\NormalTok{(estrato) }\SpecialCharTok{\%\textgreater{}\%} 
  \FunctionTok{summarise}\NormalTok{(}\AttributeTok{el\_rango=}\FunctionTok{max}\NormalTok{(precio)}\SpecialCharTok{{-}}\FunctionTok{min}\NormalTok{(precio))}
\end{Highlighting}
\end{Shaded}

\begin{verbatim}
## # A tibble: 5 x 2
##   estrato el_rango
##     <int>    <dbl>
## 1       2      103
## 2       3      225
## 3       4      610
## 4       5     1325
## 5       6     1560
\end{verbatim}

\begin{rmdnote}
El código anterior se puede leer así: Tome los datos, entonces agrúpelos por estrato, entonces haga un resumen llamado el\_rango que se obtiene como el resultado de restar el mínimo de precio al máximo de precio.
\end{rmdnote}

De los resultados podemos ver claramente que a medida que aumenta de estrato el rango del precio de los apartamentos aumenta. Apartamentos de estrato bajo tienden a tener precios similares mientras que los precios de venta para apartamentos de estratos altos tienden a ser muy diferentes entre si.

\hypertarget{varianza}{%
\section{Varianza}\label{varianza}}

La varianza es otra medida de qué tanto se alejan las observaciones \(x_i\) en relación al promedio y se mide en unidades cuadradas. Existen dos formas de calcular la varianza dependiendo de si estamos trabajando con una muestra o con la población.

La varianza \textbf{muestral} (\(S^2\)) se define así:

\[
S^2=\frac{\sum_{i=1}^{i=n}(x_i-\bar{x})^2}{n-1},
\]

donde \(\bar{x}\) representa el promedio muestral.

La varianza \textbf{poblacional} (\(\sigma^2\)) se define así:

\[
\sigma^2=\frac{\sum_{i=1}^{i=n}(x_i-\mu)^2}{n},
\]

donde \(\mu\) representa el promedio poblacional.

Para calcular la varianza muestral de una variable cuantitativa se usa la función \texttt{var}. Los argumentos básicos de la función \texttt{var} son dos y se muestran abajo.

\begin{Shaded}
\begin{Highlighting}[]
\FunctionTok{var}\NormalTok{(x, }\AttributeTok{na.rm =} \ConstantTok{FALSE}\NormalTok{)}
\end{Highlighting}
\end{Shaded}

En el parámetro \texttt{x} se indica la variable de interés para la cual se quiere calcular la varianza muestral, el parámetro \texttt{na.rm} es un valor lógico que en caso de ser \texttt{TRUE}, significa que se deben remover las observaciones con \texttt{NA}, el valor por defecto para este parámetro es \texttt{FALSE}.

\hypertarget{ejemplo-37}{%
\subsection*{Ejemplo}\label{ejemplo-37}}
\addcontentsline{toc}{subsection}{Ejemplo}

Suponga que queremos determinar cuál ubicación en la ciudad presenta mayor varianza en los precios de los apartamentos y cuántos apartamentos hay en cada ubicación.

\textbf{Solución}

Como nos interesa calcular la varianza y hacer un conteo por cada ubicación, vamos a agrupar los datos por ubicación. Para realizar lo solicitado podemos utilizar el siguiente código.

\begin{Shaded}
\begin{Highlighting}[]
\NormalTok{datos }\SpecialCharTok{\%\textgreater{}\%} 
  \FunctionTok{group\_by}\NormalTok{(ubicacion) }\SpecialCharTok{\%\textgreater{}\%} 
  \FunctionTok{summarize}\NormalTok{(}\AttributeTok{n=}\FunctionTok{n}\NormalTok{(),}
            \AttributeTok{varianza=}\FunctionTok{var}\NormalTok{(precio))}
\end{Highlighting}
\end{Shaded}

\begin{verbatim}
## # A tibble: 7 x 3
##   ubicacion          n varianza
##   <chr>          <int>    <dbl>
## 1 aburra sur       169    4169.
## 2 belen guayabal    67    2528.
## 3 centro            38    2588.
## 4 laureles          73   25351.
## 5 norte             10    1009.
## 6 occidente         69    3596.
## 7 poblado          268   84497.
\end{verbatim}

De los resultados anteriores se nota que los apartamentos ubicados en el Poblado tienen la mayor variabilidad en el precio, este resultado se confirma al dibujar un boxplot para la variable precio dada la ubicación, en la Figura \ref{fig:box1} se muestra el boxplot y se ve claramente la dispersión de los precios en el Poblado. El código usado para generar la Figura \ref{fig:box1} se presenta a continuación.

\begin{Shaded}
\begin{Highlighting}[]
\FunctionTok{with}\NormalTok{(datos, }\FunctionTok{boxplot}\NormalTok{(precio }\SpecialCharTok{\textasciitilde{}}\NormalTok{ ubicacion, }\AttributeTok{ylab=}\StringTok{\textquotesingle{}Precio (millones)\textquotesingle{}}\NormalTok{))}
\end{Highlighting}
\end{Shaded}

\begin{figure}
\centering
\includegraphics{Manual_de_R_files/figure-latex/box1-1.pdf}
\caption{\label{fig:box1}Boxplot para el precio de los apartamentos dada la ubicación.}
\end{figure}

\hypertarget{ejemplo-38}{%
\subsection*{Ejemplo}\label{ejemplo-38}}
\addcontentsline{toc}{subsection}{Ejemplo}

¿Puedo aplicar la función \texttt{var} y la función \texttt{sd} a marcos de datos?

\textbf{Solución}

La respuesta es \textbf{NO}. La función \texttt{sd} se aplica sólo a vectores mientras que la función \texttt{var} de puede aplicar tanto a vectores como a marcos de datos. Al ser aplicada a marcos de datos numéricos se obtiene una matriz en que la diagonal representa las varianzas de las de cada una de las variables mientras que arriba y abajo de la diagonal se encuentran las covarianzas entre pares de variables.

Por ejemplo, si aplicamos la función \texttt{var} al marco de datos sólo con las variables precio, área y avaluo se obtiene una matriz de dimensión \(3 \times 3\), a continuación el código usado.

\begin{Shaded}
\begin{Highlighting}[]
\NormalTok{datos }\SpecialCharTok{\%\textgreater{}\%} 
  \FunctionTok{select}\NormalTok{(precio, mt2, avaluo) }\SpecialCharTok{\%\textgreater{}\%} 
  \FunctionTok{var}\NormalTok{()}
\end{Highlighting}
\end{Shaded}

\begin{verbatim}
##          precio       mt2    avaluo
## precio 61313.15 15874.107 33055.606
## mt2    15874.11  5579.417  9508.188
## avaluo 33055.61  9508.188 28588.853
\end{verbatim}

De la salida anterior se observa que el resultado es una matriz de varianzas y covarianzas de dimensión \(3 \times 3\).

\hypertarget{desviaciuxf3n-estuxe1ndar}{%
\section{Desviación estándar}\label{desviaciuxf3n-estuxe1ndar}}

La desviación estándar es una medida de qué tanto se alejan las observaciones \(x_i\) en relación al promedio y se mide en las mismas unidades de la variable de interés. Existen dos formas de calcular la desviación estándar dependiendo de si estamos trabajando con una muestra o con la población.

La desviación estándar \textbf{muestral} (\(S\)) para \(n\) observaciones se define así:

\[
S=\sqrt{\frac{\sum_{i=1}^{i=n}(x_i-\bar{x})^2}{n-1}},
\]
donde \(\bar{x}\) representa el promedio muestral.

La desviación estándar \textbf{poblacional} (\(\sigma\)) para \(n\) observaciones se define así:

\[
\sigma=\sqrt{\frac{\sum_{i=1}^{i=n}(x_i-\mu)^2}{n}},
\]
donde \(\mu\) representa el promedio de la población.

Para calcular en R la desviación muestral (\(S\)) de una variable cuantitativa se usa la función \texttt{sd}, los argumentos básicos de la función \texttt{sd} son dos y se muestran a continuación.

\begin{Shaded}
\begin{Highlighting}[]
\FunctionTok{sd}\NormalTok{(x, }\AttributeTok{na.rm =} \ConstantTok{FALSE}\NormalTok{)}
\end{Highlighting}
\end{Shaded}

En el parámetro \texttt{x} se indica la variable de interés para la cual se quiere calcular la desviación estándar muestral, el parámetro \texttt{na.rm} es un valor lógico que en caso de ser \texttt{TRUE}, significa que se deben remover las observaciones con \texttt{NA}, el valor por defecto para este parámetro es \texttt{FALSE}.

\hypertarget{ejemplo-39}{%
\subsection*{Ejemplo}\label{ejemplo-39}}
\addcontentsline{toc}{subsection}{Ejemplo}

Calcular la desviación estándar muestral (\(S\)) para la variable precio de los apartamentos.

\textbf{Solución}

Para obtener la desviación estándar muestral (\(S\)) solicitada usamos el siguiente código:

\begin{Shaded}
\begin{Highlighting}[]
\FunctionTok{sd}\NormalTok{(}\AttributeTok{x=}\NormalTok{datos}\SpecialCharTok{$}\NormalTok{precio)}
\end{Highlighting}
\end{Shaded}

\begin{verbatim}
## [1] 247.6149
\end{verbatim}

\hypertarget{ejemplo-40}{%
\subsection*{Ejemplo}\label{ejemplo-40}}
\addcontentsline{toc}{subsection}{Ejemplo}

Calcular la desviación estándar \textbf{poblacional} (\(\sigma\)) para el siguiente conjunto de 5 observaciones: 12, 25, 32, 15, 26.

\textbf{Solución}

Recordemos que las expresiones matemáticas para obtener \(S\) y \(\sigma\) son muy similares, la diferencia está en el denominador, para \(S\) el denominador es \(n-1\) mientras que para \(\sigma\) es \(n\). Teniendo esto en cuenta podemos construir nuestra propia función llamada \texttt{Sigma} que calcule la desviación poblacional. A continuación el código para crear nuestra propia función.

\begin{Shaded}
\begin{Highlighting}[]
\NormalTok{Sigma }\OtherTok{\textless{}{-}} \ControlFlowTok{function}\NormalTok{(x) \{}
\NormalTok{  n }\OtherTok{\textless{}{-}} \FunctionTok{length}\NormalTok{(x)}
\NormalTok{  desvi }\OtherTok{\textless{}{-}} \FunctionTok{sqrt}\NormalTok{(}\FunctionTok{sum}\NormalTok{((x}\SpecialCharTok{{-}}\FunctionTok{mean}\NormalTok{(x))}\SpecialCharTok{\^{}}\DecValTok{2}\NormalTok{) }\SpecialCharTok{/}\NormalTok{ n)}
  \FunctionTok{return}\NormalTok{(desvi)}
\NormalTok{\} }
\end{Highlighting}
\end{Shaded}

Ahora para obtener la desviación estándar \textbf{poblacional} de los datos usamos el siguiente código.

\begin{Shaded}
\begin{Highlighting}[]
\NormalTok{y }\OtherTok{\textless{}{-}} \FunctionTok{c}\NormalTok{(}\DecValTok{12}\NormalTok{, }\DecValTok{25}\NormalTok{, }\DecValTok{32}\NormalTok{, }\DecValTok{15}\NormalTok{, }\DecValTok{26}\NormalTok{)}
\FunctionTok{Sigma}\NormalTok{(y)}
\end{Highlighting}
\end{Shaded}

\begin{verbatim}
## [1] 7.402702
\end{verbatim}

\hypertarget{coeficiente-de-variaciuxf3n-cv}{%
\section{\texorpdfstring{Coeficiente de variación (\(CV\)) \index{coeficiente de variación}}{Coeficiente de variación (CV) }}\label{coeficiente-de-variaciuxf3n-cv}}

El coeficiente de variación se define como \(CV=s/\bar{x}\) y es muy sencillo de obtenerlo, la función \texttt{coef\_var} mostrada abajo permite calcularlo.

\begin{Shaded}
\begin{Highlighting}[]
\NormalTok{coef\_var }\OtherTok{\textless{}{-}} \ControlFlowTok{function}\NormalTok{(x, }\AttributeTok{na.rm =} \ConstantTok{FALSE}\NormalTok{) \{}
  \FunctionTok{sd}\NormalTok{(x, }\AttributeTok{na.rm=}\NormalTok{na.rm) }\SpecialCharTok{/} \FunctionTok{mean}\NormalTok{(x, }\AttributeTok{na.rm=}\NormalTok{na.rm)}
\NormalTok{\}}
\end{Highlighting}
\end{Shaded}

\hypertarget{ejemplo-41}{%
\subsection*{Ejemplo}\label{ejemplo-41}}
\addcontentsline{toc}{subsection}{Ejemplo}

Calcular el \(CV\) para el vector \texttt{w} definido a continuación.

\begin{Shaded}
\begin{Highlighting}[]
\NormalTok{w }\OtherTok{\textless{}{-}} \FunctionTok{c}\NormalTok{(}\DecValTok{5}\NormalTok{, }\SpecialCharTok{{-}}\DecValTok{3}\NormalTok{, }\ConstantTok{NA}\NormalTok{, }\DecValTok{8}\NormalTok{, }\DecValTok{8}\NormalTok{, }\DecValTok{7}\NormalTok{)}
\end{Highlighting}
\end{Shaded}

\textbf{Solución}

Vemos que el vector \texttt{w} tiene 6 observaciones y la tercera de ellas es un \texttt{NA}. Lo correcto aquí es usar la función \texttt{coef\_var} definida antes pero indicándole que remueva los valores faltantes, para eso se usa el siguiente código.

\begin{Shaded}
\begin{Highlighting}[]
\FunctionTok{coef\_var}\NormalTok{(}\AttributeTok{x=}\NormalTok{w, }\AttributeTok{na.rm=}\NormalTok{T)}
\end{Highlighting}
\end{Shaded}

\begin{verbatim}
## [1] 0.9273618
\end{verbatim}

\hypertarget{posi}{%
\chapter{Medidas de posición}\label{posi}}

En este capítulo se mostrará cómo obtener las diferentes medidas de posición con R.

Para ilustrar el uso de las funciones se utilizará una base de datos llamada \textbf{medidas del cuerpo}, esta base de datos cuenta con 6 variables registradas a un grupo de 36 estudiantes de la universidad. Las variables son:

\begin{enumerate}
\def\labelenumi{\arabic{enumi}.}
\tightlist
\item
  \texttt{edad} del estudiante (años),
\item
  \texttt{peso} del estudiante (kilogramos),
\item
  \texttt{altura} del estudiante (centímetros),
\item
  \texttt{sexo} del estudiante (Hombre, Mujer),
\item
  \texttt{muneca}: perímetro de la muñeca derecha (centímetros),
\item
  \texttt{biceps}: perímetro del biceps derecho (centímetros).
\end{enumerate}

A continuación se presenta el código para definir la url donde están los datos, para cargar la base de datos en R y para mostrar por pantalla un encabezado (usando \texttt{head}) de la base de datos.

\begin{Shaded}
\begin{Highlighting}[]
\NormalTok{url }\OtherTok{\textless{}{-}} \StringTok{\textquotesingle{}https://raw.githubusercontent.com/fhernanb/datos/master/medidas\_cuerpo\textquotesingle{}}
\NormalTok{datos }\OtherTok{\textless{}{-}} \FunctionTok{read.table}\NormalTok{(}\AttributeTok{file=}\NormalTok{url, }\AttributeTok{header=}\NormalTok{T)}
\FunctionTok{head}\NormalTok{(datos)  }\CommentTok{\# Para ver el encabezado de la base de datos}
\end{Highlighting}
\end{Shaded}

\begin{verbatim}
##   edad peso altura   sexo muneca biceps
## 1   43 87.3  188.0 Hombre   12.2   35.8
## 2   65 80.0  174.0 Hombre   12.0   35.0
## 3   45 82.3  176.5 Hombre   11.2   38.5
## 4   37 73.6  180.3 Hombre   11.2   32.2
## 5   55 74.1  167.6 Hombre   11.8   32.9
## 6   33 85.9  188.0 Hombre   12.4   38.5
\end{verbatim}

\hypertarget{cuantiles}{%
\section{\texorpdfstring{Cuantiles \index{cuantiles} \index{quantile} \index{cuartiles} \index{deciles} \index{percentiles}}{Cuantiles     }}\label{cuantiles}}

Para obtener cualquier cuantil (cuartiles, deciles y percentiles) se usa la función \texttt{quantile}. Los argumentos básicos de la función \texttt{quantile} son tres y se muestran a continuación.

\begin{Shaded}
\begin{Highlighting}[]
\FunctionTok{quantile}\NormalTok{(x, probs, }\AttributeTok{na.rm =} \ConstantTok{FALSE}\NormalTok{)}
\end{Highlighting}
\end{Shaded}

En el parámetro \texttt{x} se indica la variable de interés para la cual se quieren calcular los cuantiles, el parámetro \texttt{probs} sirve para definir los cuantiles de interés y el parámetro \texttt{na.rm} es un valor lógico que en caso de ser \texttt{TRUE}, significa que se deben remover las observaciones con \texttt{NA}, el valor por defecto para este parámetro es \texttt{FALSE}.

\hypertarget{ejemplo-42}{%
\subsection*{Ejemplo}\label{ejemplo-42}}
\addcontentsline{toc}{subsection}{Ejemplo}

Suponga que queremos obtener el percentil 5, la mediana y el decil 8 pa la altura del grupo de estudiantes.

Se solicita el percentil 5, la mediana que es el percentil 50 y el decil 8 que corresponde al percentil 80, por lo tanto es necesario indicarle a la función \texttt{quantile} que calcule los cuantiles para las ubicaciones 0.05, 0.5 y 0.8, el código para obtener las tres medidas solicitadas es el siguiente.

\begin{Shaded}
\begin{Highlighting}[]
\FunctionTok{quantile}\NormalTok{(}\AttributeTok{x=}\NormalTok{datos}\SpecialCharTok{$}\NormalTok{altura, }\AttributeTok{probs=}\FunctionTok{c}\NormalTok{(}\FloatTok{0.05}\NormalTok{, }\FloatTok{0.5}\NormalTok{, }\FloatTok{0.8}\NormalTok{))}
\end{Highlighting}
\end{Shaded}

\begin{verbatim}
##    5%   50%   80% 
## 155.2 172.7 180.3
\end{verbatim}

\hypertarget{correl}{%
\chapter{Medidas de correlación}\label{correl}}

En este capítulo se mostrará cómo obtener el coeficiente de correlación lineal para variables cuantitativas.

\hypertarget{funciuxf3n-cor}{%
\section{\texorpdfstring{Función \texttt{cor} \index{cor} \index{correlación}}{Función cor  }}\label{funciuxf3n-cor}}

La función \texttt{cor} permite calcular el coeficiente de correlación de Pearson, Kendall o Spearman para dos variables cuantitativas. La estructura de la función es la siguiente.

\begin{Shaded}
\begin{Highlighting}[]
\FunctionTok{cor}\NormalTok{(x, y, }\AttributeTok{use=}\StringTok{"everything"}\NormalTok{,}
    \AttributeTok{method=}\FunctionTok{c}\NormalTok{(}\StringTok{"pearson"}\NormalTok{, }\StringTok{"kendall"}\NormalTok{, }\StringTok{"spearman"}\NormalTok{))}
\end{Highlighting}
\end{Shaded}

Los parámetos de la función son:

\begin{itemize}
\tightlist
\item
  \texttt{x,\ y}: vectores cuantitativos.
\item
  \texttt{use}: parámetro que indica lo que se debe hacer cuando se presenten registros \texttt{NA} en alguno de los vectores. Las diferentes posibilidades son: \texttt{everything}, \texttt{all.obs}, \texttt{complete.obs}, \texttt{na.or.complete} y \texttt{pairwise.complete.obs}, el valor por defecto es \texttt{everything}.
\item
  \texttt{method}: tipo de coeficiente de correlación a calcular, por defecto es \texttt{pearson}, otros valores posibles son \texttt{kendall} y \texttt{spearman}.
\end{itemize}

\hypertarget{ejemplo-43}{%
\subsection*{Ejemplo}\label{ejemplo-43}}
\addcontentsline{toc}{subsection}{Ejemplo}

Calcular el coeficiente de correlación de Pearson para las variables área y precio de la base de datos sobre apartamentos usados.

Lo primero que se debe hacer es cargar la base de datos usando la url apropiada. Luego de esto se usa la función \texttt{cor} sobre las variables de interés. A continuación se muestra el código necesario.

\begin{Shaded}
\begin{Highlighting}[]
\NormalTok{url }\OtherTok{\textless{}{-}} \StringTok{\textquotesingle{}https://raw.githubusercontent.com/fhernanb/datos/master/aptos2015\textquotesingle{}}
\NormalTok{datos }\OtherTok{\textless{}{-}} \FunctionTok{read.table}\NormalTok{(}\AttributeTok{file=}\NormalTok{url, }\AttributeTok{header=}\NormalTok{T)}
\FunctionTok{cor}\NormalTok{(}\AttributeTok{x=}\NormalTok{datos}\SpecialCharTok{$}\NormalTok{mt2, }\AttributeTok{y=}\NormalTok{datos}\SpecialCharTok{$}\NormalTok{precio)}
\end{Highlighting}
\end{Shaded}

\begin{verbatim}
## [1] 0.8582585
\end{verbatim}

Del resultado anterior vemos que existe una correlación de 0.8582585 entre las dos variables, eso significa que apartamentos de mayor área tienden a tener precios de venta más alto. Este resultado se ilustra en la Figura \ref{fig:disp}, se nota claramente que la nube de puntos tiene un pendiente positiva y por eso el signo del coeficiente de correlación.

A continuación el código para generar la Figura \ref{fig:disp}.

\begin{Shaded}
\begin{Highlighting}[]
\FunctionTok{with}\NormalTok{(datos, }\FunctionTok{plot}\NormalTok{(}\AttributeTok{x=}\NormalTok{mt2, }\AttributeTok{y=}\NormalTok{precio, }\AttributeTok{pch=}\DecValTok{20}\NormalTok{, }\AttributeTok{col=}\StringTok{\textquotesingle{}blue\textquotesingle{}}\NormalTok{,}
                 \AttributeTok{xlab=}\StringTok{\textquotesingle{}Área del apartamento\textquotesingle{}}\NormalTok{, }\AttributeTok{las=}\DecValTok{1}\NormalTok{,}
                 \AttributeTok{ylab=}\StringTok{\textquotesingle{}Precio del apartamento (millones COP)\textquotesingle{}}\NormalTok{))}
\end{Highlighting}
\end{Shaded}

\begin{figure}
\centering
\includegraphics{Manual_de_R_files/figure-latex/disp-1.pdf}
\caption{\label{fig:disp}Diagrama de dispersión para precio versus área de los apartamentos usados.}
\end{figure}

\hypertarget{ejemplo-44}{%
\subsection*{Ejemplo}\label{ejemplo-44}}
\addcontentsline{toc}{subsection}{Ejemplo}

Para las mismas variables del ejemplo anterior calcular los coeficientes de correlación Kendall y Spearman.

A continuación el código para obtener lo solicitado.

\begin{Shaded}
\begin{Highlighting}[]
\FunctionTok{cor}\NormalTok{(}\AttributeTok{x=}\NormalTok{datos}\SpecialCharTok{$}\NormalTok{mt2, }\AttributeTok{y=}\NormalTok{datos}\SpecialCharTok{$}\NormalTok{precio, }\AttributeTok{method=}\StringTok{\textquotesingle{}pearson\textquotesingle{}}\NormalTok{)}
\end{Highlighting}
\end{Shaded}

\begin{verbatim}
## [1] 0.8582585
\end{verbatim}

\begin{Shaded}
\begin{Highlighting}[]
\FunctionTok{cor}\NormalTok{(}\AttributeTok{x=}\NormalTok{datos}\SpecialCharTok{$}\NormalTok{mt2, }\AttributeTok{y=}\NormalTok{datos}\SpecialCharTok{$}\NormalTok{precio, }\AttributeTok{method=}\StringTok{\textquotesingle{}kendall\textquotesingle{}}\NormalTok{)}
\end{Highlighting}
\end{Shaded}

\begin{verbatim}
## [1] 0.6911121
\end{verbatim}

\begin{Shaded}
\begin{Highlighting}[]
\FunctionTok{cor}\NormalTok{(}\AttributeTok{x=}\NormalTok{datos}\SpecialCharTok{$}\NormalTok{mt2, }\AttributeTok{y=}\NormalTok{datos}\SpecialCharTok{$}\NormalTok{precio, }\AttributeTok{method=}\StringTok{\textquotesingle{}spearman\textquotesingle{}}\NormalTok{)}
\end{Highlighting}
\end{Shaded}

\begin{verbatim}
## [1] 0.860306
\end{verbatim}

\hypertarget{ejemplo-45}{%
\subsection*{Ejemplo}\label{ejemplo-45}}
\addcontentsline{toc}{subsection}{Ejemplo}

Para la base de datos de apartamentos usados, ¿cuáles de las variables cuantitativas tienen mayor correlación?

Lo primero que debemos hacer es determinar cuáles son las cuantitativas de la base de datos. Para obtener información de las variables que están almacenadas en el marco de datos llamado \texttt{datos} usamos la función \texttt{str} que muestra la estructura interna de objeto.

\begin{Shaded}
\begin{Highlighting}[]
\FunctionTok{str}\NormalTok{(datos)}
\end{Highlighting}
\end{Shaded}

\begin{verbatim}
## 'data.frame':    694 obs. of  11 variables:
##  $ precio        : num  79 93 100 123 135 140 145 160 160 175 ...
##  $ mt2           : num  43.2 56.9 66.4 61.9 89.8 ...
##  $ ubicacion     : chr  "norte" "norte" "norte" "norte" ...
##  $ estrato       : int  3 2 3 2 4 3 3 3 4 4 ...
##  $ alcobas       : int  3 2 2 3 3 3 2 3 4 3 ...
##  $ banos         : int  1 1 2 2 2 2 2 2 2 2 ...
##  $ balcon        : chr  "si" "si" "no" "si" ...
##  $ parqueadero   : chr  "si" "si" "no" "si" ...
##  $ administracion: num  0.05 0.069 0 0.13 0 0.12 0.14 0.127 0 0.123 ...
##  $ avaluo        : num  14.9 27 15.7 27 39.6 ...
##  $ terminado     : chr  "no" "si" "no" "no" ...
\end{verbatim}

Del anterior resultado vemos que las variables precio, mt2, alcobas, banos, administracion y avaluo son las variables cuantitativas, las restantes son cualitativas (nominal u ordinal). Las posiciones de las variables cuantitativas en el objeto \texttt{datos} son 1, 2, 5, 6, 9, 10, así podemos construir un marco de datos sólo con la información cuantitativa, a continuación el código usado.

\begin{Shaded}
\begin{Highlighting}[]
\NormalTok{datos.cuanti }\OtherTok{\textless{}{-}}\NormalTok{ datos[, }\FunctionTok{c}\NormalTok{(}\DecValTok{1}\NormalTok{, }\DecValTok{2}\NormalTok{, }\DecValTok{5}\NormalTok{, }\DecValTok{6}\NormalTok{, }\DecValTok{9}\NormalTok{, }\DecValTok{10}\NormalTok{)]}
\CommentTok{\# La siguiente instrucción para editar los nombres de la variables}
\FunctionTok{colnames}\NormalTok{(datos.cuanti) }\OtherTok{\textless{}{-}} \FunctionTok{c}\NormalTok{(}\StringTok{\textquotesingle{}Precio\textquotesingle{}}\NormalTok{, }\StringTok{\textquotesingle{}Área\textquotesingle{}}\NormalTok{, }\StringTok{\textquotesingle{}Alcobas\textquotesingle{}}\NormalTok{,}
                            \StringTok{\textquotesingle{}Baños\textquotesingle{}}\NormalTok{, }\StringTok{\textquotesingle{}Admon\textquotesingle{}}\NormalTok{, }\StringTok{\textquotesingle{}Avaluo\textquotesingle{}}\NormalTok{)}
\NormalTok{M }\OtherTok{\textless{}{-}} \FunctionTok{round}\NormalTok{(}\FunctionTok{cor}\NormalTok{(datos.cuanti), }\AttributeTok{digits=}\DecValTok{2}\NormalTok{)}
\NormalTok{M}
\end{Highlighting}
\end{Shaded}

\begin{verbatim}
##         Precio Área Alcobas Baños Admon Avaluo
## Precio    1.00 0.86    0.19  0.63  0.75   0.79
## Área      0.86 1.00    0.31  0.67  0.77   0.75
## Alcobas   0.19 0.31    1.00  0.35  0.16   0.15
## Baños     0.63 0.67    0.35  1.00  0.55   0.53
## Admon     0.75 0.77    0.16  0.55  1.00   0.70
## Avaluo    0.79 0.75    0.15  0.53  0.70   1.00
\end{verbatim}

El anterior resultado representa la matriz de correlaciones entre las variables cuantitativas, se observa que la mayor correlación es entre las variables precio y área del apartamento.

Es posible representar gráficamente la matriz de correlaciones \texttt{M} por medio de la función \texttt{corrplot} del paquete \textbf{corrplot}\index{corrplot} \citep{R-corrplot}, a continuación el código para obtener su representación gráfica.

\begin{Shaded}
\begin{Highlighting}[]
\FunctionTok{library}\NormalTok{(}\StringTok{\textquotesingle{}corrplot\textquotesingle{}}\NormalTok{)  }\CommentTok{\# Para cargar el paquete corrplot}
\end{Highlighting}
\end{Shaded}

\begin{verbatim}
## corrplot 0.90 loaded
\end{verbatim}

\begin{Shaded}
\begin{Highlighting}[]
\FunctionTok{corrplot.mixed}\NormalTok{(M)}
\end{Highlighting}
\end{Shaded}

\begin{figure}
\centering
\includegraphics{Manual_de_R_files/figure-latex/corplot-1.pdf}
\caption{\label{fig:corplot}Matriz de coeficientes de correlación.}
\end{figure}

En la Figura \ref{fig:corplot} se muestra la matriz con los coeficientes de correlación. En la diagonal de la Figura \ref{fig:corplot} están las variables, por encima están unos círculos de colores, entre más intensidad del color, ya sea azul o rojo, mayor es la correlación, colores ténues significan correlación baja; el tamaño de los círculos está asociado al valor absoluto de correlación. Por debajo de la diagonal se observan los valores exactos de correlación en colores.

\begin{rmdtip}
La función \texttt{corrplot} es muy versátil, se pueden obtener diferentes representaciones gráficas de la matriz de correlaciones, para conocer las diferentes posibilidades recomendamos consultar este enlace: \url{https://cran.r-project.org/web/packages/corrplot/vignettes/corrplot-intro.html}.
\end{rmdtip}

\hypertarget{ejemplo-46}{%
\subsection*{Ejemplo}\label{ejemplo-46}}
\addcontentsline{toc}{subsection}{Ejemplo}

Construya dos vectores hipotéticos con el gasto y ahorro de un grupo de 7 familias, incluya dos \texttt{NA}. Calcule el coeficiente de correlación entre \texttt{ahorro} y \texttt{gasto}, use el parámetro \texttt{use} para manejar los \texttt{NA}.

A continuación se presenta el código para crear los objetos \texttt{ahorro} y \texttt{gasto} con datos ficticios. Observe que en el primer caso donde se calcula la correlación no es posible obtener un resultado debido a que por defecto \texttt{use=\textquotesingle{}everything\textquotesingle{}} y por lo tanto usa todas las observaciones incluyendo los \texttt{NA}. En el segundo caso si se obtiene un valor para la correlación debido a que se usó \texttt{use=\textquotesingle{}complete.obs\textquotesingle{}}.

\begin{Shaded}
\begin{Highlighting}[]
\NormalTok{gasto }\OtherTok{\textless{}{-}} \FunctionTok{c}\NormalTok{(}\DecValTok{170}\NormalTok{, }\DecValTok{230}\NormalTok{, }\DecValTok{120}\NormalTok{, }\DecValTok{156}\NormalTok{, }\DecValTok{256}\NormalTok{, }\ConstantTok{NA}\NormalTok{, }\DecValTok{352}\NormalTok{)}
\NormalTok{ahorro }\OtherTok{\textless{}{-}} \FunctionTok{c}\NormalTok{(}\DecValTok{45}\NormalTok{, }\DecValTok{30}\NormalTok{, }\ConstantTok{NA}\NormalTok{, }\DecValTok{35}\NormalTok{, }\DecValTok{15}\NormalTok{, }\DecValTok{65}\NormalTok{, }\DecValTok{15}\NormalTok{)}

\FunctionTok{cor}\NormalTok{(gasto, ahorro)}
\end{Highlighting}
\end{Shaded}

\begin{verbatim}
## [1] NA
\end{verbatim}

\begin{Shaded}
\begin{Highlighting}[]
\FunctionTok{cor}\NormalTok{(gasto, ahorro, }\AttributeTok{use=}\StringTok{\textquotesingle{}complete.obs\textquotesingle{}}\NormalTok{)}
\end{Highlighting}
\end{Shaded}

\begin{verbatim}
## [1] -0.8465124
\end{verbatim}

\hypertarget{ejercicios-5}{%
\section*{EJERCICIOS}\label{ejercicios-5}}
\addcontentsline{toc}{section}{EJERCICIOS}

Use funciones o procedimientos (varias líneas) de R para responder cada una de las siguientes preguntas.

\begin{enumerate}
\def\labelenumi{\arabic{enumi}.}
\item
  Para cada uno de los estratos socioeconómicos, calcular el coeficiente de correlación lineal de Pearson para las variables precio y área de la base de datos de los apartamentos usados.
\item
  Calcular los coeficientes de correlación Pearson, Kendall y Spearman para las variables cuantitativas de la base de datos sobre medidas del cuerpo explicada en el Capítulo \ref{central}. La url con la información es la siguiente: \url{https://raw.githubusercontent.com/fhernanb/datos/master/medidas_cuerpo}
\item
  Represente gráficamente las matrices de correlación obtenidas en el ejercicio anterior.
\end{enumerate}

\hypertarget{discretas}{%
\chapter{Distribuciones discretas}\label{discretas}}

En este capítulo se mostrarán las funciones de R para distribuciones discretas.

\hypertarget{funciones-disponibles-para-distribuciones-discretas}{%
\section{\texorpdfstring{Funciones disponibles para distribuciones discretas \index{distribucionecs discretas}}{Funciones disponibles para distribuciones discretas }}\label{funciones-disponibles-para-distribuciones-discretas}}

Para cada distribución discreta se tienen 4 funciones, a continuación el listado de funciones y su utilidad.

\begin{Shaded}
\begin{Highlighting}[]
\FunctionTok{dxxx}\NormalTok{(x, ...)  }\CommentTok{\# Función de masa de probabilidad, f(x)}
\FunctionTok{pxxx}\NormalTok{(q, ...)  }\CommentTok{\# Función de distribución acumulada hasta q, F(x)}
\FunctionTok{qxxx}\NormalTok{(p, ...)  }\CommentTok{\# Cuantil para el cual P(X \textless{}= q) = p}
\FunctionTok{rxxx}\NormalTok{(n, ...)  }\CommentTok{\# Generador de números aleatorios.}
\end{Highlighting}
\end{Shaded}

En el lugar de las letras \texttt{xxx} se de debe colocar el nombre de la distribución en R, a continuación el listado de nombres disponibles para las 6 distribuciones discretas básicas.

\begin{Shaded}
\begin{Highlighting}[]
\NormalTok{binom     }\CommentTok{\# Binomial}
\NormalTok{geo       }\CommentTok{\# Geométrica}
\NormalTok{nbinom    }\CommentTok{\# Binomial negativa}
\NormalTok{hyper     }\CommentTok{\# Hipergeométrica}
\NormalTok{pois      }\CommentTok{\# Poisson}
\NormalTok{multinom  }\CommentTok{\# Multinomial}
\end{Highlighting}
\end{Shaded}

Combinando las funciones y los nombres se tiene un total de 24 funciones, por ejemplo, para obtener la función de masa de probabilidad \(f(x)\) de una binomial se usa la función \texttt{dbinom(\ )} y para obtener la función acumulada \(F(x)\) de una Poisson se usa la función \texttt{ppois(\ )}.

\hypertarget{ejemplo-binomial}{%
\subsection*{Ejemplo binomial}\label{ejemplo-binomial}}
\addcontentsline{toc}{subsection}{Ejemplo binomial}

Suponga que un grupo de agentes de tránsito sale a una vía principal para revisar el estado de los buses de transporte intermunicipal. De datos históricos se sabe que un 10\% de los buses generan una mayor cantidad de humo de la permitida. En cada jornada los agentes revisan siempre 18 buses, asuma que el estado de un bus es independiente del estado de los otros buses.

\begin{enumerate}
\def\labelenumi{\arabic{enumi})}
\tightlist
\item
  Calcular la probabilidad de que se encuentren exactamente 2 buses que generan una mayor cantidad de humo de la permitida.
\end{enumerate}

Aquí se tiene una distribucion \(Binomial(n=18, p=0.1)\) y se desea calcular \(P(X=2)\). Para obtener esta probabilidad se usa la siguiente instrucción.

\begin{Shaded}
\begin{Highlighting}[]
\FunctionTok{dbinom}\NormalTok{(}\AttributeTok{x=}\DecValTok{2}\NormalTok{, }\AttributeTok{size=}\DecValTok{18}\NormalTok{, }\AttributeTok{prob=}\FloatTok{0.10}\NormalTok{)}
\end{Highlighting}
\end{Shaded}

\begin{verbatim}
## [1] 0.2835121
\end{verbatim}

Así \(P(X=2)=0.2835\).

\begin{enumerate}
\def\labelenumi{\arabic{enumi})}
\setcounter{enumi}{1}
\tightlist
\item
  Calcular la probabilidad de que el número de buses que sobrepasan el límite de generación de gases sea al menos 4.
\end{enumerate}

En este caso interesa calcular \(P(X \geq 4)\), para obtener esta probabilidad se usa la siguiente instrucción.

\begin{Shaded}
\begin{Highlighting}[]
\FunctionTok{sum}\NormalTok{(}\FunctionTok{dbinom}\NormalTok{(}\AttributeTok{x=}\DecValTok{4}\SpecialCharTok{:}\DecValTok{18}\NormalTok{, }\AttributeTok{size=}\DecValTok{18}\NormalTok{, }\AttributeTok{prob=}\FloatTok{0.10}\NormalTok{))}
\end{Highlighting}
\end{Shaded}

\begin{verbatim}
## [1] 0.09819684
\end{verbatim}

Así \(P(X \geq 4)=0.0982\)

\begin{enumerate}
\def\labelenumi{\arabic{enumi})}
\setcounter{enumi}{2}
\tightlist
\item
  Calcular la probabilidad de que tres o menos buses emitan gases por encima de lo permitido en la norma.
\end{enumerate}

En este caso interesa \(P(X\leq3)\) lo cual es \(F(x=3)\), por lo tanto, la instrucción para obtener esta probabilidad es

\begin{Shaded}
\begin{Highlighting}[]
\FunctionTok{pbinom}\NormalTok{(}\AttributeTok{q=}\DecValTok{3}\NormalTok{, }\AttributeTok{size=}\DecValTok{18}\NormalTok{, }\AttributeTok{prob=}\FloatTok{0.10}\NormalTok{)}
\end{Highlighting}
\end{Shaded}

\begin{verbatim}
## [1] 0.9018032
\end{verbatim}

Así \(P(X\leq3)=F(x=3)=0.9018\)

\begin{enumerate}
\def\labelenumi{\arabic{enumi})}
\setcounter{enumi}{3}
\tightlist
\item
  Dibujar la función de masa de probabilidad.
\end{enumerate}

Para dibujar la función de masa de probabilidad para una \(Binomial(n=18, p=0.1)\) se usa el siguiente código.

\begin{Shaded}
\begin{Highlighting}[]
\NormalTok{x }\OtherTok{\textless{}{-}} \DecValTok{0}\SpecialCharTok{:}\DecValTok{18}  \CommentTok{\# Soporte (dominio) de la variable}
\NormalTok{Probabilidad }\OtherTok{\textless{}{-}} \FunctionTok{dbinom}\NormalTok{(}\AttributeTok{x=}\NormalTok{x, }\AttributeTok{size=}\DecValTok{18}\NormalTok{, }\AttributeTok{prob=}\FloatTok{0.1}\NormalTok{)}
\FunctionTok{plot}\NormalTok{(}\AttributeTok{x=}\NormalTok{x, }\AttributeTok{y=}\NormalTok{Probabilidad, }
     \AttributeTok{type=}\StringTok{\textquotesingle{}h\textquotesingle{}}\NormalTok{, }\AttributeTok{las=}\DecValTok{1}\NormalTok{, }\AttributeTok{lwd=}\DecValTok{6}\NormalTok{)}
\end{Highlighting}
\end{Shaded}

\begin{figure}
\centering
\includegraphics{Manual_de_R_files/figure-latex/binom1-1.pdf}
\caption{\label{fig:binom1}Función de masa de probabilidad para una \(Binomial(n=18, p=0.1)\).}
\end{figure}

En la Figura \ref{fig:binom1} se muestra la función de masa de probabilidad para la \(Binomial(n=18, p=0.1)\), de esta figura se observa claramente que la mayor parte de la probabilidad está concentrada para valores pequeños de \(X\) debido a que la probabilidad de éxito individual es \(p=0.10\). Valores de \(X \geq 7\) tienen una probabilidad muy pequeña y es por eso que las longitudes de sus barras son muy cortas.

\begin{enumerate}
\def\labelenumi{\arabic{enumi})}
\setcounter{enumi}{4}
\tightlist
\item
  Generar con 100 de una distribución \(Binomial(n=18, p=0.1)\) y luego calcular las frecuencias muestrales y compararlas con las probabilidades teóricas.
\end{enumerate}

La muestra aleatoria se obtiene con la función \texttt{rbinom} y los resultados se almacenan en el objeto \texttt{m}, por último se construye la tabla de frecuencias relativas, a continuación el código usado.

\begin{Shaded}
\begin{Highlighting}[]
\NormalTok{m }\OtherTok{\textless{}{-}} \FunctionTok{rbinom}\NormalTok{(}\AttributeTok{n=}\DecValTok{100}\NormalTok{, }\AttributeTok{size=}\DecValTok{18}\NormalTok{, }\AttributeTok{prob=}\FloatTok{0.1}\NormalTok{)}
\NormalTok{m  }\CommentTok{\# Para ver lo que hay dentro de m}
\end{Highlighting}
\end{Shaded}

\begin{verbatim}
##   [1] 0 1 4 1 3 4 2 3 1 0 3 2 1 4 1 1 1 3 0 5 0 3 5 3 2 3 1 1 2 4 1 3 2 2 3 2 2
##  [38] 3 2 3 0 2 2 2 1 1 3 1 1 1 1 1 1 1 1 3 1 2 4 4 7 2 4 0 1 2 1 1 2 3 2 1 3 1
##  [75] 1 2 3 5 1 5 0 1 2 1 4 1 1 2 1 2 0 2 2 3 3 0 1 1 3 2
\end{verbatim}

\begin{Shaded}
\begin{Highlighting}[]
\FunctionTok{prop.table}\NormalTok{(}\FunctionTok{table}\NormalTok{(m))  }\CommentTok{\# Tabla de frecuencia relativa}
\end{Highlighting}
\end{Shaded}

\begin{verbatim}
## m
##    0    1    2    3    4    5    7 
## 0.09 0.35 0.24 0.19 0.08 0.04 0.01
\end{verbatim}

A pesar de ser una muestra aleatoria de sólo 100 observaciones, se observa que las frecuencias relativas obtenidas son muy cercanas a las mostradas en la Figura \ref{fig:binom1}.

\hypertarget{ejemplo-geomuxe9trica}{%
\subsection*{Ejemplo geométrica}\label{ejemplo-geomuxe9trica}}
\addcontentsline{toc}{subsection}{Ejemplo geométrica}

En una línea de producción de bombillos se sabe que sólo el 1\% de los bombillos son defectuosos. Una máquina automática toma un bombillo y lo prueba, si el bombillo enciende, se siguen probando los bombillos hasta que se encuentre \textbf{un} bombillo defectuoso, ahí se para la línea de producción y se toman los correctivos necesarios para mejorar el proceso.

\begin{enumerate}
\def\labelenumi{\arabic{enumi})}
\tightlist
\item
  Calcular la probabilidad de que se necesiten probar 125 bombillos para encontrar el primer bombillo defectuoso.
\end{enumerate}

En la distribución geométrica, la variable \(X\) representa el número de fracasos antes de encontrar el único éxito, por lo tanto, en este caso el interés es calcular \(P(X=124)\). La instrucción para obtener esta probabiliad es la siguiente.

\begin{Shaded}
\begin{Highlighting}[]
\FunctionTok{dgeom}\NormalTok{(}\AttributeTok{x=}\DecValTok{124}\NormalTok{, }\AttributeTok{prob=}\FloatTok{0.01}\NormalTok{)}
\end{Highlighting}
\end{Shaded}

\begin{verbatim}
## [1] 0.002875836
\end{verbatim}

\begin{enumerate}
\def\labelenumi{\arabic{enumi})}
\setcounter{enumi}{1}
\tightlist
\item
  Calcular \(P(X \leq 8)\).
\end{enumerate}

En este caso interesa \(P(X \leq 50)\) lo que equivale a \(F(8)\), la instrucción para obtener la probabilidad es la siguiente.

\begin{Shaded}
\begin{Highlighting}[]
\FunctionTok{pgeom}\NormalTok{(}\AttributeTok{q=}\DecValTok{50}\NormalTok{, }\AttributeTok{prob=}\FloatTok{0.01}\NormalTok{)}
\end{Highlighting}
\end{Shaded}

\begin{verbatim}
## [1] 0.401044
\end{verbatim}

\begin{enumerate}
\def\labelenumi{\arabic{enumi})}
\setcounter{enumi}{2}
\tightlist
\item
  Encontrar el cuantil \(q\) tal que \(P(X \leq q) = 0.40\).
\end{enumerate}

En este caso interesa encontrar el cuantil \(q\) que cumpla la condición de que hasta \(q\) esté el 40\% de las observaciones, por esa razón se usa la función \texttt{qgeom} como se muestra a continuación.

\begin{Shaded}
\begin{Highlighting}[]
\FunctionTok{qgeom}\NormalTok{(}\AttributeTok{p=}\FloatTok{0.4}\NormalTok{, }\AttributeTok{prob=}\FloatTok{0.01}\NormalTok{)}
\end{Highlighting}
\end{Shaded}

\begin{verbatim}
## [1] 50
\end{verbatim}

\begin{rmdnote}
Note que las funciones \texttt{pxxx} y \texttt{qxxx} están relacionadas, \texttt{pxxx} entrega la probabilidad hasta el cuantil \(q\) mientras \texttt{qxxx} entrega el cuantil en el que se acumula \(p\) probabilidad.
\end{rmdnote}

\hypertarget{ejemplo-binomial-negativa}{%
\subsection*{Ejemplo binomial negativa}\label{ejemplo-binomial-negativa}}
\addcontentsline{toc}{subsection}{Ejemplo binomial negativa}

Una familia desea tener hijos hasta conseguir \textbf{2 niñas}, la probabilidad individual de obtener una niña es 0.5 y se supone que todos los nacimientos son individuales, es decir, un sólo bebé.

\begin{enumerate}
\def\labelenumi{\arabic{enumi})}
\tightlist
\item
  Calcular la probabilidad de que se necesiten 4 hijos, es decir, 4 nacimientos para consguir las dos niñas.
\end{enumerate}

En este problema se tiene una distribución binomial negativa con \(r=2\) niñas, los éxitos deseados por la familia. La variable \(X\) representa los fracasos, es decir los niños, hasta que se obtienen los éxitos \(r=2\) deseados.

En este caso lo que interesa es \(P(\text{familia tenga 4})\), en otras palabras interesa \(P(X=2)\), la instrucción para calcular la probabilidad es la siguiente.

\begin{Shaded}
\begin{Highlighting}[]
\FunctionTok{dnbinom}\NormalTok{(}\AttributeTok{x=}\DecValTok{2}\NormalTok{, }\AttributeTok{size=}\DecValTok{2}\NormalTok{, }\AttributeTok{prob=}\FloatTok{0.5}\NormalTok{)}
\end{Highlighting}
\end{Shaded}

\begin{verbatim}
## [1] 0.1875
\end{verbatim}

\begin{enumerate}
\def\labelenumi{\arabic{enumi})}
\setcounter{enumi}{1}
\tightlist
\item
  Calcular \(P(\text{familia tenga al menos 4 hijos})\).
\end{enumerate}

Aquí interesa calcular \(P(X \geq 2)=P(X=2)+P(X=3)+\ldots\), como esta probabilidad va hasta infinito, se debe usar el complemento así:

\[P(X \geq 2) = 1 - [P(X=0)+P(X=1)]\]
y para obtener la probabilidad solicitada se puede usar la función \texttt{dnbinom} de la siguiente manera.

\begin{Shaded}
\begin{Highlighting}[]
\DecValTok{1} \SpecialCharTok{{-}} \FunctionTok{sum}\NormalTok{(}\FunctionTok{dnbinom}\NormalTok{(}\AttributeTok{x=}\DecValTok{0}\SpecialCharTok{:}\DecValTok{1}\NormalTok{, }\AttributeTok{size=}\DecValTok{2}\NormalTok{, }\AttributeTok{prob=}\FloatTok{0.5}\NormalTok{))}
\end{Highlighting}
\end{Shaded}

\begin{verbatim}
## [1] 0.5
\end{verbatim}

Otra forma para obtener la probabilidad solicitada es por medio de la función \texttt{pnbinom} de la siguiente manera.

\begin{Shaded}
\begin{Highlighting}[]
\DecValTok{1} \SpecialCharTok{{-}} \FunctionTok{pnbinom}\NormalTok{(}\AttributeTok{q=}\DecValTok{1}\NormalTok{, }\AttributeTok{size=}\DecValTok{2}\NormalTok{, }\AttributeTok{prob=}\FloatTok{0.5}\NormalTok{)}
\end{Highlighting}
\end{Shaded}

\begin{verbatim}
## [1] 0.5
\end{verbatim}

\hypertarget{ejemplo-hipergeomuxe9trica}{%
\subsection*{Ejemplo hipergeométrica}\label{ejemplo-hipergeomuxe9trica}}
\addcontentsline{toc}{subsection}{Ejemplo hipergeométrica}

Un lote de partes para ensamblar en una empresa está formado por 100 elementos del proveedor A y 200 elementos del proveedor B. Se selecciona una muestra de 4 partes al azar sin reemplazo de las 300 para una revisión de calidad.

\begin{enumerate}
\def\labelenumi{\arabic{enumi})}
\tightlist
\item
  Calcular la probabilidad de que todas las 4 partes de la muestra sean del proveedor A.
\end{enumerate}

Aquí se tiene una situación que se puede modelar por medio de una distribución hipergeométrica con \(m=100\) éxitos en la población, \(n=200\) fracasos en la población y \(k=4\) el tamaño de la muestra. El objetivo es calcular \(P(X=4)\), para obtener esta probabilidad se usa la siguiente instrucción.

\begin{Shaded}
\begin{Highlighting}[]
\FunctionTok{dhyper}\NormalTok{(}\AttributeTok{x=}\DecValTok{4}\NormalTok{, }\AttributeTok{m=}\DecValTok{100}\NormalTok{, }\AttributeTok{k=}\DecValTok{4}\NormalTok{, }\AttributeTok{n=}\DecValTok{200}\NormalTok{)}
\end{Highlighting}
\end{Shaded}

\begin{verbatim}
## [1] 0.01185408
\end{verbatim}

\begin{enumerate}
\def\labelenumi{\arabic{enumi})}
\setcounter{enumi}{1}
\tightlist
\item
  Calcular la probabilidad de que dos o más de las partes sean del proveedor A.
\end{enumerate}

Aquí interesa \(P(X \geq 2)\), la instrucción para obtener esta probabilidad es.

\begin{Shaded}
\begin{Highlighting}[]
\FunctionTok{sum}\NormalTok{(}\FunctionTok{dhyper}\NormalTok{(}\AttributeTok{x=}\DecValTok{2}\SpecialCharTok{:}\DecValTok{4}\NormalTok{, }\AttributeTok{m=}\DecValTok{100}\NormalTok{, }\AttributeTok{k=}\DecValTok{4}\NormalTok{, }\AttributeTok{n=}\DecValTok{200}\NormalTok{))}
\end{Highlighting}
\end{Shaded}

\begin{verbatim}
## [1] 0.4074057
\end{verbatim}

\hypertarget{ejemplo-poisson}{%
\subsection*{Ejemplo Poisson}\label{ejemplo-poisson}}
\addcontentsline{toc}{subsection}{Ejemplo Poisson}

En una editorial se asume que todo libro de 250 páginas tiene en promedio 50 errores.

\begin{enumerate}
\def\labelenumi{\arabic{enumi})}
\tightlist
\item
  Encuentre la probabilidad de que en una página cualquiera no se encuentren errores.
\end{enumerate}

Este es un problema de distribución Poisson con tasa promedio de éxitos dada por:

\[\lambda=\frac{50 \quad errores}{libro}=\frac{0.2 \quad errores}{pagina}\]
El objetivo es calcular \(P(X=0)\), para obtener esta probabilidad de usa la siguiente instrucción.

\begin{Shaded}
\begin{Highlighting}[]
\FunctionTok{dpois}\NormalTok{(}\AttributeTok{x=}\DecValTok{0}\NormalTok{, }\AttributeTok{lambda=}\FloatTok{0.2}\NormalTok{)}
\end{Highlighting}
\end{Shaded}

\begin{verbatim}
## [1] 0.8187308
\end{verbatim}

Así \(P(X=0)=0.8187\).

\hypertarget{distribuciones-discretas-generales}{%
\section{Distribuciones discretas generales}\label{distribuciones-discretas-generales}}

En la práctica nos podemos encontramos con variables aleatorias discretas que no se ajustan a una de las distribuciones mostradas anteriormente, en esos casos, es posible manejar ese tipo de variables por medio de unas funciones básicas de R como se muestra en el siguiente ejemplo.

\hypertarget{ejemplo-47}{%
\subsection*{Ejemplo}\label{ejemplo-47}}
\addcontentsline{toc}{subsection}{Ejemplo}

El cangrejo de herradura hembra se caracteriza porque su caparazón se adhieren los machos de la misma especie, en la Figura \ref{fig:crab} se muestra una fotografía de este cangrejo. Los investigadores están interesado en determinar cual es el patrón de variación del número de machos sobre cada hembra, para esto, se recolectó una muestra de hembras a las cuales se les observó el color, la condición de la espina, el peso en kilogramos, el ancho del caparazón en centímetros y el número de satélites o machos sobre el caparazón, la base de datos está disponible en el siguiente \href{https://raw.githubusercontent.com/fhernanb/datos/master/crab}{enlace}. \label{crabs}

\begin{figure}

{\centering \includegraphics[width=6in]{images/crab} 

}

\caption{Fotografía del cangrejo de herradura, tomada de http://sccoastalresources.com/home/2016/6/21/a-night-of-horseshoe-crab-tagging}\label{fig:crab}
\end{figure}

\begin{enumerate}
\def\labelenumi{\arabic{enumi})}
\tightlist
\item
  Encontrar la distribución de probabilidad para la variable \texttt{Sa} que corresponde al número de machos sobre el caparazón de cada hembra.
\end{enumerate}

Primero se debe leer la base de datos usando la url suministrada y luego se construye la tabla de frecuencia relativa y se almacena en el objeto \texttt{t1}.

\begin{Shaded}
\begin{Highlighting}[]
\NormalTok{url }\OtherTok{\textless{}{-}} \StringTok{\textquotesingle{}https://raw.githubusercontent.com/fhernanb/datos/master/crab\textquotesingle{}}
\NormalTok{crab }\OtherTok{\textless{}{-}} \FunctionTok{read.table}\NormalTok{(}\AttributeTok{file=}\NormalTok{url, }\AttributeTok{header=}\NormalTok{T)}

\NormalTok{t1 }\OtherTok{\textless{}{-}} \FunctionTok{prop.table}\NormalTok{(}\FunctionTok{table}\NormalTok{(crab}\SpecialCharTok{$}\NormalTok{Sa))}
\NormalTok{t1}
\end{Highlighting}
\end{Shaded}

\begin{verbatim}
## 
##           0           1           2           3           4           5 
## 0.358381503 0.092485549 0.052023121 0.109826590 0.109826590 0.086705202 
##           6           7           8           9          10          11 
## 0.075144509 0.023121387 0.034682081 0.017341040 0.017341040 0.005780347 
##          12          14          15 
## 0.005780347 0.005780347 0.005780347
\end{verbatim}

La anterior tabla de frecuencias relativas se puede representar gráficamente usando el siguiente código.

\begin{Shaded}
\begin{Highlighting}[]
\FunctionTok{plot}\NormalTok{(t1, }\AttributeTok{las=}\DecValTok{1}\NormalTok{, }\AttributeTok{lwd=}\DecValTok{5}\NormalTok{, }\AttributeTok{xlab=}\StringTok{\textquotesingle{}Número de satélites\textquotesingle{}}\NormalTok{,}
     \AttributeTok{ylab=}\StringTok{\textquotesingle{}Proporción\textquotesingle{}}\NormalTok{)}
\end{Highlighting}
\end{Shaded}

\begin{figure}
\centering
\includegraphics{Manual_de_R_files/figure-latex/pmfcrab-1.pdf}
\caption{\label{fig:pmfcrab}Función de masa de probabilidad para el número de satélites por hembra.}
\end{figure}

\begin{enumerate}
\def\labelenumi{\arabic{enumi})}
\setcounter{enumi}{1}
\tightlist
\item
  Sea \(X\) la variable número de satélites por hembra, construir la función \(F(x)\).
\end{enumerate}

Para construir \(F(x)\) se utiliza la función \texttt{ecdf} o \emph{empirical cumulative density function}, a esta función le debe ingresar el vector con la información de la variable cuantitativa, a continuación del código usado. En la Figura \ref{fig:Fcrab} se muestra la función de distribución acumulada para para el número de satélites por hembra.

\begin{Shaded}
\begin{Highlighting}[]
\NormalTok{F }\OtherTok{\textless{}{-}} \FunctionTok{ecdf}\NormalTok{(crab}\SpecialCharTok{$}\NormalTok{Sa)}
\FunctionTok{plot}\NormalTok{(F, }\AttributeTok{las=}\DecValTok{1}\NormalTok{, }\AttributeTok{main=}\StringTok{\textquotesingle{}\textquotesingle{}}\NormalTok{)}
\end{Highlighting}
\end{Shaded}

\begin{figure}
\centering
\includegraphics{Manual_de_R_files/figure-latex/Fcrab-1.pdf}
\caption{\label{fig:Fcrab}Función de distribución acumulada para el número de satélites por hembra.}
\end{figure}

\begin{enumerate}
\def\labelenumi{\arabic{enumi})}
\setcounter{enumi}{2}
\tightlist
\item
  Calcular \(P(X \leq 9)\).
\end{enumerate}

Para obtener esta probabilidad se usa el objeto \texttt{F} que es en realidad una función, a continuación la instrucción usada.

\begin{Shaded}
\begin{Highlighting}[]
\FunctionTok{F}\NormalTok{(}\DecValTok{9}\NormalTok{)}
\end{Highlighting}
\end{Shaded}

\begin{verbatim}
## [1] 0.9595376
\end{verbatim}

Así \(P(X \leq 9)=0.9595\).

\begin{enumerate}
\def\labelenumi{\arabic{enumi})}
\setcounter{enumi}{3}
\tightlist
\item
  Calcular \(P(X > 4)\).
\end{enumerate}

Para obtener esta probabilidad se usa el hecho de que \(P(X > 4) = 1 - P(X \leq 4)\), así la instrucción a usar es.

\begin{Shaded}
\begin{Highlighting}[]
\DecValTok{1} \SpecialCharTok{{-}} \FunctionTok{F}\NormalTok{(}\DecValTok{4}\NormalTok{)}
\end{Highlighting}
\end{Shaded}

\begin{verbatim}
## [1] 0.2774566
\end{verbatim}

Por lo tanto \(P(X > 4)=0.2775\).

\begin{enumerate}
\def\labelenumi{\arabic{enumi})}
\setcounter{enumi}{4}
\tightlist
\item
  Suponga que el grupo 1 está formado por las hembras cuyo ancho de caparazón es menor o igual al ancho mediano, el grupo 2 está formado por las demás hembras. ¿Será \(F(x)\) diferente para los dos grupos?
\end{enumerate}

Para realizar esto vamos a particionar el vector \texttt{Sa} en los dos grupos de acuerdo a la nueva variable \texttt{grupo} creada como se muestra a continuacion.

\begin{Shaded}
\begin{Highlighting}[]
\NormalTok{grupo }\OtherTok{\textless{}{-}} \FunctionTok{ifelse}\NormalTok{(crab}\SpecialCharTok{$}\NormalTok{Wt }\SpecialCharTok{\textless{}=} \FunctionTok{median}\NormalTok{(crab}\SpecialCharTok{$}\NormalTok{Wt), }\StringTok{\textquotesingle{}Grupo 1\textquotesingle{}}\NormalTok{, }\StringTok{\textquotesingle{}Grupo 2\textquotesingle{}}\NormalTok{)}
\NormalTok{x }\OtherTok{\textless{}{-}} \FunctionTok{split}\NormalTok{(}\AttributeTok{x=}\NormalTok{crab}\SpecialCharTok{$}\NormalTok{Sa, }\AttributeTok{f=}\NormalTok{grupo)}
\end{Highlighting}
\end{Shaded}

El objeto \texttt{x} es una lista y para acceder a los vectores allí almacenados usamos dos corchetes \texttt{{[}{[}{]}{]}}, uno dentro del otro. Luego para calcular \(F(x)\) para los dos grupos se procede así:

\begin{Shaded}
\begin{Highlighting}[]
\NormalTok{F1 }\OtherTok{\textless{}{-}} \FunctionTok{ecdf}\NormalTok{(x[[}\DecValTok{1}\NormalTok{]])}
\NormalTok{F2 }\OtherTok{\textless{}{-}} \FunctionTok{ecdf}\NormalTok{(x[[}\DecValTok{2}\NormalTok{]])}
\end{Highlighting}
\end{Shaded}

Para obtener las dos \(F(x)\) en la misma figura se usa el código siguiente.

\begin{Shaded}
\begin{Highlighting}[]
\FunctionTok{plot}\NormalTok{(F1, }\AttributeTok{col=}\StringTok{\textquotesingle{}blue\textquotesingle{}}\NormalTok{, }\AttributeTok{main=}\StringTok{\textquotesingle{}\textquotesingle{}}\NormalTok{, }\AttributeTok{las=}\DecValTok{1}\NormalTok{)}
\FunctionTok{plot}\NormalTok{(F2, }\AttributeTok{col=}\StringTok{\textquotesingle{}red\textquotesingle{}}\NormalTok{, }\AttributeTok{add=}\NormalTok{T)}
\FunctionTok{legend}\NormalTok{(}\StringTok{\textquotesingle{}bottomright\textquotesingle{}}\NormalTok{, }\AttributeTok{legend=}\FunctionTok{c}\NormalTok{(}\StringTok{\textquotesingle{}Grupo 1\textquotesingle{}}\NormalTok{, }\StringTok{\textquotesingle{}Grupo 2\textquotesingle{}}\NormalTok{),}
       \AttributeTok{col=}\FunctionTok{c}\NormalTok{(}\StringTok{\textquotesingle{}blue\textquotesingle{}}\NormalTok{, }\StringTok{\textquotesingle{}red\textquotesingle{}}\NormalTok{), }\AttributeTok{lwd=}\DecValTok{1}\NormalTok{)}
\end{Highlighting}
\end{Shaded}

\begin{figure}
\centering
\includegraphics{Manual_de_R_files/figure-latex/Fcrabs-1.pdf}
\caption{\label{fig:Fcrabs}Función de distribución acumulada para el número de satélites por hembra diferenciando por grupo.}
\end{figure}

En la Figura \ref{fig:Fcrabs} se muestran las dos \(F(x)\), en color azul para el grupo 1 y en color rojo para el grupo 2. Se observa claramente que las curvas son diferentes antes de \(x=9\). El hecho de que la curva azul esté por encima de la roja para valores menores de 9, es decir, \(F_1(x) \geq F_2(x)\), indica que las hembras del grupo 1 tienden a tener menos satélites que las del grupo 2, esto es coherente ya que las del grupo 2 son más grandes en su caparazón.

\hypertarget{continuas}{%
\chapter{Distribuciones continuas}\label{continuas}}

En este capítulo se mostrarán las funciones de R para distribuciones continuas.

\hypertarget{funciones-disponibles-para-distribuciones-continuas}{%
\section{Funciones disponibles para distribuciones continuas}\label{funciones-disponibles-para-distribuciones-continuas}}

Para cada distribución continua se tienen 4 funciones, a continuación el listado de las funciones y su utilidad.

\begin{Shaded}
\begin{Highlighting}[]
\FunctionTok{dxxx}\NormalTok{(x, ...)  }\CommentTok{\# Función de densidad de probabilidad, f(x)}
\FunctionTok{pxxx}\NormalTok{(q, ...)  }\CommentTok{\# Función de distribución acumulada hasta q, F(x)}
\FunctionTok{qxxx}\NormalTok{(p, ...)  }\CommentTok{\# Cuantil para el cual P(X \textless{}= q) = p}
\FunctionTok{rxxx}\NormalTok{(n, ...)  }\CommentTok{\# Generador de números aleatorios.}
\end{Highlighting}
\end{Shaded}

En el lugar de las letras \texttt{xxx} se de debe colocar el nombre de la distribución en R, a continuación el listado de nombres disponibles para las 11 distribuciones continuas básicas.

\begin{Shaded}
\begin{Highlighting}[]
\NormalTok{beta     }\CommentTok{\# Beta}
\NormalTok{cauchy   }\CommentTok{\# Cauchy}
\NormalTok{chisq    }\CommentTok{\# Chi{-}cuadrada}
\NormalTok{exp      }\CommentTok{\# Exponencial}
\NormalTok{f        }\CommentTok{\# F}
\NormalTok{gamma    }\CommentTok{\# Gama}
\NormalTok{lnorm    }\CommentTok{\# log{-}normal}
\NormalTok{norm     }\CommentTok{\# normal}
\NormalTok{t        }\CommentTok{\# t{-}student}
\NormalTok{unif     }\CommentTok{\# Uniforme}
\NormalTok{weibull  }\CommentTok{\# Weibull}
\end{Highlighting}
\end{Shaded}

Combinando las funciones y los nombres se tiene un total de 44 funciones, por ejemplo, para obtener la función de densidad de probabilidad \(f(x)\) de una normal se usa la función \texttt{dnorm(\ )} y para obtener la función acumulada \(F(x)\) de una Beta se usa la función \texttt{pbeta(\ )}.

\hypertarget{ejemplo-beta}{%
\subsection*{Ejemplo beta}\label{ejemplo-beta}}
\addcontentsline{toc}{subsection}{Ejemplo beta}

Considere que una variable aleatoria \(X\) se distribuye beta con parámetros \(a=2\) y \(b=5\).

\begin{enumerate}
\def\labelenumi{\arabic{enumi})}
\tightlist
\item
  Dibuje la densidad de la distribución.
\end{enumerate}

La función \texttt{dbeta} sirve para obtener la altura de la curva de una distribución beta y combinándola con la función \texttt{curve} se puede dibujar la densidad solicitada. En la Figura \ref{fig:beta1} se presenta la densidad, observe que para la combinación de parámetros \(a=2\) y \(b=5\) la distribución es sesgada a la derecha.

\begin{Shaded}
\begin{Highlighting}[]
\FunctionTok{curve}\NormalTok{(}\FunctionTok{dbeta}\NormalTok{(x, }\AttributeTok{shape1=}\DecValTok{2}\NormalTok{, }\AttributeTok{shape2=}\DecValTok{5}\NormalTok{), }\AttributeTok{lwd=}\DecValTok{3}\NormalTok{, }\AttributeTok{las=}\DecValTok{1}\NormalTok{,}
      \AttributeTok{ylab=}\StringTok{\textquotesingle{}Densidad\textquotesingle{}}\NormalTok{)}
\end{Highlighting}
\end{Shaded}

\begin{figure}
\centering
\includegraphics{Manual_de_R_files/figure-latex/beta1-1.pdf}
\caption{\label{fig:beta1}Función de densidad para una \(Beta(2, 5)\).}
\end{figure}

\begin{enumerate}
\def\labelenumi{\arabic{enumi})}
\setcounter{enumi}{1}
\tightlist
\item
  Calcular \(P(0.3 \leq X \leq 0.7)\).
\end{enumerate}

Para obtener la probabilidad o área bajo la densidad se puede usar la función \texttt{integrate}, los límites de la integral se ingresan por medio de los parámetros \texttt{lower} y \texttt{upper}. Si la función a integrar tiene parámetros adicionales como en este caso, éstos parámetros se ingresan luego de los límites de la integral. A continuación el código necesario para obtener la probabiliad solicitada.

\begin{Shaded}
\begin{Highlighting}[]
\FunctionTok{integrate}\NormalTok{(}\AttributeTok{f=}\NormalTok{dbeta, }\AttributeTok{lower=}\FloatTok{0.3}\NormalTok{, }\AttributeTok{upper=}\FloatTok{0.7}\NormalTok{,}
          \AttributeTok{shape1=}\DecValTok{2}\NormalTok{, }\AttributeTok{shape2=}\DecValTok{5}\NormalTok{)}
\end{Highlighting}
\end{Shaded}

\begin{verbatim}
## 0.40924 with absolute error < 4.5e-15
\end{verbatim}

Otra forma de obtener la probabilidad solicitada es restando de \(F(x_{max})\) la probabilidad \(F(x_{min})\). Las probabilidades acumuladas hasta un valor dado se obtienen con la función \texttt{pbeta}, a continuación el código necesario.

\begin{Shaded}
\begin{Highlighting}[]
\FunctionTok{pbeta}\NormalTok{(}\AttributeTok{q=}\FloatTok{0.7}\NormalTok{, }\AttributeTok{shape1=}\DecValTok{2}\NormalTok{, }\AttributeTok{shape2=}\DecValTok{5}\NormalTok{) }\SpecialCharTok{{-}} \FunctionTok{pbeta}\NormalTok{(}\AttributeTok{q=}\FloatTok{0.3}\NormalTok{, }\AttributeTok{shape1=}\DecValTok{2}\NormalTok{, }\AttributeTok{shape2=}\DecValTok{5}\NormalTok{)}
\end{Highlighting}
\end{Shaded}

\begin{verbatim}
## [1] 0.40924
\end{verbatim}

De ambas formas se obtiene que \(P(0.3 \leq X \leq 0.7)=0.4092\).

\begin{rmdnote}
Recuerde que para distribuciones continuas

\[ P(a < X < b) = P(a \leq X < b) = P(a < X \leq b) = P(a \leq X \leq b)\]
\end{rmdnote}

\hypertarget{ejemplo-normal-estuxe1ndar}{%
\subsection*{Ejemplo normal estándar}\label{ejemplo-normal-estuxe1ndar}}
\addcontentsline{toc}{subsection}{Ejemplo normal estándar}

Suponga que la variable aleatoria \(Z\) se distribuye normal estándar, es decir, \(Z \sim N(0, 1)\).

\begin{enumerate}
\def\labelenumi{\arabic{enumi})}
\tightlist
\item
  Calcular \(P(Z < 1.45)\).
\end{enumerate}

Para calcular la probabilidad acumulada hasta un punto dado se usa la función \texttt{pnorm} y se evalúa en el cuantil indicado, a continuación el código usado.

\begin{Shaded}
\begin{Highlighting}[]
\FunctionTok{pnorm}\NormalTok{(}\AttributeTok{q=}\FloatTok{1.45}\NormalTok{)}
\end{Highlighting}
\end{Shaded}

\begin{verbatim}
## [1] 0.9264707
\end{verbatim}

En la Figura \ref{fig:norm1} se muestra el área sombreada correspondiente a \(P(Z < 1.45)\).

\begin{enumerate}
\def\labelenumi{\arabic{enumi})}
\setcounter{enumi}{1}
\tightlist
\item
  Calcular \(P(Z > -0.37)\).
\end{enumerate}

Para calcular la probabilidad solicitada se usa nuevamente la función \texttt{pnorm} evaluada en el cuantil dado. Como el evento de interés es \(Z > -0.37\), la probabilidad solicitada se obtiene como \texttt{1\ -\ pnorm(q=-0.37)}, esto debido a que por defecto las probabilidades entregadas por la función \texttt{pxxx} son siempre a izquierda. A continuación el código usado.

\begin{Shaded}
\begin{Highlighting}[]
\DecValTok{1} \SpecialCharTok{{-}} \FunctionTok{pnorm}\NormalTok{(}\AttributeTok{q=}\SpecialCharTok{{-}}\FloatTok{0.37}\NormalTok{)}
\end{Highlighting}
\end{Shaded}

\begin{verbatim}
## [1] 0.6443088
\end{verbatim}

En la Figura \ref{fig:norm1} se muestra el área sombreada correspondiente a \(P(Z > -0.37)\).

Otra forma para obtener la probabilidad solicitada sin hacer la resta es usar el parámetro \texttt{lower.tail} para indicar que interesa la probabilidad a la derecha del cuantil dado, a continuación un código alternativo para obtener la misma probabilidad.

\begin{Shaded}
\begin{Highlighting}[]
\FunctionTok{pnorm}\NormalTok{(}\AttributeTok{q=}\SpecialCharTok{{-}}\FloatTok{0.37}\NormalTok{, }\AttributeTok{lower.tail=}\ConstantTok{FALSE}\NormalTok{)}
\end{Highlighting}
\end{Shaded}

\begin{verbatim}
## [1] 0.6443088
\end{verbatim}

\begin{enumerate}
\def\labelenumi{\arabic{enumi})}
\setcounter{enumi}{2}
\tightlist
\item
  Calcular \(P(-1.56 < Z < 2.58)\).
\end{enumerate}

Para calcular la probabilidad solicitada se obtiene la probabilidad acumulada hasta 2.58 y de ella se resta lo acumulado hasta -1.56, a continuación el código usado.

\begin{Shaded}
\begin{Highlighting}[]
\FunctionTok{pnorm}\NormalTok{(}\AttributeTok{q=}\FloatTok{2.58}\NormalTok{) }\SpecialCharTok{{-}} \FunctionTok{pnorm}\NormalTok{(}\SpecialCharTok{{-}}\FloatTok{1.56}\NormalTok{)}
\end{Highlighting}
\end{Shaded}

\begin{verbatim}
## [1] 0.93568
\end{verbatim}

En la Figura \ref{fig:norm1} se muestra el área sombreada correspondiente a \(P(-1.56 < Z < 2.58)\).

\begin{enumerate}
\def\labelenumi{\arabic{enumi})}
\setcounter{enumi}{3}
\tightlist
\item
  Calcular el cuantil \(q\) para el cual se cumple que \(P(Z<q)=0.95\).
\end{enumerate}

Para calcular el cuantil en el cual se cumple que \(P(Z<q)=0.95\) se usa la función \texttt{qnorm}, a continuación el código usado.

\begin{Shaded}
\begin{Highlighting}[]
\FunctionTok{qnorm}\NormalTok{(}\AttributeTok{p=}\FloatTok{0.95}\NormalTok{) }
\end{Highlighting}
\end{Shaded}

\begin{verbatim}
## [1] 1.644854
\end{verbatim}

En la Figura \ref{fig:norm1} se muestra el área sombreada correspondiente a \(P(Z<q)=0.95\).

\begin{figure}
\centering
\includegraphics{Manual_de_R_files/figure-latex/norm1-1.pdf}
\caption{\label{fig:norm1}Área sombreada para los ejemplos.}
\end{figure}

\begin{rmdtip}
El parámetro \texttt{lower.tail} es muy útil para indicar si estamos trabajando una cola a izquierda o una cola a derecha.
\end{rmdtip}

\hypertarget{ejemplo-normal-general}{%
\subsection*{Ejemplo normal general}\label{ejemplo-normal-general}}
\addcontentsline{toc}{subsection}{Ejemplo normal general}

Considere un proceso de elaboración de tornillos en una empresa y suponga que el diámetro de los tornillos sigue una distribución normal con media de 10 \(mm\) y varianza de 4 \(mm^2\).

\begin{enumerate}
\def\labelenumi{\arabic{enumi})}
\tightlist
\item
  Un tornillo se considera que cumple las especificaciones si su diámetro está entre 9 y 11 mm. ¿Qué porcentaje de los tornillos cumplen las especificaciones?
\end{enumerate}

Como se solicita probabilidad se debe usar \texttt{pnorm} indicando que la media es \(\mu=10\) y la desviación de la distribución es \(\sigma=2\). A continuación el código usado.

\begin{Shaded}
\begin{Highlighting}[]
\FunctionTok{pnorm}\NormalTok{(}\AttributeTok{q=}\DecValTok{11}\NormalTok{, }\AttributeTok{mean=}\DecValTok{10}\NormalTok{, }\AttributeTok{sd=}\DecValTok{2}\NormalTok{) }\SpecialCharTok{{-}} \FunctionTok{pnorm}\NormalTok{(}\AttributeTok{q=}\DecValTok{9}\NormalTok{, }\AttributeTok{mean=}\DecValTok{10}\NormalTok{, }\AttributeTok{sd=}\DecValTok{2}\NormalTok{)}
\end{Highlighting}
\end{Shaded}

\begin{verbatim}
## [1] 0.3829249
\end{verbatim}

\begin{enumerate}
\def\labelenumi{\arabic{enumi})}
\setcounter{enumi}{1}
\tightlist
\item
  Un tornillo con un diámetro mayor a 11 mm se puede reprocesar y recuperar. ¿Cuál es el porcentaje de reprocesos en la empresa?
\end{enumerate}

Como se solicita una probabilidad a derecha se usa \texttt{lower.tail=FALSE} dentro de la función \texttt{pnorm}. A continuación el código usado.

\begin{Shaded}
\begin{Highlighting}[]
\FunctionTok{pnorm}\NormalTok{(}\AttributeTok{q=}\DecValTok{11}\NormalTok{, }\AttributeTok{mean=}\DecValTok{10}\NormalTok{, }\AttributeTok{sd=}\DecValTok{2}\NormalTok{, }\AttributeTok{lower.tail=}\ConstantTok{FALSE}\NormalTok{)}
\end{Highlighting}
\end{Shaded}

\begin{verbatim}
## [1] 0.3085375
\end{verbatim}

\begin{enumerate}
\def\labelenumi{\arabic{enumi})}
\setcounter{enumi}{2}
\tightlist
\item
  El 5\% de los tornillos más delgados no se pueden reprocesar y por lo tanto son desperdicio. ¿Qué diámetro debe tener un tornillo para ser clasificado como desperdicio?
\end{enumerate}

Aquí interesa encontrar el cuantil tal que \(P(Diametro<q)=0.05\), por lo tanto se usa la función \texttt{qnorm}. A continuación el código usado.

\begin{Shaded}
\begin{Highlighting}[]
\FunctionTok{qnorm}\NormalTok{(}\AttributeTok{p=}\FloatTok{0.05}\NormalTok{, }\AttributeTok{mean=}\DecValTok{10}\NormalTok{, }\AttributeTok{sd=}\DecValTok{2}\NormalTok{)}
\end{Highlighting}
\end{Shaded}

\begin{verbatim}
## [1] 6.710293
\end{verbatim}

\begin{enumerate}
\def\labelenumi{\arabic{enumi})}
\setcounter{enumi}{3}
\tightlist
\item
  El 10\% de los tornillos más gruesos son considerados como sobredimensionados. ¿cuál es el diámetro mínimo de un tornillo para que sea considerado como sobredimensionado?
\end{enumerate}

Aquí interesa encontrar el cuantil tal que \(P(Diametro>q)=0.10\), por lo tanto se usa la función \texttt{qnorm} pero incluyendo \texttt{lower.tail=FALSE} por ser una cola a derecha. A continuación el código usado.

\begin{Shaded}
\begin{Highlighting}[]
\FunctionTok{qnorm}\NormalTok{(}\AttributeTok{p=}\FloatTok{0.10}\NormalTok{, }\AttributeTok{mean=}\DecValTok{10}\NormalTok{, }\AttributeTok{sd=}\DecValTok{2}\NormalTok{, }\AttributeTok{lower.tail=}\ConstantTok{FALSE}\NormalTok{)}
\end{Highlighting}
\end{Shaded}

\begin{verbatim}
## [1] 12.5631
\end{verbatim}

En la Figura \ref{fig:norm2} se muestran las áreas sombreadas para cada de las anteriores preguntas.

\begin{figure}
\centering
\includegraphics{Manual_de_R_files/figure-latex/norm2-1.pdf}
\caption{\label{fig:norm2}Área sombreada para el ejemplo de los tornillos.}
\end{figure}

\hypertarget{distribuciones-continuas-generales}{%
\section{Distribuciones continuas generales}\label{distribuciones-continuas-generales}}

En la práctica nos podemos encontramos con variables aleatorias continuas que no se ajustan a una de las distribuciones mostradas anteriormente, en esos casos, es posible manejar ese tipo de variables por medio de unas funciones básicas de R como se muestra en el siguiente ejemplo.

\hypertarget{ejemplo-48}{%
\subsection*{Ejemplo}\label{ejemplo-48}}
\addcontentsline{toc}{subsection}{Ejemplo}

En este ejemplo se retomará la base de datos \texttt{crab} sobre el cangrejo de herradura hembra presentado en el capítulo anterior. La base de datos \texttt{crab} contiene las siguientes variables: el color del caparazón, la condición de la espina, el peso en kilogramos, el ancho del caparazón en centímetros y el número de satélites o machos sobre el caparazón, la base de datos está disponible en el siguiente \href{https://raw.githubusercontent.com/fhernanb/datos/master/crab}{enlace}.

\begin{enumerate}
\def\labelenumi{\arabic{enumi})}
\tightlist
\item
  Sea \(X\) la variable peso del cangrejo, dibuje la densidad para \(X\).
\end{enumerate}

Para obtener la densidad muestral de un vector cuantitativo se usa la función \texttt{density}, y para dibujar la densidad se usa la función \texttt{plot} aplicada a un objeto obtenido con \texttt{density}, a continuación el código necesario para dibujar la densidad.

\begin{Shaded}
\begin{Highlighting}[]
\NormalTok{url }\OtherTok{\textless{}{-}} \StringTok{\textquotesingle{}https://raw.githubusercontent.com/fhernanb/datos/master/crab\textquotesingle{}}
\NormalTok{crab }\OtherTok{\textless{}{-}} \FunctionTok{read.table}\NormalTok{(}\AttributeTok{file=}\NormalTok{url, }\AttributeTok{header=}\NormalTok{T)}

\FunctionTok{plot}\NormalTok{(}\FunctionTok{density}\NormalTok{(crab}\SpecialCharTok{$}\NormalTok{W), }\AttributeTok{main=}\StringTok{\textquotesingle{}\textquotesingle{}}\NormalTok{, }\AttributeTok{lwd=}\DecValTok{5}\NormalTok{, }\AttributeTok{las=}\DecValTok{1}\NormalTok{,}
     \AttributeTok{xlab=}\StringTok{\textquotesingle{}Peso (Kg)\textquotesingle{}}\NormalTok{, }\AttributeTok{ylab=}\StringTok{\textquotesingle{}Densidad\textquotesingle{}}\NormalTok{)}
\end{Highlighting}
\end{Shaded}

\begin{figure}
\centering
\includegraphics{Manual_de_R_files/figure-latex/crabcont1-1.pdf}
\caption{\label{fig:crabcont1}Función de densidad \(f(x)\) para el peso de los cangrejos.}
\end{figure}

En la Figura \ref{fig:crabcont1} se muestra la densidad para la variable peso de los cangrejos, esta densidad es bastante simétrica y el intervalo de mayor densidad está entre 22 y 30 kilogramos.

\begin{enumerate}
\def\labelenumi{\arabic{enumi})}
\setcounter{enumi}{1}
\tightlist
\item
  Dibujar \(F(x)\) para el peso del cangrejo.
\end{enumerate}

Para dibujar la función \(F(x)\) se usa la función \texttt{ecdf} y se almacena el resultado en el objeto \texttt{F}, luego se dibuja la función deseada usando \texttt{plot}. A continuación el código utilizado. En la Figura \ref{fig:crabcont2} se presenta el dibujo para \(F(x)\).

\begin{Shaded}
\begin{Highlighting}[]
\NormalTok{F }\OtherTok{\textless{}{-}} \FunctionTok{ecdf}\NormalTok{(crab}\SpecialCharTok{$}\NormalTok{W)}
\FunctionTok{plot}\NormalTok{(F, }\AttributeTok{main=}\StringTok{\textquotesingle{}\textquotesingle{}}\NormalTok{, }\AttributeTok{xlab=}\StringTok{\textquotesingle{}Peso (Kg)\textquotesingle{}}\NormalTok{, }\AttributeTok{ylab=}\StringTok{\textquotesingle{}F(x)\textquotesingle{}}\NormalTok{, }\AttributeTok{cex=}\FloatTok{0.5}\NormalTok{, }\AttributeTok{las=}\DecValTok{1}\NormalTok{)}
\end{Highlighting}
\end{Shaded}

\begin{figure}
\centering
\includegraphics{Manual_de_R_files/figure-latex/crabcont2-1.pdf}
\caption{\label{fig:crabcont2}Función acumulada \(F(x)\) para el peso de los cangrejos.}
\end{figure}

\begin{enumerate}
\def\labelenumi{\arabic{enumi})}
\setcounter{enumi}{2}
\tightlist
\item
  Calcular la probabilidad de que un cangrejo hembra tenga un peso inferior o igual a 28 kilogramos.
\end{enumerate}

Para obtener \(P(X \leq 28)\) se evalua en la función \(F(x)\) el cuantil 28 así.

\begin{Shaded}
\begin{Highlighting}[]
\FunctionTok{F}\NormalTok{(}\DecValTok{28}\NormalTok{)}
\end{Highlighting}
\end{Shaded}

\begin{verbatim}
## [1] 0.7919075
\end{verbatim}

Por lo tanto \(P(X \leq 28)=0.7919\).

\begin{enumerate}
\def\labelenumi{\arabic{enumi})}
\setcounter{enumi}{3}
\tightlist
\item
  Dibujar la función de densidad para el peso de los cangrejos hembra diferenciando por el color del caparazón.
\end{enumerate}

Como son 4 los colores de los caparazones se deben construir 4 funciones de densidad. Usando la función \texttt{split} se puede partir el vector de peso de los cangrejos según su color. Luego se construyen las cuatro densidades usando la función \texttt{density} aplicada a cada uno de los pesos, a continuación el código.

\begin{Shaded}
\begin{Highlighting}[]
\NormalTok{pesos }\OtherTok{\textless{}{-}} \FunctionTok{split}\NormalTok{(}\AttributeTok{x=}\NormalTok{crab}\SpecialCharTok{$}\NormalTok{W, }\AttributeTok{f=}\NormalTok{crab}\SpecialCharTok{$}\NormalTok{C)}
\NormalTok{f1 }\OtherTok{\textless{}{-}} \FunctionTok{density}\NormalTok{(pesos[[}\DecValTok{1}\NormalTok{]])}
\NormalTok{f2 }\OtherTok{\textless{}{-}} \FunctionTok{density}\NormalTok{(pesos[[}\DecValTok{2}\NormalTok{]])}
\NormalTok{f3 }\OtherTok{\textless{}{-}} \FunctionTok{density}\NormalTok{(pesos[[}\DecValTok{3}\NormalTok{]])}
\NormalTok{f4 }\OtherTok{\textless{}{-}} \FunctionTok{density}\NormalTok{(pesos[[}\DecValTok{4}\NormalTok{]])}
\end{Highlighting}
\end{Shaded}

Luego de tener las densidades muestrales se procede a dibujar la primera densidad con \texttt{plot}, luego se usa la funció \texttt{lines} para agregar a la densidad inicial las restantes densidades. En la Figura \ref{fig:crabcont3} se muestran las 4 densidades, una por cada color de caparazón.

\begin{Shaded}
\begin{Highlighting}[]
\FunctionTok{plot}\NormalTok{(f1, }\AttributeTok{main=}\StringTok{\textquotesingle{}\textquotesingle{}}\NormalTok{, }\AttributeTok{las=}\DecValTok{1}\NormalTok{, }\AttributeTok{lwd=}\DecValTok{4}\NormalTok{,}
     \AttributeTok{xlim=}\FunctionTok{c}\NormalTok{(}\DecValTok{18}\NormalTok{, }\DecValTok{34}\NormalTok{),}
     \AttributeTok{xlab=}\StringTok{\textquotesingle{}Peso (Kg)\textquotesingle{}}\NormalTok{, }\AttributeTok{ylab=}\StringTok{\textquotesingle{}Densidad\textquotesingle{}}\NormalTok{)}
\FunctionTok{lines}\NormalTok{(f2, }\AttributeTok{lwd=}\DecValTok{4}\NormalTok{, }\AttributeTok{col=}\StringTok{\textquotesingle{}red\textquotesingle{}}\NormalTok{)}
\FunctionTok{lines}\NormalTok{(f3, }\AttributeTok{lwd=}\DecValTok{4}\NormalTok{, }\AttributeTok{col=}\StringTok{\textquotesingle{}blue\textquotesingle{}}\NormalTok{)}
\FunctionTok{lines}\NormalTok{(f4, }\AttributeTok{lwd=}\DecValTok{4}\NormalTok{, }\AttributeTok{col=}\StringTok{\textquotesingle{}orange\textquotesingle{}}\NormalTok{)}
\FunctionTok{legend}\NormalTok{(}\StringTok{\textquotesingle{}topright\textquotesingle{}}\NormalTok{, }\AttributeTok{lwd=}\DecValTok{4}\NormalTok{, }\AttributeTok{bty=}\StringTok{\textquotesingle{}n\textquotesingle{}}\NormalTok{,}
       \AttributeTok{col=}\FunctionTok{c}\NormalTok{(}\StringTok{\textquotesingle{}black\textquotesingle{}}\NormalTok{, }\StringTok{\textquotesingle{}red\textquotesingle{}}\NormalTok{, }\StringTok{\textquotesingle{}blue\textquotesingle{}}\NormalTok{, }\StringTok{\textquotesingle{}orange\textquotesingle{}}\NormalTok{),}
       \AttributeTok{legend=}\FunctionTok{c}\NormalTok{(}\StringTok{\textquotesingle{}Color 1\textquotesingle{}}\NormalTok{, }\StringTok{\textquotesingle{}Color 2\textquotesingle{}}\NormalTok{, }\StringTok{\textquotesingle{}Color 3\textquotesingle{}}\NormalTok{, }\StringTok{\textquotesingle{}Color 4\textquotesingle{}}\NormalTok{))}
\end{Highlighting}
\end{Shaded}

\begin{figure}
\centering
\includegraphics{Manual_de_R_files/figure-latex/crabcont3-1.pdf}
\caption{\label{fig:crabcont3}Función de densidad \(f(x)\) para el peso del cangrejo diferenciando por el color.}
\end{figure}

Otra forma para dibujar las densidades es usar el paquete \textbf{ggplot2} \citep{R-ggplot2}. En la Figura \ref{fig:crabcont4} se muestra el resultado obtenido de correr el siguiente código.

\begin{Shaded}
\begin{Highlighting}[]
\FunctionTok{require}\NormalTok{(ggplot2)  }\CommentTok{\# Recuerde que primero debe instalarlo}

\NormalTok{crab}\SpecialCharTok{$}\NormalTok{Color }\OtherTok{\textless{}{-}} \FunctionTok{as.factor}\NormalTok{(crab}\SpecialCharTok{$}\NormalTok{C)  }\CommentTok{\# Para convertir en factor}

\FunctionTok{ggplot}\NormalTok{(crab, }\FunctionTok{aes}\NormalTok{(}\AttributeTok{x=}\NormalTok{W)) }\SpecialCharTok{+} 
  \FunctionTok{geom\_density}\NormalTok{(}\FunctionTok{aes}\NormalTok{(}\AttributeTok{group=}\NormalTok{Color, }\AttributeTok{fill=}\NormalTok{Color), }\AttributeTok{alpha=}\FloatTok{0.3}\NormalTok{) }\SpecialCharTok{+}
  \FunctionTok{xlim}\NormalTok{(}\DecValTok{18}\NormalTok{, }\DecValTok{34}\NormalTok{) }\SpecialCharTok{+} \FunctionTok{xlab}\NormalTok{(}\StringTok{"Peso (Kg)"}\NormalTok{) }\SpecialCharTok{+} \FunctionTok{ylab}\NormalTok{(}\StringTok{"Densidad"}\NormalTok{)}
\end{Highlighting}
\end{Shaded}

\begin{figure}
\centering
\includegraphics{Manual_de_R_files/figure-latex/crabcont4-1.pdf}
\caption{\label{fig:crabcont4}Función de densidad \(f(x)\) para el peso del cangrejo diferenciando por el color y usando ggplot2.}
\end{figure}

Para aprender más sobre el paquete \textbf{ggplot2} se recomienda consultar este \href{http://tutorials.iq.harvard.edu/R/Rgraphics/Rgraphics.html}{enlace}.

\hypertarget{loglik}{%
\chapter{Verosimilitud}\label{loglik}}

En este capítulo se mostrará como usar R para obtener la función de log-verosimilitud y estimadores por el método de máxima verosimilitud.

\hypertarget{funciuxf3n-de-verosimilitud}{%
\section{Función de verosimilitud}\label{funciuxf3n-de-verosimilitud}}

El concepto de verosimilitud fue propuesto por \citet{Fisher22} en el contexto de estimación de parámetros. En la Figura \ref{fig:fisher} se muestra una fotografía de Ronald Aylmer Fisher.

\begin{figure}

{\centering \includegraphics[width=0.4\linewidth]{images/Fisher} 

}

\caption{Fotografía de Ronald Aylmer Fisher (1890-1962).}\label{fig:fisher}
\end{figure}

La función de verosimilitud para un vector de parámetros \(\boldsymbol{\theta}\) dada una muestra aleatoria \(\boldsymbol{x}=(x_1, \ldots,x_n)^\top\) con una distribución asumida se define usualmente como:

\begin{equation}
L(\boldsymbol{\theta} | \boldsymbol{x}) = \prod_{i=1}^{n}  f(x_i | \boldsymbol{\theta}),
\label{eq:lik}
\end{equation}

donde \(x_i\) representa uno de los elementos de la muestra aleatoria y \(f(\cdot)\) es la función de masa/densidad de la distribución de la cual se obtuvo \(\boldsymbol{x}\).

\hypertarget{funciuxf3n-de-log-verosimilitud}{%
\section{Función de log-verosimilitud}\label{funciuxf3n-de-log-verosimilitud}}

La función de log-verosimilitud \(l(\boldsymbol{\theta} | \boldsymbol{x})\) se define como el logaritmo de la función de verosimilitud \(L(\boldsymbol{\theta} | \boldsymbol{x})\), es decir

\begin{equation}
l(\boldsymbol{\theta} | \boldsymbol{x}) = \log L(\boldsymbol{\theta} | \boldsymbol{x}) = \sum_{i=1}^{n} \log f(x_i | \boldsymbol{\theta})
\label{eq:loglik}
\end{equation}

\hypertarget{muxe9todo-de-muxe1xima-verosimilitud-para-estimar-paruxe1metros}{%
\section{Método de máxima verosimilitud para estimar parámetros}\label{muxe9todo-de-muxe1xima-verosimilitud-para-estimar-paruxe1metros}}

El método de máxima verosimilitud se usa para estimar los parámetros de una distribución. El objetivo de este método es encontrar los valores de \(\boldsymbol{\theta}\) que maximizan a \(L(\boldsymbol{\theta} | \boldsymbol{x})\) o a \(l(\boldsymbol{\theta} | \boldsymbol{x})\), los valores encontrados se representan usualmente por \(\hat{\boldsymbol{\theta}}\).

\begin{rmdnote}
Asumiendo un modelo estadístico parametrizado por una cantidad fija y desconocida \(\theta\), la verosimilitud \(L(\theta)\) es la probabilidad de los datos observados \(x\) como una función de \(\theta\) \citep{pawitan_2013}. Si la variable de interés es discreta se usa la probabilidad y si es continua se usa la densidad para obtener la verosimilitud.
\end{rmdnote}

\hypertarget{ejemplo-49}{%
\subsection*{Ejemplo}\label{ejemplo-49}}
\addcontentsline{toc}{subsection}{Ejemplo}

En este ejemplo vamos a considerar la distribución binomial cuya función de masa de probabilidad está dada por:

\[f(x)=P(X=x)=\binom{n}{x} p^x (1-p)^{n-x}, \quad 0<p<1, \quad n \leq 1, 2, \ldots, \quad 0 \leq x \leq n\]

La distribución binomial anterior tiene sólo un parámetro \(p\), por lo tanto en este caso se \(\theta=p\).

Suponga que se tiene el vector \texttt{rta} que corresponde a una muestra aleatoria de una distribución binomial con parámetro \(n=5\) conocido.

\begin{Shaded}
\begin{Highlighting}[]
\NormalTok{rta }\OtherTok{\textless{}{-}} \FunctionTok{c}\NormalTok{(}\DecValTok{2}\NormalTok{, }\DecValTok{2}\NormalTok{, }\DecValTok{1}\NormalTok{, }\DecValTok{1}\NormalTok{, }\DecValTok{1}\NormalTok{, }\DecValTok{1}\NormalTok{, }\DecValTok{0}\NormalTok{, }\DecValTok{2}\NormalTok{, }\DecValTok{1}\NormalTok{, }\DecValTok{2}\NormalTok{,}
         \DecValTok{1}\NormalTok{, }\DecValTok{0}\NormalTok{, }\DecValTok{1}\NormalTok{, }\DecValTok{2}\NormalTok{, }\DecValTok{1}\NormalTok{, }\DecValTok{0}\NormalTok{, }\DecValTok{0}\NormalTok{, }\DecValTok{2}\NormalTok{, }\DecValTok{2}\NormalTok{, }\DecValTok{1}\NormalTok{)}
\end{Highlighting}
\end{Shaded}

\begin{enumerate}
\def\labelenumi{\arabic{enumi})}
\tightlist
\item
  Calcular el valor de log-verosimilitud \(l(\theta)\) si asumiendo que \(p=0.30\) en la distribución binomial.
\end{enumerate}

Para obtener el valor de \(l(\theta)\) en el punto \(p=0.30\) se aplica la definición dada en la expresión \eqref{eq:loglik}. Como el problema trata de una binomial se usa entonces la función de masa \texttt{dbinom} evaluada en la muestra \texttt{rta}, el parámetro \texttt{size} como es conocido se reemplaza por el valor de cinco y en el parámetro \texttt{prob} se cambia por 0.3. Como interesa la función de log-verosimilitud se debe incluir \texttt{log=TRUE}. A continuación el código necesario.

\begin{Shaded}
\begin{Highlighting}[]
\FunctionTok{sum}\NormalTok{(}\FunctionTok{dbinom}\NormalTok{(}\AttributeTok{x=}\NormalTok{rta, }\AttributeTok{size=}\DecValTok{5}\NormalTok{, }\AttributeTok{prob=}\FloatTok{0.3}\NormalTok{, }\AttributeTok{log=}\ConstantTok{TRUE}\NormalTok{))}
\end{Highlighting}
\end{Shaded}

\begin{verbatim}
## [1] -24.55231
\end{verbatim}

Por lo tanto \(l(\theta)= -24.55\)

\begin{enumerate}
\def\labelenumi{\arabic{enumi})}
\setcounter{enumi}{1}
\tightlist
\item
  Construir una función llamada \texttt{ll} a la cual le ingrese valores del parámetro \(p\) de la binomial y que la función entregue el valor de log-verosimilitud.
\end{enumerate}

La función solicitada tiene un cuerpo igual al usado en el numeral anterior, a continuación el código necesario para crearla.

\begin{Shaded}
\begin{Highlighting}[]
\NormalTok{ll }\OtherTok{\textless{}{-}} \ControlFlowTok{function}\NormalTok{(prob) }\FunctionTok{sum}\NormalTok{(}\FunctionTok{dbinom}\NormalTok{(}\AttributeTok{x=}\NormalTok{rta, }\AttributeTok{size=}\DecValTok{5}\NormalTok{, }\AttributeTok{prob=}\NormalTok{prob, }\AttributeTok{log=}\NormalTok{T))}
\end{Highlighting}
\end{Shaded}

Vamos a probar la función en dos valores arbitrarios \(p=0.15\) y \(p=0.80\) que pertenezcan al dominio del parámetro \(p\) de la distribución binomial.

\begin{Shaded}
\begin{Highlighting}[]
\FunctionTok{ll}\NormalTok{(}\AttributeTok{prob=}\FloatTok{0.15}\NormalTok{)  }\CommentTok{\# Individual para p=0.15}
\end{Highlighting}
\end{Shaded}

\begin{verbatim}
## [1] -25.54468
\end{verbatim}

\begin{Shaded}
\begin{Highlighting}[]
\FunctionTok{ll}\NormalTok{(}\AttributeTok{prob=}\FloatTok{0.80}\NormalTok{)  }\CommentTok{\# Individual para p=0.80}
\end{Highlighting}
\end{Shaded}

\begin{verbatim}
## [1] -98.45598
\end{verbatim}

El valor de log-verosimilitud para \(p=0.15\) fue de -25.54 mientras que para \(p=0.80\) fue de -98.46.

Vamos ahora a chequear si la función \texttt{ll} está vectorizada y para esto usamos el código mostrado a continuación y deberíamos obtener un vector con los valores \texttt{c(-25.54,\ -98.56)}.

\begin{Shaded}
\begin{Highlighting}[]
\FunctionTok{ll}\NormalTok{(}\AttributeTok{prob=}\FunctionTok{c}\NormalTok{(}\FloatTok{0.15}\NormalTok{, }\FloatTok{0.80}\NormalTok{))}
\end{Highlighting}
\end{Shaded}

\begin{verbatim}
## [1] -57.31899
\end{verbatim}

No obtuvimos el resultado esperado, eso significa que nuestra función no está vectorizada. Ese problema lo podemos solucionar así:

\begin{Shaded}
\begin{Highlighting}[]
\NormalTok{ll }\OtherTok{\textless{}{-}} \FunctionTok{Vectorize}\NormalTok{(ll)}
\FunctionTok{ll}\NormalTok{(}\AttributeTok{prob=}\FunctionTok{c}\NormalTok{(}\FloatTok{0.15}\NormalTok{, }\FloatTok{0.80}\NormalTok{))}
\end{Highlighting}
\end{Shaded}

\begin{verbatim}
## [1] -25.54468 -98.45598
\end{verbatim}

Vemos que ahora que cuando se ingresa un vector a la función \texttt{ll} se obtiene un vector.

\begin{rmdnote}
Necesitamos que la función \texttt{ll} esté vectorizada para poder dibujarla y para poder optimizarla.
\end{rmdnote}

\begin{enumerate}
\def\labelenumi{\arabic{enumi})}
\setcounter{enumi}{2}
\tightlist
\item
  Dibujar la curva log-verosimilitud \(l(\theta)\), en el eje X debe estar el parámetro \(p\) del cual depende la función de log-verosimilitud.
\end{enumerate}

En la Figura \ref{fig:loglik1} se presenta la curva solicitada.

\begin{Shaded}
\begin{Highlighting}[]
\FunctionTok{curve}\NormalTok{(ll, }\AttributeTok{lwd=}\DecValTok{4}\NormalTok{, }\AttributeTok{col=}\StringTok{\textquotesingle{}dodgerblue3\textquotesingle{}}\NormalTok{,}
      \AttributeTok{xlab=}\StringTok{\textquotesingle{}Probabilidad de éxito (p)\textquotesingle{}}\NormalTok{, }\AttributeTok{las=}\DecValTok{1}\NormalTok{,}
      \AttributeTok{ylab=}\FunctionTok{expression}\NormalTok{(}\FunctionTok{paste}\NormalTok{(}\StringTok{"Probabilidad de éxito (p="}\NormalTok{, theta, }\StringTok{")"}\NormalTok{))}
\NormalTok{      )}
\FunctionTok{grid}\NormalTok{()}
\end{Highlighting}
\end{Shaded}

\begin{figure}

{\centering \includegraphics{Manual_de_R_files/figure-latex/loglik1-1} 

}

\caption{Función de log-verosimilitud para el ejemplo sobre binomial.}\label{fig:loglik1}
\end{figure}

\begin{enumerate}
\def\labelenumi{\arabic{enumi})}
\setcounter{enumi}{3}
\tightlist
\item
  Observando la Figura \ref{fig:loglik1}, ¿cuál esl el valor de \(p\) que maximiza la función de log-verosimilitud?
\end{enumerate}

Al observar la Figura \ref{fig:loglik1} se nota que el valor de \(p\) que maximiza la función log-verosimilitud está muy cerca de 0.2.

\begin{enumerate}
\def\labelenumi{\arabic{enumi})}
\setcounter{enumi}{4}
\tightlist
\item
  ¿Cuál es el valor exacto de \(p\) que maximiza la función log-verosimilitud?
\end{enumerate}

En R existe la función \texttt{optimize} que sirve para encontrar el valor que \textbf{minimiza} una función uniparamétrica en un intervalo dado, sin embargo, aquí interesa es maximimizar la función de log-verosimilitud, por esa razón se construye la función \texttt{minusll} que es el negativo de la función \texttt{ll} para así poder usar \texttt{optimize}. A continuación el código usado. \index{optimize}

\begin{Shaded}
\begin{Highlighting}[]
\NormalTok{minusll }\OtherTok{\textless{}{-}} \ControlFlowTok{function}\NormalTok{(x) }\SpecialCharTok{{-}}\FunctionTok{ll}\NormalTok{(x)}

\FunctionTok{optimize}\NormalTok{(}\AttributeTok{f=}\NormalTok{minusll, }\AttributeTok{interval=}\FunctionTok{c}\NormalTok{(}\DecValTok{0}\NormalTok{, }\DecValTok{1}\NormalTok{))}
\end{Highlighting}
\end{Shaded}

\begin{verbatim}
## $minimum
## [1] 0.229993
## 
## $objective
## [1] 23.3246
\end{verbatim}

Del resultado anterior se observa que cuando \(p=0.23\) el valor máximo de log-verosimilitud es -23.32 (negativo de \texttt{minusll}).

\hypertarget{ejemplo-50}{%
\subsection*{Ejemplo}\label{ejemplo-50}}
\addcontentsline{toc}{subsection}{Ejemplo}

Suponga que la estatura de una población se puede asumir como una normal \(N(170, 25)\). Suponga también que se genera una muestra aleatoria de 50 observaciones de la población con el objetivo de recuperar los valores de la media y varianza poblacionales a partir de la muestra aleatoria.

La muestra se va a generar con la función \texttt{rnorm} pero antes se fijará una semilla con la intención de que el lector pueda replicar el ejemplo y obtener la misma muestra aleatoria aquí generada, el código para hacerlo es el siguiente.

\begin{Shaded}
\begin{Highlighting}[]
\FunctionTok{set.seed}\NormalTok{(}\DecValTok{1235}\NormalTok{)  }\CommentTok{\# La semilla es 1235}
\NormalTok{y }\OtherTok{\textless{}{-}} \FunctionTok{rnorm}\NormalTok{(}\AttributeTok{n=}\DecValTok{50}\NormalTok{, }\AttributeTok{mean=}\DecValTok{170}\NormalTok{, }\AttributeTok{sd=}\DecValTok{5}\NormalTok{)}
\NormalTok{y[}\DecValTok{1}\SpecialCharTok{:}\DecValTok{7}\NormalTok{]  }\CommentTok{\# Para ver los primeros siete valores generados}
\end{Highlighting}
\end{Shaded}

\begin{verbatim}
## [1] 166.5101 163.5757 174.9498 170.5589 170.5710 178.4910 170.2392
\end{verbatim}

\begin{enumerate}
\def\labelenumi{\arabic{enumi})}
\tightlist
\item
  Construya la función de log-verosimilitud para los parámetros de la normal dada la muestra aleatoria \texttt{y}.
\end{enumerate}

Abajo se muestra la forma de construir la función de log-verosimilitud.

\begin{Shaded}
\begin{Highlighting}[]
\NormalTok{ll }\OtherTok{\textless{}{-}} \ControlFlowTok{function}\NormalTok{(param) \{}
\NormalTok{  media }\OtherTok{\textless{}{-}}\NormalTok{ param[}\DecValTok{1}\NormalTok{]  }\CommentTok{\# param es el vector de parámetros}
\NormalTok{  desvi }\OtherTok{\textless{}{-}}\NormalTok{ param[}\DecValTok{2}\NormalTok{] }
  \FunctionTok{sum}\NormalTok{(}\FunctionTok{dnorm}\NormalTok{(}\AttributeTok{x=}\NormalTok{y, }\AttributeTok{mean=}\NormalTok{media, }\AttributeTok{sd=}\NormalTok{desvi, }\AttributeTok{log=}\ConstantTok{TRUE}\NormalTok{))}
\NormalTok{\}}
\end{Highlighting}
\end{Shaded}

\begin{rmdwarning}
Siempre que el interés sea encontrar los valores que maximizan una función de log-verosimilitud, los parámetros de la distribución \textbf{deben} ingresar a la función \texttt{ll} como un vector. Esto se debe hacer para poder usar las funciones de búsqueda \texttt{optim} y \texttt{nlminb}.
\end{rmdwarning}

\begin{enumerate}
\def\labelenumi{\arabic{enumi})}
\setcounter{enumi}{1}
\tightlist
\item
  Dibujar la función de log-verosimilitud.
\end{enumerate}

En la Figura \ref{fig:loglik2} se muestra el gráfico de niveles para la superficie de log-verosimilitud. De esta figura se nota claramente que los valores que maximizan la superficie están alrededor de \(\mu=170\) y \(\sigma=5\).

\begin{Shaded}
\begin{Highlighting}[]
\NormalTok{ll1 }\OtherTok{\textless{}{-}} \ControlFlowTok{function}\NormalTok{(a, b) }\FunctionTok{sum}\NormalTok{(}\FunctionTok{dnorm}\NormalTok{(}\AttributeTok{x=}\NormalTok{y, }\AttributeTok{mean=}\NormalTok{a, }\AttributeTok{sd=}\NormalTok{b, }\AttributeTok{log=}\ConstantTok{TRUE}\NormalTok{))}
\NormalTok{ll1 }\OtherTok{\textless{}{-}} \FunctionTok{Vectorize}\NormalTok{(ll1)}
\NormalTok{xx }\OtherTok{\textless{}{-}} \FunctionTok{seq}\NormalTok{(}\AttributeTok{from=}\DecValTok{160}\NormalTok{, }\AttributeTok{to=}\DecValTok{180}\NormalTok{, }\AttributeTok{by=}\FloatTok{0.5}\NormalTok{)}
\NormalTok{yy }\OtherTok{\textless{}{-}} \FunctionTok{seq}\NormalTok{(}\AttributeTok{from=}\DecValTok{3}\NormalTok{, }\AttributeTok{to=}\DecValTok{7}\NormalTok{, }\AttributeTok{by=}\FloatTok{0.5}\NormalTok{)}
\NormalTok{zz }\OtherTok{\textless{}{-}} \FunctionTok{outer}\NormalTok{(}\AttributeTok{X=}\NormalTok{xx, }\AttributeTok{Y=}\NormalTok{yy, ll1)}
\FunctionTok{filled.contour}\NormalTok{(}\AttributeTok{x=}\NormalTok{xx, }\AttributeTok{y=}\NormalTok{yy, }\AttributeTok{z=}\NormalTok{zz, }\AttributeTok{nlevels=}\DecValTok{20}\NormalTok{,}
               \AttributeTok{xlab=}\FunctionTok{expression}\NormalTok{(mu), }\AttributeTok{ylab=}\FunctionTok{expression}\NormalTok{(sigma),}
               \AttributeTok{color =}\NormalTok{ topo.colors)}
\end{Highlighting}
\end{Shaded}

\begin{figure}
\centering
\includegraphics{Manual_de_R_files/figure-latex/loglik2-1.pdf}
\caption{\label{fig:loglik2}Gráfico de niveles para la función de log-verosimilitud para el ejemplo sobre normal.}
\end{figure}

\begin{enumerate}
\def\labelenumi{\arabic{enumi})}
\setcounter{enumi}{2}
\tightlist
\item
  Obtenga los valores de \(\mu\) y \(\sigma\) que maximizan la función de log-verosimilitud.
\end{enumerate}

Para obtener los valores solicitados vamos a usar la función \texttt{nlminb} que es un optimizador. A la función \texttt{nlminb} se le debe indicar por medio del parámetro \texttt{objective} la función que queremos optimizar (minimizar); el parámetro \texttt{start} es un vector con los valores iniciales para comenzar la búsqueda de \(\mu\) y \(\sigma\); los parámetros \texttt{lower} y \texttt{upper} sirven para delimitar el espacio de búsqueda. A continuación se muestra el código usado para obtener los valores que minimizan a \texttt{minusll}, es decir, los valores que maximizan la función de log-verosimilitud. \index{nlminb}

\begin{Shaded}
\begin{Highlighting}[]
\NormalTok{minusll }\OtherTok{\textless{}{-}} \ControlFlowTok{function}\NormalTok{(x) }\SpecialCharTok{{-}}\FunctionTok{ll}\NormalTok{(x)}
\FunctionTok{nlminb}\NormalTok{(}\AttributeTok{objective=}\NormalTok{minusll, }\AttributeTok{start=}\FunctionTok{c}\NormalTok{(}\DecValTok{163}\NormalTok{, }\FloatTok{3.4}\NormalTok{),}
       \AttributeTok{lower=}\FunctionTok{c}\NormalTok{(}\DecValTok{160}\NormalTok{, }\DecValTok{3}\NormalTok{), }\AttributeTok{upper=}\FunctionTok{c}\NormalTok{(}\DecValTok{180}\NormalTok{, }\DecValTok{7}\NormalTok{))}
\end{Highlighting}
\end{Shaded}

\begin{verbatim}
## $par
## [1] 170.338374   5.423529
## 
## $objective
## [1] 155.4842
## 
## $convergence
## [1] 0
## 
## $iterations
## [1] 13
## 
## $evaluations
## function gradient 
##       16       35 
## 
## $message
## [1] "relative convergence (4)"
\end{verbatim}

De la salida anterior podemos observar que los valores óptimos de \(\mu\) y \(\sigma\) son 170.338 y 5.424 respectivamente, resultado que coincide con lo observado en la Figura \ref{fig:loglik2} y con los valores reales de simulación de la muestra. Esto indica que el procedimiento de estimación de parámetros por máxima verosimilitud entrega valores insesgados de los parámetros a estimar.

Un resultado interesante de la salida anterior es que se reporta el valor mínimo que alcanza la función \texttt{minusll}, este valor fue de 155.5, por lo tanto, se puede afirmar que el valor máximo de log-verosimilitud es -155.5.

Otros resultados importantes de la salida anterior son el valor de \texttt{convergence=0} que indica que la búsqueda fue exitosa; \texttt{iterations=13} indica que se realizaron 13 pasos desde el punto inicial \texttt{start} hasta las coordenadas de optimización.

\begin{rmdnote}
En R se tienen dos funciones básicas para optimizar funciones, es decir, para encontrar los valores que minimizan una función dada. Esas dos funciones son \texttt{nliminb} y \texttt{optim}. Para optimizar en una sola dimensión se usa la función \texttt{optimize}.
\end{rmdnote}

\begin{enumerate}
\def\labelenumi{\arabic{enumi})}
\setcounter{enumi}{3}
\tightlist
\item
  ¿Hay alguna función para obtener directamente el valor que maximiza la función log-verosimilitud?
\end{enumerate}

La respuesta es si. Si la distribución estudiada es una de las distribuciones básicas se puede usar la función \texttt{fitdistr} del paquete básico \textbf{MASS}. Esta función requiere de los datos que se ingresan por medio del parámetro \texttt{x}, y de la distribución de los datos que se ingresa por medio del parámetro \texttt{densfun}. La función \texttt{fitdistr} admite 15 distribuciones diferentes para hacer la búsqueda de los parámetros que caracterizan una distribución, se sugiere consultar la ayuda de la función \texttt{fitdistr} escribiendo en la consola \texttt{help(fitdistr)}. A continuación el código usado. \index{fitdistr}

\begin{Shaded}
\begin{Highlighting}[]
\FunctionTok{require}\NormalTok{(MASS) }\CommentTok{\# El paquete ya está instalado, solo se debe cargar}
\NormalTok{res }\OtherTok{\textless{}{-}} \FunctionTok{fitdistr}\NormalTok{(}\AttributeTok{x=}\NormalTok{y, }\AttributeTok{densfun=}\StringTok{\textquotesingle{}normal\textquotesingle{}}\NormalTok{)}
\NormalTok{res}
\end{Highlighting}
\end{Shaded}

\begin{verbatim}
##       mean           sd     
##   170.3383794     5.4235271 
##  (  0.7670026) (  0.5423527)
\end{verbatim}

El objeto \texttt{res} contiene los resultados de usar \texttt{fitdistr}. En la primer línea están los valores de los parámetros que maximizan la función de log-verosimilitud, en la parte de abajo, dentro de paréntesis, están los errores estándar o desviaciones de éstos estimadores.

Al objeto \texttt{res} es de la clase \emph{fitdistr} y por lo tanto se le puede aplicar la función genérica \texttt{logLik} para obtener el valor de la log-verosimilitud. Se sugiere consultar la ayuda de la función \texttt{logLik} escribiendo en la consola \texttt{help(logLik)}. A continuación el código para usar \texttt{logLik} sobre el objeto \texttt{res}.

\begin{Shaded}
\begin{Highlighting}[]
\FunctionTok{logLik}\NormalTok{(res)}
\end{Highlighting}
\end{Shaded}

\begin{verbatim}
## 'log Lik.' -155.4842 (df=2)
\end{verbatim}

De esta última salida se observa que el valor coincide con el obtenido cuando se usó \texttt{nlminb}.

\hypertarget{ejemplo-51}{%
\subsection*{Ejemplo}\label{ejemplo-51}}
\addcontentsline{toc}{subsection}{Ejemplo}

Generar \(n=100\) observaciones de una gamma con parámetro de forma igual a 2, parámetro de tasa igual a 0.5 y luego responder las preguntas.

\begin{enumerate}
\def\labelenumi{\arabic{enumi})}
\tightlist
\item
  ¿Cómo se puede generar la muestra aleatoria solicitada?
\end{enumerate}

Para generar la muestra aleatoria (\texttt{ma}) solicitada se fijó la semilla con el objetivo de que el lector pueda obtener los mismos resultados de este ejemplo.

\begin{Shaded}
\begin{Highlighting}[]
\NormalTok{n }\OtherTok{\textless{}{-}} \DecValTok{100}
\FunctionTok{set.seed}\NormalTok{(}\DecValTok{12345}\NormalTok{)}
\NormalTok{ma }\OtherTok{\textless{}{-}} \FunctionTok{rgamma}\NormalTok{(}\AttributeTok{n=}\NormalTok{n, }\AttributeTok{shape=}\DecValTok{2}\NormalTok{, }\AttributeTok{rate=}\FloatTok{0.5}\NormalTok{)}
\end{Highlighting}
\end{Shaded}

\begin{enumerate}
\def\labelenumi{\arabic{enumi})}
\setcounter{enumi}{1}
\tightlist
\item
  Asumiendo que la muestra aleatoria proviene de una normal (lo cual es incorrecto), estime los parámetros de la distribución normal.
\end{enumerate}

\begin{Shaded}
\begin{Highlighting}[]
\NormalTok{fit1 }\OtherTok{\textless{}{-}} \FunctionTok{fitdistr}\NormalTok{(}\AttributeTok{x=}\NormalTok{ma, }\AttributeTok{densfun=}\StringTok{\textquotesingle{}normal\textquotesingle{}}\NormalTok{)}
\NormalTok{fit1}
\end{Highlighting}
\end{Shaded}

\begin{verbatim}
##      mean         sd    
##   4.3082767   2.8084910 
##  (0.2808491) (0.1985903)
\end{verbatim}

\begin{enumerate}
\def\labelenumi{\arabic{enumi})}
\setcounter{enumi}{2}
\tightlist
\item
  Asumiendo que la muestra aleatoria proviene de una gamma estime los parámetros de la distribución gamma.
\end{enumerate}

\begin{Shaded}
\begin{Highlighting}[]
\NormalTok{fit2 }\OtherTok{\textless{}{-}} \FunctionTok{fitdistr}\NormalTok{(}\AttributeTok{x=}\NormalTok{ma, }\AttributeTok{densfun=}\StringTok{\textquotesingle{}gamma\textquotesingle{}}\NormalTok{)}
\NormalTok{fit2}
\end{Highlighting}
\end{Shaded}

\begin{verbatim}
##      shape         rate   
##   2.23978235   0.51987909 
##  (0.29620136) (0.07702892)
\end{verbatim}

En la salida anterior están los valores estimados de los parámetros de la distribución por el método de máxima verosimilitud, observe la cercanía de éstos con los verdaderos valores de 2 y 0.5 para forma y tasa respectivamente.

\begin{enumerate}
\def\labelenumi{\arabic{enumi})}
\setcounter{enumi}{3}
\tightlist
\item
  Dibuje dos qqplot, uno asumiendo distribución normal y el otro distribución gamma. ¿Cuál distribución se ajusta mejor a los datos simulados?
\end{enumerate}

Para dibujar el qqplot se usa la función genérica \texttt{qqplot}, recomendamos consultar \citet{hernandez_correa} para los detalles de cómo usar esta función. Al usar \texttt{qqplot} para obtener el qqplot normal y gamma es necesario indicar los valores \(\hat{\boldsymbol{\theta}}\) obtenidos en el numeral anterior, por eso es que en el código mostrado a continuación aparece \texttt{mean=4.3083,\ sd=2.8085} en el qqplot normal y \texttt{shape=2.23978,\ rate=0.51988} en el qqplot gamma.

\begin{Shaded}
\begin{Highlighting}[]
\FunctionTok{par}\NormalTok{(}\AttributeTok{mfrow=}\FunctionTok{c}\NormalTok{(}\DecValTok{1}\NormalTok{, }\DecValTok{2}\NormalTok{))}

\FunctionTok{qqplot}\NormalTok{(}\AttributeTok{y=}\NormalTok{ma, }\AttributeTok{pch=}\DecValTok{19}\NormalTok{,}
       \AttributeTok{x=}\FunctionTok{qnorm}\NormalTok{(}\FunctionTok{ppoints}\NormalTok{(n), }\AttributeTok{mean=}\FloatTok{4.3083}\NormalTok{, }\AttributeTok{sd=}\FloatTok{2.8085}\NormalTok{),}
       \AttributeTok{main=}\StringTok{\textquotesingle{}Normal Q{-}Q Plot\textquotesingle{}}\NormalTok{,}
       \AttributeTok{xlab=}\StringTok{\textquotesingle{}Theoretical Quantiles\textquotesingle{}}\NormalTok{,}
       \AttributeTok{ylab=}\StringTok{\textquotesingle{}Sample Quantiles\textquotesingle{}}\NormalTok{)}

\FunctionTok{qqplot}\NormalTok{(}\AttributeTok{y=}\NormalTok{ma, }\AttributeTok{pch=}\DecValTok{19}\NormalTok{,}
       \AttributeTok{x=}\FunctionTok{qgamma}\NormalTok{(}\FunctionTok{ppoints}\NormalTok{(n), }\AttributeTok{shape=}\FloatTok{2.23978}\NormalTok{, }\AttributeTok{rate=}\FloatTok{0.51988}\NormalTok{),}
       \AttributeTok{main=}\StringTok{\textquotesingle{}Gamma Q{-}Q Plot\textquotesingle{}}\NormalTok{,}
       \AttributeTok{xlab=}\StringTok{\textquotesingle{}Theoretical Quantiles\textquotesingle{}}\NormalTok{,}
       \AttributeTok{ylab=}\StringTok{\textquotesingle{}Sample Quantiles\textquotesingle{}}\NormalTok{)}
\end{Highlighting}
\end{Shaded}

\begin{figure}
\centering
\includegraphics{Manual_de_R_files/figure-latex/normgamma-1.pdf}
\caption{\label{fig:normgamma}Gráfico cuantil cuantil normal y gamma para la muestra simulada.}
\end{figure}

En la Figura \ref{fig:normgamma} se muestran los qqplot solicitados. Se observa claramente que al asumir normalidad (lo cual es incorrecto), los puntos del qqplot no están alineados, mientras que al asumir distribución gamma (lo cual es correcto), los puntos si están alineados. De esta figura se concluye que la muestra \texttt{ma} puede provenir de una \(Gamma(2.23978, 0.51988)\).

\begin{rmdtip}
Para obtener el gráfico cuantil cuantil bajo normalidad se puede usar directamente la función \texttt{qqnorm}, consultar \citet{hernandez_correa} para mayores detalles.
\end{rmdtip}

\begin{rmdwarning}
En este ejemplo se eligió la mejor distribución entre dos candidatas usando una herramienta gráfica, lo que se recomienda usar algún método menos subjetivo (cuantitativo) para tomar decisiones.
\end{rmdwarning}

\begin{enumerate}
\def\labelenumi{\arabic{enumi})}
\setcounter{enumi}{4}
\tightlist
\item
  Para comparar modelos se puede utilizar el \emph{Akaike information criterion} (\(AIC\)) propuesto por \citet{Akaike74} que sirve para medir la calidad relativa de los modelos estadísticos, la expresión para calcular el indicador es \(AIC=-2 \, \hat{l}+2 \, df\), donde \(\hat{l}\) corresponde al valor de \(\log\)-verosimilitud y \(df\) corresponde al número de parámetros estimados del modelo. Siempre el modelo elegido es aquel modelo con el menor valor de \(AIC\). Calcular el \(AIC\) para los modelos asumidos normal y gamma.
\end{enumerate}

\begin{Shaded}
\begin{Highlighting}[]
\SpecialCharTok{{-}}\DecValTok{2} \SpecialCharTok{*} \FunctionTok{logLik}\NormalTok{(fit1) }\SpecialCharTok{+} \DecValTok{2} \SpecialCharTok{*} \DecValTok{2}  \CommentTok{\# AIC para modelo normal}
\end{Highlighting}
\end{Shaded}

\begin{verbatim}
## 'log Lik.' 494.3172 (df=2)
\end{verbatim}

\begin{Shaded}
\begin{Highlighting}[]
\SpecialCharTok{{-}}\DecValTok{2} \SpecialCharTok{*} \FunctionTok{logLik}\NormalTok{(fit2) }\SpecialCharTok{+} \DecValTok{2} \SpecialCharTok{*} \DecValTok{2}  \CommentTok{\# AIC para modelo gamma}
\end{Highlighting}
\end{Shaded}

\begin{verbatim}
## 'log Lik.' 466.0479 (df=2)
\end{verbatim}

De los resultados anteriores se concluye que entre los dos modelos, el mejor es el gamma porque su \(AIC=466\) es el menor de toos los \(AIC\).

\begin{rmdnote}
Modelos anidados pueden ser comparados por medio del \emph{global deviance} (\(GD\)) dado por \(GD=-2 \, \hat{l}\) y modelos no anidados por medio del \emph{Generalized Akaike information criterion} (\(GAIC\)) propuesto por \citet{Akaike83} y dado por \(GAIC=-2 \, \hat{l} + \sharp \, df\) siendo \(\sharp\) el valor de penalidad por cada parámetro adicional en el modelo; cuando \(\sharp = 2\), el \(GAIC\) coincide con el \(AIC\) y el \emph{Schwarz Bayesian criterion} (\(SBC\)) propuesto por \citet{Schwarz} se dá cuando el valor de penalidad es \(\sharp = \log(n)\) donde \(n\) es el número de observaciones del modelo; siempre el modelo elegido es aquel modelo con el menor valor de cualquiera de los criterios de información anteriores.
\end{rmdnote}

\hypertarget{score-e-informaciuxf3n-de-fisher}{%
\section{Score e Información de Fisher}\label{score-e-informaciuxf3n-de-fisher}}

En esta sección se explican los conceptos y utilidad de la función Score y la Información de Fisher.

\hypertarget{score-e-informaciuxf3n-de-fisher-en-el-caso-univariado.}{%
\subsection{Score e Información de Fisher en el caso univariado.}\label{score-e-informaciuxf3n-de-fisher-en-el-caso-univariado.}}

La función Score denotada por \(S(\theta)\) se define como la primera derivada de la función de log-verosimilitud así:

\[
S(\theta) \equiv \frac{\partial}{\partial \theta} l(\theta)
\]

y el estimador de máxima verosimilitud \(\hat{\theta}\) se encuentra solucionando la igualdad

\[
S(\theta) = 0
\]

En el valor máximo \(\hat{\theta}\) la curva \(l(\theta)\) es cóncava hacia abajo y por lo tanto la segunda derivada es negativa, así la curvatura \(I(\theta)\) se define como

\[
I(\theta) \equiv - \frac{\partial^2}{\partial \theta^2} l(\theta)
\]

Una curvatura grande \(I(\hat{\theta})\) está asociada con un gran pico en la función de log-verosimilitud y eso significa una menor incertidumbre sobre el parámetro \(\theta\) \citep{pawitan_2013}. En particular la varianza del estimador de máxima verosimilitud está dada por

\[Var(\hat{\theta})=I^{-1}(\hat{\theta})\]

\hypertarget{ejemplo-52}{%
\subsection*{Ejemplo}\label{ejemplo-52}}
\addcontentsline{toc}{subsection}{Ejemplo}

Suponga que se desea estudiar una variable que tiene distribución Poisson con parámetro \(\lambda\) desconocido. Suponga además que se tienen dos situciones:

\begin{itemize}
\tightlist
\item
  Un solo valor 5 para estimar \(\lambda\).
\item
  Cuatro valores 5, 10, 6 y 15 para estimar \(\lambda\).
\end{itemize}

Dibujar la función \(l(\lambda)\) para ambos casos e identificar la curvatura.

A continuación el código para evaluar la función \(l(\lambda)\) para cada caso.

\begin{Shaded}
\begin{Highlighting}[]
\CommentTok{\# Caso 1}
\NormalTok{w }\OtherTok{\textless{}{-}} \FunctionTok{c}\NormalTok{(}\DecValTok{5}\NormalTok{)}
\NormalTok{ll1 }\OtherTok{\textless{}{-}} \ControlFlowTok{function}\NormalTok{(lambda) }\FunctionTok{sum}\NormalTok{(}\FunctionTok{dpois}\NormalTok{(}\AttributeTok{x=}\NormalTok{w, }\AttributeTok{lambda=}\NormalTok{lambda, }\AttributeTok{log=}\NormalTok{T))}
\NormalTok{ll1 }\OtherTok{\textless{}{-}} \FunctionTok{Vectorize}\NormalTok{(ll1)}

\CommentTok{\# Caso 2}
\NormalTok{y }\OtherTok{\textless{}{-}} \FunctionTok{c}\NormalTok{(}\DecValTok{5}\NormalTok{, }\DecValTok{10}\NormalTok{, }\DecValTok{6}\NormalTok{, }\DecValTok{15}\NormalTok{)}
\NormalTok{ll2 }\OtherTok{\textless{}{-}} \ControlFlowTok{function}\NormalTok{(lambda) }\FunctionTok{sum}\NormalTok{(}\FunctionTok{dpois}\NormalTok{(}\AttributeTok{x=}\NormalTok{y, }\AttributeTok{lambda=}\NormalTok{lambda, }\AttributeTok{log=}\NormalTok{T))}
\NormalTok{ll2 }\OtherTok{\textless{}{-}} \FunctionTok{Vectorize}\NormalTok{(ll2)}
\end{Highlighting}
\end{Shaded}

En la Figura \ref{fig:loglikpoisson} se muestran las dos curvas \(l(\lambda)\) para cada uno de los casos. De la figura se observa claramente que cuando se tienen 4 observaciones la curva es más puntiaguda y por lo tanto menor incertibumbre sobre el parámetro \(\lambda\) a estimar.

\begin{figure}
\centering
\includegraphics{Manual_de_R_files/figure-latex/loglikpoisson-1.pdf}
\caption{\label{fig:loglikpoisson}Curvas de log-verosimilitud para los dos casos.}
\end{figure}

\hypertarget{muxe9todo-de-muxe1xima-verosimilitud-para-estimar-paruxe1metros-en-modelos-de-regresiuxf3n}{%
\section{Método de máxima verosimilitud para estimar parámetros en modelos de regresión}\label{muxe9todo-de-muxe1xima-verosimilitud-para-estimar-paruxe1metros-en-modelos-de-regresiuxf3n}}

En esta sección se mostrará como estimar los parámetros de un modelo de regresión general.

\hypertarget{ejemplo-53}{%
\subsection*{Ejemplo}\label{ejemplo-53}}
\addcontentsline{toc}{subsection}{Ejemplo}

Considere el modelo de regresión mostrado abajo. Simule 1000 observaciones del modelo y use la función \texttt{optim} para estimar los parámetros del modelo.

\begin{align*}
y_i &\sim N(\mu_i, \sigma^2), \\
\mu_i &= -2 + 3 x_1, \\
\sigma &= 5, \\
x_1 &\sim U(-5, 6).
\end{align*}

El código mostrado a continuación permite simular un conjunto de valores con la estructura anteior.

\begin{Shaded}
\begin{Highlighting}[]
\NormalTok{n }\OtherTok{\textless{}{-}} \DecValTok{1000}
\NormalTok{x1 }\OtherTok{\textless{}{-}} \FunctionTok{runif}\NormalTok{(}\AttributeTok{n=}\NormalTok{n, }\AttributeTok{min=}\SpecialCharTok{{-}}\DecValTok{5}\NormalTok{, }\AttributeTok{max=}\DecValTok{6}\NormalTok{)}
\NormalTok{y }\OtherTok{\textless{}{-}} \FunctionTok{rnorm}\NormalTok{(}\AttributeTok{n=}\NormalTok{n, }\AttributeTok{mean=}\SpecialCharTok{{-}}\DecValTok{2} \SpecialCharTok{+} \DecValTok{3} \SpecialCharTok{*}\NormalTok{ x1, }\AttributeTok{sd=}\DecValTok{5}\NormalTok{)}
\end{Highlighting}
\end{Shaded}

El vector de parámetros del modelo anterior es \(\boldsymbol{\theta}=(\beta_0, \beta_1, \sigma)^\top=(-2, 3, 5)^\top\), el primer elemento corresponde al intercepto, el segundo a la pendiente y el último a la desviación.

\begin{Shaded}
\begin{Highlighting}[]
\NormalTok{minusll }\OtherTok{\textless{}{-}} \ControlFlowTok{function}\NormalTok{(theta, y, x1) \{}
\NormalTok{  media }\OtherTok{\textless{}{-}}\NormalTok{ theta[}\DecValTok{1}\NormalTok{] }\SpecialCharTok{+}\NormalTok{ theta[}\DecValTok{2}\NormalTok{] }\SpecialCharTok{*}\NormalTok{ x1  }\CommentTok{\# Se define la media}
\NormalTok{  desvi }\OtherTok{\textless{}{-}}\NormalTok{ theta[}\DecValTok{3}\NormalTok{]                  }\CommentTok{\# Se define la desviación.}
  \SpecialCharTok{{-}} \FunctionTok{sum}\NormalTok{(}\FunctionTok{dnorm}\NormalTok{(}\AttributeTok{x=}\NormalTok{y, }\AttributeTok{mean=}\NormalTok{media, }\AttributeTok{sd=}\NormalTok{desvi, }\AttributeTok{log=}\ConstantTok{TRUE}\NormalTok{))}
\NormalTok{\}}
\end{Highlighting}
\end{Shaded}

Ahora vamos a usar la función \texttt{optim} para encontrar los valores que maximizan la función de log-verosimilitud, el código para hacer eso se muestra a continuación. En el parámetro \texttt{par} se coloca un vector de posibles valores de \(\boldsymbol{\theta}\) para iniciar la búsqueda, en \texttt{fn} se coloca la función de interés, en \texttt{lower} y \texttt{upper} se colocan vectores que indican los límites de búsqueda de cada parámetro, los \(\beta_k\) pueden variar entre \(-\infty\) y \(\infty\) mientras que el parámetro \(\sigma\) toma valores en el intervalo \((0, \infty)\). Como la función \texttt{minusll} tiene argumentos adicionales \texttt{y} e \texttt{x1}, estos pasan a la función \texttt{optim} al final como se muestra en el código.

\begin{Shaded}
\begin{Highlighting}[]
\NormalTok{res1 }\OtherTok{\textless{}{-}} \FunctionTok{optim}\NormalTok{(}\AttributeTok{par=}\FunctionTok{c}\NormalTok{(}\DecValTok{0}\NormalTok{, }\DecValTok{0}\NormalTok{, }\DecValTok{1}\NormalTok{), }\AttributeTok{fn=}\NormalTok{minusll, }\AttributeTok{method=}\StringTok{\textquotesingle{}L{-}BFGS{-}B\textquotesingle{}}\NormalTok{,}
              \AttributeTok{lower=}\FunctionTok{c}\NormalTok{(}\SpecialCharTok{{-}}\ConstantTok{Inf}\NormalTok{, }\SpecialCharTok{{-}}\ConstantTok{Inf}\NormalTok{, }\DecValTok{0}\NormalTok{), }\AttributeTok{upper=}\FunctionTok{c}\NormalTok{(}\ConstantTok{Inf}\NormalTok{, }\ConstantTok{Inf}\NormalTok{, }\ConstantTok{Inf}\NormalTok{), }
              \AttributeTok{y=}\NormalTok{y, }\AttributeTok{x1=}\NormalTok{x1)}
\end{Highlighting}
\end{Shaded}

En el objeto \texttt{res1} está el resultado de la optimización, para explorar los resultados usamos

\begin{Shaded}
\begin{Highlighting}[]
\NormalTok{res1}
\end{Highlighting}
\end{Shaded}

\begin{verbatim}
## $par
## [1] -1.904603  3.079600  5.014184
## 
## $value
## [1] 3031.209
## 
## $counts
## function gradient 
##       19       19 
## 
## $convergence
## [1] 0
## 
## $message
## [1] "CONVERGENCE: REL_REDUCTION_OF_F <= FACTR*EPSMCH"
\end{verbatim}

De la salida anterior se observa que el vector de parámetros estimado es \(\hat{\beta}_0 = -1.9046025\), \(\hat{\beta}_1 = 3.0796\) y \(\hat{\sigma} = 5.0141842\), se observa también que el valor de la máxima log-verosimilitud fue de -3031.2092104. Vemos entonces que el vector estimado está muy cerca del verdadero \(\boldsymbol{\theta}=(\beta_0=-2, \beta_1=3, \sigma=5)^\top\).

\begin{rmdnote}
Cuando se usa \texttt{optim} es necesario decirle que inicie la búsqueda de \(\boldsymbol{\theta}\) a partir de un lugar. Por esa razón se usó \texttt{par=c(0,\ 0,\ 1)}, esto significa que la búsqueda inicia en el tripleta \(\beta_0=0\), \(\beta_1=0\) y \(\sigma=1\).
\end{rmdnote}

En algunas ocasiones es mejor hacer la búsqueda de los parámetros en el intervalo \((-\infty, \infty)\) que en una región limitada como por ejemplo \((0, \infty)\) o \((-1, 1)\), ya que las funciones de búsqueda podrían tener problemas en los bordes de esos intervalos. Una estrategia usual en este tipo de casos es aplicar una transformación apropiada al parámetro que tiene el dominio limitado. En el presente ejemplo \(\sigma\) sólo puede tomar valores mayores que cero y una transformación de tipo \(\log\) podría ser muy útil ya que \(\log\) relaciona los reales positivos con todos los reales. La transformación para este problema sería \(\log(\sigma)=\beta_3\) o escrita de forma inversa \(\sigma=\exp(\beta_3)\). El nuevo parámetro \(\beta_3\) puede variar en \((-\infty, \infty)\) pero al ser transformado por la función exponencial este se volvería un valor apropiado para \(\sigma\). Para implementar esta variación lo único que se debe hacer es modificar la línea 3 de la función \texttt{minusll} como se muestra a continuación:

\begin{Shaded}
\begin{Highlighting}[]
\NormalTok{minusll }\OtherTok{\textless{}{-}} \ControlFlowTok{function}\NormalTok{(theta, y, x1) \{}
\NormalTok{  media }\OtherTok{\textless{}{-}}\NormalTok{ theta[}\DecValTok{1}\NormalTok{] }\SpecialCharTok{+}\NormalTok{ theta[}\DecValTok{2}\NormalTok{] }\SpecialCharTok{*}\NormalTok{ x1  }
\NormalTok{  desvi }\OtherTok{\textless{}{-}} \FunctionTok{exp}\NormalTok{(theta[}\DecValTok{3}\NormalTok{])  }\CommentTok{\# \textless{}\textless{}\textless{}\textless{}\textless{}{-}{-}{-}{-} El cambio fue aquí}
  \SpecialCharTok{{-}} \FunctionTok{sum}\NormalTok{(}\FunctionTok{dnorm}\NormalTok{(}\AttributeTok{x=}\NormalTok{y, }\AttributeTok{mean=}\NormalTok{media, }\AttributeTok{sd=}\NormalTok{desvi, }\AttributeTok{log=}\ConstantTok{TRUE}\NormalTok{))}
\NormalTok{\}}
\end{Highlighting}
\end{Shaded}

Para hacer la búsqueda se procede de forma similar, abajo el código necesario.

\begin{Shaded}
\begin{Highlighting}[]
\NormalTok{res2 }\OtherTok{\textless{}{-}} \FunctionTok{optim}\NormalTok{(}\AttributeTok{par=}\FunctionTok{c}\NormalTok{(}\DecValTok{0}\NormalTok{, }\DecValTok{0}\NormalTok{, }\DecValTok{0}\NormalTok{), }\AttributeTok{fn=}\NormalTok{minusll, }\AttributeTok{method=}\StringTok{\textquotesingle{}L{-}BFGS{-}B\textquotesingle{}}\NormalTok{,}
              \AttributeTok{y=}\NormalTok{y, }\AttributeTok{x1=}\NormalTok{x1)}
\NormalTok{res2}
\end{Highlighting}
\end{Shaded}

\begin{verbatim}
## $par
## [1] -1.904609  3.079598  1.612271
## 
## $value
## [1] 3031.209
## 
## $counts
## function gradient 
##       21       21 
## 
## $convergence
## [1] 0
## 
## $message
## [1] "CONVERGENCE: REL_REDUCTION_OF_F <= FACTR*EPSMCH"
\end{verbatim}

De la salida anterior se observa que el vector de parámetros estimado es \(\hat{\beta}_0 = -1.9046094\), \(\hat{\beta}_1 = 3.0795984\) y \(\hat{\sigma} = \exp(1.6122706)=5.0141834\), se observa también que el valor de la máxima log-verosimilitud fue de -3031.2092104. Vemos entonces que el vector estimado está muy cerca del verdadero \(\boldsymbol{\theta}=(\beta_0=-2, \beta_1=3, \sigma=5)^\top\).

\hypertarget{ejercicios-6}{%
\section*{EJERCICIOS}\label{ejercicios-6}}
\addcontentsline{toc}{section}{EJERCICIOS}

\begin{enumerate}
\def\labelenumi{\arabic{enumi})}
\tightlist
\item
  Al inicio del Capítulo \ref{central} se presentó la base de datos sobre medidas del cuerpo, consulte la explicación sobre la base de datos y responda lo siguiente.
\end{enumerate}

\begin{itemize}
\tightlist
\item
  Si se asume que la \texttt{edad} tiene distribución normal, ¿cuáles son los estimadores de máxima verosimilitud para \(\mu\) y \(\sigma\)?
\item
  Como el histograma para la edad muestra un sesgo a la derecha se podría pensar que la distribución gamma sería una buena candidata para explicar las edades observadas. Asumiendo una distribución gamma, ¿cuáles son los estimadores de máxima verosimilitud para los parámetros?
\item
  ¿Cuál de los dos modelos es más apropiado para explicar la variable de interés? Calcule el \(AIC\) para decidir.
\end{itemize}

\begin{enumerate}
\def\labelenumi{\arabic{enumi})}
\setcounter{enumi}{1}
\tightlist
\item
  En el capítulo \ref{discretas} se presentó un ejemplo donde se usó la base de datos sobre cangrejos hembra. Consulte la explicación sobre la base de datos y responda lo siguiente.
\end{enumerate}

\begin{itemize}
\tightlist
\item
  Suponga que el número de satélites sobre cada hembra es una variable que se distribuye Poisson. Construya en R la función de log-verosimilitud \(l\), dibuje la función \(l\) y encuentre el estimador de máxima verosimilitud de \(\lambda\).
\item
  Repita el ejercicio anterior asumiendo que el número de satélites se distribuye binomial negativo.
\item
  ¿Cuál de los dos modelos es más apropiado para explicar la variable de interés? Calcule el \(AIC\) para decidir.
\end{itemize}

\begin{enumerate}
\def\labelenumi{\arabic{enumi})}
\setcounter{enumi}{2}
\tightlist
\item
  Al inicio del Capítulo \ref{varia} se presentó la base de datos sobre apartamentos usados en Medellín, consulte la explicación sobre la base de datos y responda lo siguiente.
\end{enumerate}

\begin{itemize}
\tightlist
\item
  Dibuje una densidad para la variable área del apartamento.
\item
  Describa lo encontrado en esa densidad.
\item
  ¿Qué distribuciones de 2 parámetros podrían explicar el comportamiento del área de los apartamentos? Mencione al menos 3.
\item
  Para cada una de las distribuciones anteriores dibuje un gráfico de contornos o calor para la función de log-verosimilitud y estime los parámetros de la distribución elegida.
\item
  ¿Cuál de los dos modelos es más apropiado para explicar la variable de interés? Calcule el \(AIC\) para decidir.
\end{itemize}

\begin{enumerate}
\def\labelenumi{\arabic{enumi})}
\setcounter{enumi}{3}
\tightlist
\item
  Considere el siguiente modelo de regresión.
\end{enumerate}

\begin{align*}
y_i &\sim Gamma(shape_i, scale_i), \\
\log(shape_i) &= 3 - 7 x_1, \\
\log(scale_i) &= 3 - 1 x_2, \\
x_1 &\sim U(0, 1), \\
x_2 &\sim Poisson(\lambda=3)
\end{align*}

\begin{itemize}
\tightlist
\item
  Simule 100 observaciones del modelo anterior.
\item
  Escriba el vector de parámetros del problema.
\item
  Construya la función \texttt{minusll} para el problema.
\item
  Use la función \texttt{optim} para estimar los parámetros del problema.
\end{itemize}

\hypertarget{select_dist}{%
\chapter{Selección de la distribución}\label{select_dist}}

En este capítulo se mostrará cómo usar R para obtener obtener el listado de las distribuciones que mejor se ajustan a una variable.

\hypertarget{funciuxf3n-fitdists}{%
\section{\texorpdfstring{Función \texttt{fitDists}}{Función fitDists}}\label{funciuxf3n-fitdists}}

La función \texttt{fitDist} del paquete \textbf{gamlss} \index{gamlss} permite explorar las distribuciones que mejor explican un conjunto de datos.

La función \texttt{fitDist} tiene la siguiente estructura:

\begin{Shaded}
\begin{Highlighting}[]
\FunctionTok{fitDist}\NormalTok{(y, }\AttributeTok{k =} \DecValTok{2}\NormalTok{, }
      \AttributeTok{type =} \FunctionTok{c}\NormalTok{(}\StringTok{"realAll"}\NormalTok{, }\StringTok{"realline"}\NormalTok{, }\StringTok{"realplus"}\NormalTok{,}
               \StringTok{"real0to1"}\NormalTok{, }\StringTok{"counts"}\NormalTok{, }\StringTok{"binom"}\NormalTok{))}
\end{Highlighting}
\end{Shaded}

El parámetro \texttt{y} sirve para ingresar el vector con la información; \texttt{k=2} es la penalización por cada parámetro estimado para calcular el \(GAIC\), por defecto es 2; y el parámetro \texttt{type} sirve para indicar el tipo de distribución, los posibles valores son:

\begin{itemize}
\tightlist
\item
  \texttt{realAll}: para hacer la búsqueda en todas las distribuciones disponibles en \textbf{gamlss}.
\item
  \texttt{realline}: para variables en \(\Re\).
\item
  \texttt{realplus}: para variables en \(\Re^+\).
\item
  \texttt{real0to1}: para variables en el intervalo \((0, 1)\).
\item
  \texttt{counts}: para variables de conteo.
\item
  \texttt{binom}: para variables de tipo binomial.
\end{itemize}

\hypertarget{ejemplo-54}{%
\subsection*{Ejemplo}\label{ejemplo-54}}
\addcontentsline{toc}{subsection}{Ejemplo}

Generar \(n=100\) observaciones de una gamma con parámetro \(\mu=2\) y parámetro \(\sigma=0.5\) y verificar si la función \texttt{fitDist} logra identificar que los datos fueron generados de una distribución gamma. Use \(k=2\) para calcular el \(AIC\).

\textbf{Solución}

Para generar la muestra aleatoria solicitada se fijó la semilla con el objetivo de que el lector pueda obtener los mismos resultados de este ejemplo. En este ejemplo vamos a usar la función \texttt{rGA} del paquete \textbf{gamlss} para simular la muestra aleatoria \texttt{ma}.

\begin{Shaded}
\begin{Highlighting}[]
\FunctionTok{library}\NormalTok{(gamlss)}
\NormalTok{n }\OtherTok{\textless{}{-}} \DecValTok{500}
\FunctionTok{set.seed}\NormalTok{(}\DecValTok{12345}\NormalTok{)}
\NormalTok{ma }\OtherTok{\textless{}{-}} \FunctionTok{rGA}\NormalTok{(}\AttributeTok{n=}\NormalTok{n, }\AttributeTok{mu=}\DecValTok{2}\NormalTok{, }\AttributeTok{sigma=}\FloatTok{0.5}\NormalTok{)}
\end{Highlighting}
\end{Shaded}

Para ver los datos simulados vamos a construir un histograma sencillo y en el eje horizontal se van a destacar los datos usando una especie de ``tapete'' con la función \texttt{rug}.

\begin{Shaded}
\begin{Highlighting}[]
\FunctionTok{hist}\NormalTok{(}\AttributeTok{x=}\NormalTok{ma, }\AttributeTok{freq=}\ConstantTok{FALSE}\NormalTok{, }\AttributeTok{main=}\StringTok{""}\NormalTok{)}
\FunctionTok{rug}\NormalTok{(}\AttributeTok{x=}\NormalTok{ma, }\AttributeTok{col=}\StringTok{"deepskyblue3"}\NormalTok{)}
\end{Highlighting}
\end{Shaded}

\begin{figure}
\centering
\includegraphics{Manual_de_R_files/figure-latex/histDist0-1.pdf}
\caption{\label{fig:histDist0}Histograma para la muestra simulada con la densidad de una Gamma(mu=4.308, sigma=0.6682).}
\end{figure}

Se va a usar la función \texttt{fitDist} con \texttt{type=\textquotesingle{}realplus\textquotesingle{}} porque se sabemos que la muestra aleatoria tiene valores en \(\Re^+\). Los resultados de almacenan en el objeto \texttt{modelos} y para obtener la lista de los mejores modelos con su respectivo \(AIC\) se escribe en la consola \texttt{modelos\$fits}. Abajo el código usado.

\begin{Shaded}
\begin{Highlighting}[]
\NormalTok{modelos }\OtherTok{\textless{}{-}} \FunctionTok{fitDist}\NormalTok{(}\AttributeTok{y=}\NormalTok{ma, }\AttributeTok{type=}\StringTok{\textquotesingle{}realplus\textquotesingle{}}\NormalTok{, }\AttributeTok{k=}\DecValTok{2}\NormalTok{)}
\end{Highlighting}
\end{Shaded}

\begin{verbatim}
##   |                                                                              |                                                                      |   0%  |                                                                              |===                                                                   |   4%  |                                                                              |======                                                                |   9%  |                                                                              |=========                                                             |  13%  |                                                                              |============                                                          |  17%  |                                                                              |===============                                                       |  22%  |                                                                              |==================                                                    |  26%  |                                                                              |=====================                                                 |  30%  |                                                                              |========================                                              |  35%  |                                                                              |===========================                                           |  39%  |                                                                              |==============================                                        |  43%  |                                                                              |=================================                                     |  48%  |                                                                              |=====================================                                 |  52%  |                                                                              |========================================                              |  57%  |                                                                              |===========================================                           |  61%  |                                                                              |==============================================                        |  65%  |                                                                              |=================================================                     |  70%  |                                                                              |====================================================                  |  74%  |                                                                              |=======================================================               |  78%  |                                                                              |==========================================================            |  83%Error in solve.default(oout$hessian) : 
##   Lapack routine dgesv: system is exactly singular: U[4,4] = 0
##   |                                                                              |=============================================================         |  87%Error in solve.default(oout$hessian) : 
##   Lapack routine dgesv: system is exactly singular: U[4,4] = 0
##   |                                                                              |================================================================      |  91%  |                                                                              |===================================================================   |  96%  |                                                                              |======================================================================| 100%
\end{verbatim}

\begin{Shaded}
\begin{Highlighting}[]
\NormalTok{modelos}\SpecialCharTok{$}\NormalTok{fits}
\end{Highlighting}
\end{Shaded}

\begin{verbatim}
##       GA       GG    BCCGo     BCCG      GIG      GB2    BCPEo     BCPE 
## 1368.503 1368.561 1369.284 1369.284 1370.503 1370.562 1370.967 1370.967 
##      BCT     BCTo      WEI     WEI2     WEI3   exGAUS    LOGNO   LOGNO2 
## 1371.284 1371.284 1376.365 1376.365 1376.365 1383.357 1401.925 1401.925 
##       IG   IGAMMA      EXP  PARETO2       GP PARETO2o 
## 1418.277 1488.333 1738.337 1740.337 1740.338 1740.339
\end{verbatim}

De la lista anterior se observa que la función gamma está en el primer lugar con un \(AIC=1368.5027744\) con el menor \(AIC\). Esto significa que la distribución gamma explica mejor los datos de la muestra, y esto coincide con la realidad, ya que la muestra fue generada de una distribución gamma.

\begin{rmdnote}
En la salida anterior se observan unos mensajes de error que no deben causar preocupación. Esos errores se deben al proceso de estimación de parámetros con algunas de las distribuciones que no aparecen en la lista final.
\end{rmdnote}

Para obtener los valores estimados de \(\mu\) y \(\sigma\) se usa el siguiente código.

\begin{Shaded}
\begin{Highlighting}[]
\NormalTok{modelos}\SpecialCharTok{$}\NormalTok{mu}
\end{Highlighting}
\end{Shaded}

\begin{verbatim}
## [1] 2.088272
\end{verbatim}

\begin{Shaded}
\begin{Highlighting}[]
\NormalTok{modelos}\SpecialCharTok{$}\NormalTok{sigma}
\end{Highlighting}
\end{Shaded}

\begin{verbatim}
## [1] 0.4945352
\end{verbatim}

Por último vamos a dibujar el histograma para la muestra aleatoria y vamos a agregar la densidad de la distribución gamma identificada como la distribución que mejor explica el comportamiento de la variable. Para hacer lo deseado se usa la función \texttt{histDist} del paquete \textbf{gamlss}, sólo es necesario ingresar los datos y el nombre de la distribución. Abajo el código usado.

\begin{Shaded}
\begin{Highlighting}[]
\NormalTok{h }\OtherTok{\textless{}{-}} \FunctionTok{histDist}\NormalTok{(}\AttributeTok{y=}\NormalTok{ma, }\AttributeTok{family=}\NormalTok{GA, }\AttributeTok{main=}\StringTok{\textquotesingle{}\textquotesingle{}}\NormalTok{, }\AttributeTok{xlab=}\StringTok{\textquotesingle{}x\textquotesingle{}}\NormalTok{, }\AttributeTok{ylab=}\StringTok{\textquotesingle{}Densidad\textquotesingle{}}\NormalTok{,}
              \AttributeTok{line.col=}\StringTok{\textquotesingle{}deepskyblue3\textquotesingle{}}\NormalTok{, }\AttributeTok{line.wd=}\DecValTok{4}\NormalTok{, }\AttributeTok{ylim=}\FunctionTok{c}\NormalTok{(}\DecValTok{0}\NormalTok{, }\FloatTok{0.45}\NormalTok{))}
\FunctionTok{rug}\NormalTok{(}\AttributeTok{x=}\NormalTok{ma, }\AttributeTok{col=}\StringTok{"deepskyblue3"}\NormalTok{)}
\end{Highlighting}
\end{Shaded}

\begin{figure}
\centering
\includegraphics{Manual_de_R_files/figure-latex/histDist1-1.pdf}
\caption{\label{fig:histDist1}Histograma para la muestra simulada con la densidad de una Gamma(mu=2.088, sigma=0.495).}
\end{figure}

En la Figura \ref{fig:histDist1} se presenta el histograma para muestra aleatoria y la densidad de la gamma que mejor explica estos datos. Se observa claramente que la curva de densidad azul acompaña la forma del histograma.

\hypertarget{normalidad}{%
\chapter{Estudiando la normalidad}\label{normalidad}}

En este capítulo se mostrará cómo utilizar las herramientas de R para estudiar la normalidad univariada de un conjunto de datos.

\hypertarget{consideraciones-iniciales}{%
\section{Consideraciones iniciales}\label{consideraciones-iniciales}}

En estadística hay una gran cantidad de modelos, pruebas y procedimientos que tienen como supuesto la normalidad, por lo tanto, se hace necesario contar con herramientas que nos guíen para responder si se cumple o no el supuesto de normalidad.

Para estudiar si una muestra aleatoria proviene de una población con distribución normal se disponen de tres herramientas que se listan a continuación.

\begin{enumerate}
\def\labelenumi{\arabic{enumi}.}
\tightlist
\item
  Histograma y/o densidad.
\item
  Gráficos cuantil cuantil (QQplot).
\item
  Pruebas de hipótesis.
\end{enumerate}

Al construir un histograma y/o densidad para la variable de interés se puede evaluar visualmente la simetría de la distribución de los datos. Si se observa una violación clara de la simetría (sesgo a uno de los lados) o si se observa una distribución con más de una moda, eso sería indicio de que la muestra no proviene de una población normal. Por otra parte, si se observa simetría en los datos, esto NO garantiza que la muestra aleatoria proviene de una población normal y se hace necesario recurrir a otras herramientas específicas para estudiar normalidad como lo son los gráficos QQplot y pruebas de hipótesis.

A continuación se presentan varias secciones donde se profundiza sobre el uso de cada de las tres herramientas anteriormente listadas para estudiar la normalidad.

\hypertarget{histograma-y-densidad}{%
\section{Histograma y densidad}\label{histograma-y-densidad}}

El histograma y el gráfico de densidad son herramientas muy útiles porque sirven para mostrar la distribución, la simetría, el sesgo, variabilidad, moda, mediana y observaciones atípicas de un conjunto de datos. Para explorar la normalidad de un conjunto de datos lo que se busca es que el histograma o gráfico de densidad presenten un patrón más o menos simétrico. Para obtener detalles de la construcción de histogramas y gráficos de densidad se recomienda consultar \citet{hernandez_correa}.

A continuación se presentan dos ejemplos, uno con datos simulados y otro con datos reales, con los cuales se muestra la utilidad del histograma y gráfico de densidad al explorar la normalidad.

\hypertarget{ejemplo-densidad-con-datos-simulados}{%
\subsection*{Ejemplo densidad con datos simulados}\label{ejemplo-densidad-con-datos-simulados}}
\addcontentsline{toc}{subsection}{Ejemplo densidad con datos simulados}

Simular 4 muestra aleatorias de una \(N(0, 1)\) con tamaños de muestra \(n\)=10, 100, 1000 y 10000; para cada una de las muestras construir los gráficos de densidad y analizar si son simétricos.

Lo primero que se debe hacer es definir el vector \texttt{n} con los valores de los tamaños de muestra, luego dentro de una sentencia \texttt{for} se simula cada muestra y se dibuja su densidad.

\begin{Shaded}
\begin{Highlighting}[]
\FunctionTok{par}\NormalTok{(}\AttributeTok{mfrow=}\FunctionTok{c}\NormalTok{(}\DecValTok{2}\NormalTok{, }\DecValTok{2}\NormalTok{))}
\NormalTok{n }\OtherTok{\textless{}{-}} \FunctionTok{c}\NormalTok{(}\DecValTok{10}\NormalTok{, }\DecValTok{100}\NormalTok{, }\DecValTok{1000}\NormalTok{, }\DecValTok{10000}\NormalTok{)}
\ControlFlowTok{for}\NormalTok{ (i }\ControlFlowTok{in}\NormalTok{ n) \{}
\NormalTok{  x }\OtherTok{\textless{}{-}} \FunctionTok{rnorm}\NormalTok{(i)}
  \FunctionTok{plot}\NormalTok{(}\FunctionTok{density}\NormalTok{(x), }\AttributeTok{main=}\FunctionTok{bquote}\NormalTok{(}\SpecialCharTok{\textasciitilde{}}\NormalTok{ n }\SpecialCharTok{==}\NormalTok{ .(i)),}
       \AttributeTok{ylab=}\StringTok{\textquotesingle{}Densidad\textquotesingle{}}\NormalTok{, }\AttributeTok{col=}\StringTok{\textquotesingle{}blue3\textquotesingle{}}\NormalTok{, }\AttributeTok{xlab=}\StringTok{\textquotesingle{}x\textquotesingle{}}\NormalTok{, }\AttributeTok{las=}\DecValTok{1}\NormalTok{, }\AttributeTok{lwd=}\DecValTok{4}\NormalTok{)}
\NormalTok{\}}
\end{Highlighting}
\end{Shaded}

\begin{figure}
\centering
\includegraphics{Manual_de_R_files/figure-latex/normalidad1-1.pdf}
\caption{\label{fig:normalidad1}Densidad para 4 muestras de una N(0, 1) con diferente tamaño de muestra.}
\end{figure}

En la Figura \ref{fig:normalidad1} se muestran las cuatro densidades, de esta figura se observa que, a pesar de haber generado las muestras de una normal estándar, las densidades no son perfectamente simétricas, sólo para el caso de tamaño de muestra \(n=10000\) la densidad muestral fue bastante simétrica. Esto significa que el gráfico de densidad se debe usar con precaución para decidir sobre la normalidad de un conjunto de datos, como regla práctica se aconseja lo siguiente:

\begin{itemize}
\tightlist
\item
  Si la densidad muestral es muy pero muy asimétrica se debe desconfiar de la normalidad de los datos.
\item
  Si la densidad muestral es más o menos simétrica, se deben usar otras herramientas como QQplot o pruebas de hipótesis para obtener una mejor conclusión.
\end{itemize}

\begin{rmdwarning}
Si un histograma o densidad es muy asimétrico o sesgado, es evidencia en contra de la normalidad de los datos. Si el histograma o densidad presentan simetría, esto no es garantía de la normalidad de los datos.
\end{rmdwarning}

\hypertarget{ejemplo-densidad-del-peso-corporal}{%
\subsection*{Ejemplo densidad del peso corporal}\label{ejemplo-densidad-del-peso-corporal}}
\addcontentsline{toc}{subsection}{Ejemplo densidad del peso corporal}

Considerando la base de datos \textbf{medidas del cuerpo} presentada en el Capítulo \ref{central}, se desea saber si el peso corporal, tanto de hombres y mujeres, tiene una distribución normal.

Para hacer la exploración lo primero es cargar la base de datos usando el siguiente código.

\begin{Shaded}
\begin{Highlighting}[]
\NormalTok{url }\OtherTok{\textless{}{-}} \StringTok{\textquotesingle{}https://raw.githubusercontent.com/fhernanb/datos/master/medidas\_cuerpo\textquotesingle{}}
\NormalTok{datos }\OtherTok{\textless{}{-}} \FunctionTok{read.table}\NormalTok{(}\AttributeTok{file=}\NormalTok{url, }\AttributeTok{header=}\NormalTok{T)}
\end{Highlighting}
\end{Shaded}

La variable \texttt{peso} del objeto \texttt{datos} contiene la información sobre el peso corporal de ambos sexos, debemos entonces partir o dividir esta información diferenciando entre hombres y mujeres, para esto usamos la función \texttt{split} de la siguiente forma.

\begin{Shaded}
\begin{Highlighting}[]
\NormalTok{pesos }\OtherTok{\textless{}{-}} \FunctionTok{split}\NormalTok{(datos}\SpecialCharTok{$}\NormalTok{peso, datos}\SpecialCharTok{$}\NormalTok{sexo)}
\NormalTok{pesos  }\CommentTok{\# Para ver los elementos de la lista pesos}
\end{Highlighting}
\end{Shaded}

\begin{verbatim}
## $Hombre
##  [1] 87.3 80.0 82.3 73.6 74.1 85.9 73.2 76.3 65.9 90.9 89.1 62.3 82.7 79.1 98.2
## [16] 84.1 83.2 83.2
## 
## $Mujer
##  [1] 51.6 59.0 49.2 63.0 53.6 59.0 47.6 69.8 66.8 75.2 55.2 54.2 62.5 42.0 50.0
## [16] 49.8 49.2 73.2
\end{verbatim}

El objeto \texttt{pesos} es una lista con dos elementos, el primero contiene los pesos de los hombres mientras que el segundo contiene los pesos de las mujeres. Note que \texttt{pesos} es un objeto mientras que \texttt{peso} es el nombre usado para la variable peso corporal en la base de datos.

Para explorar la normalidad de los pesos se dibujan dos densidades, una para el peso de hombres y otra para el peso de las mujeres, a continuación el código utilizado.

\begin{Shaded}
\begin{Highlighting}[]
\FunctionTok{plot}\NormalTok{(}\FunctionTok{density}\NormalTok{(pesos}\SpecialCharTok{$}\NormalTok{Hombre), }\AttributeTok{lwd=}\DecValTok{3}\NormalTok{, }\AttributeTok{col=}\StringTok{\textquotesingle{}blue\textquotesingle{}}\NormalTok{,}
     \AttributeTok{xlim=}\FunctionTok{c}\NormalTok{(}\DecValTok{30}\NormalTok{, }\DecValTok{110}\NormalTok{), }\AttributeTok{main=}\StringTok{\textquotesingle{}\textquotesingle{}}\NormalTok{, }\AttributeTok{las=}\DecValTok{1}\NormalTok{,}
     \AttributeTok{xlab=}\StringTok{\textquotesingle{}Peso (kg)\textquotesingle{}}\NormalTok{, }\AttributeTok{ylab=}\StringTok{\textquotesingle{}Densidad\textquotesingle{}}\NormalTok{)}
\FunctionTok{lines}\NormalTok{(}\FunctionTok{density}\NormalTok{(pesos}\SpecialCharTok{$}\NormalTok{Mujer), }\AttributeTok{lwd=}\DecValTok{3}\NormalTok{, }\AttributeTok{col=}\StringTok{\textquotesingle{}deeppink\textquotesingle{}}\NormalTok{)}
\FunctionTok{legend}\NormalTok{(}\StringTok{\textquotesingle{}topleft\textquotesingle{}}\NormalTok{, }\AttributeTok{legend=}\FunctionTok{c}\NormalTok{(}\StringTok{\textquotesingle{}Hombres\textquotesingle{}}\NormalTok{, }\StringTok{\textquotesingle{}Mujeres\textquotesingle{}}\NormalTok{),}
       \AttributeTok{lwd=}\DecValTok{3}\NormalTok{, }\AttributeTok{col=}\FunctionTok{c}\NormalTok{(}\StringTok{\textquotesingle{}blue\textquotesingle{}}\NormalTok{, }\StringTok{\textquotesingle{}deeppink\textquotesingle{}}\NormalTok{), }\AttributeTok{bty=}\StringTok{\textquotesingle{}n\textquotesingle{}}\NormalTok{)}
\end{Highlighting}
\end{Shaded}

\begin{figure}
\centering
\includegraphics{Manual_de_R_files/figure-latex/normalidad2-1.pdf}
\caption{\label{fig:normalidad2}Densidad para el peso corporal de hombres y mujeres.}
\end{figure}

En la Figura \ref{fig:normalidad2} se muestran las dos densidades, en la figura no se observa una evidencia clara de sesgo, lo que se observa es que la densidad para los hombres es un poco más simétrica que la densidad para las mujeres. De estos resultados no se puede rechazar la premisa de que pesos corporales provienen de una distribución normal, lo recomendable es construir QQplot y aplicar prueba de hipótesis.

\hypertarget{gruxe1ficos-cuantil-cuantil}{%
\section{\texorpdfstring{Gráficos cuantil cuantil \index{QQplot} \index{qqnorm} \index{qqline}}{Gráficos cuantil cuantil   }}\label{gruxe1ficos-cuantil-cuantil}}

Los gráficos cuantil cuantil (QQplot) son una herramienta gráfica para explorar si un conjunto de datos o muestra proviene de una población con cierta distribución, en particular aquí nos interesan para estudiar la normalidad de un conjunto de datos. La función \texttt{qqnorm} sirve para construir el QQplot y la función \texttt{qqline} agrega una línea de referencia que ayuda a interpretar el gráfico QQplot, para obtener una explicación de cómo construir este gráfico se recomienda ver el video disponible en este \href{https://www.youtube.com/watch?v=kx_o9rnI4DE}{enlace}.

En la Figura \ref{fig:normalidad3} se muestra un ejemplo de un QQplot y de sus partes, los puntos y la línea de referencia. Si se tuviese una muestra distribuída perfectamente normal, se esperaría que los puntos estuviesen perfectamente alineados con la línea de referencia, sin embargo, las muestran con las que se trabajan en la práctica casi nunca presentan este comportamiento aún si fueron obtenidas de una población normal. En la práctica se aceptan alejamientos del patrón lineal para aceptar que los datos si provienen de una población normal.

\begin{figure}
\centering
\includegraphics{Manual_de_R_files/figure-latex/normalidad3-1.pdf}
\caption{\label{fig:normalidad3}Ejemplo de un QQplot.}
\end{figure}

A continuación se presentan cuatro ejemplos, dos con datos simulados y otro con datos reales para mostrar la utilidad del qqplot al explorar la normalidad.

\hypertarget{ejemplo-1-qqplot-con-datos-simulados}{%
\subsection*{Ejemplo 1 QQplot con datos simulados}\label{ejemplo-1-qqplot-con-datos-simulados}}
\addcontentsline{toc}{subsection}{Ejemplo 1 QQplot con datos simulados}

Simular 4 muestra aleatorias de una \(N(0, 1)\) con tamaños de muestra \(n\)=10, 30, 50 y 100, para cada una de ellas construir el QQplot.

Lo primero que se debe hacer es definir el vector \texttt{n} con los valores del tamaño de muestra, luego dentro de una sentencia \texttt{for} se simula cada muestra \texttt{x} y por último se dibuja el QQplot para cada muestra, a continuación el código utilizado.

\begin{Shaded}
\begin{Highlighting}[]
\FunctionTok{par}\NormalTok{(}\AttributeTok{mfrow=}\FunctionTok{c}\NormalTok{(}\DecValTok{2}\NormalTok{, }\DecValTok{2}\NormalTok{))}
\NormalTok{n }\OtherTok{\textless{}{-}} \FunctionTok{c}\NormalTok{(}\DecValTok{10}\NormalTok{, }\DecValTok{30}\NormalTok{, }\DecValTok{50}\NormalTok{, }\DecValTok{100}\NormalTok{)}
\ControlFlowTok{for}\NormalTok{ (i }\ControlFlowTok{in}\NormalTok{ n) \{}
\NormalTok{  x }\OtherTok{\textless{}{-}} \FunctionTok{rnorm}\NormalTok{(i)}
  \FunctionTok{qqnorm}\NormalTok{(x, }\AttributeTok{main=}\FunctionTok{bquote}\NormalTok{(}\SpecialCharTok{\textasciitilde{}}\NormalTok{ n }\SpecialCharTok{==}\NormalTok{ .(i)))}
  \FunctionTok{qqline}\NormalTok{(x)  }\CommentTok{\# Para agregar la linea de referencia}
\NormalTok{\}}
\end{Highlighting}
\end{Shaded}

\begin{figure}
\centering
\includegraphics{Manual_de_R_files/figure-latex/normalidad4-1.pdf}
\caption{\label{fig:normalidad4}QQplot para 4 muestras de una N(0, 1) con diferente tamaño de muestra.}
\end{figure}

La Figura \ref{fig:normalidad4} muestra que, a pesar de haber simulado cada muestra \texttt{x} de una \(N(0, 1)\), los puntos no se alinean de forma perfecta, esto significa que en la práctica se debe ser prudente con la interpretación de un QQplot, un alejamiento del patrón lineal NO significa que la muestra no provenga de una población normal.

\hypertarget{ejemplo-2-qqplot-con-datos-simulados}{%
\subsection*{Ejemplo 2 QQplot con datos simulados}\label{ejemplo-2-qqplot-con-datos-simulados}}
\addcontentsline{toc}{subsection}{Ejemplo 2 QQplot con datos simulados}

Simular 1 muestra aleatoria con \(n=50\) de cada una de las siguientes poblaciones Poisson(5), NBinom(5, 0.5), Gamma(2, 3) y Weibull(1, 3), para cada una de las muestras construir el QQplot para explorar la normalidad de las muestras.

Las muestras de cada una de las poblaciones se generan con las funciones \texttt{rpois}, \texttt{rnbinom}, \texttt{rgamma} y \texttt{rweibull} especificando los parámetros. A continuación el código necesario para obtener los QQplot solicitados.

\begin{Shaded}
\begin{Highlighting}[]
\NormalTok{m1 }\OtherTok{\textless{}{-}} \FunctionTok{rpois}\NormalTok{(}\AttributeTok{n=}\DecValTok{50}\NormalTok{, }\AttributeTok{lambda=}\DecValTok{5}\NormalTok{)}
\NormalTok{m2 }\OtherTok{\textless{}{-}} \FunctionTok{rnbinom}\NormalTok{(}\AttributeTok{n=}\DecValTok{50}\NormalTok{, }\AttributeTok{size=}\DecValTok{5}\NormalTok{, }\AttributeTok{prob=}\FloatTok{0.35}\NormalTok{)}
\NormalTok{m3 }\OtherTok{\textless{}{-}} \FunctionTok{rgamma}\NormalTok{(}\AttributeTok{n=}\DecValTok{50}\NormalTok{, }\AttributeTok{shape=}\DecValTok{2}\NormalTok{, }\AttributeTok{scale=}\DecValTok{3}\NormalTok{)}
\NormalTok{m4 }\OtherTok{\textless{}{-}} \FunctionTok{rweibull}\NormalTok{(}\AttributeTok{n=}\DecValTok{50}\NormalTok{, }\AttributeTok{shape=}\DecValTok{1}\NormalTok{, }\AttributeTok{scale=}\DecValTok{3}\NormalTok{)}
\FunctionTok{par}\NormalTok{(}\AttributeTok{mfrow=}\FunctionTok{c}\NormalTok{(}\DecValTok{2}\NormalTok{, }\DecValTok{2}\NormalTok{))}
\FunctionTok{qqnorm}\NormalTok{(m1, }\AttributeTok{main=}\StringTok{\textquotesingle{}Poisson(5)\textquotesingle{}}\NormalTok{)}
\FunctionTok{qqline}\NormalTok{(m1)}
\FunctionTok{qqnorm}\NormalTok{(m2, }\AttributeTok{main=}\StringTok{\textquotesingle{}NBinom(5, 0.35)\textquotesingle{}}\NormalTok{)}
\FunctionTok{qqline}\NormalTok{(m1)}
\FunctionTok{qqnorm}\NormalTok{(m3, }\AttributeTok{main=}\StringTok{\textquotesingle{}Gamma(2, 3)\textquotesingle{}}\NormalTok{)}
\FunctionTok{qqline}\NormalTok{(m1)}
\FunctionTok{qqnorm}\NormalTok{(m4, }\AttributeTok{main=}\StringTok{\textquotesingle{}Weibull(1, 3)\textquotesingle{}}\NormalTok{)}
\FunctionTok{qqline}\NormalTok{(m1)}
\end{Highlighting}
\end{Shaded}

\begin{figure}
\centering
\includegraphics{Manual_de_R_files/figure-latex/normalidad6-1.pdf}
\caption{\label{fig:normalidad6}QQplot para muestras generadas de poblaciones Poisson, Binomial Negativa, Gamma y Weibull.}
\end{figure}

En la Figura \ref{fig:normalidad6} se muestran los cuatro QQplot para cada una de las muestras generadas de las distribuciones indicadas. Se observa claramente que los puntos del QQplot no está alineados, esto es una clara evidencia de que las muestras NO provienen de poblaciones normales. Otro aspecto interesante a resaltar es el patrón de escalera que se observa para las muestras generadas de poblaciones discretas (Poisson y Binomial Negativa).

\begin{rmdwarning}
Se debe tener cuidado al concluir con un QQplot, lo que para una persona puede estar alineado, para otra puede no estarlo. El QQplot es un gráfico exploratorio de normalidad.
\end{rmdwarning}

\hypertarget{ejemplo-qqplot-para-peso-corporal}{%
\subsection*{Ejemplo QQplot para peso corporal}\label{ejemplo-qqplot-para-peso-corporal}}
\addcontentsline{toc}{subsection}{Ejemplo QQplot para peso corporal}

Retomando la base de datos \textbf{medidas del cuerpo} presentada en el Capítulo \ref{central}, se desea saber si el peso corporal, tanto de hombres y mujeres, tiene una distribución normal usando QQplots.

Para hacer la exploración lo primero es cargar la base de datos si aún no se ha cargado.

\begin{Shaded}
\begin{Highlighting}[]
\NormalTok{url }\OtherTok{\textless{}{-}} \StringTok{\textquotesingle{}https://raw.githubusercontent.com/fhernanb/datos/master/medidas\_cuerpo\textquotesingle{}}
\NormalTok{datos }\OtherTok{\textless{}{-}} \FunctionTok{read.table}\NormalTok{(}\AttributeTok{file=}\NormalTok{url, }\AttributeTok{header=}\NormalTok{T)}
\end{Highlighting}
\end{Shaded}

La variable \texttt{peso} del objeto \texttt{datos} contiene la información sobre el peso corporal de ambos sexos, debemos entonces partir o dividir esta información diferenciando entre hombres y mujeres, para esto usamos la función \texttt{split} de la siguiente forma.

\begin{Shaded}
\begin{Highlighting}[]
\NormalTok{pesos }\OtherTok{\textless{}{-}} \FunctionTok{split}\NormalTok{(datos}\SpecialCharTok{$}\NormalTok{peso, datos}\SpecialCharTok{$}\NormalTok{sexo)}
\NormalTok{pesos}
\end{Highlighting}
\end{Shaded}

\begin{verbatim}
## $Hombre
##  [1] 87.3 80.0 82.3 73.6 74.1 85.9 73.2 76.3 65.9 90.9 89.1 62.3 82.7 79.1 98.2
## [16] 84.1 83.2 83.2
## 
## $Mujer
##  [1] 51.6 59.0 49.2 63.0 53.6 59.0 47.6 69.8 66.8 75.2 55.2 54.2 62.5 42.0 50.0
## [16] 49.8 49.2 73.2
\end{verbatim}

El objeto \texttt{pesos} es una lista con dos elementos, el primero contiene los pesos de los hombres mientras que el segundo contiene los pesos de las mujeres. Note que \texttt{pesos} es un objeto mientras que \texttt{peso} es el nombre usado para la variable peso corporal en la base de datos.

Para explorar la normalidad de los pesos se dibujan dos densidades, una para el peso de hombres y otra para el peso de las mujeres, a continuación el código utilizado.

\begin{Shaded}
\begin{Highlighting}[]
\FunctionTok{par}\NormalTok{(}\AttributeTok{mfrow=}\FunctionTok{c}\NormalTok{(}\DecValTok{1}\NormalTok{, }\DecValTok{2}\NormalTok{))}
\FunctionTok{qqnorm}\NormalTok{(pesos}\SpecialCharTok{$}\NormalTok{Hombre, }\AttributeTok{pch=}\DecValTok{20}\NormalTok{,}
       \AttributeTok{main=}\StringTok{\textquotesingle{}QQplot para peso de hombres\textquotesingle{}}\NormalTok{)}
\FunctionTok{qqline}\NormalTok{(pesos}\SpecialCharTok{$}\NormalTok{Hombre)}

\FunctionTok{qqnorm}\NormalTok{(pesos}\SpecialCharTok{$}\NormalTok{Mujer, }\AttributeTok{pch=}\DecValTok{20}\NormalTok{,}
       \AttributeTok{main=}\StringTok{\textquotesingle{}QQplot para peso de mujeres\textquotesingle{}}\NormalTok{)}
\FunctionTok{qqline}\NormalTok{(pesos}\SpecialCharTok{$}\NormalTok{Mujer)}
\end{Highlighting}
\end{Shaded}

\begin{figure}
\centering
\includegraphics{Manual_de_R_files/figure-latex/normalidad5-1.pdf}
\caption{\label{fig:normalidad5}QQplot para el peso corporal de hombres y mujeres.}
\end{figure}

La Figura \ref{fig:normalidad5} muestra el QQplot para el peso corporal de hombres y mujeres, de la figura se observa que los puntos no están tan desalineados, lo cual cual nos lleva a no rechazar la hipótesis de normalidad.

\begin{rmdwarning}
Se debe interpretar con precaución el QQplot, del ejemplo con datos simulados se vió que a pesar de haber generado las muestras de una \(N(0,1)\) los QQplot no siempre están perfectamente alineados.
\end{rmdwarning}

\hypertarget{ejemplo-qqplot-con-bandas}{%
\subsection*{Ejemplo QQplot con bandas}\label{ejemplo-qqplot-con-bandas}}
\addcontentsline{toc}{subsection}{Ejemplo QQplot con bandas}

Construir QQplot con bandas de confianza para el peso corporal de hombres y mujeres con los datos del ejemplo anterior. ¿Se puede afirma que los pesos corporales provienen de una distribución normal?

Para construir este tipo de QQplot se usa la función \texttt{qqplot} del paquete \textbf{car}. A continuación el código para construir el gráfico solicitado.

\begin{Shaded}
\begin{Highlighting}[]
\FunctionTok{require}\NormalTok{(car)}
\FunctionTok{par}\NormalTok{(}\AttributeTok{mfrow=}\FunctionTok{c}\NormalTok{(}\DecValTok{1}\NormalTok{, }\DecValTok{2}\NormalTok{))}
\FunctionTok{qqPlot}\NormalTok{(pesos}\SpecialCharTok{$}\NormalTok{Hombre, }\AttributeTok{pch=}\DecValTok{20}\NormalTok{, }\AttributeTok{ylab=}\StringTok{\textquotesingle{}Peso (Kg)\textquotesingle{}}\NormalTok{,}
       \AttributeTok{main=}\StringTok{\textquotesingle{}QQplot para peso de hombres\textquotesingle{}}\NormalTok{)}
\end{Highlighting}
\end{Shaded}

\begin{verbatim}
## [1] 12 15
\end{verbatim}

\begin{Shaded}
\begin{Highlighting}[]
\FunctionTok{qqPlot}\NormalTok{(pesos}\SpecialCharTok{$}\NormalTok{Mujer, }\AttributeTok{pch=}\DecValTok{20}\NormalTok{, }\AttributeTok{ylab=}\StringTok{\textquotesingle{}Peso (Kg)\textquotesingle{}}\NormalTok{,}
       \AttributeTok{main=}\StringTok{\textquotesingle{}QQplot para peso de mujeres\textquotesingle{}}\NormalTok{)}
\end{Highlighting}
\end{Shaded}

\begin{figure}
\centering
\includegraphics{Manual_de_R_files/figure-latex/normalidad7-1.pdf}
\caption{\label{fig:normalidad7}QQplot con bandas de confianza para el peso corporal de hombres y mujeres.}
\end{figure}

\begin{verbatim}
## [1] 10 18
\end{verbatim}

En la Figura \ref{fig:normalidad7} se muestra el QQplot solicitado, como los puntos del QQplot están dentro de las bandas se puede aceptar que los pesos corporales provienen de una población normal.

La función \texttt{qqPlot} tiene varios parámetros adicionales que recomendamos consultar en la ayuda de la función \texttt{help(qqPlot)}.

\hypertarget{pruebas-de-normalidad}{%
\section{Pruebas de normalidad}\label{pruebas-de-normalidad}}

Una forma menos subjetiva de explorar la normalidad de un conjunto de datos es por medio de las pruebas de normalidad. Las hipótesis para este tipo de pruebas son:

\begin{equation}
\begin{split}
&H_0: \text{la muestra proviene de una población normal.} \\
&H_A: \text{la muestra NO proviene de una población normal.}
\end{split}
\end{equation}

En la literatura estadística se reportan varias pruebas, algunas de ellas se listan a continuación.

\begin{enumerate}
\def\labelenumi{\arabic{enumi}.}
\tightlist
\item
  Prueba Shapiro-Wilk con la función \texttt{shapiro.test}.
\item
  Prueba Anderson-Darling con la función \texttt{ad.test} del paquete \textbf{nortest}.
\item
  Prueba Cramer-von Mises con la función \texttt{cvm.test} del paquete \textbf{nortest}.
\item
  Prueba Lilliefors (Kolmogorov-Smirnov) con la función \texttt{lillie.test} del paquete \textbf{nortest}.
\item
  Prueba Pearson chi-square con la función \texttt{pearson.test} del paquete \textbf{nortest}.
\item
  Prueba Shapiro-Francia con la función \texttt{sf.test} del paquete \textbf{nortest}).
\end{enumerate}

\hypertarget{ejemplo-con-datos-simulados}{%
\subsection*{Ejemplo con datos simulados}\label{ejemplo-con-datos-simulados}}
\addcontentsline{toc}{subsection}{Ejemplo con datos simulados}

Generar una muestra aleatoria con \(n=100\) de una \(N(150, 25)\) y aplicar las pruebas de normalidad Shapiro-Wilk y Anderson-Darling con un nivel de significancia del 3\%.

Lo primero es generar la muestra aleatoria \texttt{x} así:

\begin{Shaded}
\begin{Highlighting}[]
\NormalTok{x }\OtherTok{\textless{}{-}} \FunctionTok{rnorm}\NormalTok{(}\AttributeTok{n=}\DecValTok{100}\NormalTok{, }\AttributeTok{mean=}\DecValTok{150}\NormalTok{, }\AttributeTok{sd=}\DecValTok{5}\NormalTok{)}
\end{Highlighting}
\end{Shaded}

Para aplicar la prueba Shapiro-Wilk se usa la función \texttt{shapiro.test} al vector \texttt{x} así:

\begin{Shaded}
\begin{Highlighting}[]
\FunctionTok{shapiro.test}\NormalTok{(x)}
\end{Highlighting}
\end{Shaded}

\begin{verbatim}
## 
##  Shapiro-Wilk normality test
## 
## data:  x
## W = 0.98102, p-value = 0.1596
\end{verbatim}

De la salida anterior se tiene que el valor-P para la prueba fue de 0.2 y que es mayor al nivel de significancia 3\%, lo cual indica que no hay evidencias para rechazar la hipótesis nula de normalidad.

Para aplicar la prueba Anderson-Darling se usa la función \texttt{ad.test} al vector \texttt{x} así:

\begin{Shaded}
\begin{Highlighting}[]
\FunctionTok{require}\NormalTok{(nortest)  }\CommentTok{\# Se debe haber instalado nortest}
\FunctionTok{ad.test}\NormalTok{(x)}
\end{Highlighting}
\end{Shaded}

\begin{verbatim}
## 
##  Anderson-Darling normality test
## 
## data:  x
## A = 0.60684, p-value = 0.1119
\end{verbatim}

De la salida anterior se tiene que el valor-P para la prueba fue de 0.1 y que es mayor al nivel de significancia 3\%, esto indica que no hay evidencias para rechazar la hipótesis nula de normalidad.

\hypertarget{ejemplo-normalidad-para-peso-corporal}{%
\subsection*{Ejemplo normalidad para peso corporal}\label{ejemplo-normalidad-para-peso-corporal}}
\addcontentsline{toc}{subsection}{Ejemplo normalidad para peso corporal}

Retomando la base de datos \textbf{medidas del cuerpo} presentada en el Capítulo \ref{central}, se desea saber si el peso corporal, tanto de hombres y mujeres, tiene una distribución normal usando las pruebas normalidad Shapiro-Wilks y Anderson-Darling con un nivel de significancia del 5\%.

Lo primero es cargar la base de datos si aún no se ha cargado.

\begin{Shaded}
\begin{Highlighting}[]
\NormalTok{url }\OtherTok{\textless{}{-}} \StringTok{\textquotesingle{}https://raw.githubusercontent.com/fhernanb/datos/master/medidas\_cuerpo\textquotesingle{}}
\NormalTok{datos }\OtherTok{\textless{}{-}} \FunctionTok{read.table}\NormalTok{(}\AttributeTok{file=}\NormalTok{url, }\AttributeTok{header=}\NormalTok{T)}
\end{Highlighting}
\end{Shaded}

La variable \texttt{peso} del objeto \texttt{datos} contiene la información sobre el peso corporal de ambos sexos, debemos entonces partir o dividir esta información diferenciando entre hombres y mujeres, para esto usamos la función \texttt{split} de la siguiente forma.

\begin{Shaded}
\begin{Highlighting}[]
\NormalTok{pesos }\OtherTok{\textless{}{-}} \FunctionTok{split}\NormalTok{(datos}\SpecialCharTok{$}\NormalTok{peso, datos}\SpecialCharTok{$}\NormalTok{sexo)}
\NormalTok{pesos}
\end{Highlighting}
\end{Shaded}

\begin{verbatim}
## $Hombre
##  [1] 87.3 80.0 82.3 73.6 74.1 85.9 73.2 76.3 65.9 90.9 89.1 62.3 82.7 79.1 98.2
## [16] 84.1 83.2 83.2
## 
## $Mujer
##  [1] 51.6 59.0 49.2 63.0 53.6 59.0 47.6 69.8 66.8 75.2 55.2 54.2 62.5 42.0 50.0
## [16] 49.8 49.2 73.2
\end{verbatim}

Para aplicar la prueba Shapiro-Wilk se usa la función \texttt{shapiro.test}. Como el objeto \texttt{pesos} es una lista se debe usar la función \texttt{lapply} para aplicar \texttt{shapiro.test} a la lista, a continuación el código usado.

\begin{Shaded}
\begin{Highlighting}[]
\FunctionTok{lapply}\NormalTok{(pesos, shapiro.test)}
\end{Highlighting}
\end{Shaded}

\begin{verbatim}
## $Hombre
## 
##  Shapiro-Wilk normality test
## 
## data:  X[[i]]
## W = 0.97803, p-value = 0.9274
## 
## 
## $Mujer
## 
##  Shapiro-Wilk normality test
## 
## data:  X[[i]]
## W = 0.94709, p-value = 0.3812
\end{verbatim}

De la salida anterior se observa que ambos valores-P fueron mayores al nivel de significancia 5\%, por lo tanto, se puede concluir que ambas muestras provienen de poblaciones con distribución normal.

Para aplicar la prueba Anderson-Darling se usa la función \texttt{ad.test} del paquete \textbf{nortest}. Como el objeto \texttt{pesos} es una lista se debe usar la función \texttt{lapply} para aplicar \texttt{ad.test} a la lista, a continuación el código usado.

\begin{Shaded}
\begin{Highlighting}[]
\FunctionTok{require}\NormalTok{(nortest) }\CommentTok{\# Se debe haber instalado nortest}
\FunctionTok{lapply}\NormalTok{(pesos, shapiro.test)}
\end{Highlighting}
\end{Shaded}

\begin{verbatim}
## $Hombre
## 
##  Shapiro-Wilk normality test
## 
## data:  X[[i]]
## W = 0.97803, p-value = 0.9274
## 
## 
## $Mujer
## 
##  Shapiro-Wilk normality test
## 
## data:  X[[i]]
## W = 0.94709, p-value = 0.3812
\end{verbatim}

De la salida anterior se observa que ambos valores-P fueron mayores al nivel de significancia 5\%, por lo tanto, se puede concluir que ambas muestras provienen de poblaciones con distribución normal.

Al usar las pruebas Shapiro-Wilks y Anderson-Darling se concluye que no hay evidencias para pensar que los pesos corporales de hombres y mujeres no provienen de una población normal.

\hypertarget{ejercicios-7}{%
\section*{EJERCICIOS}\label{ejercicios-7}}
\addcontentsline{toc}{section}{EJERCICIOS}

\begin{enumerate}
\def\labelenumi{\arabic{enumi}.}
\tightlist
\item
  Para la base de datos \textbf{medidas del cuerpo} presentada en el Capítulo \ref{central}, explorar si la variable estatura, diferenciada por hombres y mujeres, tiene una distribución normal.
\end{enumerate}

\hypertarget{ic}{%
\chapter{Intervalos de confianza}\label{ic}}

En este capítulo se muestran las funciones que hay disponibles en R para construir intervalos de confianza para:

\begin{enumerate}
\def\labelenumi{\arabic{enumi}.}
\tightlist
\item
  la media \(\mu\),
\item
  la proporción \(p\),
\item
  la varianza \(\sigma^2\),
\item
  la diferencia de medias \(\mu_1-\mu_2\) para muestras independientes y dependientes (o pareadas),
\item
  la diferencia de proporciones \(p_1 - p_2\), y
\item
  la razón de varianzas \(\sigma_1^2 / \sigma_2^2\).
\end{enumerate}

Para ilustrar el uso de las funciones se utilizará la base de datos \textbf{medidas del cuerpo} presentada en el Capítulo \ref{central}.

\hypertarget{funciuxf3n-t.test}{%
\section{\texorpdfstring{Función \texttt{t.test}}{Función t.test}}\label{funciuxf3n-t.test}}

La función \texttt{t.test} se usa para calcular intervalos de confianza para la media y diferencia de medias, con muestras independientes y dependientes (o pareadas). La función y sus argumentos son los siguientes:

\begin{Shaded}
\begin{Highlighting}[]
\FunctionTok{t.test}\NormalTok{(x, }\AttributeTok{y =} \ConstantTok{NULL}\NormalTok{,}
       \AttributeTok{alternative =} \FunctionTok{c}\NormalTok{(}\StringTok{"two.sided"}\NormalTok{, }\StringTok{"less"}\NormalTok{, }\StringTok{"greater"}\NormalTok{),}
       \AttributeTok{mu =} \DecValTok{0}\NormalTok{, }\AttributeTok{paired =} \ConstantTok{FALSE}\NormalTok{, }\AttributeTok{var.equal =} \ConstantTok{FALSE}\NormalTok{,}
       \AttributeTok{conf.level =} \FloatTok{0.95}\NormalTok{, ...)}
\end{Highlighting}
\end{Shaded}

\hypertarget{intervalo-de-confianza-bilateral-para-la-media-mu}{%
\subsection{\texorpdfstring{Intervalo de confianza bilateral para la media \(\mu\)}{Intervalo de confianza bilateral para la media \textbackslash mu}}\label{intervalo-de-confianza-bilateral-para-la-media-mu}}

Para calcular intervalos de confianza bilaterales para la media a partir de la función \texttt{t.test} es necesario definir 2 argumentos:

\begin{itemize}
\tightlist
\item
  \texttt{x}: vector numérico con los datos.
\item
  \texttt{conf.level}: nivel de confianza a usar, por defecto es 0.95.
\end{itemize}

Los demás argumentos se usan cuando se desea obtener intervalos de confianza para diferencia de media con muestras independientes y dependientes (o pareadas).

\hypertarget{ejemplo-55}{%
\subsection*{Ejemplo}\label{ejemplo-55}}
\addcontentsline{toc}{subsection}{Ejemplo}

Suponga que se quiere obtener un intervalo de confianza bilateral del 90\% para la altura promedio de los hombres de la base de datos \textbf{medidas del cuerpo}.

Para calcular el intervalo de confianza, primero se carga la base de datos usando la url apropiada, luego se crea un subconjunto de \texttt{datos} y se aloja en el objeto \texttt{hombres} como sigue a continuación:

\begin{Shaded}
\begin{Highlighting}[]
\NormalTok{url }\OtherTok{\textless{}{-}} \StringTok{\textquotesingle{}https://raw.githubusercontent.com/fhernanb/datos/master/medidas\_cuerpo\textquotesingle{}}
\NormalTok{datos }\OtherTok{\textless{}{-}} \FunctionTok{read.table}\NormalTok{(}\AttributeTok{file=}\NormalTok{url, }\AttributeTok{header=}\NormalTok{T)}
\NormalTok{hombres }\OtherTok{\textless{}{-}}\NormalTok{ datos[datos}\SpecialCharTok{$}\NormalTok{sexo}\SpecialCharTok{==}\StringTok{"Hombre"}\NormalTok{, ]}
\end{Highlighting}
\end{Shaded}

Una vez leídos los datos, se analiza la normalidad de la variable altura de los hombres, a partir de un QQplot y un histograma

\begin{Shaded}
\begin{Highlighting}[]
\FunctionTok{par}\NormalTok{(}\AttributeTok{mfrow=}\FunctionTok{c}\NormalTok{(}\DecValTok{1}\NormalTok{, }\DecValTok{2}\NormalTok{))}
\FunctionTok{require}\NormalTok{(car)  }\CommentTok{\# Debe instalar antes el paquete car}
\FunctionTok{qqPlot}\NormalTok{(hombres}\SpecialCharTok{$}\NormalTok{altura, }\AttributeTok{pch=}\DecValTok{19}\NormalTok{,}
       \AttributeTok{main=}\StringTok{\textquotesingle{}QQplot para la altura de hombres\textquotesingle{}}\NormalTok{,}
       \AttributeTok{xlab=}\StringTok{\textquotesingle{}Cuantiles teóricos\textquotesingle{}}\NormalTok{,}
       \AttributeTok{ylab=}\StringTok{\textquotesingle{}Cuantiles muestrales\textquotesingle{}}\NormalTok{)}
\end{Highlighting}
\end{Shaded}

\begin{verbatim}
## [1] 5 8
\end{verbatim}

\begin{Shaded}
\begin{Highlighting}[]
\FunctionTok{hist}\NormalTok{(hombres}\SpecialCharTok{$}\NormalTok{altura, }\AttributeTok{freq=}\ConstantTok{TRUE}\NormalTok{,}
     \AttributeTok{main=}\StringTok{\textquotesingle{}Histograma para la altura de hombres\textquotesingle{}}\NormalTok{,}
     \AttributeTok{xlab=}\StringTok{\textquotesingle{}Altura (cm)\textquotesingle{}}\NormalTok{,}
     \AttributeTok{ylab=}\StringTok{\textquotesingle{}Frecuencia\textquotesingle{}}\NormalTok{)}
\end{Highlighting}
\end{Shaded}

\begin{figure}
\centering
\includegraphics{Manual_de_R_files/figure-latex/ic1-1.pdf}
\caption{\label{fig:ic1}QQplot e histograma para la altura de los hombres.}
\end{figure}

En la Figura \ref{fig:ic1} se muestra el QQplot e histograma para la variable altura, de estas figuras no se observa un claro patrón normal, sin embargo, al aplicar la prueba Shapiro-Wilk a la muestra de alturas de los hombres se obtuvo un valor-P de 0.3599, por lo tanto, se asume que la muestra de alturas provienen de una población normal.

Una vez chequeado el supuesto de normalidad se puede usar la función \texttt{t.test} sobre la variable de interés para construir el intervalo de confianza. El resultado de usar \texttt{t.test} es una lista, uno de los elementos de esa lista es justamente el intevalo de confianza y para extraerlo es que se usa \texttt{\$conf.int} al final de la instrucción. A continuación se muestra el código utilizado.

\begin{Shaded}
\begin{Highlighting}[]
\FunctionTok{t.test}\NormalTok{(}\AttributeTok{x=}\NormalTok{hombres}\SpecialCharTok{$}\NormalTok{altura, }\AttributeTok{conf.level=}\FloatTok{0.90}\NormalTok{)}\SpecialCharTok{$}\NormalTok{conf.int}
\end{Highlighting}
\end{Shaded}

\begin{verbatim}
## [1] 176.4384 181.7172
## attr(,"conf.level")
## [1] 0.9
\end{verbatim}

A partir del resultado obtenido se puede concluir, con un nivel de confianza del 90\%, que la altura promedio de los estudiantes hombres se encuentra entre 176.4 cm y 181.7 cm.

\hypertarget{intervalo-de-confianza-bilateral-para-la-diferencia-de-medias-mu_1-mu_2-de-muestras-independientes}{%
\subsection{\texorpdfstring{Intervalo de confianza bilateral para la diferencia de medias (\(\mu_1-\mu_2\)) de muestras independientes}{Intervalo de confianza bilateral para la diferencia de medias (\textbackslash mu\_1-\textbackslash mu\_2) de muestras independientes}}\label{intervalo-de-confianza-bilateral-para-la-diferencia-de-medias-mu_1-mu_2-de-muestras-independientes}}

Para construir intervalos de confianza bilaterales para la diferencia de medias (\(\mu_1-\mu_2\)) de muestras independientes se usa la función \texttt{t.test} y es necesario definir 5 argumentos:

\begin{itemize}
\tightlist
\item
  \texttt{x}: vector numérico con la información de la muestra 1,
\item
  \texttt{y}: vector numérico con la información de la muestra 2,
\item
  \texttt{paired=FALSE}: indica que el intervalo de confianza se hará para muestras independientes, en el caso de que sean dependientes (o pareadas) este argumento será \texttt{paired=TRUE},
\item
  \texttt{var.equal=FALSE}: indica que las varianzas son desconocidas y diferentes, si la varianzas se pueden considerar iguales se coloca \texttt{var.equal=TRUE}.
\item
  \texttt{conf.level}: nivel de confianza.
\end{itemize}

\hypertarget{ejemplo-56}{%
\subsection*{Ejemplo}\label{ejemplo-56}}
\addcontentsline{toc}{subsection}{Ejemplo}

Se quiere saber si existe diferencia estadísticamente significativa entre las alturas de los hombres y las mujeres. Para responder esto se va a construir un intervalo de confianza del \(95\%\) para la diferencia de las altura promedio de los hombres y de las mujeres (\(\mu_{hombres}-\mu_{mujeres}\)).

Para construir el intervalo de confianza, primero se carga la base de datos usando la url apropiada, luego se crean dos subconjuntos de datos y se alojan en los objetos \texttt{hombres} y \texttt{mujeres} como sigue a continuación:

\begin{Shaded}
\begin{Highlighting}[]
\NormalTok{url }\OtherTok{\textless{}{-}} \StringTok{\textquotesingle{}https://raw.githubusercontent.com/fhernanb/datos/master/medidas\_cuerpo\textquotesingle{}}
\NormalTok{datos }\OtherTok{\textless{}{-}} \FunctionTok{read.table}\NormalTok{(}\AttributeTok{file=}\NormalTok{url, }\AttributeTok{header=}\NormalTok{T)}
\NormalTok{hombres }\OtherTok{\textless{}{-}}\NormalTok{ datos[datos}\SpecialCharTok{$}\NormalTok{sexo}\SpecialCharTok{==}\StringTok{"Hombre"}\NormalTok{, ]}
\NormalTok{mujeres }\OtherTok{\textless{}{-}}\NormalTok{ datos[datos}\SpecialCharTok{$}\NormalTok{sexo}\SpecialCharTok{==}\StringTok{"Mujer"}\NormalTok{, ]}
\end{Highlighting}
\end{Shaded}

Una vez leídos los datos, se analiza la normalidad de la variable altura de los hombres y las mujeres, a partir de un QQplot y un histograma

\begin{Shaded}
\begin{Highlighting}[]
\FunctionTok{par}\NormalTok{(}\AttributeTok{mfrow=}\FunctionTok{c}\NormalTok{(}\DecValTok{2}\NormalTok{,}\DecValTok{2}\NormalTok{))}
\FunctionTok{require}\NormalTok{(car)  }\CommentTok{\# Debe instalar antes el paquete car}
\FunctionTok{qqPlot}\NormalTok{(hombres}\SpecialCharTok{$}\NormalTok{altura, }\AttributeTok{pch=}\DecValTok{19}\NormalTok{, }\AttributeTok{las=}\DecValTok{1}\NormalTok{, }\AttributeTok{main=}\StringTok{\textquotesingle{}QQplot\textquotesingle{}}\NormalTok{,}
       \AttributeTok{xlab=}\StringTok{\textquotesingle{}Cuantiles teóricos\textquotesingle{}}\NormalTok{, }\AttributeTok{ylab=}\StringTok{\textquotesingle{}Cuantiles muestrales\textquotesingle{}}\NormalTok{)}
\end{Highlighting}
\end{Shaded}

\begin{verbatim}
## [1] 5 8
\end{verbatim}

\begin{Shaded}
\begin{Highlighting}[]
\FunctionTok{hist}\NormalTok{(hombres}\SpecialCharTok{$}\NormalTok{altura, }\AttributeTok{las=}\DecValTok{1}\NormalTok{, }\AttributeTok{xlab=}\StringTok{\textquotesingle{}Altura\textquotesingle{}}\NormalTok{, }\AttributeTok{ylab=}\StringTok{\textquotesingle{}Frecuencia\textquotesingle{}}\NormalTok{,}
     \AttributeTok{main=}\StringTok{\textquotesingle{}Histograma para la altura de hombres\textquotesingle{}}\NormalTok{)}

\FunctionTok{qqPlot}\NormalTok{(mujeres}\SpecialCharTok{$}\NormalTok{altura, }\AttributeTok{pch=}\DecValTok{19}\NormalTok{, }\AttributeTok{las=}\DecValTok{1}\NormalTok{, }\AttributeTok{main=}\StringTok{\textquotesingle{}QQplot\textquotesingle{}}\NormalTok{,}
       \AttributeTok{xlab=}\StringTok{\textquotesingle{}Cuantiles teóricos\textquotesingle{}}\NormalTok{, }\AttributeTok{ylab=}\StringTok{\textquotesingle{}Cuantiles muestrales\textquotesingle{}}\NormalTok{)}
\end{Highlighting}
\end{Shaded}

\begin{verbatim}
## [1] 16  9
\end{verbatim}

\begin{Shaded}
\begin{Highlighting}[]
\FunctionTok{hist}\NormalTok{(mujeres}\SpecialCharTok{$}\NormalTok{altura, }\AttributeTok{las=}\DecValTok{1}\NormalTok{, }\AttributeTok{xlab=}\StringTok{\textquotesingle{}Altura\textquotesingle{}}\NormalTok{, }\AttributeTok{ylab=}\StringTok{\textquotesingle{}Frecuencia\textquotesingle{}}\NormalTok{,}
     \AttributeTok{main=}\StringTok{\textquotesingle{}Histograma para la altura de mujeres\textquotesingle{}}\NormalTok{)}
\end{Highlighting}
\end{Shaded}

\begin{figure}
\centering
\includegraphics{Manual_de_R_files/figure-latex/ic2-1.pdf}
\caption{\label{fig:ic2}QQplot e histograma para la altura de hombres y mujeres.}
\end{figure}

De la Figura \ref{fig:ic2} se puede concluir que las alturas de los estudiantes hombres y mujeres siguen una distribución normal. Al aplicar la prueba Shapiro-Wilk para estudiar la normalidad de la altura se encontró un valor-P de 0.3599 para el grupo de hombres y un valor-P de 0.5921 para el grupo de mujeres, esto confirma que se cumple el supuesto de normalidad.

Como se cumple el supuesto de normalidad se puede usar la función \texttt{t.test} para construir el intervalo de confianza requerido. A continuación se muestra el código

\begin{Shaded}
\begin{Highlighting}[]
\FunctionTok{t.test}\NormalTok{(}\AttributeTok{x=}\NormalTok{hombres}\SpecialCharTok{$}\NormalTok{altura, }\AttributeTok{y=}\NormalTok{mujeres}\SpecialCharTok{$}\NormalTok{altura,}
       \AttributeTok{paired=}\ConstantTok{FALSE}\NormalTok{, }\AttributeTok{var.equal=}\ConstantTok{FALSE}\NormalTok{,}
       \AttributeTok{conf.level =} \FloatTok{0.95}\NormalTok{)}\SpecialCharTok{$}\NormalTok{conf.int}
\end{Highlighting}
\end{Shaded}

\begin{verbatim}
## [1] 10.05574 20.03315
## attr(,"conf.level")
## [1] 0.95
\end{verbatim}

A partir del intervalo de confianza anterior se puede concluir, con un nivel de confianza del \(95\%\), que la altura promedio de los hombres es superior a la altura promedio de las mujeres, ya que el intervalo de confianza \textbf{NO} incluye el cero y por ser positivos sus limites se puede afirmar con un nivel de confianza del \(95\%\) que \(\mu_{hombres} > \mu_{mujeres}\).

\hypertarget{intervalo-de-confianza-bilateral-para-la-diferencia-de-medias-mu_1-mu_2-de-muestras-dependientes-o-pareadas}{%
\subsection{\texorpdfstring{Intervalo de confianza bilateral para la diferencia de medias (\(\mu_1-\mu_2\)) de muestras dependientes o pareadas}{Intervalo de confianza bilateral para la diferencia de medias (\textbackslash mu\_1-\textbackslash mu\_2) de muestras dependientes o pareadas}}\label{intervalo-de-confianza-bilateral-para-la-diferencia-de-medias-mu_1-mu_2-de-muestras-dependientes-o-pareadas}}

Para construir intervalos de confianza bilaterales para la diferencia de medias de muestras dependientes a partir de la función \texttt{t.test} es necesario definir 4 argumentos:

\begin{itemize}
\tightlist
\item
  \texttt{x}: vector numérico con la información de la muestra 1,
\item
  \texttt{y}: vector numérico con la información de la muestra 2,
  \texttt{paired=TRUE} indica que el intervalo de confianza se hará para muestras dependientes o pareadas.
\item
  \texttt{conf.level}: nivel de confianza.
\end{itemize}

\hypertarget{ejemplo-57}{%
\subsection*{Ejemplo}\label{ejemplo-57}}
\addcontentsline{toc}{subsection}{Ejemplo}

Los desórdenes musculoesqueléticos del cuello y hombro son comunes entre empleados de oficina que realizan tareas repetitivas mediante pantallas de visualización. Se reportaron los datos de un estudio para determinar si condiciones de trabajo más variadas habrían tenido algún impacto en el movimiento del brazo. Los datos que siguen se obtuvieron de una muestra de \(n=16\) sujetos. Cada observación es la cantidad de tiempo, expresada como una proporción de tiempo total observado, durante el cual la elevación del brazo fue de menos de 30 grados. Las dos mediciones de cada sujeto se obtuvieron con una separación de 18 meses. Durante este período, las condiciones de trabajo cambiaron y se permitió que los sujetos realizaran una variedad más amplia de tareas. ¿Sugieren los datos que el tiempo promedio verdadero durante el cual la elevación es de menos de 30 grados luego del cambio difiere de lo que era antes? Calcular un intervalo de confianza del \(95\%\) para responder la pregunta.

\begin{longtable}[]{@{}lllllllll@{}}
\toprule
Sujeto & 1 & 2 & 3 & 4 & 5 & 6 & 7 & 8 \\
\midrule
\endhead
Antes & 81 & 87 & 86 & 82 & 90 & 86 & 96 & 73 \\
Después & 78 & 91 & 78 & 78 & 84 & 67 & 92 & 70 \\
Diferencia & 3 & -4 & 8 & 4 & 6 & 19 & 4 & 3 \\
\bottomrule
\end{longtable}

\begin{longtable}[]{@{}lllllllll@{}}
\toprule
Sujeto & 9 & 10 & 11 & 12 & 13 & 14 & 15 & 16 \\
\midrule
\endhead
Antes & 74 & 75 & 72 & 80 & 66 & 72 & 56 & 82 \\
Después & 58 & 62 & 70 & 58 & 66 & 60 & 65 & 73 \\
Diferencia & 16 & 13 & 2 & 22 & 0 & 12 & -9 & 9 \\
\bottomrule
\end{longtable}

Para construir el intervalo de confianza primero se crean dos vectores con los datos y se nombran \texttt{Antes} y \texttt{Despues}, luego se calcula la diferencia y se aloja en el vector \texttt{Diferencia}, como sigue a continuación:

\begin{Shaded}
\begin{Highlighting}[]
\NormalTok{Antes   }\OtherTok{\textless{}{-}} \FunctionTok{c}\NormalTok{(}\DecValTok{81}\NormalTok{, }\DecValTok{87}\NormalTok{, }\DecValTok{86}\NormalTok{, }\DecValTok{82}\NormalTok{, }\DecValTok{90}\NormalTok{, }\DecValTok{86}\NormalTok{, }\DecValTok{96}\NormalTok{, }\DecValTok{73}\NormalTok{,}
             \DecValTok{74}\NormalTok{, }\DecValTok{75}\NormalTok{, }\DecValTok{72}\NormalTok{, }\DecValTok{80}\NormalTok{, }\DecValTok{66}\NormalTok{, }\DecValTok{72}\NormalTok{, }\DecValTok{56}\NormalTok{, }\DecValTok{82}\NormalTok{)}
\NormalTok{Despues }\OtherTok{\textless{}{-}} \FunctionTok{c}\NormalTok{(}\DecValTok{78}\NormalTok{, }\DecValTok{91}\NormalTok{, }\DecValTok{78}\NormalTok{, }\DecValTok{78}\NormalTok{, }\DecValTok{84}\NormalTok{, }\DecValTok{67}\NormalTok{, }\DecValTok{92}\NormalTok{, }\DecValTok{70}\NormalTok{,}
             \DecValTok{58}\NormalTok{, }\DecValTok{62}\NormalTok{, }\DecValTok{70}\NormalTok{, }\DecValTok{58}\NormalTok{, }\DecValTok{66}\NormalTok{, }\DecValTok{60}\NormalTok{, }\DecValTok{65}\NormalTok{, }\DecValTok{73}\NormalTok{)}
\NormalTok{Diferencia }\OtherTok{\textless{}{-}}\NormalTok{ Antes }\SpecialCharTok{{-}}\NormalTok{ Despues}
\end{Highlighting}
\end{Shaded}

En seguida se analiza la normalidad de la variable \texttt{Diferencia} de los cambios en las condiciones de trabajo, a partir de un qqplot y una densidad.

\begin{Shaded}
\begin{Highlighting}[]
\FunctionTok{par}\NormalTok{(}\AttributeTok{mfrow=}\FunctionTok{c}\NormalTok{(}\DecValTok{1}\NormalTok{,}\DecValTok{2}\NormalTok{))}
\FunctionTok{require}\NormalTok{(car)}
\FunctionTok{qqPlot}\NormalTok{(Diferencia, }\AttributeTok{pch=}\DecValTok{19}\NormalTok{, }\AttributeTok{main=}\StringTok{\textquotesingle{}QQplot para Diferencias\textquotesingle{}}\NormalTok{, }\AttributeTok{las=}\DecValTok{1}\NormalTok{, }
       \AttributeTok{xlab=}\StringTok{\textquotesingle{}Cuantiles teóricos\textquotesingle{}}\NormalTok{, }\AttributeTok{ylab=}\StringTok{\textquotesingle{}Cuantiles muestrales\textquotesingle{}}\NormalTok{)}
\end{Highlighting}
\end{Shaded}

\begin{verbatim}
## [1] 15 12
\end{verbatim}

\begin{Shaded}
\begin{Highlighting}[]
\FunctionTok{plot}\NormalTok{(}\FunctionTok{density}\NormalTok{(Diferencia), }\AttributeTok{main=}\StringTok{\textquotesingle{}Densidad para Diferencias\textquotesingle{}}\NormalTok{, }\AttributeTok{las=}\DecValTok{1}\NormalTok{,}
     \AttributeTok{xlab=}\StringTok{\textquotesingle{}Diferencia de tiempo\textquotesingle{}}\NormalTok{, }\AttributeTok{ylab=}\StringTok{\textquotesingle{}Densidad\textquotesingle{}}\NormalTok{)}
\end{Highlighting}
\end{Shaded}

\begin{figure}
\centering
\includegraphics{Manual_de_R_files/figure-latex/ic3-1.pdf}
\caption{\label{fig:ic3}QQplot y densidad para Diferencias.}
\end{figure}

De la Figura \ref{fig:ic3} se observa que la diferencia de los tiempos sigue una distribución normal, debido a que en el QQplot se observa un patron lineal y la densidad muestra una forma cercana a la simétrica.

Luego de chequear la normalidad de la variable \texttt{Diferencia} se usa la función \texttt{t.test} para construir el intervalo. A continuación se muestra el código utilizado.

\begin{Shaded}
\begin{Highlighting}[]
\FunctionTok{t.test}\NormalTok{(}\AttributeTok{x=}\NormalTok{Antes, }\AttributeTok{y=}\NormalTok{Despues, }\AttributeTok{paired=}\ConstantTok{TRUE}\NormalTok{, }\AttributeTok{conf.level=}\FloatTok{0.95}\NormalTok{)}\SpecialCharTok{$}\NormalTok{conf.int}
\end{Highlighting}
\end{Shaded}

\begin{verbatim}
## [1]  2.362371 11.137629
## attr(,"conf.level")
## [1] 0.95
\end{verbatim}

A partir del resultado obtenido se puede concluir con un nivel de confianza del \(95\%\), que el tiempo promedio verdadero durante el cual la elevación es de menos de 30 grados luego del cambio difiere de lo que era antes del mismo. Como el intervalo de confianza es \(2.362< \mu_D < 11.138\), esto indica que \(\mu_{antes} - \mu_{despues}>0\) y por lo tanto \(\mu_{antes} > \mu_{despues}\).

\hypertarget{intervalo-de-confianza-unilateral-para-la-media-mu}{%
\subsection{\texorpdfstring{Intervalo de confianza unilateral para la media \(\mu\)}{Intervalo de confianza unilateral para la media \textbackslash mu}}\label{intervalo-de-confianza-unilateral-para-la-media-mu}}

Para construir intervalos de confianza unilaterales se usa el argumento \texttt{alternative\ =\ \textquotesingle{}less\textquotesingle{}} o \texttt{alternative=\textquotesingle{}greater\textquotesingle{}}, a continuación un ejemplo.

\hypertarget{ejemplo-58}{%
\subsection*{Ejemplo}\label{ejemplo-58}}
\addcontentsline{toc}{subsection}{Ejemplo}

Simule una muestra aleatoria de una \(N(18, 3)\) y calcule un intervalo de confianza unilateral superior del \(90\%\) para la media

\begin{Shaded}
\begin{Highlighting}[]
\NormalTok{x }\OtherTok{\textless{}{-}} \FunctionTok{rnorm}\NormalTok{(}\DecValTok{50}\NormalTok{, }\AttributeTok{mean =} \DecValTok{18}\NormalTok{, }\AttributeTok{sd =}\DecValTok{3}\NormalTok{)}
\FunctionTok{t.test}\NormalTok{(x, }\AttributeTok{alternative =} \StringTok{"greater"}\NormalTok{, }\AttributeTok{conf.level =} \FloatTok{0.90}\NormalTok{)}\SpecialCharTok{$}\NormalTok{conf.int}
\end{Highlighting}
\end{Shaded}

\begin{verbatim}
## [1] 17.58728      Inf
## attr(,"conf.level")
## [1] 0.9
\end{verbatim}

En el resultado anterior se muestra el intervalo de confianza unilateral.

\hypertarget{funciuxf3n-var.test}{%
\section{\texorpdfstring{Función \texttt{var.test}}{Función var.test}}\label{funciuxf3n-var.test}}

Para construir intervalos de confianza para la varianza se usa la función \texttt{var.test} del paquete \textbf{stests} \citep{R-usefultools} disponible en el repositorio \href{https://github.com/}{GitHub}.

Para instalar el paquete \textbf{stests} desde GitHub se debe copiar el siguiente código en la consola de R:

\begin{Shaded}
\begin{Highlighting}[]
\ControlFlowTok{if}\NormalTok{ (}\SpecialCharTok{!}\FunctionTok{require}\NormalTok{(}\StringTok{\textquotesingle{}devtools\textquotesingle{}}\NormalTok{)) }\FunctionTok{install.packages}\NormalTok{(}\StringTok{\textquotesingle{}devtools\textquotesingle{}}\NormalTok{)}
\NormalTok{devtools}\SpecialCharTok{::}\FunctionTok{install\_github}\NormalTok{(}\StringTok{\textquotesingle{}fhernanb/stests\textquotesingle{}}\NormalTok{, }\AttributeTok{force=}\ConstantTok{TRUE}\NormalTok{)}
\end{Highlighting}
\end{Shaded}

Una vez instalado el paquete \textbf{stests} se puede usar la función \texttt{var.test}, la cual es una generalización de la función \texttt{var.test} del paquete \textbf{stats} y por esa razón aparece el siguiente mensaje en la consola cuando se invoca el paquete.

\begin{Shaded}
\begin{Highlighting}[]
\FunctionTok{library}\NormalTok{(stests)}
\DocumentationTok{\#\# }
\DocumentationTok{\#\# The following object is masked from ‘package:stats’:}
\DocumentationTok{\#\# }
\DocumentationTok{\#\#     var.test}
\end{Highlighting}
\end{Shaded}

Si usted desea usar \texttt{var.test} del paquete \textbf{stats} o \texttt{var.test} del paquete \textbf{stests}, puede invocar las funciones explícitamente así:

\begin{Shaded}
\begin{Highlighting}[]
\NormalTok{stats}\SpecialCharTok{::}\FunctionTok{var.test}\NormalTok{()   }\CommentTok{\# Para usar la fución del paquete stats}
\NormalTok{stests}\SpecialCharTok{::}\FunctionTok{var.test}\NormalTok{()  }\CommentTok{\# Para usar la fución del paquete stests}
\end{Highlighting}
\end{Shaded}

\begin{rmdnote}
Recuerde que:
- \texttt{stats::var.test()} sólo sirve para 1 población.
- \texttt{stests::var.test()} sirve para 1 o 2 poblaciones.
\end{rmdnote}

\hypertarget{intervalo-de-confianza-bilateral-para-la-varianza-sigma2}{%
\subsection{\texorpdfstring{Intervalo de confianza bilateral para la varianza \(\sigma^2\)}{Intervalo de confianza bilateral para la varianza \textbackslash sigma\^{}2}}\label{intervalo-de-confianza-bilateral-para-la-varianza-sigma2}}

Para calcular intervalos de confianza bilaterales para la varianza \(\sigma^2\) a partir de la función \texttt{var.test} es necesario definir 2 argumentos:

\begin{itemize}
\tightlist
\item
  \texttt{x}: vector numérico con la información de la muestra,
\item
  \texttt{conf.level}: nivel de confianza.
\end{itemize}

\hypertarget{ejemplo-59}{%
\subsection*{Ejemplo}\label{ejemplo-59}}
\addcontentsline{toc}{subsection}{Ejemplo}

Considerando la información del ejemplo de Intervalos de confianza bilaterales para la media, construir un intervalo de confianza del 98\% para la varianza de la altura de los estudiantes hombres.

\begin{Shaded}
\begin{Highlighting}[]
\FunctionTok{require}\NormalTok{(stests)  }\CommentTok{\# Para cargar el paquete}
\NormalTok{res }\OtherTok{\textless{}{-}}\NormalTok{ stests}\SpecialCharTok{::}\FunctionTok{var.test}\NormalTok{(}\AttributeTok{x=}\NormalTok{hombres}\SpecialCharTok{$}\NormalTok{altura, }\AttributeTok{conf.level=}\FloatTok{0.98}\NormalTok{)}
\NormalTok{res}\SpecialCharTok{$}\NormalTok{conf.int}
\end{Highlighting}
\end{Shaded}

\begin{verbatim}
## [1]  21.08468 109.93095
## attr(,"conf.level")
## [1] 0.98
\end{verbatim}

El intervalo de confianza del \(98\%\) indica que la varianza de la estatura de los estudiantes hombres se encuentra entre 21.08 y 109.93 \(cm^{2}\).

\hypertarget{intervalo-de-confianza-bilateral-para-la-razuxf3n-de-varianzas-sigma_12-sigma_22}{%
\subsection{\texorpdfstring{Intervalo de confianza bilateral para la razón de varianzas \(\sigma_1^2 / \sigma_2^2\)}{Intervalo de confianza bilateral para la razón de varianzas \textbackslash sigma\_1\^{}2 / \textbackslash sigma\_2\^{}2}}\label{intervalo-de-confianza-bilateral-para-la-razuxf3n-de-varianzas-sigma_12-sigma_22}}

Para calcular intervalos de confianza bilaterales para la razón de varianzas a partir de la función \texttt{var.test} es necesario definir 3 argumentos:

\begin{itemize}
\tightlist
\item
  \texttt{x}: vector numérico con la información de la muestra 1,
\item
  \texttt{y}: vector numérico con la información de la muestra 2,
\item
  \texttt{conf.level}: nivel de confianza.
\end{itemize}

\hypertarget{ejemplo-60}{%
\subsection*{Ejemplo}\label{ejemplo-60}}
\addcontentsline{toc}{subsection}{Ejemplo}

Usando la información del ejemplo de diferencia de medias para muestras independientes se quiere obtener un intervalo de confianza del \(95\%\) para la razón de las varianzas de las alturas de los estudiantes hombres y mujeres.

\begin{Shaded}
\begin{Highlighting}[]
\FunctionTok{var.test}\NormalTok{(}\AttributeTok{x=}\NormalTok{hombres}\SpecialCharTok{$}\NormalTok{altura, }\AttributeTok{y=}\NormalTok{mujeres}\SpecialCharTok{$}\NormalTok{altura,}
         \AttributeTok{conf.level=}\FloatTok{0.95}\NormalTok{)}\SpecialCharTok{$}\NormalTok{conf.int}
\end{Highlighting}
\end{Shaded}

\begin{verbatim}
## [1] 0.2327398 1.6632830
## attr(,"conf.level")
## [1] 0.95
\end{verbatim}

El intervalo de confianza del \(95\%\) indica que la razón de varianzas se encuentra entre 0.2327 y 1.6633. Puesto que el intervalo de confianza incluye el 1 se concluye que las varianzas de las alturas de los hombres y las mujeres son iguales.

\begin{rmdwarning}
¿Notó que las funciones \texttt{var.test} y \texttt{var.test} son diferentes?

\texttt{var.test} sirve para construir IC para \(\sigma^2\).

\texttt{var.test} sirve para construir IC para \(\sigma_1^2 / \sigma_2^2\).
\end{rmdwarning}

\hypertarget{funciuxf3n-prop.test}{%
\section{\texorpdfstring{Función \texttt{prop.test}}{Función prop.test}}\label{funciuxf3n-prop.test}}

La función \texttt{prop.test} se usa para calcular intervalos de confianza para la porporción y diferencia de proporciones. La función y sus argumentos son los siguientes:

\begin{Shaded}
\begin{Highlighting}[]
\FunctionTok{prop.test}\NormalTok{(x, n, }\AttributeTok{p=}\ConstantTok{NULL}\NormalTok{,}
          \AttributeTok{alternative=}\FunctionTok{c}\NormalTok{(}\StringTok{"two.sided"}\NormalTok{, }\StringTok{"less"}\NormalTok{, }\StringTok{"greater"}\NormalTok{),}
          \AttributeTok{conf.level=}\FloatTok{0.95}\NormalTok{, }\AttributeTok{correct=}\ConstantTok{TRUE}\NormalTok{)}
\end{Highlighting}
\end{Shaded}

\hypertarget{intervalo-de-confianza-bilateral-para-la-proporciuxf3n-p}{%
\subsection{\texorpdfstring{Intervalo de confianza bilateral para la proporción \(p\)}{Intervalo de confianza bilateral para la proporción p}}\label{intervalo-de-confianza-bilateral-para-la-proporciuxf3n-p}}

Para calcular intervalos de confianza bilaterales para la proporción a partir de la función \texttt{prop.test} es necesario definir 3 argumentos: \texttt{x} considera el conteo de éxitos, \texttt{n} indica el número de eventos o de forma equivalente corresponde a la longitud de la variable que se quiere analizar, y \texttt{conf.level} corresponde al nivel de confianza.

\hypertarget{ejemplo-61}{%
\subsection*{Ejemplo}\label{ejemplo-61}}
\addcontentsline{toc}{subsection}{Ejemplo}

El gerente de una estación de televisión debe determinar en la ciudad qué porcentaje de casas tienen más de un televisor. Una muestra aleatoria de 500 casas revela que 275 tienen dos o más televisores. ¿Cuál es el intervalo de confianza del 90\% para estimar la proporción de todas las casas que tienen dos o más televisores?

\begin{Shaded}
\begin{Highlighting}[]
\FunctionTok{prop.test}\NormalTok{(}\AttributeTok{x=}\DecValTok{275}\NormalTok{, }\AttributeTok{n=}\DecValTok{500}\NormalTok{, }\AttributeTok{conf.level=}\FloatTok{0.90}\NormalTok{)}\SpecialCharTok{$}\NormalTok{conf.int}
\end{Highlighting}
\end{Shaded}

\begin{verbatim}
## [1] 0.5122310 0.5872162
## attr(,"conf.level")
## [1] 0.9
\end{verbatim}

A partir del resultado obtenido se puede concluir, con un nivel de confianza del \(90\%\), que la proporción \(p\) de casas que tienen dos o más televisores se encuentra entre 0.5122 y 0.5872.

\hypertarget{intervalo-de-confianza-bilateral-para-la-diferencia-de-proporciones-p_1---p_2}{%
\subsection{\texorpdfstring{Intervalo de confianza bilateral para la diferencia de proporciones \(p_1 - p_2\)}{Intervalo de confianza bilateral para la diferencia de proporciones p\_1 - p\_2}}\label{intervalo-de-confianza-bilateral-para-la-diferencia-de-proporciones-p_1---p_2}}

Para construir intervalos de confianza bilaterales para la proporción a partir de la función \texttt{prop.test} es necesario definir 3 argumentos:

\begin{itemize}
\tightlist
\item
  \texttt{x}: vector con el conteo de éxitos de las dos muestras,
\item
  \texttt{n}: vector con el número de ensayos,
\item
  \texttt{conf.level}: nivel de confianza.
\end{itemize}

\hypertarget{ejemplo-62}{%
\subsection*{Ejemplo}\label{ejemplo-62}}
\addcontentsline{toc}{subsection}{Ejemplo}

Se quiere determinar si un cambio en el método de fabricación de una piezas ha sido efectivo o no. Para esta comparación se tomaron 2 muestras, una antes y otra después del cambio en el proceso y los resultados obtenidos son los siguientes.

\begin{longtable}[]{@{}lll@{}}
\toprule
Num piezas & Antes & Después \\
\midrule
\endhead
Defectuosas & 75 & 80 \\
Analizadas & 1500 & 2000 \\
\bottomrule
\end{longtable}

Construir un intervalo de confianza del 90\% para decidir si el cambio tuvo efecto positivo o no.

\begin{Shaded}
\begin{Highlighting}[]
\FunctionTok{prop.test}\NormalTok{(}\AttributeTok{x=}\FunctionTok{c}\NormalTok{(}\DecValTok{75}\NormalTok{, }\DecValTok{80}\NormalTok{), }\AttributeTok{n=}\FunctionTok{c}\NormalTok{(}\DecValTok{1500}\NormalTok{, }\DecValTok{2000}\NormalTok{), }\AttributeTok{conf.level=}\FloatTok{0.90}\NormalTok{)}\SpecialCharTok{$}\NormalTok{conf.int}
\end{Highlighting}
\end{Shaded}

\begin{verbatim}
## [1] -0.002314573  0.022314573
## attr(,"conf.level")
## [1] 0.9
\end{verbatim}

A partir del resultado obtenido se puede concluir, con un nivel de confianza del \(90\%\), que la diferencia de proporción de defectos (\(p_1 - p_2\)) se encuentra entre -0.002315 y 0.022315. Como el cero está dentro del intervalo se concluye que el cambio en el método de fabricación no ha disminuído el porcentaje de defectos.

\hypertarget{ph}{%
\chapter{Prueba de hipótesis}\label{ph}}

En este capítulo se muestran las funciones que hay disponibles en R para realizar prueba de hipótesis para:

\begin{enumerate}
\def\labelenumi{\arabic{enumi}.}
\tightlist
\item
  la media \(\mu\),
\item
  la proporción \(p\),
\item
  la varianza \(\sigma^2\),
\item
  la diferencia de medias \(\mu_1-\mu_2\) para muestras independientes y dependientes (o pareadas),
\item
  la diferencia de proporciones \(p_1 - p_2\), y
\item
  la razón de varianzas \(\sigma_1^2 / \sigma_2^2\).
\end{enumerate}

\hypertarget{prueba-de-hipuxf3tesis-para-mu-de-una-poblaciuxf3n-normal}{%
\section{\texorpdfstring{Prueba de hipótesis para \(\mu\) de una población normal}{Prueba de hipótesis para \textbackslash mu de una población normal}}\label{prueba-de-hipuxf3tesis-para-mu-de-una-poblaciuxf3n-normal}}

Para realizar este tipo de prueba se puede usar la función \texttt{t.test} que tiene la siguiente estructura.

\begin{Shaded}
\begin{Highlighting}[]
\FunctionTok{t.test}\NormalTok{(x, }\AttributeTok{y =} \ConstantTok{NULL}\NormalTok{,}
       \AttributeTok{alternative =} \FunctionTok{c}\NormalTok{(}\StringTok{"two.sided"}\NormalTok{, }\StringTok{"less"}\NormalTok{, }\StringTok{"greater"}\NormalTok{),}
       \AttributeTok{mu =} \DecValTok{0}\NormalTok{, }\AttributeTok{paired =} \ConstantTok{FALSE}\NormalTok{, }\AttributeTok{var.equal =} \ConstantTok{FALSE}\NormalTok{,}
       \AttributeTok{conf.level =} \FloatTok{0.95}\NormalTok{, ...)}
\end{Highlighting}
\end{Shaded}

Los argumentos a definir dentro de \texttt{t.test} para hacer la prueba son:

\begin{itemize}
\tightlist
\item
  \texttt{x}: vector numérico con los datos.
\item
  \texttt{alternative}: tipo de hipótesis alterna. Los valores disponibles son \texttt{"two.sided"} cuando la hipótesis alterna es \(\neq\), \texttt{"less"} para el caso \(<\) y \texttt{"greater"} para \(>\).
\item
  \texttt{mu}: valor de referencia de la prueba.
\item
  \texttt{conf.level}: nivel de confianza para reportar el intervalo de confianza asociado (opcional).
\end{itemize}

\hypertarget{ejemplo-63}{%
\subsection*{Ejemplo}\label{ejemplo-63}}
\addcontentsline{toc}{subsection}{Ejemplo}

Para verificar si el proceso de llenado de bolsas de café con 500 gramos está operando correctamente se toman aleatoriamente muestras de tamaño diez cada cuatro horas. Una muestra de bolsas está compuesta por las siguientes observaciones: 502, 501, 497, 491, 496, 501, 502, 500, 489, 490.

¿Está el proceso llenando bolsas conforme lo dice la envoltura? Use un nivel de significancia del 5\%.

\emph{Solución}

Lo primero es explorar si la muestra proviene de una distribución normal, para eso ingresamos los datos y aplicamos la prueba Anderson-Darling por medio de la función \texttt{ad.test} disponible en el paquete \textbf{nortest} \citep{R-nortest} como se muestra a continuación.

\begin{Shaded}
\begin{Highlighting}[]
\NormalTok{contenido }\OtherTok{\textless{}{-}} \FunctionTok{c}\NormalTok{(}\DecValTok{510}\NormalTok{, }\DecValTok{492}\NormalTok{, }\DecValTok{494}\NormalTok{, }\DecValTok{498}\NormalTok{, }\DecValTok{492}\NormalTok{,}
               \DecValTok{496}\NormalTok{, }\DecValTok{502}\NormalTok{, }\DecValTok{491}\NormalTok{, }\DecValTok{507}\NormalTok{, }\DecValTok{496}\NormalTok{) }

\FunctionTok{require}\NormalTok{(nortest) }\CommentTok{\# Se debe haber instalado antes nortest}
\FunctionTok{ad.test}\NormalTok{(contenido)}
\end{Highlighting}
\end{Shaded}

\begin{verbatim}
## 
##  Anderson-Darling normality test
## 
## data:  contenido
## A = 0.49161, p-value = 0.1665
\end{verbatim}

Como el valor-P de la prueba Anderson-Darling es 20\% y mayor que el nivel de significancia del 5\%, se puede asumir que la muestra proviene de una población normal.

Luego de haber explorado la normalidad retornamos al problema de interés que se puede resumir así:

\[H_0: \mu = 500 \quad gr\]
\[H_1: \mu \neq 500 \quad gr\]
La prueba de hipótesis se puede realizar usando la función \texttt{t.test} por medio del siguiente código.

\begin{Shaded}
\begin{Highlighting}[]
\FunctionTok{t.test}\NormalTok{(contenido, }\AttributeTok{alternative=}\StringTok{\textquotesingle{}two.sided\textquotesingle{}}\NormalTok{,}
       \AttributeTok{conf.level=}\FloatTok{0.95}\NormalTok{, }\AttributeTok{mu=}\DecValTok{500}\NormalTok{)}
\end{Highlighting}
\end{Shaded}

\begin{verbatim}
## 
##  One Sample t-test
## 
## data:  contenido
## t = -1.0629, df = 9, p-value = 0.3155
## alternative hypothesis: true mean is not equal to 500
## 95 percent confidence interval:
##  493.1176 502.4824
## sample estimates:
## mean of x 
##     497.8
\end{verbatim}

Como el valor-P es 30\% y mayor que el nivel de significancia 5\%, no se rechaza la hipótesis nula, es decir, las evidencias no son suficientes para afirmar que el proceso de llenando no está cumpliendo con lo impreso en la envoltura.

\hypertarget{prueba-de-hipuxf3tesis-para-mu-con-muestras-grandes}{%
\section{\texorpdfstring{Prueba de hipótesis para \(\mu\) con muestras grandes}{Prueba de hipótesis para \textbackslash mu con muestras grandes}}\label{prueba-de-hipuxf3tesis-para-mu-con-muestras-grandes}}

\hypertarget{ejemplo-64}{%
\subsection*{Ejemplo}\label{ejemplo-64}}
\addcontentsline{toc}{subsection}{Ejemplo}

Se afirma que los automóviles recorren en promedio más de 20000 kilómetros por año pero usted cree que el promedio es en realidad menor. Para probar tal afirmación se pide a una muestra de 100 propietarios de automóviles seleccionada de manera aleatoria que lleven un registro de los kilómetros que recorren.

¿Estaría usted de acuerdo con la afirmación si la muestra aleatoria indicara un promedio de 19500 kilómetros y una desviación estándar de 3900 kilómetros? Utilice un valor P en su conclusión y use una significancia del 3\%.

\emph{Solución}

En este problema interesa:

\[H_0: \mu \ge 20000 \quad km\]
\[H_1: \mu < 20000 \quad km\]
Para este tipo de pruebas no hay una función de R que haga los cálculos, por esta razón uno mismo debe escribir una líneas de código para obtener los resultados deseados, a continuación las instrucciones para calcular el estadístico y su valor-P.

\begin{Shaded}
\begin{Highlighting}[]
\NormalTok{xbarra }\OtherTok{\textless{}{-}} \DecValTok{19500}  \CommentTok{\# Datos del problema}
\NormalTok{desvia }\OtherTok{\textless{}{-}} \DecValTok{3900}   \CommentTok{\# Datos del problema}
\NormalTok{n }\OtherTok{\textless{}{-}} \DecValTok{100}         \CommentTok{\# Datos del problema}
\NormalTok{mu }\OtherTok{\textless{}{-}} \DecValTok{20000}      \CommentTok{\# Media de referencia}

\NormalTok{est }\OtherTok{\textless{}{-}}\NormalTok{ (xbarra }\SpecialCharTok{{-}}\NormalTok{ mu) }\SpecialCharTok{/}\NormalTok{ (desvia }\SpecialCharTok{/} \FunctionTok{sqrt}\NormalTok{(n))}
\NormalTok{est  }\CommentTok{\# Para obtener el valor del estadístico}
\end{Highlighting}
\end{Shaded}

\begin{verbatim}
## [1] -1.282051
\end{verbatim}

\begin{Shaded}
\begin{Highlighting}[]
\FunctionTok{pnorm}\NormalTok{(est)  }\CommentTok{\# Para obtener el valor{-}P}
\end{Highlighting}
\end{Shaded}

\begin{verbatim}
## [1] 0.09991233
\end{verbatim}

Como el valor-P es mayor que el nivel de significancia 3\%, no hay evidencias suficientes para pensar que ha disminuido el recorrido anual promedio de los autos.

\hypertarget{prueba-de-hipuxf3tesis-para-la-proporciuxf3n-p}{%
\section{\texorpdfstring{Prueba de hipótesis para la proporción \(p\)}{Prueba de hipótesis para la proporción p}}\label{prueba-de-hipuxf3tesis-para-la-proporciuxf3n-p}}

Existen varias pruebas para estudiar la propoción \(p\) de una distribución binomial, a continuación el listado de las más comunes.

\begin{enumerate}
\def\labelenumi{\arabic{enumi}.}
\tightlist
\item
  Prueba de \href{https://en.wikipedia.org/wiki/Wald_test}{Wald},
\item
  Prueba \(\Chi^2\) de \href{https://en.wikipedia.org/wiki/Pearson\%27s_chi-squared_test\#Fairness_of_dice}{Pearson},
\item
  Prueba \href{https://en.wikipedia.org/wiki/Binomial_test}{binomial exacta}.
\end{enumerate}

\hypertarget{prueba-de-wald}{%
\subsection{Prueba de Wald}\label{prueba-de-wald}}

Esta prueba se recomienda usar cuando se tiene un tamaño de muestra \(n\) suficientemente grande para poder usar la distribución normal para aproximar la distribución binomial.

En esta prueba el estadístico está dado por

\[z=\frac{\hat{p}-p_0}{\sqrt{\frac{p_0(1-p_0)}{n}}},\]
donde \(\hat{p}\) es la proporción muestral calculada como el cociente entre el número de éxitos \(x\) observados en los \(n\) ensayos y \(p_0\) es el valor de referencia de las hipótesis. El estadístico \(z\) tiene distribución \(N(0, 1)\) cuando \(n \to \infty\).

Para realizar esta prueba en R no hay una función y debemos escribir la líneas de código para obtener el estadístico y el valor-P de la prueba. A continuación se muestra un ejemplo de cómo proceder para aplicar la prueba de Wald.

\hypertarget{ejemplo-65}{%
\subsection*{Ejemplo}\label{ejemplo-65}}
\addcontentsline{toc}{subsection}{Ejemplo}

Un fabricante de un quitamanchas afirma que su producto quita 90\% de todas las manchas. Para poner a prueba esta afirmación se toman 200 camisetas manchadas de las cuales a solo 174 les desapareció la mancha. Pruebe la afirmación del fabricante a un nivel \(\alpha=0.05\).

\emph{Solución}

En este problema interesa probar lo siguiente:

\[H_0: p = 0.90\]
\[H_1: p < 0.90\]

Del anterior conjunto de hipótesis se observa que el valor de referencia de la prueba es \(p_0=0.90\). De la información inicial se tiene que de las \(n=200\) pruebas se observó que en \(x=174\) la mancha desapareció, con esta información se puede calcular el estadístico \(z\) así:

\begin{Shaded}
\begin{Highlighting}[]
\NormalTok{z }\OtherTok{\textless{}{-}}\NormalTok{ (}\DecValTok{174}\SpecialCharTok{/}\DecValTok{200} \SpecialCharTok{{-}} \FloatTok{0.90}\NormalTok{) }\SpecialCharTok{/} \FunctionTok{sqrt}\NormalTok{(}\FloatTok{0.90} \SpecialCharTok{*}\NormalTok{ (}\DecValTok{1} \SpecialCharTok{{-}} \FloatTok{0.90}\NormalTok{) }\SpecialCharTok{/} \DecValTok{200}\NormalTok{)}
\NormalTok{z  }\CommentTok{\# Para obtener el valor del estadístico}
\end{Highlighting}
\end{Shaded}

\begin{verbatim}
## [1] -1.414214
\end{verbatim}

Para obtener el valor-P de la prueba debemos tener en cuenta el sentido en la hipótesis alternativa \(H_1: p < 0.90\), por esa razón el valor-P será \(P(Z<z)\) y para obtenerlo usamos el siguiente código

\begin{Shaded}
\begin{Highlighting}[]
\FunctionTok{pnorm}\NormalTok{(}\AttributeTok{q=}\NormalTok{z, }\AttributeTok{lower.tail=}\ConstantTok{TRUE}\NormalTok{)  }\CommentTok{\# Para obtener el valor{-}P}
\end{Highlighting}
\end{Shaded}

\begin{verbatim}
## [1] 0.0786496
\end{verbatim}

El valor-P obtenido se puede representar gráficamente en la Figura \ref{fig:wald1}.

Como el valor-P obtenido fue mayor que el nivel de significancia \(\alpha=0.05\) se concluye que no hay evidencias suficientes para rechazar la hipótesis nula.

\begin{figure}
\centering
\includegraphics{Manual_de_R_files/figure-latex/wald1-1.pdf}
\caption{\label{fig:wald1}Representación del Valor-P para la prueba Wald.}
\end{figure}

\newpage

\hypertarget{prueba-chi2-de-pearson}{%
\subsection{\texorpdfstring{Prueba \(\Chi^2\) de Pearson}{Prueba \textbackslash Chi\^{}2 de Pearson}}\label{prueba-chi2-de-pearson}}

Para realizar la prueba \(\Chi^2\) de Pearson se usa la función \texttt{prop.test} que tiene la siguiente estructura.

\begin{Shaded}
\begin{Highlighting}[]
\FunctionTok{prop.test}\NormalTok{(x, n, }\AttributeTok{p =} \ConstantTok{NULL}\NormalTok{,}
          \AttributeTok{alternative =} \FunctionTok{c}\NormalTok{(}\StringTok{"two.sided"}\NormalTok{, }\StringTok{"less"}\NormalTok{, }\StringTok{"greater"}\NormalTok{),}
          \AttributeTok{conf.level =} \FloatTok{0.95}\NormalTok{, }\AttributeTok{correct =} \ConstantTok{TRUE}\NormalTok{)}
\end{Highlighting}
\end{Shaded}

Los argumentos a definir dentro de \texttt{prop.test} para hacer la prueba son:

\begin{itemize}
\tightlist
\item
  \texttt{x}: número de éxitos en la muestra.
\item
  \texttt{n}: número de observaciones en la muestra.
\item
  \texttt{alternative}: tipo de hipótesis alterna. Los valores disponibles son \texttt{"two.sided"} cuando la alterna es \(\neq\), \texttt{"less"} para el caso \(<\) y \texttt{"greater"} para \(>\).
\item
  \texttt{p}: valor de referencia de la prueba.
\item
  \texttt{correct}: valor lógico para indicar si se usa la corrección de Yates.
\item
  \texttt{conf.level}: nivel de confianza para reportar el intervalo de confianza asociado (opcional).
\end{itemize}

\hypertarget{ejemplo-66}{%
\subsection*{Ejemplo}\label{ejemplo-66}}
\addcontentsline{toc}{subsection}{Ejemplo}

Un fabricante de un quitamanchas afirma que su producto quita 90\% de todas las manchas. Para poner a prueba esta afirmación se toman 200 camisetas manchadas de las cuales a solo 174 les desapareció la mancha. Pruebe la afirmación del fabricante a un nivel \(\alpha=0.05\).

\emph{Solución}

En este problema interesa probar lo siguiente:

\[H_0: p = 0.90\]
\[H_1: p < 0.90\]
La forma de usar la función \texttt{prop.test} para realizar la prueba se muestra a continuación.

\begin{Shaded}
\begin{Highlighting}[]
\FunctionTok{prop.test}\NormalTok{(}\AttributeTok{x=}\DecValTok{174}\NormalTok{, }\AttributeTok{n=}\DecValTok{200}\NormalTok{, }\AttributeTok{p=}\FloatTok{0.9}\NormalTok{, }\AttributeTok{alternative=}\StringTok{\textquotesingle{}less\textquotesingle{}}\NormalTok{,}
          \AttributeTok{conf.level=}\FloatTok{0.95}\NormalTok{, }\AttributeTok{correct=}\ConstantTok{FALSE}\NormalTok{)}
\end{Highlighting}
\end{Shaded}

\begin{verbatim}
## 
##  1-sample proportions test without continuity correction
## 
## data:  174 out of 200, null probability 0.9
## X-squared = 2, df = 1, p-value = 0.07865
## alternative hypothesis: true p is less than 0.9
## 95 percent confidence interval:
##  0.0000000 0.9042273
## sample estimates:
##    p 
## 0.87
\end{verbatim}

Como el valor-P (con valor de 0.07865 pero repotado en la salida como 0.08) es mayor que \(\alpha\) no se rechaza la hipótesis nula y se concluye que no hay evidencias suficientes para rechazar la hipótesis nula.

\hypertarget{prueba-binomial-exacta}{%
\subsection{Prueba binomial exacta}\label{prueba-binomial-exacta}}

Para realizar la prueba binomial exacta se usa la función \texttt{binom.test} que tiene la siguiente estructura.

\begin{Shaded}
\begin{Highlighting}[]
\FunctionTok{binom.test}\NormalTok{(x, n, }\AttributeTok{p =} \FloatTok{0.5}\NormalTok{,}
           \AttributeTok{alternative =} \FunctionTok{c}\NormalTok{(}\StringTok{"two.sided"}\NormalTok{, }\StringTok{"less"}\NormalTok{, }\StringTok{"greater"}\NormalTok{),}
           \AttributeTok{conf.level =} \FloatTok{0.95}\NormalTok{)}
\end{Highlighting}
\end{Shaded}

Los argumentos a definir dentro de \texttt{binom.test} para hacer la prueba son:

\begin{itemize}
\tightlist
\item
  \texttt{x}: número de éxitos en la muestra.
\item
  \texttt{n}: número de observaciones en la muestra.
\item
  \texttt{alternative}: tipo de hipótesis alterna. Los valores disponibles son \texttt{"two.sided"} cuando la alterna es \(\neq\), \texttt{"less"} para el caso \(<\) y \texttt{"greater"} para \(>\).
\item
  \texttt{p}: valor de referencia de la prueba.
\item
  \texttt{conf.level}: nivel de confianza para reportar el intervalo de confianza asociado (opcional).
\end{itemize}

\hypertarget{ejemplo-67}{%
\subsection*{Ejemplo}\label{ejemplo-67}}
\addcontentsline{toc}{subsection}{Ejemplo}

Un asadero de pollos asegura que 90\% de sus órdenes se entregan en menos de 10 minutos. En una muestra de 20 órdenes, 17 se entregaron dentro de ese lapso. ¿Puede concluirse en el nivel de significancia 0.05, que menos de 90\% de las órdenes se entregan en menos de 10 minutos?

\emph{Solución}

En este problema interesa probar lo siguiente:

\[H_0: p = 0.90\]
\[H_1: p < 0.90\]
La forma de usar la función \texttt{binom.test} para realizar la prueba se muestra a continuación.

\begin{Shaded}
\begin{Highlighting}[]
\FunctionTok{binom.test}\NormalTok{(}\AttributeTok{x=}\DecValTok{17}\NormalTok{, }\AttributeTok{n=}\DecValTok{20}\NormalTok{, }\AttributeTok{p=}\FloatTok{0.9}\NormalTok{, }\AttributeTok{alternative=}\StringTok{"less"}\NormalTok{)}
\end{Highlighting}
\end{Shaded}

\begin{verbatim}
## 
##  Exact binomial test
## 
## data:  17 and 20
## number of successes = 17, number of trials = 20, p-value = 0.3231
## alternative hypothesis: true probability of success is less than 0.9
## 95 percent confidence interval:
##  0.0000000 0.9578306
## sample estimates:
## probability of success 
##                   0.85
\end{verbatim}

Como el valor-P (reportado como 0.3 pero con valor de 0.3231) es mayor que \(\alpha\) no se rechaza la hipótesis nula y se concluye que no hay evidencias suficientes para rechazar la hipótesis nula.

\hypertarget{prueba-de-hipuxf3tesis-para-la-varianza-sigma2-de-una-poblaciuxf3n-normal}{%
\section{\texorpdfstring{Prueba de hipótesis para la varianza \(\sigma^2\) de una población normal}{Prueba de hipótesis para la varianza \textbackslash sigma\^{}2 de una población normal}}\label{prueba-de-hipuxf3tesis-para-la-varianza-sigma2-de-una-poblaciuxf3n-normal}}

Para realizar este tipo de prueba se usa la función \texttt{var.test} del paquete \textbf{usefultools} \citep{R-usefultools} disponible en el repositorio \href{https://github.com/}{GitHub}. La función \texttt{var.test} tiene la siguiente estructura.

\begin{Shaded}
\begin{Highlighting}[]
\FunctionTok{var.test}\NormalTok{(x, }\AttributeTok{alternative =} \StringTok{"two.sided"}\NormalTok{,}
         \AttributeTok{null.value =} \DecValTok{1}\NormalTok{, }\AttributeTok{conf.level =} \FloatTok{0.95}\NormalTok{)}
\end{Highlighting}
\end{Shaded}

Los argumentos a definir dentro de \texttt{var.test} para hacer la prueba son:

\begin{itemize}
\tightlist
\item
  \texttt{x}: vector numérico con los datos.
\item
  \texttt{alternative}: tipo de hipótesis alterna. Los valores disponibles son \texttt{"two.sided"} cuando la alterna es \(\neq\), \texttt{"less"} para el caso \(<\) y \texttt{"greater"} para \(>\).
\item
  \texttt{null.value}: valor de referencia de la prueba.
\item
  \texttt{conf.level}: nivel de confianza para reportar el intervalo de confianza asociado (opcional).
\end{itemize}

Para instalar el paquete \textbf{usefultools} desde GitHub se debe copiar el siguiente código en la consola de R:

\begin{Shaded}
\begin{Highlighting}[]
\ControlFlowTok{if}\NormalTok{ (}\SpecialCharTok{!}\FunctionTok{require}\NormalTok{(}\StringTok{\textquotesingle{}devtools\textquotesingle{}}\NormalTok{)) }\FunctionTok{install.packages}\NormalTok{(}\StringTok{\textquotesingle{}devtools\textquotesingle{}}\NormalTok{)}
\NormalTok{devtools}\SpecialCharTok{::}\FunctionTok{install\_github}\NormalTok{(}\StringTok{\textquotesingle{}fhernanb/usefultools\textquotesingle{}}\NormalTok{, }\AttributeTok{force=}\ConstantTok{TRUE}\NormalTok{)}
\end{Highlighting}
\end{Shaded}

\hypertarget{ejemplo-68}{%
\subsection*{Ejemplo}\label{ejemplo-68}}
\addcontentsline{toc}{subsection}{Ejemplo}

Para verificar si el proceso de llenado de bolsas de café está operando con la variabilidad permitida se toman aleatoriamente muestras de tamaño diez cada cuatro horas. Una muestra de bolsas está compuesta por las siguientes observaciones: 502, 501, 497, 491, 496, 501, 502, 500, 489, 490. El proceso de llenado está bajo control si presenta un varianza de 40 o menos. ¿Está el proceso llenando bolsas conforme lo dice la envoltura? Use un nivel de significancia del 5\%.

\emph{Solución}

En un ejemplo anterior se comprobó que la muestra proviene de una población normal así que se puede proceder con la prueba de hipótesis sobre \(\sigma^2\).

En este ejemplo nos interesa estudiar el siguiente conjunto de hipótesis

\[H_0: \sigma^2 \leq 40\]
\[H_1: \sigma^2 > 40\]
La prueba de hipótesis se puede realizar usando la función \texttt{var.test} por medio del siguiente código.

\begin{Shaded}
\begin{Highlighting}[]
\NormalTok{contenido }\OtherTok{\textless{}{-}} \FunctionTok{c}\NormalTok{(}\DecValTok{510}\NormalTok{, }\DecValTok{492}\NormalTok{, }\DecValTok{494}\NormalTok{, }\DecValTok{498}\NormalTok{, }\DecValTok{492}\NormalTok{,}
               \DecValTok{496}\NormalTok{, }\DecValTok{502}\NormalTok{, }\DecValTok{491}\NormalTok{, }\DecValTok{507}\NormalTok{, }\DecValTok{496}\NormalTok{)}

\FunctionTok{require}\NormalTok{(usefultools)  }\CommentTok{\# Ya debe estar instalado}
\FunctionTok{var.test}\NormalTok{(}\AttributeTok{x=}\NormalTok{contenido, }\AttributeTok{alternative=}\StringTok{\textquotesingle{}greater\textquotesingle{}}\NormalTok{,}
         \AttributeTok{null.value=}\DecValTok{40}\NormalTok{, }\AttributeTok{conf.level=}\FloatTok{0.95}\NormalTok{)}
\end{Highlighting}
\end{Shaded}

\begin{verbatim}
## 
##  X-squared test for variance
## 
## data:  contenido
## X-squared = 9.64, df = 9, p-value = 0.3804
## alternative hypothesis: true variance is greater than 40
## 95 percent confidence interval:
##    0.000 115.966
## sample estimates:
## variance of x 
##      42.84444
\end{verbatim}

Como el valor-P es mayor que el nivel de significancia 5\%, no se rechaza la hipótesis nula, es decir, las evidencias no son suficientes para afirmar que la varianza del proceso de llenado es mayor que 40 unidades.

\hypertarget{prueba-de-hipuxf3tesis-para-el-cociente-de-varianzas-sigma_12-sigma_22}{%
\section{\texorpdfstring{Prueba de hipótesis para el cociente de varianzas \(\sigma_1^2 / \sigma_2^2\)}{Prueba de hipótesis para el cociente de varianzas \textbackslash sigma\_1\^{}2 / \textbackslash sigma\_2\^{}2}}\label{prueba-de-hipuxf3tesis-para-el-cociente-de-varianzas-sigma_12-sigma_22}}

Para realizar este tipo de prueba se puede usar la función \texttt{var.test}.

\hypertarget{ejemplo-69}{%
\subsection*{Ejemplo}\label{ejemplo-69}}
\addcontentsline{toc}{subsection}{Ejemplo}

Se realiza un estudio para comparar dos tratamientos que se aplicarán a frijoles crudos con el objetivo de reducir el tiempo de cocción. El tratamiento T1 es a base de bicarbonato de sodio, el T2 es a base de cloruro de sodio o sal común. La variable respuesta es el tiempo de cocción en minutos. Los datos se muestran abajo. ¿Son las varianzas de los tiempos iguales o diferentes? Usar \(\alpha=0.05\).

\textbf{T1}: 76, 85, 74, 78, 82, 75, 82.

\textbf{T2}: 57, 67, 55, 64, 61, 63, 63.

\emph{Solución}

En este problema interesa probar si las varianzas poblacionales son iguales o no, por esta razón el cociente de \(\sigma_{T1}^2 / \sigma_{T2}^2\) se iguala al valor de 1 que será el valor de referencia de la prueba.

\[H_0: \sigma_{T1}^2 / \sigma_{T2}^2 = 1\]
\[H_1: \sigma_{T1}^2 / \sigma_{T2}^2 \neq 1\]

Para ingresar los datos se hace lo siguiente:

\begin{Shaded}
\begin{Highlighting}[]
\NormalTok{T1 }\OtherTok{\textless{}{-}} \FunctionTok{c}\NormalTok{(}\DecValTok{76}\NormalTok{, }\DecValTok{85}\NormalTok{, }\DecValTok{74}\NormalTok{,}\DecValTok{78}\NormalTok{, }\DecValTok{82}\NormalTok{, }\DecValTok{75}\NormalTok{, }\DecValTok{82}\NormalTok{) }
\NormalTok{T2 }\OtherTok{\textless{}{-}} \FunctionTok{c}\NormalTok{(}\DecValTok{57}\NormalTok{, }\DecValTok{67}\NormalTok{, }\DecValTok{55}\NormalTok{, }\DecValTok{64}\NormalTok{, }\DecValTok{61}\NormalTok{, }\DecValTok{63}\NormalTok{, }\DecValTok{63}\NormalTok{)}
\end{Highlighting}
\end{Shaded}

Primero se debe explorar si las muestras provienen de una población normal y para esto se construyen los QQplot que se muestran en la Figura \ref{fig:frijoles1}, a continuación el código para generar la Figura \ref{fig:frijoles1}.

\begin{Shaded}
\begin{Highlighting}[]
\NormalTok{q1 }\OtherTok{\textless{}{-}} \FunctionTok{qqnorm}\NormalTok{(T1, }\AttributeTok{plot.it=}\ConstantTok{FALSE}\NormalTok{)}
\NormalTok{q2 }\OtherTok{\textless{}{-}} \FunctionTok{qqnorm}\NormalTok{(T2, }\AttributeTok{plot.it=}\ConstantTok{FALSE}\NormalTok{)}
\FunctionTok{plot}\NormalTok{(}\FunctionTok{range}\NormalTok{(q1}\SpecialCharTok{$}\NormalTok{x, q2}\SpecialCharTok{$}\NormalTok{x), }\FunctionTok{range}\NormalTok{(q1}\SpecialCharTok{$}\NormalTok{y, q2}\SpecialCharTok{$}\NormalTok{y), }\AttributeTok{type=}\StringTok{"n"}\NormalTok{, }\AttributeTok{las=}\DecValTok{1}\NormalTok{,}
     \AttributeTok{xlab=}\StringTok{\textquotesingle{}Theoretical Quantiles\textquotesingle{}}\NormalTok{, }\AttributeTok{ylab=}\StringTok{\textquotesingle{}Sample Quantiles\textquotesingle{}}\NormalTok{)}
\FunctionTok{points}\NormalTok{(q1, }\AttributeTok{pch=}\DecValTok{19}\NormalTok{)}
\FunctionTok{points}\NormalTok{(q2, }\AttributeTok{col=}\StringTok{"red"}\NormalTok{, }\AttributeTok{pch=}\DecValTok{19}\NormalTok{)}
\FunctionTok{qqline}\NormalTok{(T1, }\AttributeTok{lty=}\StringTok{\textquotesingle{}dashed\textquotesingle{}}\NormalTok{)}
\FunctionTok{qqline}\NormalTok{(T2, }\AttributeTok{col=}\StringTok{"red"}\NormalTok{, }\AttributeTok{lty=}\StringTok{"dashed"}\NormalTok{)}
\FunctionTok{legend}\NormalTok{(}\StringTok{\textquotesingle{}topleft\textquotesingle{}}\NormalTok{, }\AttributeTok{legend=}\FunctionTok{c}\NormalTok{(}\StringTok{\textquotesingle{}T1\textquotesingle{}}\NormalTok{, }\StringTok{\textquotesingle{}T2\textquotesingle{}}\NormalTok{), }\AttributeTok{bty=}\StringTok{\textquotesingle{}n\textquotesingle{}}\NormalTok{,}
       \AttributeTok{col=}\FunctionTok{c}\NormalTok{(}\StringTok{\textquotesingle{}black\textquotesingle{}}\NormalTok{, }\StringTok{\textquotesingle{}red\textquotesingle{}}\NormalTok{), }\AttributeTok{pch=}\DecValTok{19}\NormalTok{)}
\end{Highlighting}
\end{Shaded}

\begin{figure}
\centering
\includegraphics{Manual_de_R_files/figure-latex/frijoles1-1.pdf}
\caption{\label{fig:frijoles1}QQplot para los tiempos de cocción.}
\end{figure}

De la Figura \ref{fig:frijoles1} se observa que los puntos están bastante alineados lo cual nos lleva a pensar que las muestras si provienen de una población normal, para estar más seguros se aplicará una prueba formal para estudiar la normalidad.

A continuación el código para aplicar la prueba de normalidad Kolmogorov-Smirnov a cada una de las muestras.

\begin{Shaded}
\begin{Highlighting}[]
\FunctionTok{require}\NormalTok{(nortest)  }\CommentTok{\# Se debe tener instalado}
\FunctionTok{lillie.test}\NormalTok{(T1)}\SpecialCharTok{$}\NormalTok{p.value}
\end{Highlighting}
\end{Shaded}

\begin{verbatim}
## [1] 0.520505
\end{verbatim}

\begin{Shaded}
\begin{Highlighting}[]
\FunctionTok{lillie.test}\NormalTok{(T2)}\SpecialCharTok{$}\NormalTok{p.value}
\end{Highlighting}
\end{Shaded}

\begin{verbatim}
## [1] 0.3952748
\end{verbatim}

Del QQplot mostrado en la Figura \ref{fig:frijoles1} y las pruebas de normalidad se observa que se puede asumir que las poblaciones son normales.

La función \texttt{var.test} se puede usar para probar \(H_0\), a continuación el código para realizar la prueba.

\begin{Shaded}
\begin{Highlighting}[]
\FunctionTok{var.test}\NormalTok{(T1, T2, }\AttributeTok{null.value=}\DecValTok{1}\NormalTok{, }\AttributeTok{alternative=}\StringTok{"two.sided"}\NormalTok{,}
         \AttributeTok{conf.level=}\FloatTok{0.95}\NormalTok{)}
\end{Highlighting}
\end{Shaded}

\begin{verbatim}
## 
##  F test to compare two variances
## 
## data:  T1  and  T2
## F = 1.011, num df = 6, denom df = 6, p-value = 0.9897
## alternative hypothesis: true ratio of variances is not equal to 1
## 95 percent confidence interval:
##  0.1737219 5.8838861
## sample estimates:
## ratio of variances 
##           1.011019
\end{verbatim}

Como el valor-P es 0.9897 (reportado como 1 en la salida anterior), muy superior al nivel \(\alpha\) de significancia 5\%, se puede concluir que las varianzas son similares.

\hypertarget{ejemplo-70}{%
\subsection*{Ejemplo}\label{ejemplo-70}}
\addcontentsline{toc}{subsection}{Ejemplo}

El arsénico en agua potable es un posible riesgo para la salud. Un artículo reciente reportó concentraciones de arsénico en agua potable en partes por billón (ppb) para diez comunidades urbanas y diez comunidades rurales. Los datos son los siguientes:

\textbf{Urbana}: 3, 7, 25, 10, 15, 6, 12, 25, 15, 7

\textbf{Rural}: 48, 44, 40, 38, 33, 21, 20, 12, 1, 18

\emph{Solución}

¿Son las varianzas de las concentraciones iguales o diferentes? Usar \(\alpha=0.05\).

En este problema interesa probar:

\[H_0: \sigma_{Urb}^2 / \sigma_{Rur}^2 = 1\]
\[H_1: \sigma_{Urb}^2 / \sigma_{Rur}^2 \neq 1\]

Para ingresar los datos se hace lo siguiente:

\begin{Shaded}
\begin{Highlighting}[]
\NormalTok{urb }\OtherTok{\textless{}{-}} \FunctionTok{c}\NormalTok{(}\DecValTok{3}\NormalTok{, }\DecValTok{7}\NormalTok{, }\DecValTok{25}\NormalTok{, }\DecValTok{10}\NormalTok{, }\DecValTok{15}\NormalTok{, }\DecValTok{6}\NormalTok{, }\DecValTok{12}\NormalTok{, }\DecValTok{25}\NormalTok{, }\DecValTok{15}\NormalTok{, }\DecValTok{7}\NormalTok{)}
\NormalTok{rur }\OtherTok{\textless{}{-}} \FunctionTok{c}\NormalTok{(}\DecValTok{48}\NormalTok{, }\DecValTok{44}\NormalTok{, }\DecValTok{40}\NormalTok{, }\DecValTok{38}\NormalTok{, }\DecValTok{33}\NormalTok{, }\DecValTok{21}\NormalTok{, }\DecValTok{20}\NormalTok{, }\DecValTok{12}\NormalTok{, }\DecValTok{1}\NormalTok{, }\DecValTok{18}\NormalTok{)}
\end{Highlighting}
\end{Shaded}

Primero se debe explorar si las muestras provienen de una población normal, para esto se construyen los QQplot mostrados en la Figura \ref{fig:ars1}.

\begin{figure}
\centering
\includegraphics{Manual_de_R_files/figure-latex/ars1-1.pdf}
\caption{\label{fig:ars1}QQplot para las concentraciones de arsénico.}
\end{figure}

A continuación el código para aplicar la prueba de normalidad Kolmogorov-Smirnov, a continuación el código usado.

\begin{Shaded}
\begin{Highlighting}[]
\FunctionTok{require}\NormalTok{(nortest)  }\CommentTok{\# Se debe tener instalado}
\FunctionTok{lillie.test}\NormalTok{(urb)}\SpecialCharTok{$}\NormalTok{p.value}
\end{Highlighting}
\end{Shaded}

\begin{verbatim}
## [1] 0.5522105
\end{verbatim}

\begin{Shaded}
\begin{Highlighting}[]
\FunctionTok{lillie.test}\NormalTok{(rur)}\SpecialCharTok{$}\NormalTok{p.value}
\end{Highlighting}
\end{Shaded}

\begin{verbatim}
## [1] 0.6249628
\end{verbatim}

Del QQplot mostrado en la Figura \ref{fig:ars1} y las pruebas de normalidad se observa que se pueden asumir poblaciones normales.

La función \texttt{var.test} se puede usar para probar \(H_0\), a continuación el código para realizar la prueba.

\begin{Shaded}
\begin{Highlighting}[]
\FunctionTok{var.test}\NormalTok{(urb, rur, }\AttributeTok{null.value=}\DecValTok{1}\NormalTok{, }\AttributeTok{alternative=}\StringTok{"two.sided"}\NormalTok{,}
         \AttributeTok{conf.level=}\FloatTok{0.95}\NormalTok{)}
\end{Highlighting}
\end{Shaded}

\begin{verbatim}
## 
##  F test to compare two variances
## 
## data:  urb  and  rur
## F = 0.24735, num df = 9, denom df = 9, p-value = 0.04936
## alternative hypothesis: true ratio of variances is not equal to 1
## 95 percent confidence interval:
##  0.06143758 0.99581888
## sample estimates:
## ratio of variances 
##          0.2473473
\end{verbatim}

Como el valor-P es 0.0493604 (reportado como 0.05 en la salida anterior) y es menor que el nivel de significancia \(\alpha=0.05\), se puede concluir que las varianzas no son iguales.

\begin{rmdwarning}
¿Notó que las funciones \texttt{var.test} y \texttt{var.test} son diferentes?

\texttt{var.test} sirve para prueba de hipótesis sobre \(\sigma^2\).

\texttt{var.test} sirve para prueba de hipótesis sobre \(\sigma_1^2 / \sigma_2^2\).
\end{rmdwarning}

\hypertarget{prueba-de-hipuxf3tesis-para-la-diferencia-de-medias-mu_1-mu_2-con-varianzas-iguales}{%
\section{\texorpdfstring{Prueba de hipótesis para la diferencia de medias \(\mu_1-\mu_2\) con varianzas iguales}{Prueba de hipótesis para la diferencia de medias \textbackslash mu\_1-\textbackslash mu\_2 con varianzas iguales}}\label{prueba-de-hipuxf3tesis-para-la-diferencia-de-medias-mu_1-mu_2-con-varianzas-iguales}}

Para realizar este tipo de prueba se puede usar la función \texttt{t.test} que tiene la siguiente estructura.

\begin{Shaded}
\begin{Highlighting}[]
\FunctionTok{t.test}\NormalTok{(x, }\AttributeTok{y =} \ConstantTok{NULL}\NormalTok{,}
       \AttributeTok{alternative =} \FunctionTok{c}\NormalTok{(}\StringTok{"two.sided"}\NormalTok{, }\StringTok{"less"}\NormalTok{, }\StringTok{"greater"}\NormalTok{),}
       \AttributeTok{mu =} \DecValTok{0}\NormalTok{, }\AttributeTok{paired =} \ConstantTok{FALSE}\NormalTok{, }\AttributeTok{var.equal =} \ConstantTok{FALSE}\NormalTok{,}
       \AttributeTok{conf.level =} \FloatTok{0.95}\NormalTok{, ...)}
\end{Highlighting}
\end{Shaded}

Los argumentos a definir dentro de \texttt{t.test} para hacer la prueba son:

\begin{itemize}
\tightlist
\item
  \texttt{x}: vector numérico con la información de la muestra 1,
\item
  \texttt{y}: vector numérico con la información de la muestra 2,
\item
  \texttt{alternative}: tipo de hipótesis alterna. Los valores disponibles son \texttt{"two.sided"} cuando la alterna es \(\neq\), \texttt{"less"} para el caso \(<\) y \texttt{"greater"} para \(>\).
\item
  \texttt{mu}: valor de referencia de la prueba.
\item
  \texttt{var.equal=TRUE}: indica que las varianzas son desconocidas pero iguales.
\item
  \texttt{conf.level}: nivel de confianza para reportar el intervalo de confianza asociado (opcional).
\end{itemize}

\hypertarget{ejemplo-71}{%
\subsection*{Ejemplo}\label{ejemplo-71}}
\addcontentsline{toc}{subsection}{Ejemplo}

Retomando el ejemplo de los fríjoles, ¿existen diferencias entre los tiempos de cocción de los fríjoles con T1 y T2? Usar un nivel de significancia del 5\%.

Primero se construirá un boxplot comparativo para los tiempos de cocción diferenciando por el tratamiento que recibieron. Abajo el código para obtener en este caso el boxplot. En la Figura \ref{fig:frijoles2} se muestra el boxplot, de esta figura se observa que las cajas de los boxplot no se traslapan, esto es un indicio de que las medias poblacionales, \(\mu_1\) y \(\mu_2\), son diferentes, se observa también que el boxplot para el tratamiento T1 está por encima del T2.

\begin{Shaded}
\begin{Highlighting}[]
\NormalTok{datos }\OtherTok{\textless{}{-}} \FunctionTok{data.frame}\NormalTok{(}\AttributeTok{tiempo=}\FunctionTok{c}\NormalTok{(T1, T2), }\AttributeTok{trat=}\FunctionTok{rep}\NormalTok{(}\DecValTok{1}\SpecialCharTok{:}\DecValTok{2}\NormalTok{, }\AttributeTok{each=}\DecValTok{7}\NormalTok{))}
\FunctionTok{boxplot}\NormalTok{(tiempo }\SpecialCharTok{\textasciitilde{}}\NormalTok{ trat, }\AttributeTok{data=}\NormalTok{datos, }\AttributeTok{las=}\DecValTok{1}\NormalTok{,}
        \AttributeTok{xlab=}\StringTok{\textquotesingle{}Tratamiento\textquotesingle{}}\NormalTok{, }\AttributeTok{ylab=}\StringTok{\textquotesingle{}Tiempo (min)\textquotesingle{}}\NormalTok{)}
\end{Highlighting}
\end{Shaded}

\begin{figure}
\centering
\includegraphics{Manual_de_R_files/figure-latex/frijoles2-1.pdf}
\caption{\label{fig:frijoles2}Boxplot para los tiempos de cocción dado el tratamiento.}
\end{figure}

En este problema interesa estudiar el siguiente conjunto de hipótesis.

\[H_0: \mu_1  - \mu_2 = 0\]
\[H_1: \mu_1  - \mu_2 \neq 0\]

El código para realizar la prueba es el siguiente:

\begin{Shaded}
\begin{Highlighting}[]
\FunctionTok{t.test}\NormalTok{(}\AttributeTok{x=}\NormalTok{T1, }\AttributeTok{y=}\NormalTok{T2, }\AttributeTok{alternative=}\StringTok{"two.sided"}\NormalTok{, }\AttributeTok{mu=}\DecValTok{0}\NormalTok{, }
       \AttributeTok{paired=}\ConstantTok{FALSE}\NormalTok{, }\AttributeTok{var.equal=}\ConstantTok{TRUE}\NormalTok{, }\AttributeTok{conf.level=}\FloatTok{0.97}\NormalTok{)}
\end{Highlighting}
\end{Shaded}

\begin{verbatim}
## 
##  Two Sample t-test
## 
## data:  T1 and T2
## t = 7.8209, df = 12, p-value = 4.737e-06
## alternative hypothesis: true difference in means is not equal to 0
## 97 percent confidence interval:
##  11.94503 22.91212
## sample estimates:
## mean of x mean of y 
##  78.85714  61.42857
\end{verbatim}

De la prueba se obtiene un valor-P muy pequeño, por lo tanto, podemos concluir que si hay diferencias significativas entre los tiempos promedios de cocción con T1 y T2, resultado que ya se sospechaba al observar la Figura \ref{fig:frijoles2}.

Si el objetivo fuese elegir el tratamiento que minimice los tiempos de cocción se recomendaría el tratamiento T2, remojo de fríjoles en agua con sal.

\hypertarget{prueba-de-hipuxf3tesis-para-la-diferencia-de-medias-mu_1-mu_2-con-varianzas-diferentes}{%
\section{\texorpdfstring{Prueba de hipótesis para la diferencia de medias \(\mu_1-\mu_2\) con varianzas diferentes}{Prueba de hipótesis para la diferencia de medias \textbackslash mu\_1-\textbackslash mu\_2 con varianzas diferentes}}\label{prueba-de-hipuxf3tesis-para-la-diferencia-de-medias-mu_1-mu_2-con-varianzas-diferentes}}

\hypertarget{ejemplo-72}{%
\subsection*{Ejemplo}\label{ejemplo-72}}
\addcontentsline{toc}{subsection}{Ejemplo}

Retomando el ejemplo de la concentración de arsénico en el agua, ¿existen diferencias entre las concentraciones de arsénico de la zona urbana y rural? Usar un nivel de significancia del 5\%.

Primero se construirá un boxplot comparativo para las concentraciones de arsénico diferenciando por la zona donde se tomaron las muestras. Abajo el código para obtener en este caso el boxplot. En la Figura \ref{fig:ars2} se muestra el boxplot, de esta figura se observa que las cajas de los boxplot no se traslapan, esto es un indicio de que las medias poblacionales, \(\mu_1\) y \(\mu_2\), son diferentes, se observa también que el boxplot para la zona rural está por encima del de la zona urbana.

\begin{Shaded}
\begin{Highlighting}[]
\NormalTok{datos }\OtherTok{\textless{}{-}} \FunctionTok{data.frame}\NormalTok{(}\AttributeTok{Concentracion=}\FunctionTok{c}\NormalTok{(urb, rur),}
                    \AttributeTok{Zona=}\FunctionTok{rep}\NormalTok{(}\FunctionTok{c}\NormalTok{(}\StringTok{\textquotesingle{}Urbana\textquotesingle{}}\NormalTok{, }\StringTok{\textquotesingle{}Rural\textquotesingle{}}\NormalTok{), }\AttributeTok{each=}\DecValTok{10}\NormalTok{))}
\FunctionTok{boxplot}\NormalTok{(Concentracion }\SpecialCharTok{\textasciitilde{}}\NormalTok{ Zona, }\AttributeTok{data=}\NormalTok{datos, }\AttributeTok{las=}\DecValTok{1}\NormalTok{,}
        \AttributeTok{xlab=}\StringTok{\textquotesingle{}Zona\textquotesingle{}}\NormalTok{, }\AttributeTok{ylab=}\StringTok{\textquotesingle{}Concentración arsénico (ppb)\textquotesingle{}}\NormalTok{)}
\end{Highlighting}
\end{Shaded}

\begin{figure}
\centering
\includegraphics{Manual_de_R_files/figure-latex/ars2-1.pdf}
\caption{\label{fig:ars2}Boxplot para las concentaciones de arsénico dada la zona.}
\end{figure}

En este problema interesa estudiar el siguiente conjunto de hipótesis.

\[H_0: \mu_1  - \mu_2 = 0\]
\[H_1: \mu_1  - \mu_2 \neq 0\]

El código para realizar la prueba es el siguiente:

\begin{Shaded}
\begin{Highlighting}[]
\FunctionTok{t.test}\NormalTok{(}\AttributeTok{x=}\NormalTok{urb, }\AttributeTok{y=}\NormalTok{rur, }\AttributeTok{alternative=}\StringTok{"two.sided"}\NormalTok{, }\AttributeTok{mu=}\DecValTok{0}\NormalTok{, }
       \AttributeTok{paired=}\ConstantTok{FALSE}\NormalTok{, }\AttributeTok{var.equal=}\ConstantTok{FALSE}\NormalTok{, }\AttributeTok{conf.level=}\FloatTok{0.95}\NormalTok{)}
\end{Highlighting}
\end{Shaded}

\begin{verbatim}
## 
##  Welch Two Sample t-test
## 
## data:  urb and rur
## t = -2.7669, df = 13.196, p-value = 0.01583
## alternative hypothesis: true difference in means is not equal to 0
## 95 percent confidence interval:
##  -26.694067  -3.305933
## sample estimates:
## mean of x mean of y 
##      12.5      27.5
\end{verbatim}

De la prueba se obtiene un valor-P pequeño, por lo tanto, podemos concluir que si hay diferencias significativas entre las concentraciones de arsénico del agua entre las dos zonas, resultado que ya se sospechaba al observar la Figura \ref{fig:ars2}. La zona que presenta mayor concentración media de arsénico en el agua es la rural.

\begin{rmdtip}
Para todas las pruebas se incluyó un intervalo de confianza, revise si la conclusión obtenida con el IC coincide con la obtenida con PH.
\end{rmdtip}

\hypertarget{prueba-de-hipuxf3tesis-para-la-diferencia-de-proporciones-p_1---p_2}{%
\section{\texorpdfstring{Prueba de hipótesis para la diferencia de proporciones \(p_1 - p_2\)}{Prueba de hipótesis para la diferencia de proporciones p\_1 - p\_2}}\label{prueba-de-hipuxf3tesis-para-la-diferencia-de-proporciones-p_1---p_2}}

Para realizar pruebas de hipótesis para la proporción se usa la función \texttt{prop.test} y es necesario definir los siguientes argumentos:

\begin{itemize}
\tightlist
\item
  \texttt{x}: vector con el conteo de éxitos de las dos muestras,
\item
  \texttt{n}: vector con el número de ensayos de las dos muestras,
\item
  \texttt{alternative}: tipo de hipótesis alterna. Los valores disponibles son \texttt{"two.sided"} cuando la alterna es \(\neq\), \texttt{"less"} para el caso \(<\) y \texttt{"greater"} para \(>\).
\item
  \texttt{p}: valor de referencia de la prueba.
\item
  \texttt{conf.level}: nivel de confianza para reportar el intervalo de confianza asociado (opcional).
\end{itemize}

\hypertarget{ejemplo-73}{%
\subsection*{Ejemplo}\label{ejemplo-73}}
\addcontentsline{toc}{subsection}{Ejemplo}

Se quiere determinar si un cambio en el método de fabricación de una piezas ha sido efectivo o no. Para esta comparación se tomaron 2 muestras, una antes y otra después del cambio en el proceso y los resultados obtenidos son los siguientes.

\begin{longtable}[]{@{}lll@{}}
\toprule
Num piezas & Antes & Después \\
\midrule
\endhead
Defectuosas & 75 & 80 \\
Analizadas & 1500 & 2000 \\
\bottomrule
\end{longtable}

Realizar una prueba de hipótesis con un nivel de significancia del 10\%.

En este problema interesa estudiar el siguiente conjunto de hipótesis.

\[H_0: p_{antes}  - p_{despues} = 0\]
\[H_1: p_{antes}  - p_{despues} > 0\]

Para realizar la prueba se usa la función \texttt{prop.test} como se muestra a continuación.

\begin{Shaded}
\begin{Highlighting}[]
\FunctionTok{prop.test}\NormalTok{(}\AttributeTok{x=}\FunctionTok{c}\NormalTok{(}\DecValTok{75}\NormalTok{, }\DecValTok{80}\NormalTok{), }\AttributeTok{n=}\FunctionTok{c}\NormalTok{(}\DecValTok{1500}\NormalTok{, }\DecValTok{2000}\NormalTok{),}
          \AttributeTok{alternative=}\StringTok{\textquotesingle{}greater\textquotesingle{}}\NormalTok{, }\AttributeTok{conf.level=}\FloatTok{0.90}\NormalTok{)}
\end{Highlighting}
\end{Shaded}

\begin{verbatim}
## 
##  2-sample test for equality of proportions with continuity correction
## 
## data:  c(75, 80) out of c(1500, 2000)
## X-squared = 1.7958, df = 1, p-value = 0.09011
## alternative hypothesis: greater
## 90 percent confidence interval:
##  0.0002765293 1.0000000000
## sample estimates:
## prop 1 prop 2 
##   0.05   0.04
\end{verbatim}

Del reporte anterior se observa que el Valor-P es 9\%, por lo tanto no hay evidencias suficientes para pensar que el porcentaje de defectuosos después del cambio ha disminuído.

\hypertarget{prueba-de-hipuxf3tesis-para-la-diferencia-de-medias-pareadas}{%
\section{Prueba de hipótesis para la diferencia de medias pareadas}\label{prueba-de-hipuxf3tesis-para-la-diferencia-de-medias-pareadas}}

\hypertarget{ejemplo-74}{%
\subsection*{Ejemplo}\label{ejemplo-74}}
\addcontentsline{toc}{subsection}{Ejemplo}

Diez individuos participaron de programa para perder peso corporal por medio de una dieta. Los voluntarios fueron pesados antes y después de haber participado del programa y los datos en libras aparecen abajo. ¿Hay evidencia que soporte la afirmación de la dieta disminuye el peso medio de los participantes? Usar nivel de significancia del 5\%.

\begin{longtable}[]{@{}ll@{}}
\toprule
Sujeto & 001 002 003 004 005 006 007 008 009 010 \\
\midrule
\endhead
Antes & 195 213 247 201 187 210 215 246 294 310 \\
Después & 187 195 221 190 175 197 199 221 278 285 \\
\bottomrule
\end{longtable}

Primero se debe explorar si las diferencias de peso (antes-después) provienen de una población normal, para esto se construye el QQplot mostrado en la Figura \ref{fig:pesos1}. De la figura no se observa un alejamiento serio de la recta de referencia, por lo tanto se puede asumir que las diferencias se distribuyen en forma aproximadamente normal.

\begin{Shaded}
\begin{Highlighting}[]
\NormalTok{antes }\OtherTok{\textless{}{-}} \FunctionTok{c}\NormalTok{(}\DecValTok{195}\NormalTok{, }\DecValTok{213}\NormalTok{, }\DecValTok{247}\NormalTok{, }\DecValTok{201}\NormalTok{, }\DecValTok{187}\NormalTok{, }\DecValTok{210}\NormalTok{, }\DecValTok{215}\NormalTok{, }\DecValTok{246}\NormalTok{, }\DecValTok{294}\NormalTok{, }\DecValTok{310}\NormalTok{)}
\NormalTok{despu }\OtherTok{\textless{}{-}} \FunctionTok{c}\NormalTok{(}\DecValTok{187}\NormalTok{, }\DecValTok{195}\NormalTok{, }\DecValTok{221}\NormalTok{, }\DecValTok{190}\NormalTok{, }\DecValTok{175}\NormalTok{, }\DecValTok{197}\NormalTok{, }\DecValTok{199}\NormalTok{, }\DecValTok{221}\NormalTok{, }\DecValTok{278}\NormalTok{, }\DecValTok{285}\NormalTok{)}
\NormalTok{dif }\OtherTok{\textless{}{-}}\NormalTok{ antes }\SpecialCharTok{{-}}\NormalTok{ despu}
\FunctionTok{qqnorm}\NormalTok{(dif, }\AttributeTok{pch=}\DecValTok{19}\NormalTok{, }\AttributeTok{main=}\StringTok{\textquotesingle{}\textquotesingle{}}\NormalTok{)}
\FunctionTok{qqline}\NormalTok{(dif)}
\end{Highlighting}
\end{Shaded}

\begin{figure}
\centering
\includegraphics{Manual_de_R_files/figure-latex/pesos1-1.pdf}
\caption{\label{fig:pesos1}QQplot para las diferencias de peso.}
\end{figure}

Se puede aplicar la prueba de normalidad Kolmogorov-Smirnov para estudiar si las diferencias \texttt{dif} provienen de una población normal, esto se puede realizar por medio del siguiente código.

\begin{Shaded}
\begin{Highlighting}[]
\FunctionTok{require}\NormalTok{(nortest)}
\FunctionTok{lillie.test}\NormalTok{(dif)}
\end{Highlighting}
\end{Shaded}

\begin{verbatim}
## 
##  Lilliefors (Kolmogorov-Smirnov) normality test
## 
## data:  dif
## D = 0.19393, p-value = 0.3552
\end{verbatim}

De la salida anterior se observa que el valor-P de la prueba es grande por lo tanto se puede asumir que las diferencias se distribuyen en forma aproximadamente normal.

En este problema interesa estudiar el siguiente conjunto de hipótesis.

\[H_0: \mu_{antes}  - \mu_{despues} = 0\]
\[H_1: \mu_{antes}  - \mu_{despues} > 0\]

El código para realizar la prueba es el siguiente:

\begin{Shaded}
\begin{Highlighting}[]
\FunctionTok{t.test}\NormalTok{(}\AttributeTok{x=}\NormalTok{antes, }\AttributeTok{y=}\NormalTok{despu, }\AttributeTok{alternative=}\StringTok{"greater"}\NormalTok{, }\AttributeTok{mu=}\DecValTok{0}\NormalTok{, }
       \AttributeTok{paired=}\ConstantTok{TRUE}\NormalTok{, }\AttributeTok{conf.level=}\FloatTok{0.95}\NormalTok{)}
\end{Highlighting}
\end{Shaded}

\begin{verbatim}
## 
##  Paired t-test
## 
## data:  antes and despu
## t = 8.3843, df = 9, p-value = 7.593e-06
## alternative hypothesis: true difference in means is greater than 0
## 95 percent confidence interval:
##  13.2832     Inf
## sample estimates:
## mean of the differences 
##                      17
\end{verbatim}

De la prueba se obtiene un valor-P pequeño, por lo tanto, podemos concluir que el peso \(\mu_{antes}\) es mayor que \(\mu_{despues}\), en otras palabras, la dieta si ayudó a disminuir el peso corporal.

\hypertarget{aproxint}{%
\chapter{Aproximación de integrales}\label{aproxint}}

En este capítulo se mostrará cómo aproximar integrales en una y varias dimensiones.

\hypertarget{aproximaciuxf3n-de-laplace-unidimensional}{%
\section{Aproximación de Laplace unidimensional}\label{aproximaciuxf3n-de-laplace-unidimensional}}

Esta aproximación es útil para obtener el valor de una integral usando la expansión de Taylor para una función \(f(x)\) unimodal en \(\Re\), en otras palabras lo que interesa es:
\[ I = \int_{-\infty}^{\infty} f(x) d(x)\]
Al hacer una expansión de Taylor de segundo orden para \(\log(f(x))\) en su moda \(x_0\) el resultado es:
\[ \log(f(x)) \approx \log(f(x_0)) + \frac{\log(f)^\prime(x_0)}{1!} (x-x_0) + \frac{\log(f)^{\prime \prime}(x_0)}{2!} (x-x_0)^2 \]
El segundo término de la suma se anula porque \(\log(f)^\prime(x_0)=0\) por ser \(x_0\) el valor donde está el máximo de \(\log(f(x))\). La expresión anterior se simplifica en:
\[ \log(f(x)) \approx \log(f(x_0)) + \frac{\log(f)^{\prime \prime}(x_0)}{2!} (x-x_0)^2 \]
al aislar \(f(x)\) se tiene que

\begin{equation} \label{fx}
f(x) \approx f(x_0)  \exp \left( -\frac{c}{2} (x-x_0)^2 \right)
\end{equation}

donde \(c=-\frac{d^2}{dx^2} \log(f(x)) \bigg|_{x=x_0}\).

La expresión \ref{fx} se puede reescribir de manera que aparezca el núcleo de la función de densidad de la distribución normal con media \(x_0\) y varianza \(1/c\), a continuación la expresión

\[
f(x) \approx f(x_0) \frac{\sqrt{2 \pi / c}}{\sqrt{2 \pi / c}}  \exp \left( -\frac{1}{2} \left( \frac{x-x_0}{1/\sqrt{c}} \right)^2 \right)
\]
Así al calcular la integral de \(f(x)\) en \(\Re\) se tiene que:
\begin{equation} \label{aprox_laplace}
I = \int_{-\infty}^{\infty} f(x) d(x) = f(x_0) \sqrt{2 \pi / c}
\end{equation}

\hypertarget{ejemplo-75}{%
\subsection*{Ejemplo}\label{ejemplo-75}}
\addcontentsline{toc}{subsection}{Ejemplo}

Calcular la integral de \(f(x)=\exp \left( -(x-1.5)^2 \right)\) en \(\Re\) utilizando la aproximación de Laplace.

Primero vamos a dibujar la función \(f(x)\) para ver en dónde está su moda \(x_0\).

\begin{Shaded}
\begin{Highlighting}[]
\NormalTok{fun }\OtherTok{\textless{}{-}} \ControlFlowTok{function}\NormalTok{(x) }\FunctionTok{exp}\NormalTok{(}\SpecialCharTok{{-}}\NormalTok{(x}\FloatTok{{-}1.5}\NormalTok{)}\SpecialCharTok{\^{}}\DecValTok{2}\NormalTok{)}
\FunctionTok{curve}\NormalTok{(fun, }\AttributeTok{from=}\SpecialCharTok{{-}}\DecValTok{5}\NormalTok{, }\AttributeTok{to=}\DecValTok{5}\NormalTok{, }\AttributeTok{ylab=}\StringTok{\textquotesingle{}f(x)\textquotesingle{}}\NormalTok{, }\AttributeTok{las=}\DecValTok{1}\NormalTok{)}
\end{Highlighting}
\end{Shaded}

\begin{figure}
\centering
\includegraphics{Manual_de_R_files/figure-latex/aprox1-1.pdf}
\caption{\label{fig:aprox1}Perfil de la función f(x).}
\end{figure}

Visualmente se nota que la moda está cerca del valor 1.5 y para determinar numéricamente el valor de la moda \(x_0\) se usa la función \texttt{optimize}, los resultados se almacenan en el objeto \texttt{res}. El valor de la moda corresponde al elemento \texttt{maximum} del objeto \texttt{res}.

\begin{Shaded}
\begin{Highlighting}[]
\NormalTok{res }\OtherTok{\textless{}{-}} \FunctionTok{optimize}\NormalTok{(fun, }\AttributeTok{interval=}\FunctionTok{c}\NormalTok{(}\SpecialCharTok{{-}}\DecValTok{10}\NormalTok{, }\DecValTok{10}\NormalTok{), }\AttributeTok{maximum=}\ConstantTok{TRUE}\NormalTok{)}
\NormalTok{res}
\end{Highlighting}
\end{Shaded}

\begin{verbatim}
## $maximum
## [1] 1.499997
## 
## $objective
## [1] 1
\end{verbatim}

Para determinar el valor de \(c\) de la expresión \ref{aprox_laplace} se utiliza el siguiente código.

\begin{Shaded}
\begin{Highlighting}[]
\FunctionTok{require}\NormalTok{(}\StringTok{"numDeriv"}\NormalTok{)}
\NormalTok{constant }\OtherTok{\textless{}{-}} \SpecialCharTok{{-}} \FunctionTok{as.numeric}\NormalTok{(}\FunctionTok{hessian}\NormalTok{(fun, res}\SpecialCharTok{$}\NormalTok{maximum))}
\end{Highlighting}
\end{Shaded}

Para obtener la aproximación de la integral se usa la expresión \ref{aprox_laplace} y para tener un punto de comparación se evalua la integral usando la función \texttt{integrate}, a continuación el código.

\begin{Shaded}
\begin{Highlighting}[]
\FunctionTok{fun}\NormalTok{(res}\SpecialCharTok{$}\NormalTok{maximum) }\SpecialCharTok{*} \FunctionTok{sqrt}\NormalTok{(}\DecValTok{2}\SpecialCharTok{*}\NormalTok{pi}\SpecialCharTok{/}\NormalTok{constant)}
\end{Highlighting}
\end{Shaded}

\begin{verbatim}
## [1] 1.772454
\end{verbatim}

\begin{Shaded}
\begin{Highlighting}[]
\FunctionTok{integrate}\NormalTok{(fun, }\SpecialCharTok{{-}}\ConstantTok{Inf}\NormalTok{, }\ConstantTok{Inf}\NormalTok{)  }\CommentTok{\# Para comparar}
\end{Highlighting}
\end{Shaded}

\begin{verbatim}
## 1.772454 with absolute error < 1.5e-06
\end{verbatim}

De los anteriores resultados vemos que la aproximación es buena.

\hypertarget{curio}{%
\chapter{Curiosidades}\label{curio}}

En este capítulo se mostrarán algunos procedimientos de R para solucionar problemas frecuentes.

\hypertarget{cuxf3mo-verificar-si-un-paquete-no-estuxe1-instalado-para-instalarlo-de-forma-automuxe1tica}{%
\section{¿Cómo verificar si un paquete no está instalado para instalarlo de forma automática?}\label{cuxf3mo-verificar-si-un-paquete-no-estuxe1-instalado-para-instalarlo-de-forma-automuxe1tica}}

Muchas veces compartimos código de R con otros colegas y si ellos no tienen instalados ciertos paquetes el código no funcionará. Para evitar ese problema podemos colocar al inicio del código unas líneas que chequeen si ciertos paquetes están instalados o no, si están instalados, se cargan esos paquetes y caso contrario, el código instala los paquetes y luego los carga, todo de forma automática sin que el usuario tenga que identificar los paquetes que le faltan.

\hypertarget{ejemplo-76}{%
\subsection*{Ejemplo}\label{ejemplo-76}}
\addcontentsline{toc}{subsection}{Ejemplo}

El código mostrado abajo revisa si los paquetes \texttt{knitr}, \texttt{png} y \texttt{markdown} están instalados e instala los ausentes y luego carga todos los paquetes que estén en el vector \texttt{packages}.

\begin{Shaded}
\begin{Highlighting}[]
\NormalTok{packages }\OtherTok{\textless{}{-}}  \FunctionTok{c}\NormalTok{(}\StringTok{"knitr"}\NormalTok{, }\StringTok{"png"}\NormalTok{, }\StringTok{"markdown"}\NormalTok{)}
\NormalTok{package.check }\OtherTok{\textless{}{-}} \FunctionTok{lapply}\NormalTok{(packages, }\AttributeTok{FUN =} \ControlFlowTok{function}\NormalTok{(x) \{}
  \ControlFlowTok{if}\NormalTok{ (}\SpecialCharTok{!}\FunctionTok{require}\NormalTok{(x, }\AttributeTok{character.only =} \ConstantTok{TRUE}\NormalTok{)) \{}
    \FunctionTok{install.packages}\NormalTok{(x, }\AttributeTok{dependencies =} \ConstantTok{TRUE}\NormalTok{)}
    \FunctionTok{library}\NormalTok{(x, }\AttributeTok{character.only =} \ConstantTok{TRUE}\NormalTok{)}
\NormalTok{  \}}
\NormalTok{\})}
\end{Highlighting}
\end{Shaded}


  \bibliography{book.bib,packages.bib}

\backmatter
\printindex

\end{document}
